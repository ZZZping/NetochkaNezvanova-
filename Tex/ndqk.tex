\documentclass[12pt, UTF8]{ctexbook}
\usepackage{geometry}
\usepackage[all, PDF]{xy}
\usepackage{indentfirst}
\usepackage{graphicx}
\usepackage{subfigure}
\usepackage{xcolor}
\usepackage{soulutf8}

% subtitle
% Неточка Незванова
\newcommand\subtitle[1]{{\normalsize #1}} %为了保险,最好使用两层大括号

% use russian for author name
\usepackage[OT2,T1]{fontenc}

\setlength{\parindent}{2em}

\geometry{a4paper,centering,scale==0.75, left=2.5cm,right=2cm,top=2.54cm,bottom=2.54cm}

\begin{document}
\title{涅朵奇卡·涅茨瓦诺娃 \\ \subtitle{\fontencoding{OT2}\selectfont Netoqka Nezvanova}}
\author{陀思妥耶夫斯基 \\ \small\fontencoding{OT2}\selectfont F\"edor Miha\u{i}luvhq Dostoevski\u{i}}
\maketitle
\newpage

\section*{一}
\par 我记忆中没有我的生父的印象。他死的时候我才两岁。我母亲又嫁了别人。这次再醮给她带来了很多痛苦,尽管她改嫁是出于爱情。我的继父是个乐师。他的命运很不寻常:这是我认识的人中间最古怪、最奇特的一个。在我童年时代最初的印象中,他留下的痕迹太深刻了,这对我一生都有影响。为了便于理解我要讲的故事,我先在此概述一下他的履历。下面我要讲的一切,都是后来我从大名鼎鼎的小提琴家B那里知道的,他是我继父年轻时的伙伴和密友。
\par 我的继父姓叶非莫夫。他出生于一位非常有钱的地主的村庄,继父的父亲是个穷乐师,度过漫长的漂泊生涯之后在这位地主的庄上落了户,受雇假如他的乐队。这位地主生活极其阔绰,平生最爱音乐,而且爱的成瘾。据说,他从来不离开自己的村庄,连莫斯科也不去,可是有一回突然出国到一处矿泉疗养地去了,而且只去了几个星期,唯一的目的就是去听一位赫赫有名的小提琴家的演奏,因为报上说他要在那个疗养地举行三场音乐会。这位地主拥有一个相当不坏的私人乐队,他几乎把全部收入都花在这上头。我的继父刚进这个乐队时吹单簧管。他在二十二岁那年结识了一个奇怪的人物。在他们那个县份,住着一位富有的伯爵,可是他为了养一个私人戏班子不惜于倾家荡产。这位伯爵因为自己的意大利出生的乐队长行为不端而把他辞退了。乐队长的品性确实不好。他被解雇以后,更是潦倒不堪,老是在乡下小酒店里喝的酩酊大醉,有时候索性祈求施舍,全省谁也不愿意给他一个职位。我的继父竟跟这样一个人交了朋友。这种奇怪的交往实在难以解释,因为谁也看不出我的继父由于学朋友的样在行为方面有什么变化,甚至起初不准他跟那个意大利人厮混的地主,后来对他们的友谊睁一只眼闭一只眼。最后,乐队长突然死了。他是清晨被农民在堤坝旁边的水沟里面发现的。经过验尸,确定他死于中风。他的遗物存放在我的继父那里,我继父当即出示文件,证明他有充分的权力继承这些遗物,因为死者留下一张亲笔所写的字条,指定叶菲莫夫为自己遗产的继承人。遗产包括一件黑色燕尾服和一把小提琴:燕尾服由死者保存的很仔细,因为他始终抱有觅得一席之位的希望;小提琴看上去却很平常。没有人对这比遗产提出什么争议。可是过了若干时日,伯爵乐队里的首席小提琴手带着铂爵的信来见地主。伯爵在信上于叶菲莫夫情商,却他让出意大利人身后留下的那把提琴,因为伯爵很想把它买下来给自己的乐队使用。伯爵愿意出三千卢布,还说已派人去请过叶果尔·叶菲莫夫多次,以便当面了结这笔交易,但他执意不肯。伯爵最后写道,提琴货真,但他的价钱也实足不假,绝不会少一个子儿,并认为叶菲莫夫的顽固是一种多疑的表现,生怕成交是欺负他老实和外行,所以伯爵动了气,请地主开导开导叶菲莫夫。
\par 地主派人把我继父叫去。
\par “你为什么不肯出让提琴?”他问道。“你又用不着它。人家出你三千卢布,这是十足的价钱,要是你以为人家会出更高的价钱,可就错了,伯爵不会欺骗你的。”
\par 叶菲莫夫回答着说,他自己不想去见伯爵,如果要他去,那就只能按主人的意志办;提琴他不愿意卖给伯爵,如果硬要从他这里把提琴抢走,那也只能按住人的意志办。
\par 很明显,他还祥的回答触到了地主性格中最敏感的一根弦。事情是这样的. 地主一向自豪地说他懂得怎样对待他的乐师, 因为他们个个都是真正的艺术家,因此他的乐队并不但笔伯爵的高明,甚至同京城里的乐队相比也不逊色。
\par“好!” ,地主说。“我通知伯爵, 说你不愿卖琴就是不愿, 因为卖与不卖的权利完全在你,懂呜?不过我要问你:你要提琴干嘛? 你的乐器是单簧管. 虽然你的单簧管吹得相当蹩脚。把琴让给我吧。我出三千卢布。(谁知道此琴是这样的一件乐器!)”
\par 叶菲莫夫莞尔一笑。
\par “不,老爷,我不卖给您,”他答道,“当然,您可以······”
\par “难道我在逼你,难道我在强迫你?!”地主终于沉不住气叫了起来,偏偏事情是当着伯爵的乐师之面发生的,来者据此情景可能推断:地主对他的乐队的全体乐师都不给好看待。“滚开,没良心的东西!从今以后别让我再看见你!要是没有我,冲你那支吹得这样糟糕的单簧管,你能上哪儿混饭吃?你在我这里有吃有穿,还领薪俸;你过的是上等人的日子,把你当艺术家看待,可你根本不想明白这一点,简直无知无觉。滚开,别待在此地招我生气!”
\par 地主总是把他生气的对象从自己身边赶走,因为对自己不放心,怕他的火爆性子发作。而他是说什么也不愿意对他的“艺术家”过于严历的(他管自己的乐师们都叫“艺术家”)。
\par 买卖没有成交,事情似乎到此为止了,不料一个月以后,伯爵的小提琴手忽然大启讼端:他本人出首告发我继父应对意大利人之死负责,说我继父怀着自私的目的把他致于死地,为的是占有价值可观的遗产。伯爵的乐师声称遗嘱是在硬逼软骗之下写出来的,并表示能为这项指控提供人证。伯爵劝之再三,地主也为我的继父说情,但什么也不能动摇告发者的主意。人家把情况摊在他面前:法医对乐队长尸体所做的检验是正确的,硬要告发岂非违背明摆着的事实,也许是因为得不到曾经为他洽购的那件名贵乐器而怀恨在心,想泄私愤。伯爵的乐师一意孤行,还赌神罚咒地说自己是对的,说中风并非酗酒引起,而是中毒造成的,要求复查。乍看起来,他的论点似乎颇有道理。不用说,事情还是闹开了。叶菲莫夫被抓起来关进城里的监狱。这桩官司打起来以后,引起了全省的注意。案件进展很快,结果查明伯爵的乐师犯有诬告罪。判决给了他应得的惩罚,但他始终坚持自己的看法是对的。最后,他承认自己没有任何证据,他提出的论点是他自己臆造的,但他捏造所有这些事实是根据一种假设、一种猜想行事的,因为直到复查结束、正式确认叶菲莫夫先生无罪为止,他还坚信不幸的乐队长之死是叶菲莫夫造成的。不过,对此人的判决没有来得及执行,因为他突然患脑炎发了疯,接着就死在监狱医院里。
\par 整个这桩公案的始末,地主的行为是光明磊落的。他为我继父出力,仿佛我继父是他的亲生儿子似的。他曾几次到监狱里去探望我的继父,安慰他,给他钱;得悉叶菲莫夫喜欢抽烟,就给他带去最好的雪茄;宣告我继父无罪释放时,他让乐队全体成员大事庆贺。地主把叶菲莫夫这个案子看作是关系到整个乐队的事情,因为他对自己的乐师的品行即使不比他们的才能看的更重,至少是不相上下。过了整整一年,忽然有消息在省里传开,说是一位著名的法国小提琴家途径省城,打算举行几场音乐会。地主马上开始设法请他来做客。事情进行的很顺,法国人答应前来。对他的来临已经做好一切准备,还邀请了县里几乎所有的知名人士,不料情况陡起变化。
\par 一天早晨,有人来报告说,叶菲莫夫不知去向。开始到各处去寻找,可是杳无踪影。乐队缺了一只单簧管岂不急死人?在叶菲莫夫失踪三天之后,地主忽然接到法国人写来的信,那位小提琴家在信中傲慢地拒绝了地主的邀请,并且表示(当然是拐弯抹角地),今后在跟那些私人拥有乐队的老爷们打交道时要格外谨慎,看到真正的天才处在根本不知道其价值地人监督之下实在太杀风景,临了还说,叶菲莫夫的例子足以证实他所言不虚,此人是位真正地艺术家,是他在俄国遇到的最好的小提琴演奏家。
\par 地主读了这封信大为惊愕。他这一气直气的他发昏章第十一。什么?他对叶菲莫夫如此关怀备至,如此厚爱有加,而就是这个叶菲莫夫竟昧着良心向一位欧洲的艺术家,向他高度敬重其见解的人如此恶毒地诋毁他!此外,这封信在另一点上也是令人不解的;信中说,叶菲莫夫是个有真正天才的艺术家,他是位小提琴家,可是人们竟不能发现他的才华,强迫他演奏另一种乐器。这一切使地主惊讶万分,他当下准备进城去面晤法国人,忽然伯爵派人送来一封便简,请地主立刻到他那里去,并说他了解全部情况,那位路过的演奏家此时在他家里,叶菲莫夫也在,伯爵闻说叶菲莫夫的诽谤后大为震惊,已下令不准他离开,信上最后说,之所以必须请地主前去,还因为叶菲莫夫的指责甚至涉及伯爵本人;兹事体大,必须尽快加以澄清云云。
\par 地主马上赶往伯爵家中,随即同法国人见面,向他介绍了我继父的全部身世,并说他没料想到叶菲莫夫竟有这般了不起的才华,相反,叶菲莫夫在他那里只是一个很蹩脚的单簧管手,现在他头一回听说,这个离他而去的乐师竟是位小提琴家。地主还说,叶菲莫夫并不是农奴,他享有完全的自由,如果他确实受到束缚,任何时候都可以离开他家。法国人感到很奇怪。他们把叶菲莫夫叫来,他简直跟过去判若两人:态度傲慢,答话带着讥笑,并坚持自己向法国人所说的情况属实。这一切使伯爵恼怒到了极点,他当面骂我的继父是混蛋、造谣中伤的小人,应该得到最可耻的下场。
\par “请放心,伯爵大人,敝人跟阁下并非初次相交,对阁下颇有了解,”我的继父说,“多蒙阁下恩典,在下几乎受到刑事处分。敝人知道,阿列克塞·尼基福雷奇——府上过去的乐师——告发在下是受了何人的嗾使。”
\par 听到这样骇人听闻的责难,伯爵肺都快气炸了。他好不容易才控制住自己;但此时厅堂里一位有事来见的伯爵的官员宣称他不能听任这一切不了了之,说叶菲莫夫这种侮辱性的物理态度包含着恶毒的、不公正的指控和污蔑,所以他谨请允许他在伯爵府第里立即逮捕叶菲莫夫。法国人也表示极度的慷慨,说他无法理解这种丧尽天良的负义行为。于是我的继父暴跳如雷地回答,他宁可接受处分、审判,哪怕再来一次刑事侦讯,也强似迄今为止他在地主乐队里过的那种日子,由于他极度贫困,不能更早离开那里;说完,他就跟逮捕他的官员一起走出厅堂。他被锁在一间偏僻的屋子里,说是明天把他押送进城。
\par 将近午夜时分,拘押我继父物的屋子门打开了。进来的是地主。他穿着睡袍,趿着拖鞋,双手打着一盏点亮的灯笼。看来他睡不着,痛心的焦虑迫使他在这般时分离开床衾。叶果尔也没睡,他惊讶地望着进来的地主。地主把灯笼放好,怀着十分激动的心情坐在他对面的一把椅子上。
\par “叶果尔,”他对叶菲莫夫说,“你为什么要这样伤我的心?”
\par 叶菲莫夫不答。地主又问了一遍,他的话流露出某种深刻的感情,一种奇怪的忧伤。
\par “天知道我为什么要这样伤您的心,老爷!”我继父终于一甩手答道。“想必是鬼迷住了我的心窍!我自己也不知道是谁推动着我这样干!反正我不能在您那儿再待下去,不能再待下去······。魔鬼把我缠住了!”
\par “叶果尔!”地主又开言道。“回到我那儿去吧,我把一切都忘掉,什么都原谅你。听着:你可以当我的首席乐师;我给你一份跟旁人不一样的薪金······”
\par “不,老爷,不,您别说了:我不能在您那儿待下去!我告诉您,魔鬼缠上了我。我要是再待下去,会放火烧掉您的房子;有时候我苦闷得只恨爹娘不该把我生下来!现在我自己也不能担保,老爷,您还是别管我吧。这都是从那个魔鬼跟我结交后开始的······”
\par “谁?”地主问。
\par “就是那个像狗一样咽了气的意大利人,这条狗导出都不受欢迎。”
\par “这么说,叶果鲁什卡,是他教你拉琴的喽?”
\par “是的!他教会了我许多东西,把我引向毁灭。我还是从来没有见到他的好。”
\par “难道他是个小提琴高手,叶果鲁什卡?”
\par “不,他自己知道的不多,可是教的挺好。我是自己学会的;他只不过是示范而已,——这比正规的办法容易,要是我撒谎,就让我这支胳臂烂掉。现在我自己也不知道要什么。老爷,您问我:‘叶果尔卡!你要什么?我什么都能给你,’——可是,老爷,我一句话也没法回答您,一位我自己不知道要什么。老爷,您还是别管我吧,下次再说。我要对自己干一件惊人的事情,好让我被远远地打发走,事情才能了结!”
\par “叶果尔!”地主在沉默片刻后说。“我不能这样撇下你不管。既然你不愿意在我那里干下去,你可以走;你是自由人,我不能强留;但我现在不肯就这样离开你。你用你的琴拉一首曲子给我听听,叶果尔,拉吧!看在上帝的份上,拉一首!我不是命令你,你要明白我的意思,我不强迫你;我含着眼泪请求你:叶果鲁什卡,看在上帝的份上,把你为法国人演奏的曲子也拉给我听听!吐一吐你的心曲!你很固执,我也很固执;要知道我也有自己的脾气,叶果鲁什卡!我心中是有你的,你心中也该像我一样才对。除非你自愿且乐意地把为法国人演奏的曲子拉给我听,否则我日子没法过。”
\par “行,那就照办!”叶菲莫夫说。“老爷,我发过誓永远不在您面前演奏,单单不为您演奏,可现在我心上的束缚解除了。我可以为您演奏不过这是第一次,也是最后一次,老爷,以后您在任何时候、任何地方再也听不到我的演奏,哪怕向我许一千卢布的愿也没用。”
\par 于是他拿起琴来,开始演奏他自己所作的俄罗斯民歌变奏曲。据B说,这套变奏曲是他的第一部、也是写的最好的小提琴作品,此后他任何曲子从来都没有演奏的这样好,这样充满灵感。这位地主听音乐本来就没法不动心,这回索性放声痛哭。演奏完毕时,他从椅子上站起来,掏出三百卢布交给我的继父,说:
\par “现在你走吧,叶果尔。我把你从这里放出去,伯爵那一头我去对付;不过你听着:往后你可别跟我遇上。你面前的道路宽又广,万一咱俩在路上碰头,对我对你都不好受。好啦,那就分手吧!······等一下!我对你还有一句临别赠言,很简单:别喝酒,要用功,莫骄傲!我是像你的亲生父亲一样对你说话。注意,我重复一下:要用过,别喝酒,一旦你开始借酒浇愁(叫人愁苦的事将来多的很!)——那就前功尽弃,非完蛋不可,也许你自己会像那个意大利人一样在不定什么地方的水沟里咽气。好啦,现在分手吧!······等一等,吻我一下!”
\par 他们相互吻了一下,随后我的继父就被放出去。
\par 他刚获得自由,先是立即在附近一个县城里把三百卢布胡乱花光,同时跟一帮最堕落、最下流的无赖交上朋友,后来一个人落得穷愁潦倒、孤苦无依,不得不加入一个小地方的流动戏班子,在可怜巴巴的乐队里拉第一个小提琴,或许是唯一小提琴。这一切跟他原先的设想不太吻合,他本来打算尽快到彼得堡去深造,谋到一个好位子,不折不扣地把自己造成一个艺术家。但是,小乐队里的生活很不如意。我继父不久就跟江湖戏班子的班主闹翻,并且离开了那里。那时他彻底泄了气,甚至不顾一切走出了深深刺痛他的自尊心的一步。他写了一封信给前面提到的那位地主,向它描述了自己的境况,请他资助。信的语气还相当要面子,但是没有回音。于是他再写一封,措辞上极尽卑躬屈膝之能事,称地主为自己的恩人,把他尊为真正的艺术鉴赏家,目的还是求他帮助。回音总算来了。地主寄来一百卢布和由他贴身侍从代笔的寥寥数行复信,叫叶菲莫夫今后再也不要向他提出任何请求。继父得到了这点钱,当即想动身去彼得堡,可是付账还债以后,只剩下那么一点点钱,彼得堡之行根本无法考虑。他仍旧留在小地方,重新加入一个小地方乐队,后来在那里又待不下去,如此不断地换来换去,心中念念不忘能快一点去彼得堡,其实却在小地方跑了整整六年。后来,他忽然大起恐慌。他绝望地发现,在不规则的、贫困的生活不断折磨下,他的才华遭到了不知多大的损失,于是在一个早上抛下班主,拿起提琴,几乎靠乞讨走到彼得堡。他在某处的一个楼顶上住下,在那里第一次遇到了B,彼时B刚从德国来,也想为自己开辟前程。他们很快就交了朋友,B至今还满怀深情回忆他们相交一场。他俩都是青年,怀抱一样的希望,有着相同的目标。但B的青春还刚刚开始,他经历的贫困和苦痛尚少;撇开这些不谈,他首先是一个日耳曼人,在奔向目标的道路上能坚持不懈、持之以恒,充分意识到自己的力量,并且对于自己能有多大的作为几乎早有成竹在胸。可是他的伙伴叶菲莫夫已经三十岁;他已经疲倦、困乏,整整七年不得不东飘西泊,在小地方的戏班子和地主的私人乐队里混口饭吃,耐心既完全丧失,最初旺盛的精力也消耗殆尽。过去支撑着他的只有一个永远不变的固定观念——好歹得摆脱窘境,积一笔钱到彼得堡去。但这个观念事模糊的、朦胧的;这是一种不可违拗的内心的召唤,随着岁月的流逝,这呼声在叶菲莫夫心中也不象最初那样清晰了,当他来到彼得堡时,几乎已经处于无意识状态,只是按照夙愿和反复思量这次进京的老习惯行事而,几乎连自己也不知道要在京城里干什么。他的热情近似歇斯底里,带有怄气和阵发的性质,似乎他想用这种热情欺骗自己,借以使自己相信,他身上最初的精力、最初的热情、最初的灵感尚未枯竭。这股不断迸发的劲头个不冷不热、有条有理的B震动很大;他感到目眩神迷,把我继父当作未来伟大的天才音乐家看待。他不能想象这位伙伴将来的命运会是什么别的样子。但不久B就睁开眼睛把他看透了。他清楚地看到,所有这些阵发地狂热、焦躁的情绪无非是想到自己怀才不遇而不自觉地表现出来的绝望的挣扎;说到底,甚至他的才也许一开始就没有什么了不起,而多是盲目和不切实际的自信、浅薄的自负以及不断幻想自己是盖世奇才的白日梦。“但是,”B如此说,“我对我这位伙伴奇异的天性不能不表示惊讶。我眼看着在病态的强烈欲望与内心的软弱无力之间不断进行凶猛的生死搏斗。这个不幸的人整整七年光靠幻想将来成名聊以自慰,甚至在不知不觉中丢掉了我们这些技艺中最起码的东起,甚至丧失了最基本的业务能力。偏偏在他乱糟糟的想象中无时无刻不在为未来构思气吞山河的宏大计划。他不唯要成为第一流的天才,成为世界上数一数二的小提琴家;他不唯已经把自己当作这样的天才,——他还想成为作曲家,事实上他根本不懂得对位法。但最使我惊讶的是,”B又说,“这个人尽管绝对无能,尽管在技艺方面只是极其贫乏,然而他对艺术却有非常深刻、非常明晰、可以说是出于本能的立即。他的艺术感和鉴赏力是那样高超,无怪乎会失去自知之明,看不到本能决定他是个深刻的艺术批评家,却把自己当作艺术大师、天才演奏家。有时候他用毫无学术味道的粗言俗语能对我说出极其深刻的道理,使我大惑不解:他从来不看书报,什么也不学,可是这一切他是通过什么方式领悟的呢?我在自修过程中,”B接着说,“有许多地方得益于他和他的指点。至于我本人,”B继续说,“我对自己的看法是稳定的。我也热爱本行艺术,虽然从我一开始走上这条道路就有自知之明,知道在某种意义上我只能当一名艺术的苦力;但我感到自豪的是没有象一个懒惰的奴隶那样把自己的拿点天赋埋没掉,相反把它扩到了一百倍;如果说,现在人们夸奖我演奏时干净利落,惊叹功夫到家,那末,这一切都得归功于坚持不懈的努力,归功于有自知之明,宁可把自己看的渺小,永远力戒骄傲,力戒过早的自满,力戒懒惰,因为懒惰是这种自满情绪的必然结果。”
\par B也曾尝试向自己最初非常佩服的伙伴进几句忠言,但结果只是白白惹他生气。他们之间的关系开始疏远了。不久B注意到,淡漠、苦闷和无聊开始愈来愈频繁地控制叶菲莫夫,而他热情冲动的次数则愈来愈少,过后出现的是一种阴沉凄凉、灰心丧气的状态。再后来,叶菲莫夫干脆把琴放下,有时一连几个星期不去碰它。这与彻底的堕落已相去不远,很快,这个不幸的人染上了所有的恶习。当初地主告诫他的事情果然发生了,他开始纵酒无度。B瞧着他无法不感到震惊,他的忠告不起作用,他甚至不敢开口。渐渐地,叶菲莫夫落到恬不知耻的地步:他竟心安理得地花B的钱过日子,这样做甚至好像有充分权力似的。其时维持生活的钱行将告罄;B靠教课竭力苦撑,或者受雇在商人、日耳曼人、小官吏家的晚会上演奏,尽管报酬很少,但他们总能给几个钱。叶菲莫夫根本不愿看到伙伴的难处,对他疾言厉色,有时几个星期不跟他说一句话。一次,B用及其婉转的语气劝他不要过于轻视那把琴,以免对此乐器完全荒废;不料叶菲莫夫大发脾气,声言他将故意永远不去碰自己的琴,那副架势好像会有人跪下来求他似的。另一次,B在一个晚会上演奏需要有个搭档,就请叶菲莫夫跟他合作。这一邀竟使叶菲莫夫暴跳如雷。他怒气冲冲地宣称自己不是街头提琴师,不会像B那样卑鄙,在完全不能赏识他的功夫和才华地臭商人面前拉琴,贬低崇高的艺术。B听罢没回答一句话,出门演奏去了,但叶菲莫夫在伙伴走后对这次邀请反复思考,认为这一切无非暗示他在花B的钱过日子,想让他知道,叫他尝试挣点钱。等B回来以后,叶菲莫夫忽然斥责他行为卑鄙,并表示一分钟也不能跟他待在一起。他确实有两天不知去向,但第三天回来了,仿佛什么也没有发生似的,又继续过原来地那种生活。
\par B本来想结束这种不像话的生活,跟他的伙伴一刀两断,仅仅由于旧的习惯和友谊,加上B瞧着那个堕落的人心中老大不忍,才没有这样做。后来,他们终于分道扬镳。命运向B作了微笑:他找到一座强有力的靠山,成功地举行了一场出色的音乐会。那时他已经是个优秀的艺术家,他那蒸蒸日上的名气旋即给他带来歌剧院乐队里面的一个席位,在那里他很快就取得了完全应该取得的成功。分手的时候,他给了叶菲莫夫一些钱,含着眼泪恳求他回到正路上来。直到现在,B想起他来还是抑制不住一种特殊的感情。跟叶菲莫夫相交一场是他青年时代最深刻的印象之一。他们曾一起开始向自己的目标进军,彼此曾有过非常热烈的好感,叶菲莫夫的乖张性格和十分显著的缺点本身甚至使B对他产生更加强烈的感情。B了解他;他把叶菲莫夫看得透亮,事先就知道这一切将以上面告终。分袂之际,他们相互拥抱,两人都哭了。当时叶菲莫夫流着眼泪呜呜咽咽地说,他是个彻底完蛋的,最最不幸的人,这一点他早就知道了,但现在才看清自己的末路。
\par “我没有才华!”他的除了结论,说时面无人色。
\par B大大地为之心动。
\par “听着,叶果尔·彼得罗维奇,”他对我继父说,“你何苦自暴自弃呢?你的绝望只能毁了你自己;你既没有耐性,又没有勇气。刚才你在灰心丧气的情绪控制下说自己没有才华。不对!你有才华,你可以相信我的话。你有才华。单凭你对艺术的感受和理解,我就看出这一点。我可以用你的全部生活向你证明这一点。你不是把你过去的经历告诉过我吗?当初你也曾不自觉地陷于同样绝望地境地。那时,你的第一位老师——你曾经对我讲过他很多故事的那个怪人——最先在你身上激发起对艺术的爱,最先察觉到你的才华。当时你也产生了强烈而痛心的感觉,就跟现在一样。但你自己不知道你是怎么搞的。你在地主家里待不下去,你自己也不知道你到底要什么。你的老师死得太早。他撇下你的时候你只有一些模模糊糊的志向,主要的是没有使你认清自己。你觉得你需要走另一条更宽广的路,应该向另外的目标进军,可是你不懂得该如何去达到目的,于是在苦闷中痛恨当时你周围的一切。你六年贫困的岁月没有虚度;你在学,在想,在认识自己和自己的能力,现在你对艺术,对自己的使命有了理解。我的朋友,需要耐性和勇气。等待着你的命运要比我的更值得羡慕:你的艺术家气质超过我一百倍,只要上帝把我的耐性的十分之一赐给你就够了。要用功,别喝酒,正像那位好心的地主对你说过的那样,主要的是你得重新起步,从头开始。什么东西使你苦恼?贫困?但贫困能造就艺术家。事业的起点总是和穷字分不开的。现在还没有人把你放在眼里,谁都不愿意认识你;人世间就是这样。你等着,一旦人们知道你有才能,可气的事情还不止这些呢。妒忌、卑劣的小心眼儿、特别是种种荒唐的蠢事将比贫穷更加变本加厉地往你身上压下来。才华需要同情,需要有人理解,可是你知道取得一点点成就,你会看到,包围着你的将是些什么样的面孔。他们会把你靠艰苦的劳动、咬紧牙关、挨饿熬夜练出来的功夫说的一文不值,对你嗤之以鼻。你未来的伙伴们不会鼓励你、安慰你;他们不会向你指出你的真善美,但会幸灾乐祸地挑你的每一处毛病,偏偏把你不好和不对的地方指向你,表面上冷冰冰地瞧不起你,心里却象过节一样庆祝你犯的每一个错误(好像有人能不犯错误似的!)。你生性傲慢,往往在不适当的场合逞骄,可能会得罪自尊心很强的小人,那就糟了——你只有一个人,而他们人多;他们会像针刺那样折磨你。甚至我也已经开始有此感受。你得立刻振作精神!你还不算太穷,你的日子能过下去,不要嫌弃粗活,有柴就劈,象我在不足道地生意人晚会上劈柴一样。但你缺乏耐性,你有急躁的毛病;你不够朴实,耍小聪明太多,想得太多,脑筋动得太多;你口头上大言不惭,临到需要拿起琴弓地时候又胆怯了。你自尊心太强,胆量又太小。勇敢一些,耐性一些,耐心等待,学着点儿,如果你不寄望于自己的力量,那就碰碰运气;你身上有激情,有感觉。也许碰运气能达到目的,即使达不到,也不妨碰碰运气,反正不会失去什么,因为奖赏实在太大了。这个意义上说,老兄,咱们的运气是了不起的大事。”
\par 叶菲莫夫怀着很深的感情听过去伙伴这番话。但在B往下说的过程中,叶菲莫夫苍白的脸色逐渐泛红,两腮出现了血色;他的眼睛里燃起了少有的勇气和希望。不久,这种高尚的勇气转换为自负,接着又变成平时那份狂妄,最后,当B快要结束这番规劝的时候,叶菲莫夫已经心不在焉,听得不耐烦了。不过他还是热烈地和B握手,向他表示感谢,并且拿出从自暴自弃和灰心丧气迅速跃向极端傲慢狂妄那种老脾气,用过于自信的口气叫他的朋友不必为他的命运操心,说他知道该如何安排自己的未来,他希望不久也能找到一座靠山,举行音乐会,那时便可一下子名利双收。B耸耸肩膀,但没有给过去的伙伴泼冷水他们就此分手,虽然分别的时间不长——这是不言而喻的。叶菲莫夫把给他的钱立刻花的精光,又去要了第二次,然后是第三次,然后是第四次······第十次,最后B实在忍不住了,便推说不在家。从此叶菲莫夫下落不明。
\par 几年过去了。有一次B排练归来,在一条陋巷里肮脏的小酒店门口碰见一个衣衫褴褛的醉汉再叫他的名字。那人是叶菲莫夫。他大大变了样,脸上浮肿、发黄;显然,放荡的生活在他身上打下了不可磨灭的烙印。B非常高兴,没说上两句话,就被他拖进小酒店。到了那意见偏僻、乌黑的小屋子里,B才看清楚他的摸样。叶菲莫夫的衣着几乎全是破烂,一双靴子坏得不像话。衬衣的前胸沾满酒污。他的头发开始斑白、脱落。
\par “你怎么啦?你现在哪里?”B问。
\par 叶菲莫夫窘态毕露,起初甚至有些发慌,语无伦次,答非所问,以致B以为他神经错乱了呢。后来,叶菲莫夫承认,不喝一点伏特加他就没办法说话,可是小酒店里对他早就不相信了。他这样说时脸是红的,尽管竭力做一些豪放的手势给自己壮胆;但造成的印象却是厚颜无耻、娇柔做作,简直令人不忍卒睹;善良的B看到自己原先的忧虑果然完全成为事实,又动了恻隐之心。他先吩咐把伏特加端上来。叶菲莫夫由于感激脸上顿时变样,他激动得含着眼泪准备吻他恩人的手。进餐的时候、B惊讶万分地得悉这个不幸的人结了婚。但更使他惊讶的是了解到,妻子竟然构成了叶菲莫夫的全部灾难和悲哀,结婚彻底摧残了他的才华。
\par “怎么会的呢?”B问。
\par “老弟,我已经两年没拿起提琴了,”叶菲莫夫答道。“简直是个村妇、厨娘、毫无教养的粗俗女人。别提她了!······我们一天到晚打架,旁的什么也不干。”
\par “既然这样,你为什么要结婚?”
\par “当时没东西吃啊。我结识了她:她有千把卢布,我就不管三七二十一结了婚。是她爱上我的。她自己缠住我不放。又没人撺掇她!钱花完了、喝光了,老弟,哪儿还有什么才华!全都落了空!”
\par B看得出,叶菲莫夫似乎急于向他表白自己没有过错。
\par “我把一切扔下了,”叶菲莫夫又说。他向B表示,在这以前他的琴艺几乎达到了完美的境界;虽然从我一开始走上这条道路就有自知之明B是全城数一数二的小提琴家,可是跟他比起来还差得远呢,如果叶菲莫夫愿意的话。
\par “那你为什么不干呢?”B诧异地问。“你该找份事情做啊!”
\par “没意思!”叶菲莫夫一甩手说。“你们那里有谁懂得一点点皮毛?你们懂得什么?懂个屁!你们只会从什么芭蕾音乐中抽一段五去来胡闹一通。优秀的小提琴家你们既没有见过,也没有听过。何必去碰你们呢;你们爱怎么干,就怎么干吧!”
\par 说到这里,叶菲莫夫又把手一甩,身体在椅子上一晃,因为他已经醉得相当厉害。后来他邀请B上他家去;但是B谢绝了,只问了他的住址,答应明天就去看他。叶菲莫夫此刻酒醉饭饱,已经用嘲弄的目光瞧他过去的伙伴,千方百计用话刺他。他们离座起身时,叶菲莫夫连忙把B的贵重皮裘递给他,做出以卑事尊的样子。经过第一间屋子,他就停下来向酒店里的掌柜、伙计和客人介绍B是全城首屈一指和独一无二的小提琴家。总而言之,此时此刻他的表演十分令人恶心。
\par 第二天上午,B在顶楼上找到了他,当时我们全家就住在那么一间屋子里,过着极度穷困的生活。我那时只有四岁,可是我母亲改嫁叶菲莫夫已有两年。我母亲是个不幸的女人。从前她当过家庭教师,受过很好的教育,长得也漂亮,可是由于穷,嫁给了一个老公务员,也就是我的生父。他们在一起只过了一年。我的生父突然去世以后,为数无多的遗产由他的各个继承人瓜分,留下母亲和我,还有就是她分得的一点点钱。带着一个还不会走路的婴儿再去当家庭教师谈何容易。这时,她在一个偶然的机缘下遇到了叶菲莫夫,的确爱上了他。我母亲富于热情和幻想,把叶菲莫夫当作了不起的天才,相信了他前程似锦的那些自命不凡的话,想到有幸成为一个天才的精神支柱和生活指导,我母亲得意非凡,就嫁给了他。不出一个月,她的理想和希望统统化作泡影,她面前留下的只是寒碜的现实。叶菲莫夫跟我母亲结婚也许确实看在她有千把卢布这一点上,等到钱一花完,便叉起两支胳膊,似乎欣然找到一个借口,立即向所有的人宣称,结婚毁了他的才华,说他在窄闷的屋子里面对饥饿的一家子没法工作,说在这样的条件下头脑里不可能产生歌曲和音乐,看来他命中注定得受这份罪云云。后来,好像他自己也确信自己的牢骚发得有理,似乎还为有了新的口实感到高兴。看样子,这位不幸的、毁掉的天才自己在寻找表面的理由。以便把所有的挫折、所有的灾难一古脑儿推到那上头。至于接受这样一个可怕的念头:他在艺术上早已经完蛋,而且是永远地完蛋了——他做不到。他象抵抗恶梦似地向这个可怕的结论作拼死的挣扎,及至现实把他压倒、使他偶尔有儿分钟睁开眼睛的时候,他觉得自己马上就会吓得发疯。对于长期以来构成他全部生活的信念,他不能这样轻易地放弃,直到最后一分钟依然认为那一分钟还没有过去。在彷徨的时刻,他就借熏人的酒味浇胸中之块垒。可能,他自己也不知道这时候妻子对于他是多么需要。这是一个活的借口,的确,我继父对这样一个设想几乎着了魔,认为一旦他把坑了他的妻子葬入坟墓,一切都将走上正轨。可怜的妈妈不理解他。作为一个十足的幻想家,在冷酷无情的现实中刚迈出第一步,她就受不了,变得暴躁易怒,动辄骂人,时刻跟存心折磨她取乐的丈夫吵架,不断逼着他工作。但是,我继父的盲目自大、固定观念和荒谬见解使他变得几乎毫无心肝和麻木不仁。他只是笑着赌咒决不拿起提琴,除非妻子死去,并以狠心的直率态度向她宣布这一决心。不管出现什么情况,妈妈对他的炽热的爱至死不渝,可是这样的生活她忍受不了。她变得老是生病、多灾多难,身心处在没完没了的痛苦之中,陈了这一切不幸之外,一家人的吃饭问题全要她一个人操心。她开始做菜,承接包饭。但是丈夫偷偷地从她那儿把钱都弄走,逼得她常常开不出人家包的饭。当B来访的时候,她正在洗衣服,并且把一件旧的连衫裙重新染色。我们就一直这样在我们的顶楼上凑合着过。
\par 我们家的贫穷使B吃惊。
\par “你呀,完全是胡说八道,”他对我继父说,“这儿哪有什么摧残才华的事情?她明明在养活你,你在干什么?”
\par “什么也不干!”继父回答说。
\par 但B还不知道妈妈所有的灾难。丈夫经常把形形色色捣乱撒野的浪荡鬼一大帮一大帮地带到家里来,那时真可以说是无所不为!
\par B对他过去的伙伴劝说了半天,最后表示,如果叶菲莫夫再不改过,他决不提供任何帮助,还直截了当地说不给他钱,因为他又会把钱喝光,临了他要叶菲莫夫用提琴拉点儿什么给他听听,看看能为叶菲莫夫想些什么办法。等我继父去取琴的时候,B悄悄地要把钱给我母亲,可是她不拿。她这是第一次落到接受施舍的地步!于是B就把钱交给我,可怜的妈妈哭了。继父取了琴来,但要求先给他一点伏特加,说否则没法演奏。于是就派人去买伏特加。酒喝下去以后,劲头就上来了。
\par “看在朋友交情上,我给你拉一首我自己的作品。”他对B说,并从柜子底下拖出厚厚一本尘封的练习簿。
\par “这些都是我自己写的,”他指着本子说。“你不妨瞧瞧!老弟,这可不是你们的那些芭蕾音乐!”
\par B默默地翻阅了几页,然后打开他自己带着的乐谱,要我继父把自己的作品放在一边,从他带来的乐谱中拉几段给他听听。
\par 继父有些动气,不过由于担心失去新的靠山,还是照B的吩咐做了。于是B发现,在他们分手期间,他从前的伙伴确实下了很多功夫,进步不小,虽然他吹说打从结婚以后就没有拿过琴。我那可怜的母亲高兴得什么似的。她瞧着丈夫,重又为他感到自豪。善良的B也衷心欣喜,决定设法安置我继父。那时B已经认识许多有地位的人物,他当即开始去托人情,向人推荐这位可怜的伙伴,但事前要他保证好自为之。B先掏钱把他的衣着搞得象样些,带着他去见几位名人,因为B打算给他谋的位子关键在那些人物身上。叶菲莫夫只是口头上神气十足,其实恐怕他万分乐意地接受了老朋友的建议。据B说,我继父拚命讨好他,生怕失去他的照应,那种阿谀奉迎、卑躬屈膝的丑态使B为他害臊。叶菲莫夫明白别人想把他引上正途,甚至酒也不喝了。后来,在一个剧团的乐队里总算给他谋到一个位子。他以良好的成绩通过了考试,因为一个月刻苦努力他就把一年半荒废的东西补了回来,并保证要继续练功和一丝不苟地对待新的职责。但我们家的境况毫无改善。继父的薪金他一个子儿也不给妈妈,全都自己花掉,跟他很快又交上的一批新朋友一起喝光、吃光。他结交的大都是剧团职员、合唱队员、龙套演员,总之都是些他可以在其中高踞首席的人,对于真正有才能的却避不交往。他已使那些新朋友对他产生某种特殊的敬意,刚认识就向他们宣传,他是个被埋没的人,他有伟大的天才,是妻子断送了他,并说他们的乐队指挥根本不懂音乐。他嘲笑乐队里所有的演奏员,嘲笑排演的剧目,还嘲笑上演的几部歌剧的作者。后来,他开始侈谈一种新的音乐理论,总之,使整个乐队都感到腻烦!他跟同事、指挥一一闹翻,对上司傲慢无礼,成了出名的一个最不安分、最爱争吵、同时又最卑琐的人,弄得人人讨厌。
\par 这样一个不足道的人,这样一个成事不足、败事有余的演奏员,偏偏这样踌躇满志,这样自命不凡、目空一切,确实是非常奇特的怪现象。
\par 最后,继父跟B也翻了脸;他编造了最拙劣的谣言当作明明白白的事实抛出去,对B进行极其恶毒的诽谤。他在乐队里鬼混了半年之后,终于因玩忽职守、行为失检被撵走。但他没有随即离开那个地方。不久,有人看到他身上又跟过去一样破破烂烂,因为比较像样的衣服又都变卖和抵押了。他去找过去的同事,不管人家欢迎不欢迎这样的客人,在他们面前散播流言,搬弄是非,哭诉日子难过,叫所有的人去看看他那恶魔般的妻子。当然,有人爱听,有人乐于给这位遭逐的同伴灌上几杯,让他大放厥词。何况他的嘴皮子总是那样尖刻,往往击中要害,而夹杂在话中的那股冲天的怨气和各种放肆的怪论,自有一部分人欣赏。他被当作一个神经失常的小丑,闲得无聊时不妨叫他胡言乱语扯上一通。人们喜欢当着他的面谈论某一位新来的小提琴家,故意逗他。叶菲莫夫听到这话,脸上就会变色,怯生生地打听是谁来了,谁是崭露头角的天才,并立即开始妒忌他的声誉。大概从这个时候开始,他才真正陷入经常性的精神错乱,也就是形成了一个固定观念:他是首屈一指的小提琴家,至少在彼得堡是无双的,但他命运多舛,怀才不遇,由于种种妒贤忌能的阴谋,他始终不为人所了解,至今没没无闻。他对末了这一点甚至颇为得意,因为有这样一些人就是喜欢把自己看作受欺压者,喜欢把抱怨挂在嘴上,或在暗中崇拜自己得不到承认的伟大聊以自慰。彼得堡所有的小提琴手他个个了如指掌,在他看来,其中没有一人能与他匹敌。凡是知道这个疯疯癫癫的可怜虫的,不管是行家还是逢场作戏的票友,都喜欢在他面前谈论某某才华出众的著名小提琴家,好让他也发表自己的看法。他们没有人能如此巧妙地用如此夸张的漫画手法描绘当代著名的音乐家。甚至遭到他这般挖苦的那些艺术家也有些怕他,因为知道他那张嘴够损的,承认他的指摘有根据,他的见解有道理,如果确实该骂的话。人们惯常在剧场的走廊里和后台看到他。职员们对他不加拦阻,因为少不得这样一个人,于是他成了个国产的忒耳西忒斯【荷马的史诗《伊里亚特》中人物,他别有用心地劝兵临特洛伊城下的希腊人不要打完仗就回去,通常用来比喻喋喋不休地喜欢跟大家抬扛的人。莎士比亚的历史剧《特洛伊罗斯与克瑞西达》也写了这个丑陋和好谩骂的希腊人。】。这样的生活持续了两三年,最后,他连扮演这样一个角色也惹得人人厌烦。接下来我继父遭到毫不含糊的驱逐,在他生前的最后两年似乎销声匿迹了,哪儿也看不到他。不过,B遇见过他两次,可是瞧着他那副狼狈相,在B的心中同情又压倒了厌恶。B招呼他,但我继父动了气,假装没听见,把一顶破旧不堪的帽子拉得遮住眼睛,打旁边走过去。后来,在一个不知什么大节日,一清早有人向B通报,说他以前的伙伴叶菲莫夫来向他拜节。B出去见他。叶菲莫夫喝得醉醺醺地站在那儿,开始行大幅度鞠躬礼,几乎碰到地上,嘴唇不住牵动,固执地不肯走进房间。他的举动的意思是说:我们这种没有才能的人,怎么能同您这样的名流交往,我们这种小人物能到此听差的所在,来向您拜个节,行个礼就走,已经心满意足。总之,这一切恶劣、无聊之至,令人作呕。此后B很久没有见到他,直至发生这幕惨剧结束他可悲的、病态的、堕落的一生为止。他的生命是以可怕的方式结束的。这幕惨剧不仅与我童年时代最初的印象紧紧连在一起,甚至和我的一生也有密切关系。事情是这样发生的······。但我首先必须说明,我的童年时代是怎么一回事,在我最初的印象中留下如此痛苦的痕迹、把我那可怜的妈妈逼死的这个人对于我又意味着什么。

\newpage
\section*{二}
\par 我自己记得起来的事情开始得很晚,大约在八岁以后。我不知道八岁以前的事情怎么没有给我留下一点清晰的印象,否则现在我也不至于回忆不起来。但从八岁半开始的一切,我都记得很清楚,一天又一天接连不断,仿佛这以后的事情顶多发生在昨天。诚然,这以前的事情我也能朦朦胧胧想起一些来:在幽暗的角落里,古老的圣像旁老是点着一灯如豆;此外,有一次我在街上给马撞倒了,据后来人家告诉我,我因此躺在床上病了三个月;还有,在这次卧病期间,我夜里在和我同睡的妈妈身旁醒来,对于我梦见的异象、深夜的寂寥和在屋角制造声响的耗子忽然怕得要命,我躲在被窝里吓得哆嗦了一宿,但不敢叫醒妈妈,——现在我根据这一点推断,我对她比什么都怕。但从我一下子开始意识到自己的时刻起,我的头脑发育得很快、很突然,许多完全不是孩子应该有的印象对我却可怕地易于接受。我觉得一切在自己面前豁然开朗,一切很快都变得明白易懂。我对自己开始记得很清楚的那个时期,在我头脑里留下了强烈而痛苦的印象。这印象后来天天再现,并且一天比一天深刻;它给我生活在父母身边的那段时期,从而也给我的整个童年时代抹上一层阴暗而奇怪的色彩。
\par 现在我觉得.我好象从酣睡中骤然醒来(不过,当时我的感受自然没有这样明确)。我来到一间很大的屋子里,天花板很低,空气很坏,又不干净。墙壁的颜色灰不溜丢,屋角是一只老大的俄式炉子;窗子临街,或者毋宁说是向着对面一幢房子的屋顶,短而且宽,犹如几道裂缝。窗台离地板相当高,我记得,我必须搁一把椅子、一张板凳才勉强够得到窗口,家里没人的时候我喜欢坐在那儿。从我们的住处看得见半个城;我们就住在一幢极大的六层楼房的屋顶下面。我们的全部家具陈设只有一张满是败絮和灰尘的漆皮破沙发、一张白木桌子、两把椅子、妈妈的床褥、角落里一口放东西的小橱、一只老是歪向一边的抽屉柜和几扇纸糊的破屏风。
\par 记得是在黄昏时分,一切都乱七八糟,东西扔满一地:刷子、抹布、我们的木质器皿、一只碎瓶子,还有不知道什么别的物件。我记得妈妈恼火得厉害,不知为什么在哭。继父坐在屋角,照例穿着那件很破的常礼服。他用嘲笑的口气回答妈妈,使她更加恼怒,于是刷子和器皿又纷纷摔到地上。我哭叫起来,向他俩身边扑过去。我害怕极了,紧紧搂住爸爸,用我的身体保护他。天知道为什么,反正我觉得妈妈在对他发无名之火,他没有过错;我想代替他请求原谅,代替他承受无论什么样的惩罚。我对妈妈怕得要死,并以为大家都这样怕她。妈妈起先愣了一下,接着抓住我的一条胳膊,把我拉到屏风后面去。我的胳膊在床上撞得生疼,但是恐惧比疼痛更厉害,所以我连眉头也没皱一下。还记得,妈妈开始指着我用痛苦和激动的语调对父亲不知说些什么(我在下面的叙述中将称他父亲.因为我在很久以后才知道他不是我的生父)。这场戏历时有两个钟点,我战战兢兢地等待着,竭力揣测这一切将以什么告终。后来,争吵总算平静下来,妈妈不知到什么地方去了。这时爸爸把我叫去,吻我,在我头上抚摩,把我抱到大腿上,我紧紧地、甜蜜地偎在他胸前。这也许是我第一次领略到亲人的疼爱;可能真因为如此,从那个时期开始,我才对一切都记得那么清楚。我也看得出,我是因为卫护父亲而赢得他的亲热,恐怕也是在那个时候,我第一次惊异地想到,是妈妈害他吃了很多苦。从此这个念头就永远留在我头脑里,而且一天比一天更使我感到气愤。
\par 从这一刻起,我开始无限地爱我的父亲,但这是一种奇特的爱,完全不象小孩的感情。我可以说,这更象是母亲的怜惜孩子的感情,如果这样来形容我的爱在一个小孩身上不是十分可笑的话。我觉得父亲总是那样可怜,那样受欺凌、遭践踏、吃苦头,以致在我看来,如果不发疯似地爱他,不安慰他,不跟他亲热,不千方百计为他着想,那简直是可怕的、不近人情的事情。但我至今仍不明白,为什么偏偏是我会产生父亲在世上受苦倒霉这样的想法!这是谁向我灌输的?我小小年纪,对于他个人的蹇滞怎么可能有些许的理解?然而我却能理解,尽管在想象中把一切都按我自己的方式重新加以解释、改制;但直到现在我还是想不出,我头脑里怎么会形成这样的印象。也许妈妈对我太严厉了,所以我对父亲抱有好感,以为他和我一样受苦,同病相怜。
\par 我已经谈了最初从婴儿的梦境中惊醒的情形,谈了我一生中最初的举动。我的心从最初的一刹那起就受到伤害,我的头脑也就以不可思议和十分消耗精力的速度开始发展。我已经不能满足于浮面的印象。我开始思考、推论、观察;但这种观察发生得太早、太反常了,故所我的想象不能不把一切都按自己的方式加以改制,于是我一下子进入了一个特殊的世界。我周围的一切变得象父亲经常对我讲的神奇的童话故事,而在那个时候我不可能不把它当作百分之百的真实。奇怪的概念产生了。我很清楚地了解,——但我不知道怎样会了解的,——我生活在一个奇怪的家庭,我父母跟当时我见到过的那些人完全不一样。“为什么,”我心想,“为什么在我看来别人连外表也跟我的父母不一样?为什么我发现别人脸上有笑容,而在我们这个角落里从来不笑,从来不快活,为什么这一点立刻使我吃惊?”是什么力量、什么原因促使我这个才九岁的小孩如此用心地看和用心地听?傍晚,我用妈妈的掩襟旧棉袄往自己的破衣衫外面一裹,拿着铜子儿到小店里买几戈比的食糖、茶叶和面包,在我们的楼梯上或街上会遇到一些人,我总是仔细听他们说的每一句话。我懂得了,但记不得是怎样懂得的,反正在我们那个角落里永远是无法忍受的悲哀。我绞尽脑汁,竭力想猜透为什么会这样的原因,也不知道是什么人帮我对这一切按自己的方式作出了解答;总之,我责怪妈妈,认为她是我父亲的冤家。这里我又要说明一下:我不明白,如此荒唐的概念在我的想象中是怎样形成的。我在多大的程度上爱父亲,也就在多大的程度上恨我那可怜的母亲。有关这一切的回忆,直到现在还深深地、痛苦地折磨着我。但另外有件事情比前面那一件更加促进我奇怪地靠拢父亲。有一次,晚上九点多钟,妈妈差我到小店里去买酵母,爸爸不在家。我回家的时候摔倒在街上,把一碗酵母全洒了。我首先想到的是妈妈会大发脾气。同时,我的左胳膊疼得非常厉害,使我站也站不起来。行人纷纷止步站在我周围;一位老婆婆开始把我扶起,可是在旁边跑过的一个男孩却用钥匙敲我的脑袋。后来,人家把我扶了起来,我捡起碗的碎片,勉强拖着两条腿,摇摇晃晃回家去。忽然,我看见了爸爸。他站在我们对面那幢豪华的楼房前的人群里。这幢房屋为一些身价颇高的人所有,装饰得富丽堂皇,台阶旁停着许多马车,乐声从窗户里边飘到街上。我抓住爸爸的衣襟,给他看打破的碗,开始哭哭啼啼地说,我不敢去见妈妈。我好象确信他会卫护我的。但是,为什么我会确信,是谁向我暗示,是谁教我认定他比妈妈更喜欢我呢?为什么我走到他跟前时并不害怕?他拉住我的手开始安慰我,然后说他要让我看什么事情,并把我抱起来。我什么也看不见,因为他抓住了我摔伤的一支胳膊,使我疼得要命;但我没有叫喊,怕扫他的兴。他一再问我看见了什么。我竭力想迎合他,就回答说我看到红色的帷幕。当他要把我抱到更靠近那幢房子的街道另一边去时,我不知为什么突然哭了起来,搂住他,要求赶快回到楼上妈妈那里去。我记得,当时爸爸的怜爱开始叫我感到有些沉重,因为我那么喜欢的两个人中的一个疼我、爱我,而对另一个人我甚至不敢走近她,这种状况我受不了。但妈妈几乎完全没有生气,就打发我去睡觉。我记得,胳膊疼得愈来愈凶,使我发了寒热。不过我为事情这样顺利结束特别感到高兴。这一夜我梦见的始终是对面那幢挂着红色帷幕的房子。
\par 第二天醒来,我首先想到的、首先关心的是挂着红色帷幕的房子。妈妈刚走出院子,我爬到窗台上,开始望着那幢房屋。它早已激起了我作为一个孩子的好奇心。我特别喜欢在华灯初上的傍晚看它,那时这幢房子灯烛辉煌,整块大玻璃后面的紫红色帘幔开始射出一种特别的、血一般的闪光。台阶旁几乎不断有豪华的马车来到,拉车的都是雄赳赳的骏马,大门口的吆喝、忙乱、马车的彩灯、坐车前来的盛装妇女——一切都吸引着我的好奇心。这一切在我幼小的心灵中本来就具有某种帝王气派和神话色彩。如今,我在那里见到父亲以后,这幢富丽的房子在我心目中变得加倍神奇美妙。如今,在我受到震惊的想象中开始产生一些古怪的念头和猜测。在父亲和母亲这样的怪人中间,我自己会变成这样稀奇古怪的一个孩子,我倒觉得是很自然的。他俩性格的对比特别使我吃惊。比方说,我诧异于妈妈老是为我们的穷家业操心忙碌,老是责备父亲,说只有她一个人为一家子劳累;我不禁向自己提出一个问题:究竟为什么爸爸一点不帮帮她,究竟为什么他象个外人住在我们家?妈妈的某些话使我对此有了点儿概念,我惊讶地了解到:爸爸是位艺术家(这个词儿一直留在我记忆中),爸爸是个有才华的人。于是在我的想象中立即形成一个概念:艺术家是跟别人不一样的特殊的人。也许是父亲的举止本身使我产生这个想法;也许我听到了一些现在已经忘怀的什么话;反正有一次父亲怀着特殊的感情当我的面说了一些话的时候,我觉得他的话出奇地可以理解。他说,总有一天他将不再过穷日子,他自己将成为富豪;还说,只有等妈妈死了以后,他才能复活。我记得,听了这些话我最初害怕极了。我不能再待在房间里,一个人跑到寒冷的过道里去,用胳膊肘抵着窗子,双手掩面放声大哭。但后来,我对这件事经常反复地思考过后,我对父亲这个骇人听闻的愿望习惯下来过后,想象力忽然帮了我的忙。再说,我自己也不能长期为不明真相而苦恼,我非得作出某种假设不可。于是,——我不知道这一切是怎样开的头,——但最后我认为,等妈妈死了以后,爸爸将离开这个凄凉的住所,带着我到别的地方去。但是去哪儿?——我直到最后还是没能想清楚,反正我认定我们将一道走。只记得,凡是我能用来点缀我将跟他前往的那个地方的一切,凡是我的想象力所能创造的一切辉煌、华丽、瑰奇的东西,在这些幻想中全部都用上了。我觉得,那时我们将立刻成为富人;我不再被差遣到小店里去买东西,这是我非常不愿意干的,因为我走出家门,邻近一幢房屋的孩子老是欺负我,而这是我极其害怕的,特别当我拿着牛奶或黄油的时候,知道万一洒了要受到严厉的惩罚。后来我在幻想中确定,爸爸将马上给自己做些好衣服,我们将住进一幢漂亮的房子,而此刻,这幢挂红色帷幕的豪华楼房,和爸爸在房子门前的相遇,他要让我看里边的什么事情——这一切都为我的想象提供帮助。在我的头脑里立即形成一种设想:我们正是要住进这幢房子,并将在里边永远生活得同过节一样,永远幸福顺遂。从此,每天晚上我怀着紧张的好奇心,隔窗展望这幢对我来说具有魔力的房屋,加上车水马龙的盛况和我从未见过的服饰华丽的宾客,我仿佛听到从窗户里边飘出美妙的乐声;我凝望着窗帘上时隐时现的人影,努力想猜透那里在做什么,——我总觉得那里是天堂和一年到头的节日。我憎恨我们贫寒的住所,憎恨我自己所穿的破衣衫。有一次,我照例爬到窗台上,妈妈向我叱喝,命我从窗台上下来,我当即想到:她就是不让我看那幢房子,不让我想它,她妒忌我们的幸福,这一回她要从中作梗······。整个晚上,我一直用怀疑的目光留神注意着妈妈。
\par 对于妈妈这样一个永远受苦受难的人,在我身上怎么会产生如此狠心的态度呢?如今我才了解她悲惨的一生,回想起这个苦命人我痛心难禁。即使在当时,在我那阴暗而古怪的童年时代,在我生命初期发育如此反常的时代,我的心也常常被痛苦和怜悯攥紧,接着,忧虑、惶惑和怀疑便落入我的心田。当时良心即已在我身上起来反抗,我常常怀着痛苦和内疚的心情感觉到自己对妈妈是不公平的。但我们母女俩好象彼此挺疏远的,我连一次也不记得自己曾和她表示亲昵。现在,哪怕是最琐屑的回忆也往往刺痛和震撼我的心灵。记得有一天(现在我要讲的事情当然是琐碎、庸俗、不足道的,但正是这样的回忆使我特别难过,留在我脑海中的印象也最为痛苦),——有一天傍晚,父亲不在家,妈妈要差我到小店里去给她买茶叶和食糖。但她反复思量,老是拿不定主意,一边出声地数铜币——她只能调度这点可怜巴巴的钱。我想她数了有半个钟点,可还是数不好。有时候,妈妈甚至会陷入一种无意识的状态,想必是悲哀的缘故。现在我回忆起来,她一边数,一边老是念念有词,声音不大,疾徐有节,仿佛是无意间漏出话来;她的嘴唇和面颊没有血色,手哆嗦个不停,在她一个人自言自语的时候,又老是摇头。
\par “不,不要了,”她看看我以后说,“我还是躺下睡觉吧。啊?涅朵琦卡,你想睡吗?”
\par 我不做声;于是她把我的头抬起一点,用那么安详、那么亲切的目光望着我,她的脸绽开了那样充满母爱的笑容,使我的心顿时隐隐作痛,怦怦直跳。加上她叫我涅朵琦卡,这意味着此刻她特别喜欢我。这个叫法是她自己发明的,把我的教名安娜深情地改成涅朵琦卡这样一个小名;当她这样叫我的时候,就意味着她要和我亲热一番。我深受感动,我想搂住她,依偎在她身边,和她一起哭一场。她面色苍白,接着在我头上抚摩了很久,——也许已经是做着机械的动作,忘了在和我亲热,一边不断地说着:“我的孩子,小安娜,涅朵琦卡!”我的眼泪夺眶欲出,但我竭力忍住,坚持不哭。我似乎顽固地不愿在她面前宣泄自己的感情,虽然自己十分难过。这不可能是我的心肠在自然而然地趋于冷酷。单凭对我严厉这一点,她不可能使我跟她对立得这么厉害。不!是我对父亲的那种不可思议的、异乎寻常的爱在那里作祟。有时夜里我在屋角短短的床垫上冰冷的被窝里醒来,老是感到一种莫名的恐惧。我迷迷糊糊地回想起,不久前我更小一些的时候,我和妈妈睡在一起,不象现在那样害怕夜里醒来;只要挨到她身边,把眼睛眯起来,紧紧搂住她——马上又睡着了。我毕竟感觉到不能不暗自偷偷地爱她。后来我注意到,许多孩子往往畸形地缺乏感情;如果他们爱谁的话,就爱得异乎寻常。我的情况亦然如此。
\par 我们家里偶尔也会接连几个星期保持死一般的沉寂。父亲和母亲吵腻了,我照旧生活在他们之间,老是沉默、思量、忧伤,老是在我的幻想中追求着什么。我瞧着他俩,完全理解他们的相互关系。我理解他们这种深刻、永久的敌意,理解笼罩在我们乱糟糟的家中的愁云惨雾,——当然,前因后果我未加探究,我能理解多少就理解多少。在冬季漫长的晚上,我往往缩在某个角落里,连续几个小时贪婪地观察他们,注视父亲的脸,力图猜透他在想些什么,是什么事情这样吸引着他的心思。后来我又对妈妈感到惊异、害怕。她会不知疲倦地在房间里来回走上几个钟头,甚至夜里失眠症折磨着她的时候也常常起来一边走,一边自言自语,仿佛屋子里只有她一个人,时而把一双手摊开,时而交叉在胸前,时而又在可怕的、无穷尽的郁悒中拚命地扭绞。有时眼泪在她脸上潸潸地流,也许她常常自己也不明白这眼泪是哪儿来的,因为她不时陷入无知觉的状态。她患有一种痼疾,但她完全不予理睬。
\par 我记得,我的孤独和我不敢打破的缄默使我愈来愈感到沉重难挨。我开始懂事到那时已有整整一年,我老是在思忖,在憧憬,在暗中用一下子萌生的迷离恍惚的空想折磨自己。我像幽居在密林中孤僻成性。后来,爸爸最先注意到了,把我叫到自己跟前,问我为什么这样凝视着他。我记不得向他回答了些什么,只记得他听了以后若有所思,最后瞧着我说,明天他要带识字课本回来开始教我认字。我焦急地期待着这识字课本,足足想了一夜,对于这识字课本是怎么回事还不甚了了。第二天终于来临,父亲真的开始教我。我根据三言两语就明白了对我提出的要求,所以学得很快,因为我知道这样能讨他喜欢。这在我当时的生活中是最幸福的一段时间。当他夸奖我悟性好,在我头上抚摩、吻我的时候,我立刻会高兴得流泪。父亲渐渐地喜欢我了;我已经有勇气和他交谈,我们常常会连续谈上几个小时而不觉得厌烦,虽则他对我说的话有时候我一句也不懂。但我不知怎的有些怕他,唯恐他以为我和他在一起感到无聊,所以竭力向他表示我全都明白。久而久之,晚上和我坐在一起在他已成为一种习惯。只要暮色渐浓,他回到家里,我马上拿着识字课本走到他跟前去。他让我坐在他对面的小凳上,教完了课,就开始读一本不知什么书。我什么也不懂,但是不停地哈哈大笑,估计这样他会得到极大的欣慰。的确,他对我很感兴趣,他瞧我笑得这么欢自己也乐了。就在那个时期,有一天教课完毕,他开始给我讲童话故事。这是我听到的第一个童话。我象着了迷似地坐着,听得心急如焚,随着情节的发展飞到遥远的地方,到故事终了大喜若狂。不仅是童话本身如此吸引着我,——不,我把一切都信以为真,同时纵任自己丰富的想象力驰骋天际,并随即把臆想和现实混杂在一起。我想象中马上会出现挂红色帘幔的房子,不知怎的就象剧中人登场似地,那里有自己给我讲这个故事的父亲,有阻挠我们俩不知到哪里去的妈妈,最后——或许毋宁说首先——是我,连带着我的胡思乱想和满脑瓜荒诞离奇的幻影,——所有这一切在我头脑里彻底搅混,不久便形成最最纷乱的一团糟,以致我有若干时间完全失去了分寸,完全失去了现实感,天知道生活在什么地方。这时我迫不及待地想跟父亲谈等待着我们的未来,谈他自己的期望,谈有朝一日我们离开这间顶楼以后他将带我一同前去的地方。从我这方面说,我确信这一切很快就能实现,但是如何实现,这一切又将是怎么样的——我不知道,并为此苦苦思索,伤透脑筋。有时——尤其是在晚上——我会觉得爸爸马上就要偷偷地向我丢个眼色,把我叫到过道里去,那时我准备瞒过妈妈,顺手拿起我的识字课本,还有从不知何时就没配框子贴在我家墙上的一张蹩脚的平版石印画(那是我决意一定要拿走的),然后我和爸爸悄悄地逃往某一个地方,再也不回家,再也不到妈妈这里来。有一次妈妈不在家,我选择父亲情绪特别好的机会,——那是在他稍许喝了点酒的时候,——走到他身边跟他闲聊,目的是想旋即把谈话转到我心爱的题目上去。我终于设法使他笑了起来,于是我紧紧搂着他,心吓得直发颤,象是准备谈一件神秘而可怕的事似的,开始没头没脑、语无伦次地问他,我们要去什么地方,还要等多久,我们该带些什么,我们将如何生话,还有,我们是不是到挂红色帷幕的房子里去?
\par “房子?红色的帷幕?怎么回事?傻丫头,你在说什么胡话?”
\par 于是我比先前更加害怕地开始向他解释,等妈妈死了,我们再也不住顶楼,他将把我带到一个地方去,我们俩将变得富有而幸福,临了还要他相信,这一切是他自己向我许诺的。我在说服他的时候,自己完全相信,我父亲先前确实说过这句话,至少我认为如此。
\par “妈妈?死了?妈妈什么时候死?”他愕然望着我连连问道,两道杂有些许霜华的浓眉顿时皱紧,脸色也有些变了。“你在说些什么,可怜的傻丫头······”
\par 接着他开始骂我,对我讲上很久,说我是个糊涂的孩子,什么也不懂······我不记得还说些什么,反正他非常扫兴。
\par 他的责备我一句也不明白,不明白他是多么痛苦,因为我把他在气头上饱含酸辛对妈妈说的话都听进去,记牢了,甚至在心中已经想得很多。不管他当时是怎么个人,不管他本人的行为荒唐到什么程度,但是这一切必然使他吃惊。尽管我完全不明白他为什么生气,不过我感到伤心万分,我哭了,在我看来,等待着我们的一切太重要了,叫我一个糊涂孩子既不敢说,也不敢想这件事。此外,虽然我没有一下子明白他的意思,但我隐隐约约感觉到自己对不起妈妈。恐惧随即向我袭来,疑惑乘虚潜入心坎。他看到我痛哭流涕、苦恼不堪的情状,便开始安慰我,用衣袖给我擦去眼泪,叫我别哭。我们俩默默地坐了一段时间;他颦蹙愁眉,似乎在考虑什么事情,后来又开始对我说话,但是,无论我怎样集中注意,他说的一切我都感到极其费解。根据这番话至今还留在我记忆中的只言片语推测,他当时向我解释,他是何许样人,他是多么伟大的艺术家,可是谁也不了解,他是一个才华出众的人。我还记得,他问我懂不懂,我当然作出肯定的回答,然后他要我应对,他有没有才华?我答道:“有才华”,他听了莞尔一笑,大概到末了他自己也觉得可笑:竟然跟我谈这样一个对他说来十分严肃的题目。卡尔·菲奥多雷奇的来到打断了我们的谈话,爸爸指着他对我说:
\par “可是卡尔·菲奥多雷奇连半点才华也没有。”
\par 于是我笑了,情绪完全好转。
\par 这位卡尔·菲奥多雷奇是个饶有兴味的人物。在我一生的那个时期,我见到的人少得可怜,所以怎么也忘不了他。我现在还记得很清楚:他是个日耳曼人,姓迈耶尔,出生在德国,到俄国来一心一意想进彼得堡的芭蕾舞团。但他的舞技实在不行,因此他甚至没有被录用当群舞演员,只在话剧团跑跑龙套。他在福丁布拉斯【 莎士比亚的悲剧《哈姆雷特》中的挪威王子。】的扈从中间担任没有台词的角色,或者扮演维罗那骑士之一,跟二十来人一起举着硬纸板做的短剑齐声高呼:“誓为国王而死!”不过,世上恐怕没有一个演员象这位卡尔·菲奥多雷奇那样一片丹心忠于自己所演的角色。他一生最大的不幸和悲哀是他没有进入芭蕾舞团。他把芭蕾艺术看得高于世上一切艺术,并在某种意义上对之一往情深,正象爸爸对小提琴一往情深那样。他和爸爸还在话剧团同事的时候就结识了,此后这位退休的龙套一直不离开他。他俩经常见面,两个人都悲叹自己命乖运舛,怀才不遇。卡尔·菲奥多雷奇可算是世上最重感情的人,对我继父怀有最热烈、无私的友情,但爸爸对他似乎没有什么特别的好感,无非当他一般的熟人而已,因为也没有别人可以相与。除此之外,爸爸凭着自己目空一切的性格,怎么也无法理解芭蕾也算一门艺术,从而每每把那个可怜的日耳曼人气得直哭。爸爸知道他有这根脆弱的心弦,老是去触动它,当不幸的卡尔·菲奥多雷奇气冲冲地提出反驳的时候,爸爸就笑他。后来我从B那里听到许多有关这个卡尔·菲奥多雷奇的事情,B把他叫做纽伦堡饭桶。关于他跟父亲的友谊,B讲了很多;反正他们不时碰头,在一起喝上几杯,便一起开始怨命,哀叹他们的才能被埋没。我记得这种会面的情景,还记得,我瞧着这一对畸人,有时也会抽抽搭搭地哭起来,自己却不知道为什么哭。这总是发生在妈妈不在家的时候,因为那个日耳曼人怕得她要命,他到我们家来照例先在过道里站一会,看看有没有人出来,万一了解到妈妈在家,立刻从楼梯上跑下去。他每次都带来一些德文诗歌,热情洋溢地念给我们俩听,然后一边朗诵,一边译成不通的俄语,想让我们领会诗意。这会逗得爸爸乐不可支,我也往往把眼泪都笑出来。但是有一回他们俩弄到一部俄国人写的作品,它点燃了这一对朋友身上如火如荼的激情,以后他们聚在一起,几乎老是读这部作品。我记得那是一位负有盛名的俄国作家所写的诗剧【指库柯尔尼克1834年所作的描写一个画家的戏剧幻想曲《贾科博·桑纳扎雷》。库柯尔尼克的“盛名”来自他于同年发表的忠君剧本《上帝的手拯救了祖国》。】。这本书开头的几行我背得很熟,在事隔数年之后,我偶然看到这本书,并不费力气就把它认出来了。这部诗剧写的是一个伟大的画家的不幸遭遇,名字叫杰纳罗还不知是贾科博,他在某一页上大叫:“我得不到公认!”在另一页上喊道:“我得到了公认!”一会儿说:“我没有才华!”可是才相隔几行又说:“我有才华!”结尾是非常凄惨的。这个剧当然是庸俗透顶的作品;奇怪的是它对这两个读者却产生十分幼稚和可悲的影响,他们从主人公身上发现和自己有许多相似之处。记得卡尔·菲奥多雷奇有时会激动得从座位上跳起来,跑到房间的另一角去,情恳辞切、声泪俱下地请求爸爸和我(他总是用法文的“小姐”称呼我)就他和命运、世人究竟孰是孰非作出即决裁判。他还当场跳起舞来,一边表演各种各样的舞姿,一边喊着要我们立刻告诉他,他是不是一个艺术家,能不能说他不是,也就是说他没有才华?爸爸也会一下子高兴起来,悄悄地向我眼睛,大概是预先通知我,他这就要拿日耳曼人大大地开开心。我觉得太可乐了,但爸爸向我扬一扬手,我只好勉强忍住不笑,差点儿没把自己憋死。即使到了现在,只要一想起来,我还是不能不笑。这个可怜的卡尔·菲奥多雷奇此刻好象在我眼前。他个儿长得奇小,又非常瘦,头发已经花白,一个红通通的鹰钩鼻沾着烟末子,两条罗圈腿怪难看的,尽管如此,他对自己这两条腿的构造似乎颇为得意,还穿着紧身裤。他做完最后一个跳跃动作,摆好姿势向我们伸出双手含笑致意,就象舞蹈演员在舞台上做完一套动作面带笑容亮相那样,这时爸爸有几秒钟默不作声,仿佛拿不定主意发表评论,故意让得不到公认的舞蹈家保持原来的姿势,可怜他仅靠一条腿竭力维持平衡,弄得左右摇晃。最后,爸爸现出一本正经的表情望着我,好象邀请我充当不偏不倚的证人听他的评论;与此同时,舞蹈家也向我投来胆怯、哀求的目光。
\par “不,卡尔·菲奥多雷奇,你怎么也不行!”爸爸终于说,一边还装得他自己也不愿道破这痛苦的真理。于是从卡尔·菲奥多雷奇胸中迸出一声真正的喟叹;但他倏忽之间又打起精神来,做着加快的手势重新要求予以重视,说他跳的舞别有章法,恳请我们再评判一次。然后他跑到另一个角落又跳起来,有时跳得卖力极了,脑袋甚至碰到天花板撞得生疼,但他拿出斯巴达人的精神英勇地熬住疼痛,重新停下来亮相,重新含笑向我们伸出颤巍巍的双手,重新请求决定他的命运。但爸爸不为所动,依旧绷着脸回答:
\par “不,卡尔·菲奥多雷奇,看来你命该如此:你怎么也不行!”
\par  这时我再也忍耐不住,开始笑得前俯后仰,爸爸也跟着我笑起来。卡尔·菲奥多雷奇这才发现人家在拿他开心,气得面红耳赤,眼睛里噙着泪水对爸爸说:
\par “你是个背心(信)起(弃)义的碰(朋)友!”
\par 他怀着深深受到伤害的感情说这句话,当时虽然可笑,但后来却使我为这个可怜的人觉得难受。
\par 说完,他拿起帽子从我们家跑出去,指着皇天后土发誓永不再来。但这样的决裂并不持久;过几天他又到我们家里来,又开始读那个享有盛名的诗剧,又是涕泗滂沱,然后天真的卡尔·菲奥多雷奇又请我们裁判他和世人、命运的是非,不过这回一定要本着真正的朋友情谊认真裁判,不再拿他开心。
\par 有一次,妈妈差我到小店里去买什么东西,我小心翼翼地拿着找给我的一个银币回家。登上梯阶时,我遇见父亲正要走出家门。我冲他笑了起来,因为我在他面前总抑制不住自己的感情,他俯身吻了我一下.发现我手里拿着一个银币······。我忘了交代一点:我对他的面部表情太熟悉了,他心里想要什么,我几乎都能一目了然。如果他闷闷不乐.我也为之心碎。当他身无分文,因而已经成了瘾的酒连一滴也喝不成的时候,他最容易犯愁,也愁得最厉害。但我在梯阶上遇见他的那个时刻,我觉得他有点异样。他的眼珠子变浑浊了,目光游荡飘忽;起初他没有注意到我;可是当他看见我手里有一个银币的时候,脸突然涨红,随后又泛白,本想伸手把钱拿走,可马上又缩了回去。显然,他内心在进行斗争。最后,他大概战胜了自己,命我上楼去,自己往下走了几级,但一下子又站住,急忙把我叫回去。
\par 他显得非常不好意思的样子。
\par “听着,涅朵琦卡。”他说,“把这钱给我,回头我还给你。好吗?你一定肯给爸爸的,对不对?你不是心肠挺好的吗,涅朵琦卡?”
\par 我好象预感到了这一着。但在最初的一刹那,我想到妈妈会非常生气,不禁有些胆怯,而更重要的是我本能地为自己和父亲感到羞耻,所以没有把钱交给他。他在瞬息之问看出了这一点,赶紧说:
\par “那就不要了,不要了!······”
\par “不,不,爸爸,你拿去;我就说钱丢了,说是邻居的孩子把钱抢走了。”
\par “好,好;我知道你是个聪明的小姑娘,”他说着,哆嗦的嘴唇现出微笑;一旦他感觉到手里有钱,并不掩饰自己的欣喜。“你的心地真好,你是我的小天使!让我吻一下你的小手。”
\par 他抓住我的一只手想亲吻,但我很快把手抽了回来。我被一种怜悯的感觉控制住了,羞愧愈来愈叫我难受。我撇下父亲.慌慌张张跑上楼去,甚至没有和他告别。我走进房间的时候,一种过去我不知道的苦恼使我两腮灼热,心突突地跳。不过我还是放大胆子对妈妈说,我把钱掉在雪地里了,怎么也找不到。我估计至少要挨一顿打,但这倒没有发生。妈妈起先确乎气得要命,因为我们实在太穷。她冲我大叫大嚷,但紧接着就好象改变了主意,不再骂我,只说我一点也不麻利,又不尽心,说我这样不爱惜她的钱,可见我不那么爱她。这句话最使我痛心,即使我挨了一顿打,也不至于如此。但妈妈对我已经有所了解。她已经注意到我过于敏感,动不动便会进入亢奋状态,所以沉痛地指责我不爱她,想用这样的办法给我较大的震动,促使我今后多留点儿神。
\par 傍晚,到了爸爸应当回来的时候,我跟往常一样在过道里等他。这一次我处在极大的不安之中。我的情绪被痛苦地折磨着我的良心的感觉搅乱了。最后,我看见父亲回来,高兴极了,以为我的心情会因此而轻松一些。他带着几分醉意,但一见到我,马上现出神秘、尴尬的表情,接着把我拉到角落里,胆怯地望着我们的房门,从兜里掏出他买的一个糖酱饼,开始悄声告诫我往后切莫自己拿钱和瞒着妈妈把钱藏起来,这是恶劣的、可耻的、极坏的;还说刚才这样做是因为爸爸非常需要钱用,但他会归还的,以后我可以说钱找到了,不过私自拿妈妈的钱是可耻的,叫我今后决不可存这个念头,要是我听他的话,他还要给我买糖酱饼;末了,他甚至还补充几句,要我心疼妈妈,因为妈妈身体很不好,怪可怜的,她一个人为我们大家干活。我听着心里非常害怕,浑身发抖,眼泪忍不住要掉下来。我震惊得话也说不出,动也动不了。最后,他走进房间里去,叫我别哭,这件事不要对妈妈说。我注意到他自己也窘得厉害。整个晚上我心情惶恐,破题儿第一遭不敢朝父亲看,不敢走到他身边去。他大概也在避免和我目光交接。妈妈在房间里走来走去,一边按她的老习惯神志不清似地自言自语。这一天她感到特别不舒服,大概病又发作了。内心的痛苦终于使我发起寒热来了。到了夜里,我睡不安稳。病态的幻象老缠着我。后来我实在受不了,伤心地哭了起来。我的哭声把妈妈吵醒;她叫了我一声,问我怎么啦。我不回答,可是哭得更加伤心。于是她点了蜡烛,走到我床边,开始安慰我,以为我做梦吓着了。“唉,你呀,真是个蠢丫头!”她说。“直到现在梦见了什么还哭鼻子。好啦,别哭了!······”她吻了我一下,叫我去跟她睡在一起。但我不要,我不敢抱住她,不敢到她那里去。我在难以想象的苦恼中折腾。我想把什么都告诉她。我已经打算开口,但想到爸爸和他的禁令,欲言又止。“你也真可怜,涅朵琦卡!”妈妈说着安置我睡下,用她的旧披风把我裹起来,因为她发现我浑身直打寒颤,“八成你将来也象我一样多病!”这时她充满忧伤地望着我,我架不住她的目光,只得把眼睛眯起来,转过头去。我不记得自己是怎样睡着的,但朦胧中听到可怜的妈妈还说了很久催我入眠的话。我还从来没有忍受过比这更难熬的折磨。我的心简直给挤得感到疼痛。第二天早晨,我觉得好些了。我开始跟爸爸说话,故意不提昨天的事,因为我料想这样他一定会非常愉快。他当即高兴起来,而在这以前,他自己也老是皱眉蹙额望着我。现在看到我挺快活的样子,他满心欢喜,几乎象小孩子一样感到满足。不久,妈妈从家里出去,他便再也不克制自己。他开始热烈地吻我,使我沉浸在一种歇斯底里的狂喜之中,又哭又笑。后来他说要给我看一件了不起的好东西,说我见了这件东西一定非常喜欢,算是给我的奖励,因为我是个聪明而又善良的小姑娘。于是他解开背心的扣子,取下用黑色的细绳套在他脖子上的一把钥匙。接着,他神秘地对我瞧瞧,似乎想从我的眼睛里看出他认为我一定会感觉到的全部喜悦;他打开箱子,备加小心地从里边取出一只形状很奇怪的黑匣子,以前我从来没有看见过他有这件东西。他战战兢兢拿起这只匣子,顿时象换了一个人;他的笑容不见了,脸上骤然出现一种庄严的表情。随后,他用钥匙打开那只神秘的匣子,从中取出一件我从未见过的东西,它的形状非常奇特。他诚惶诚恐地把它拿到手里,说这是他的小提琴,他演奏的乐器。于是他开始郑重其事地低声对我说好多好多话,但我不懂他说些什么,只记得有我已经知道的一些词语,——说他是个艺术家,他有才华,将来他要演奏小提琴,总有一天我们都将成为富人,得到很大很大的幸福。热泪挤满了他的眼眶,顺着腮帮子直淌。我感动极了。最后他吻了吻提琴,让我也吻它一下。他见我很想仔细看看那把琴,便把我带到妈妈床前,把琴放在我手里,但我看到他紧张得全身发抖,生怕我把它摔坏了。我接过提琴,碰了一下上面的弦,琴弦发出轻微的声响。
\par “这是音乐!”我朝爸爸望了望说。
\par “对,对,是音乐!”他高兴地搓搓手加以肯定。“你是个聪明的孩子,你是个善良的孩子!”但我看得出,他在夸奖和欣喜的同时,并没有忘记为他的琴担心,弄得我也紧张起来,赶忙把琴还给他。提琴仍旧和刚才一样谨慎唯恐不周地放进匣子,匣子锁上后再放回到箱子里,爸爸重又抚摩着我的脑袋,答应以后每次都给我看小提琴,只要我同现在一样懂事、善良、听话。就这样,小提琴驱散了我们共同的忧伤。不过到了晚上,爸爸在离家外出的时候,悄悄地叮嘱我,要我记住昨天他对我说的那些话。
\par 我就这样在我们的存身之所渐渐成长,我的爱——不,应该说我的痴情,因为我不知道还有什么确切的字眼可以表达我对父亲的这种遏制不住的、我自己也颇以为苦的感情,——逐步发展到了丝毫经不起刺激的过敏状态。我只有一种乐趣——想他;只有一个心愿——做一切会带给他哪怕是一点点欢欣的事情。我不知有多少次在楼梯上等候他回来,往往冻得瑟瑟抖,脸色发青,目的只是为了早一眨眼的工夫知道他已到家,只是为了尽快看他一眼。逢到他对我稍加怜爱的时候.我就大喜欲狂。与此同时,我又常常为自己如此固执地冷淡可怜的妈妈而感到痛心;有时我望着她,哀伤和怜悯使我肝肠寸断。他们长期处于敌对状态,我不能视若无睹,必须在他们两人之间作出抉择,必须站在某人一边;结果我站到了这个半疯子的一边,原因仅仅在于:他在我心目中是那么可怜,那么卑微,而且最初曾那么不可思议地刺激我的幻想。不过,谁能作出判断呢?我眷恋他,也许正因为他怪得很,连外表也如此,而且不象妈妈那样老是绷着脸,他差不多是个疯子,他身上往往会表现出类乎杂耍丑角的姿态,做出一些孩子气的举动,说到底,也许正因为我不那么怕他,甚至不那么尊敬他,不象对妈妈那样。他更象是我的平辈。渐渐地.我感到甚至主动权在我这一边,我逐步使他听命于我,他已经少不了我。我为此暗暗感到自豪,心中洋洋得意,由于明白他少不了我,有时我甚至向他撒娇。的确,我这种异态的好感有点儿象罗曼司······。但这部罗曼司是注定不能持久的,不久我就失去了父母。他们的生活最后以一幕可怕的惨剧告终,它在我的回忆中创巨痛深。事情是这样发生的——
\newpage
\section*{三}
\par 当时有一条非同小可的新闻轰动了整个彼得堡。据传,大名鼎鼎的小提琴家C——茨途经此地。彼得堡的音乐界纷纷大起忙头。歌唱家、优伶、诗人、画家、音乐爱好者乃至那些从来不喜欢音乐而且一向既谦逊又自豪地声称一个音符也不懂的人,无不如饥似渴地竞相设法弄到入场券。有热情而且花得起二十五卢布买票的大有人在,然而举行音乐会的场子连这些人的十分之一也容纳不下;C——茨在欧洲素享盛誉,到老来赢得无数桂冠,他的才华象永不凋落的鲜花,据说近来他已难得公开演奏,还有人声言他这是最后一次周游全国,以后将告别乐坛——所有这些因素都产生一定的作用。总而言之,这个消息引起了强烈而深刻的反响。
\par 我已经说过,每一位初次来访或者多少有点名气的小提琴家到彼得堡,都会在我继父身上造成极不愉快的影响。他总是急于赶在头里去听访问演出,以便尽早了解人家的艺术造诣。他听到人们对来访者的赞扬,甚至往往感到痛苦,直要到他能够挑出这位新来的提琴手技艺上的瑕疵,用刻薄的口吻到处发表自己的评论,才得宽心。可怜这个疯疯颠颠的人认为全世界只有一个天才、一位艺术家,此人当然就是他自己。但是,乐坛奇才C——茨来访的传闻对他产生的震动很大。必须说明一点,最近十年来,彼得堡没有听到过一位才华出众的名家,更不用说能与C——茨匹敌的好手了;故所我父亲对欧洲第一流艺术家的演奏水平毫无概念。
\par 后来别人告诉我,C——茨要来访问演出的消息刚一传开,人们又在剧场的后台看见我父亲。据说,他异常激动地来到那里,急煎煎地打听C——茨和即将举行的音乐会的情况。大家已好久没见他到后台来,他的出现甚至引起了一点骚动。有人故意用挑逗的口气对他说;“叶果尔·彼得罗维奇,这回您老将听到的可不是什么芭蕾舞曲,而是管保您坐立不安的音乐!”据说,他听了这番揶揄,面色刷地变白,不过仍然带着歇斯底里的笑容答道:“那还得走着瞧;远来的和尚未必都会讲好经;C——茨一向在巴黎,法国人把他捧上了天,可是大家知道法国人的话究竟有几分可信!”如此等等。当时在场者发出哄堂大笑,可怜的父亲自尊心受到了伤害,但他沉住气又添上几句,说他并没有发表什么具体的意见,还是走着瞧,反正后天快到了,那时,种种神乎其神的说法究竟是否属实便可分晓。
\par 据B所述,这天傍晚他遇见了有名的票友X公爵——这是一位深通音律、酷爱艺术的人。他们在一起边走边谈小提琴家访问演出的事,忽然B在一条街的拐角上看见我父亲站在商店的橱窗前凝视那里用大号铅字印着C——茨举行独奏音乐会消息的海报。
\par “您看到那个人没有?”B指着我父亲向公爵说。
\par “他是谁?”公爵问。
\par “他的事您已经听说了。这就是我对您提到过不止一回的那个叶菲莫夫,以前您甚至帮过他的忙。”
\par “哦,有意思!”公爵说。“您谈过许许多多有关他的事情。据说这个人很有意思。我倒想听听他的演奏。”
\par “这不值得,”B答道,“而且怪难受的。我不知道您觉得如何,反正我听了总感到揪心的沉痛。他的一生是一部可怕而荒谬的悲剧.我深深地同情他,不管此人多么下流,在我心中对他的好感还没有泯灭。公爵,您说这个人大概挺有意思。不错,但他给人的印象太痛苦了。首先,他是个疯子;其次,这个疯子犯有三项罪行,因为,除了自己以外,他还毁了另外两个人:他的妻子和女儿。我了解他,如果他确实相信自己犯了罪,会立地倒毙。可怕就可怕在他已经有八年几乎相信了这一点,而他和自己的良心也斗争了八年,直要到在这一点上完全确信——不是几乎相信——为止。”
\par “您说他境况很不好?”公爵问。
\par “是的,但贫穷目前对他差不多是一种幸福,因为贫穷可以做他的口实。现在他可以向任何人声称,正是贫穷阻碍着他,如果他有钱的话,他就有时间,就无须乎操心,人们一定马上会发现他是位了不起的艺术家。他结婚时抱有一个奇怪的希望,以为他妻子的一千卢布能帮助他站稳脚跟。他这一举动象个空想家、诗人,其实他一生都如此行事。您可知道,整整八年他不住口地说的是什么?他硬说他的不幸都是妻子造成的,说妻子妨碍他。他两手一叉,什么也不干。可要是他没有这个妻子,他将是世上天字第一号的可怜虫。他已有好多年没有拿起琴来,——您知道为什么?因为他每次拿起琴弓的时候,自己内心不得不承认,他是个废物,是个零,不是什么艺术家。如今琴弓搁在一边,他至少还保存着一线渺茫的希望——也许并非如此。他沉湎于幻想之中,总以为将来会出现奇迹,他可以一下子变成世界上最最出名的人。他的座右铭是:aut Caesar,aut nihil【拉丁语:要末做恺撒大帝,要末被人瞧不起。】,仿佛恺撒是可以在转眼间突然变出来似的。他念念不忘的是一举成名。如果这样的心情成了一个艺术工作者的主要和唯一的动力,那他便不是一位艺术家,因为他已经丧失了主要的艺术本能,也就是丧失了对艺术的爱,不再因为艺术并非其他、并非名气、仅仅因为它是艺术而爱它。但是C——茨则相反:他一拿起琴弓,除了他的音乐之外,对他来说世上什么都不存在。继琴弓之后,他首先关心的事情是钱,大概第三位才是名气。不过他很少考虑名气······。您可知道,这个不幸的人此刻在想些什么?”B指着叶菲莫夫又说。“现在占据他头脑的是世界上最愚蠢、最无聊、最可怜而又最可笑的一个问题,那就是:究竟他比C——茨高明,还是C——茨比他高明,除此以外什么都不在他心上,因为他还以为自己是全世界首屈一指的音乐家。谁要是能使他信服他根本不是艺术家,我可以对您说,他会象遭到雷击一般立地倒毙,要知道,他把整个一生都奉献给了一个根深蒂固的固定观念,因为最初他确曾表现出真正的天赋,一旦要他放弃这样的固定观念,那太可怕了。”
\par “他听了C——茨的演奏以后不知会怎么样,这倒是耐人寻味的,”公爵道。
\par “是啊。”B若有所思地说。“不过,他震动过后立刻故态复萌;他的疯狂能盖过事实真相,他马上会想出某种理由来解嘲。”
\par “您认为是这样吗?”公爵道。
\par 这时,他们快要走到叶菲莫夫近旁。我父亲想悄悄地溜之大吉,但B叫住了他,同他交谈起来。B问他去不去听C——茨的音乐会。父亲淡漠地表示不一定,说他有比听外国演奏家的音乐会更重要的事情,不过,到时候要是有空的话,去听听倒也不妨。说到这里,他情虚地向B和公爵瞥了很快的一眼,并且怀着戒心莞尔一笑,然后举帽点头,推说无暇多谈,打旁边走了过去。
\par 但我在前一天便已知道父亲的心事。尽管他究竟为什么事情苦恼我不知道,但我看得出他心神不宁得可怕;甚至妈妈也注意到这情形。当时她正病得厉害,几乎迈不开腿脚。父亲一忽儿回家来,一忽儿又出去。早上有三四位客人来找他,都是他过去的同事,这使我大为诧异,因为自从爸爸完全离开剧院以后,大家都跟我们断绝了往来,除了卡尔·菲奥多雷奇,我几乎从未见过旁人到我们家来。末了,卡尔·菲奥多雷奇气急败坏地跑来,并带来一份海报。我留神听,仔细看,对于这一切感到坐立不安,仿佛造成这派惶惑气氛以及我从爸爸脸上看到的忐忑情状都是我一人之过。我很想弄清楚他们谈些什么,当时我第一次听到C——茨的名字。后来我明白了,要见到这位C——茨,至少得花十五卢布。我还记得,爸爸大概沉不住气了,便把手一甩,说他知道这些洋玩意儿和吹得神乎其神的奇才是什么货色,C——茨他也知道,无非都是些犹太佬,存心来诓俄国人的钱.因为俄国人会凭空相信任何无稽之谈,更何况是法国人大吹大擂的事情。我已经懂得什么叫做没有才华。客人们笑了起来,不久都走了,撇下爸爸如坐针毡。我明白他在为某一件事情生那个C——茨的气,为了讨他的好,驱散他的愁闷,我走到桌子旁边,拿起那份海报来拼读,把C——茨的名字念出声来。尔后,我笑了起来,看看坐在椅子上若有所思的爸爸,说,“这人想必跟卡尔·菲奥多雷奇一个样:他大概也是怎么也不行的。”爸爸打了个寒颤,象是吓了一跳,他从我手中把海报抢去,又是叫嚷,又是跺脚,拿起帽子走出房门,但旋又回来,把我叫到过道里去,吻了我一下,开始以一种紧张和隐藏着恐惧的神态对我说,我是个懂事而善良的孩子,想必不愿意扫他的兴,说他指望我帮他一次大忙,但究竟帮什么忙他没说。再者,我听着他这样说只觉得怪不好受;我看得出,他的言语和疼爱并非出于真心,这一切都使我震惊。我开始为他苦恼,为他忧虑。
\par 次日吃饭的时候——那已经是音乐会的前夕,爸爸沮丧到了极点。他变得面目全非,不住地瞧瞧我,又瞧瞧妈妈。后来,他竟跟妈妈谈起什么事情来了,这使我大为纳罕,因为他几乎从来不跟妈妈交谈。饭后他对我表示异样的亲热:他不时用各种借口把我叫到过道里去,一边四顾张望,似乎生怕被人撞见,一边总是抚摩着我的脑袋,吻我,对我说,我是个好心的孩子,我是个听话的孩子,说我一定爱爸爸,一定能做到他要我去做的事情。凡此种种,无不引起我难以忍受的惆怅。直到他第十次把我叫到楼梯那儿去,我才明白是怎么回事。他带着无可奈何的痛苦表情不安地东张西望,问我知道不知道,昨天早晨妈妈拿回来的二十五个卢布放在什么地方?我听他问起此事,吓得发了呆。但正在这个当儿,楼梯上有人发出声响,爸爸惊恐地撇下我跑了出去。他到晚上才回家,神态尴尬,忧心忡忡,在椅子上默默地坐下来,频频用不好意思的目光看我。我感到一阵莫名的恐惧,有意避开他的视线。后来,在床上躺了一整天的妈妈把我叫去,给了我几个铜币,差我到小店里去给她买点儿茶叶和白糖。我们家喝茶是非常难得的事,除非妈妈身体很不好,有寒热,否则她舍不得把钱花在这种按我们的境况来说是奢侈的项目上。我拿了钱走到过道里拔腿就跑,仿佛唯恐有人追上来。但我担心的事情还是发生了:爸爸在街上追上了我,并把我带回到楼梯边。
\par “涅朵琦卡!“他用发颤的声音开口道。“我的小宝贝!你听着,把这点钱给我,我明天就······”
\par “爸爸!爸爸!”我叫着跪下来求他。“爸爸!我不能!这不行!得给妈妈买茶叶······。不能把妈妈的钱拿走,怎么也不行!下次我去拿······”
\par “那末你是不肯喽?你不肯?”他恶狠狠地向着我悄声说。“这么说,你是不愿意爱我喽?那好吧!现在我不管你了。你跟妈妈待在一起,我要离开你们,我不带你走。听见没有,你这个坏心的小丫头!你听见没有?”
\par “爸爸!”我满怀恐惧嚷道。“把钱拿去,给!现在我该怎么办呢?”我扭绞着双手,揪住他的衣襟说。“妈妈会哭的,妈妈又会骂我的!”
\par 他大概没料到会遇上这样的阻力,可钱还是拿去了,最后,由于受不了我的怨言和哭声,他把我撂在楼梯上,自己跑了下去。我只得上楼,可是到了我们家的房门口,再也支撑不住;我不敢进去,也不能进去;我的心受到剧烈的震荡,象一潭被搅浑的水。我用双手掩面扑到窗前,就象第一次从父亲口中听到他但愿妈妈早死时那样。我神思恍惚,不敢动弹,哆嗦着谛听楼梯上的任何些微声响。后来我听到有人匆匆上楼来。这是他;我能分辨出他的脚步声。
\par “你在这儿?”他压低嗓门问。
\par 我急忙向他跑去。
\par “拿去!”他喝道,一边把钱往我手中塞。“拿去!把钱拿回去!如今我不是你父亲了,你听见没有?如今我不愿做你的父亲了,你听见没有?如今我不愿做你的父亲了!你爱妈妈胜过爱我,你就到妈妈那儿去吧!我不愿意认识你!”说完,他把我推开,又跑下楼去了。我哭着去追他。
\par “爸爸!好爸爸!我听话就是了!”我喊着。“我爱你胜过爱妈妈!把钱拿去,给!”
\par 但他已经听不见我的叫喊;我连他的影儿也看不到。整个晚上我失魂落魄,不住打着冷战。我记得妈妈似乎在向我说些什么,几次把我叫去;我好象处在昏迷状态,什么也听不见,什么也看不清。最后,一切以歇斯底里发作而告终:我又是哭,又是叫;妈妈吓得不知如何是好。她让我躺在她被窝里,我搂住她的脖子,哆嗦着每分钟都担心发生什么事情,也不记得是怎样入睡的。这样过了一宿。第二天上午,我很晚才醒来,妈妈已经不在家里。这个时候她总是出去干自己的事。爸爸那儿好象来了个外人,他们俩在高谈阔论。我勉强等到客人走后,屋里只剩下我们俩了,就哭着扑到父亲跟前,为昨天的事求他原谅我。
\par “你答应我仍跟以前一样做一个聪明的孩子吗?”他声色俱厉地问我。
\par “我答应,爸爸,一定!”我回答说。“我告诉你,妈妈的钱放在什么地方。她的钱就在这抽屉里的一只匣子里,昨天还放在那里。”
\par “昨天还在?在哪儿?”他嚷着全身猛地一震,从椅子上直立起来。“放在什么地方?”
\par “钱锁起来了,爸爸!”我说。“你等着:晚上妈妈会差我去兑破的,因为我看到零钱都花完了。”
\par “我需要十五个卢布,涅朵琦卡!你听见没有?只要十五卢布!今天你给我弄来;明天我就全还给你。现在我去给你买果汁糖,买榛子······还要给你买一个洋娃娃······明天也买······我每天都带好东西回来,只要你做一个聪明的小姑娘!”
\par “不要,爸爸,不要!我不要你送东西;我也不要吃好东西;你买了,我也要还给你的!”我哭着喊道,因为顷刻间我的心整个儿疼痛如绞。在这一刹那,我感觉到了他并不怜惜我,并不疼爱我,因为他看不到我是多么爱他,以为我给他效劳图的是果汁糖或洋娃娃。在这一刹那,我尽管只是个小孩,却把他彻底看透了,并且已经觉得,这种想法对我造成了不可平复的伤害,我已经不可能爱他,我失去了我原来的那个爸爸。我的许诺使他欢欣雀跃;他看到我为了他一切都能豁出去,一切都愿意干,至于这“一切”当时对于我究竟包含多少内容,只有上帝知道。我明白,这点钱对于可怜的妈妈意味着什么,我知道,丢了这点钱她会懊丧得病倒,所以我的心灵在发出追悔的惨叫。但他什么也看不出来;他把我当做三岁的婴孩,其实我全明白。他的欢欣超越了一切界限,他吻我,劝我别哭,向我许愿,说我们今天就离开妈妈远走高飞,——无疑是迎合我一贯的梦想,最后他从兜里掏出一份海报,开始说服我相信,今天他要去见的那个人是他的冤家,是他的死对头,但他的冤家对头决不会得逞。他跟我谈起自己的敌人来,他本人倒十足象个小孩子。他注意到我不象往常听他说话时那样面露笑容,只是默默地听着,他便拿起帽子走出房间,因为他急于到什么地方去;但临走的时候,他又吻了我一次,讪讪地笑着向我点点头,似乎对我不大放心,又象是竭力不让我改变主意。
\par 我已经说过,他的精神有些失常;但这在前一天就看得出来。他需要钱买音乐会的入场券,而这场音乐会对他来说将是决定一切的。他仿佛预感到这场音乐会将要决定他的整个命运,但他急昏了,所以昨天竟要把几个铜币从我手里夺走,好象这点钱够买票似的。他在吃饭的时候神态更加反常。他在位子上完全坐不定,什么也不吃,一刻不停地站起来,随即又坐下,象是改变了主意,忽而拿起帽子,似乎准备出去,忽而又一下子变得出奇地心不在焉,老是喃喃自语,接着突然看看我,对我眼睛,打打手势,迫不及待地希望尽快把钱弄到手,对于我直到现在还没有从妈妈那儿拿到钱有些生气。甚至妈妈也注意到了这些反常的表现,望着他直纳闷儿。我简直象个被判处死刑的囚犯。饭后,我躲到角落里,象害疟疾似地全身发抖,一分钟一分钟直数到通常妈妈差我去买东西的时候。我一辈子也没有度过更难捱的时刻;这段时间将永远留在我的回忆中。当时我说得上是百感交集!在某几分钟内,一个人的意识经历的感受比几年还多。我觉得自己的行为非常要不得;其实,正是他自己启发了我的善良本性——当初他第一次懦怯地把我推向恶的一边,自己吓了一跳,便向我说明我的行为非常要不得。难道他不明白,要蒙蔽一颗渴望自觉地体味印象的心灵是多么困难?何况这颗心灵对于善与恶已经感觉得很多,思索得也很多。我明明懂得,显然有极端不得已的难处迫使他又一次驱使我去做坏事,从而牺牲我的可怜而得不到保护的童年,即便再度动摇我那立足未稳的良心也在所不惜。此刻,我缩在角落里独自寻思;他为什么要许诺奖赏我已经自愿决定去做的事情?过去所不知道的种种新的感受、新的意向和新的问题成堆地在我头脑中浮起,我给这些新问题折腾得苦不堪言。后来,我忽然开始为妈妈着想;我想象着她丢失这最后一点辛苦挣来的钱会伤心到什么程度。这样等着,等着,妈妈终于放下勉力支撑着在干的活,把我叫去。我哆嗦着往她那儿走。她从柜子里取出钱来交给我,说:“去吧,涅朵琦卡;不过,看在上帝份上,别再象前些日子那样让人少绐了找头,也别稀里糊涂地丢了。”我用哀求的目光看看父亲,但他点点头,向我露出赞许的笑容,急煎煎地搓着手。钟敲六下,而音乐会定于七点开始。这一番等待也够他受的。
\par 我走到楼梯上停下来等他。激动和焦急使他置一切必要的谨慎于不顾,紧跟在我后面跑了出来。我把钱交给他,楼梯上暗得很,我看不清他的脸,但我感觉到他接过钱时浑身在发抖。我发了呆一般站在那里一动也不动;直到他差我上楼去把他的帽子拿来,我才如梦初醒。他自己甚至不愿进去。
\par “爸爸!难道······你不跟我一起去吗?”我用断断续续的声音问,心中还惦记着我最后的一线希望,希望他会保护我。
\par “不······你先一个人待着······知道吗?等一下,等一下!”他忽然想起了什么,急忙说。“等一下,我这就去买好东西给你;你先去把我的帽子拿来。”
\par 我的心象给一只冰冷的手蓦地揪住。我发出一声惊呼,把他推开,急忙跑上楼去。我走进屋子的时候脸无人色,此刻我如果说钱被人抢走了,妈妈会相信我的.但这时节我什么也说不出来。一阵绝望引起的歇斯底里使我扑倒在妈妈的被窝上,双手捂住面孔。一分钟以后,房门不好意思地发出呀的一声,爸爸走了进来。他是来取帽子的。
\par “钱呢?”妈妈骤然叫嚷起来,她一下子就猜到发生了不寻常的事情。“钱在哪儿?说呀!你说呀!”她把我从床上拖起来,让我站在屋子中央。
\par 我默不作声,低头看着地上,我几乎闹不清自己究竟是怎么一回事,也闹不清别人在对我做什么。
\par “钱呢?”妈妈扔下了我,突然转向正要拿起帽子的爸爸,又高声问道。“钱在哪儿?”她再问一遍。“啊!她把钱给了你?你这个目无神明的害人精!你这个恶魔!你要把她也毁掉!连她这样一个孩子你也不放过?!不成!你不能走!”
\par 转眼间,她跑到门口,把房门从里边锁上,钥匙藏在身边。
\par “你说!老老实实承认!”她开始向我问罪,由于愤激过度,几乎嗓子也哑了。“老老实实承认一切!说,你说呀!要不······我简直不知道会把你怎么处!”
\par 她抓住我的两只手扭绞着拷问我。她气得快发疯了。在这一刹那,我发誓保持沉默,一句话也不提到爸爸,但还抱着万一的希望最后一次举目看他······。他只要对我瞥上一眼,只要听他说一句我巴巴地盼着、在心中祈求他说的话,无论忍受怎样的痛楚,无论遭到怎样的拷问,我也是幸福的······。可是,我的天哪!他却用无情的威胁性手势禁止我开口,殊不知此时此刻来自任何人的其他威吓对我都不起作用。我觉得咽喉好象被堵塞了,气喘不过来,两腿一软,就倒在地板上人事不省······。我又象昨天那样发生了一次神经性的休克。
\par 我是在我家房门上忽然响起叩门声时惊醒的。妈妈用钥匙开了门,我看见一个穿号衣的人进来惊讶地向我们一一环顾,问哪位是叶菲莫夫乐师。继父说他就是。于是那个听差递交了一封便简,说他是此刻正在公爵宅第的B派来的。信封内有一张C——茨音乐会的请柬。
\par 一名身穿华丽号衣的听差奉主人——一位公爵——之命来找穷乐师叶菲莫夫,——这件事在瞬息间给妈妈留下了强烈的印象。在这个故事的一开始我谈到过她的性格,这个可怜的女人始终爱着父亲。现在,尽管经受了整整八年接连不断的忧患和困苦,她的心仍然未变:她仍然可以爱他!天知道,也许现在她突然看到丈夫时来运转了。即使是某种希望的一点点影子对她也会产生影响。可能,她也多少感染到她那宝贝丈夫毫不动摇的自信亦未可知!再说,这种自信也不可能对她这样一个脆弱的女人毫无影响,根据公爵的垂青,她在倏忽之间可以为丈夫设想出上千种前程。才一眨眼的工夫,她已经准备跟他重归于好,她可以原宥他把一家子的生活弄到这般光景,即使考虑到他刚刚犯下的又一罪恶——牺牲她唯一的孩子——这件事情,在重新燃起的热情冲动下,在新希望的鼓舞下,她也可以把这桩罪恶缩小为普通的失检行为,看成是穷极无聊的生活、走投无路的境况所造成的懦怯表现。痴情在她身上还没有泯灭,此时此刻她又愿意为她那堕落的丈夫提供宽恕和同情。
\par 父亲一时手忙脚乱起来;公爵和B的关注也使他震悚。他直接走到妈妈跟前,对她悄悄地说了些什么,妈妈便从屋子里走了出去。两分钟以后,她带了兑破的钱回来,爸爸立即给了信差一个银卢布,来人颇有礼貌地鞠一躬后走了。妈妈出去了不多一会,拿来一只熨斗,取出丈夫最体面的一个胸衬动手把它烫平。她亲手把一条白色的麻纱领带在丈夫脖子上系好,这条领带连同还是父亲刚进剧院任职时做的一件黑色燕尾服(虽然穿得很旧了)保存着备而不用已不知有几许时日。装束停当后,父亲拿起帽子,但临走时要了一杯水喝;他面色苍白,精神疲惫地坐到椅子上。水是我递给他的,也许,憎恶的感觉重又潜入妈妈的心房,使她最初的热情冲动冷却了下来。
\par 爸爸走了出去;屋子里剩下我们俩。我退到角落里,一声不吭,久久地望着妈妈。我从未见过她这样愤激:她的嘴唇发颤,苍白的面颊一下子烧得通红,每隔一会儿就全身哆嗦。尔后,她的悲苦开始通过怨诉、啜泣和哀叹发泄出来。
\par “这都怪我,都怪我这个苦命的人!”她自言自语道。“她将来怎么办呢?我死了以后.她怎么办呢?”她站住了继续说,这个念头象闪电一般把她击中在房间中央。“涅朵琦卡!我的孩子!我可怜的涅朵琦卡!苦命的孩子!”她把我抱起来,神经质地搂着我说。“我活着尚且不能把你抚养照看好,将来能把你托给谁呢?哦,你不懂我的意思!你懂吗?我刚才说的话你能记住吗,涅朵琦卡?往后你不会忘记吧?”
\par “不会,不会,好妈妈!”我把十指交叠在一起求她放心。
\par 她长久地、紧紧地把我搂在怀里,想起一旦要跟我分离就颤栗不已。我的心在破裂。
\par “妈妈!妈妈!”我呜咽着说。“你为什么······为什么不喜欢爸爸?”别的话都哽住了说不出来。
\par 但听得从她胸中迸出一声呻吟。接着,又一阵揪心的悲伤驱使着她在屋子里来回走动。
\par “可怜哪,我可怜的孩子!我没注意到她已经长大了!她知道,全都知道!我的天!她得到的是什么印象,看到的是什么榜样!”她在绝望之余,又拼命绞自己的双手。
\par 后来她走到我跟前,怀着疯狂的爱吻我,吻我的手,在上面洒下许多泪水,求我原谅······。象这样的痛苦我从来没有看到过······。最后,她似已心力交瘁,陷入昏昏沉沉的状态。如此过了足有一个钟点。然后她疲乏不堪地站起来,叫我去睡。我走到自己的角落里,钻进被窝,可是没法入睡。我为妈妈深感苦恼,我也为爸爸深感苦恼。我焦急地等着他回来。一想起爸爸,我就被一种恐怖攫住。过了半个小时,妈妈拿起烛台向我走过来,看看我睡着了没有。为了安她的心,我眯上眼睛,假装已经入睡。她对我察看了一番,轻手轻脚走到食橱前,打开橱门给自己倒了一杯酒。她把酒喝下去以后,自己上床睡觉,让蜡烛点着留在桌上,门也不上锁,逢到爸爸晚归时照例如此。
\par 我迷迷糊糊躺着,但是不能成眠。我刚合上眼睛,立即被一可怕的幻象惊醒过来,吓得直哆嗦。我的忧伤愈来愈加剧。我想叫喊,但喊声在我胸中给堵住了。直至夜已深了,我听得我家的房门被推开。我不记得过了多久,但到我忽然完全睁开眼睛的时候,我看见了爸爸。我似乎觉得他的脸色白得可怕。他坐在紧挨房门的一把椅子上,似乎在沉思。死一般的岑寂笼罩着屋子。行将泪尽的残烛用惨淡的微光照着我们的住所。
\par 我看了好大一会工夫,可爸爸还在老地方没有离开,他一动不动地坐着,姿势始终未变,耷拉着脑袋,胳膊直僵僵地支在膝盖上。我几次想叫他,可是没能出声。我的麻痹状态还没有结束。最后,他猛然醒来,把头一抬,离座起身。他在屋子中央站了有好几分钟,象是在下决心做一件事情!随后突然走到妈妈床边凝神听了一会,等到确信她睡着了以后,便走向放着他的小提琴的那只箱子。他开了箱子的锁,取出黑色的琴匣,把它放在桌上;接着他又四顾张望,他的眼神迷茫,飘忽不定,我还从来没有发现过他这般模样。
\par 他刚拿起琴来,旋又把它放下,转身锁上房门。过后,他发现食橱打开着,便悄悄地走到橱前,看到一只杯子和酒,就倒出来喝了。于是他第三次去拿提琴,但第三次把它放下,并走到妈妈床边。我吓得不敢动弹,只好静观其变。
\par 他倾听了很久很久,然后一下子掀开被子,开始用一只手触摸妈妈的面孔。我打了个寒战。他再次俯下身去,几乎把脑袋贴着她,但当他最后一次竖起上身的时候,他那惨白的脸上似乎掠过一丝微笑。他轻轻地、小心地给睡着的妈妈盖好被子,盖住她的头、脚······我感到一种前所未知的恐怖而开始发抖,妈妈的状态叫我害怕,她的酣睡叫我害怕,我不安地望着那毫无动静的轮廓,她的肢体隔着被子显示着棱角毕现的线条······。一个可怕的念头象一道电光在我头脑里闪过。
\par 一切都安排好以后,父亲又走到橱前,把剩下的酒统统喝光。他周身颤栗着向桌子那边走去。他脸上一丝血色都没有,旁人简直认不出他来。他重又拿起提琴。我看见过这把琴,知道它是什么东西,但这时我预料会发生吓人的、可怕的、怪异的情况······因而最初的几个琴音使我猛吃一惊。爸爸开始拉琴了。但琴声焦躁遽促;他不时停下来,状似在搜索记忆;终于,他痛苦地颓然放下琴弓,用异样的目光看看床上。那里有什么东西总叫他放不下心来。他又走到床边······。他的一举一动都落在我眼里,我怀着恐怖的心情屏息凝神注视着他。
\par 忽然,他匆匆忙忙地开始在手边寻找什么东西,于是,刚才那个可怕的念头又象闪电似地烫了我一下。我想起来了:妈妈为什么睡得这样熟?为什么爸爸用手触摸她的面孔她也不醒来?末了,我看到爸爸将所有我们的衣裳凡是能找到的一股脑儿拖去,包括妈妈的棉袄、他自己的旧外套、睡袍乃至我脱下的衣杉,把妈妈盖在一大堆衣服下面完全看不出来;她始终一动不动地躺着,身体的任何部分都没有半点反应。
\par 她睡得好熟啊!
\par 干完了以后,他好象松了口气。这下再也没有什么妨碍他了,但有件事情还是叫他放不下心来。他把蜡烛搬了个地方,脸朝门站着,这样床上的景象可以看也不去看一眼。最后,他拿起提琴,心一横,用弓猛击琴弦······。音乐开始了。
\par 然而这不是音乐······。我一切都记得清清楚楚,直到最后的一刹那,当时震动我注意力的一切我都记得。不,这不是我后来曾有机会听到的音乐!这不是小提琴的声音,而象是某人极其可怕的嗓门在我们幽暗的住所里第一次发出巨响。要末我得的印象是不正确的、病态的,要末我亲眼目睹的一切使我的感官受到震荡,容易产生可怕的、无比痛苦的反应,但我坚信我听到了呻吟,是活人的哀叫、号哭;通过这些声音宣泄出来的是不折不扣的绝望。临了,凡是恸哭中包含的全部惨痛、苦楚中包含的全部凄怆、灰心中包含的全部哀伤似乎一下子聚合在一起,当凝集着这一切的凄厉的结尾和弦终于拉响的时候······我再也支持不住了,——我开始颤抖,泪水从我的眼眶里迸涌,随着一声绝望的惨叫,我扑到爸爸身边,双手抱住他。他惊呼之余,放下了提琴。
\par 他茫然若失站着有一分钟光景。后来,他的眼珠子开始转动,目光向左右两侧游荡,他象是在寻找什么,突然抓住那把提琴,在我头上高高举起······再过一会儿,他也许会把我当场打死。
\par “爸爸!”我冲他高声叫唤。“爸爸!”
\par 听到我的声音,他象一张树叶瑟瑟发抖,并且倒退了两步。
\par “啊!原来还有你呢!我以为全都完了!原来还有你也和我一起留下!”他狂叫着夹住我的臂膀把我举到空中。
\par “爸爸!"我又发出呼唤。“看在上帝份上,别吓我!我害怕!哇——畦!”
\par 我的哭声使他一愣。他把我轻轻放到地板上,默默地对我看了半晌,似乎在辨认和追忆什么印象。骤然间,他象是倒了个过儿,仿佛被一个可怕的念头吓了一大跳,——从他变得模糊的眼睛里溅出了泪花;他俯身向着我,开始端详我的面庞。
\par “爸爸!”我处在恐怖的折磨下对他说。“别这样看我,爸爸!咱们离开这儿!快走!咱们逃走吧!”
\par “对,逃走,咱们逃走!是时候了!咱们走,涅朵琦卡!快,快!”于是他着了忙,好象这才刚刚想到他该怎么办。他匆忙四顾,见地板上有妈妈的一方巾帕,便捡起来放进兜里,又看见一只系带的软帽,也把它拾起来藏在身边,状似在准备出远门的行装,把他用得着的东西全都带走。
\par 我转眼就穿好自己的衣服,也急急忙忙开始把我觉得路上大概用得着的东西通通带走。
\par “好了没有?好了没有?”父亲问。“都准备好了吗?快!快!”
\par 我胡乱打好一个包裹,系上一方头巾,我们俩已经准备走出去了,这时我忽然想起,应该把墙上的一张画也带走。爸爸当即表示同意。此时他挺安分,说话声音很轻,只是催我快走。画挂得很高,我们两人掇来一把椅子,再往上面放一张板凳爬上去,费了不少工夫总算把画拿下来。现在,我们做好了远行的一切准备。他携起我的手,我们已经迈开脚步,但忽然爸爸又叫我站住。他把自己的脑门揉了好久,似乎在回忆还有什么没做。后来,他大概想起了他应当做的事情,便取出放在妈妈枕头底下的一串钥匙,开始在柜子里匆匆寻找什么东西。最后,他回到我身旁,拿来从抽屉里找到的一些钱。
\par “给,你把这点钱拿去,藏好了,”他压低了嗓门对我说,“别丢了,记住,记住!”
\par 他把钱先放在我手中,接着又拿回去塞在我怀里。我记得,当这些银币触到我的身体时,我打了个寒颤,我仿佛直到那时才明白钱是什么东西。现在我们又作好了准备,可是他忽然又把我叫住。
\par “涅朵琦卡!”他对我说,一边似在费力地思索。“我的孩子,我忘了······那是一件什么事情?······反正是必须做的······我记不得了······对,对!我想起来了,是这么回事!······你过来,涅朵琦卡!”
\par 他把我带到供神像的一个角落里,叫我跪下。
\par “祈祷吧,我的孩子,祈祷吧!这对你有好处!······是的,会有好处的,”他指着神像,奇怪地望着我,向我喃喃低语。“祈祷吧,祈祷吧!”他用一种恳请、央求的语调说。
\par 我双膝跪下,两手合握,满怀已经完全把我抓住的恐怖和绝望仆倒在地,如此俯卧有好几分钟,仿佛呼吸已经停止。我努力把自己的全部思想、全部感情集中起来作祈祷,但恐惧还是一再把我压倒。我架不住忧伤的困扰,微微抬起身子。我已经不想跟爸爸走了,我怕他,我想留下。结果,憋在心中折磨着我的那个问题还是从我胸膛里冲了出来。
\par “爸爸,”我泪汪汪地说,“那末妈妈呢?······妈妈怎么啦?她在哪儿?我的妈妈在哪儿?······”
\par 我再也说不下去,便放声大哭。
\par 他也含泪望着我。于是,他拉着我的手,带我走到床边,扒开胡乱扔着的衣服堆,把被子掀去。我的上帝!她僵卧在那儿,已经冰凉发青。我仆到她身上,抱住她的尸体,自己几乎失去了知觉。父亲让我跪下。
\par “向她行个礼,孩子!”他说。“跟她告别······”
\par 我鞠了一躬。父亲和我一起鞠了躬······。他面容惨白,频频翕动嘴唇念念有词地嘟囔着些什么。
\par “这不能怪我.涅朵琦卡,不能怪我。”他颤颤巍巍地指着尸体对我说。“你听着,这不能怪我;这不是我的错。记住,涅朵琦卡!”
\par “爸爸,咱们走吧,”我惶惶然低声道。“该走了!”
\par “是的,现在该走了,早该走了!”说完,他紧紧抓住我的手,慌忙走出房间。“走,这就出发!谢天谢地,谢天谢地,现在一切都结束了!”
\par 我们下了楼梯;睡眼惺忪的扫院人为我们开了大门,一边用怀疑的眼光瞧着我们;爸爸似乎生怕他问长问短,率先逃出大门,我差点儿追他不上。我们走完屋前的那条街,来到运河的堤岸上。石块铺就的路面上夜来下过一场雪,此时还飘着零星小雪。天很冷,我连骨头架子都一起打着哆嗦,只顾死命地抓住爸爸的燕尾服衣襟跟在他后面跑。他腋下夹着小提琴,不时停下来扶一扶胳肢窝里的琴匣。
\par 我们走了约莫有一刻钟,最后,他顺着便道的斜坡折向一条沟渠,在末尾一座石墩上坐下。旁边是一个冰窟窿,与我们相去仅在咫尺之间。四周一个人也没有。天哪!当时一下子占据我心怀的那种可怕的感觉,直到现在我还记得。整整一年我梦寐以求的理想终于实现了。我们离开了我们那个凄凉的住所······。但这和我所盼望的、梦想的难道是一回事?我对那个人的爱大不同于儿童的感情,我也为他的幸福作过设想,可是在我儿童的想象中构思的难道是这样的图景?此刻最使我感到内疚的是妈妈。“我们为什么孤零零把她撇下?”我心想。“为什么象抛弃废物一样把她的躯体扔下不管?”我记得,这一点最叫我坐立不安,心神不宁。
\par “爸爸!”我实在忍受不了心中这个疙瘩的折磨,还是开了口。“爸爸!”
\par “什么事?”他生硬地问。
\par “爸爸,我们为什么把妈妈丢在那里?我们为什么把她抛弃?”我哭了起来。“爸爸!我们回去吧!我们去叫人看看她。”
\par “对,对!”他猛然一震,嚷着从石墩上站起来,象是想出了一个新的主意可以把他的犹豫扫除干净。“对,涅朵琦卡,不应该这样;应该去看妈妈,她在那边冷得很!你到她那儿去,涅朵琦卡,去吧,那里并不暗,那里有蜡烛;别怕,你去叫个人看看她,然后再到我这儿来,你一个人去,我在此地等你······。我不走开,一步也不离开。”
\par 我转身就走,可是刚踏上便道,我的心就好象给什么东西刺了一下······。我回过头去,见他已经从另一边跑了,在这个时刻扔下我一个人,自己逃跑!我没命地叫起来,怀着极度的恐惧向他追去。我追得上气不接下气;他愈跑愈快······眼看着快要从我的视野中消失。路上我看到他的一顶帽子,这是他逃跑时失落的,我把帽子捡起来,继续追赶。我奔得行即气闭,两腿发软。我觉得自己正落入狼狈不堪的境地,我总以为这是一场梦,有时内心会产生和我在梦中逃避别人追赶时完全相同的感受;我两腿发抖,结果被人追上,我倒下去不省人事。痛苦的心情简直要把我撕裂;我可怜他,一想到他不穿大氅,不戴帽子,撇开我,撇开他心爱的孩子逃跑······我的心一阵阵作痛。我要追上他,只是为了再一次热烈地吻他,叫他不要怕我,让他放心,既然他不愿意,我可以不跟随他,我可以一个人回到妈妈那儿去。后来,我看见他拐向某一条街。我跑到那条街的转角上,也跟着他拐弯,并且还能分辨在我前边的那个身影······。这时,我的体力已经不支,又是哭,又是喊。我记得自己在奔跑中曾跟两个路人相撞,他们在便道中央站住脚,莫名其妙地望着我们俩。
\par “爸爸!爸爸!”我发出最后一次呼喊,接着突然在便道上滑了一下,跌倒在一幢房子的大门旁。我感到自己这一跤摔得满脸都是血。再过一刹那,我便失去了知觉。
\par 我苏醒过来时,身在又暖又软的被窝里,只见旁边好些和蔼可亲的面容都在欢迎我恢复知觉。我看到一位鼻梁上架着眼镜的老太太,一位个子很高的先生深表同情地望着我,还有一位年轻美丽的女士,末了是一位白发苍苍的老者,他扼住我的手腕,一边在看表。这次醒来是我新生活的开始。我在奔跑时曾遇上其中的一位,他是X公爵,我就跌倒在他的宅第大门旁。给我父亲送C——茨的音乐会请柬去的正是这位公爵。公爵经过一再查访,才弄清楚我是谁。他了解这一离奇的事件后颇为所动,决定把我收留在自己家里,和他自己的孩子一起抚养。他们四出寻找爸爸的下落,打听到他在城外大发癫狂的时候被人拦住了。他被送进一所医院,两天以后在那里死去。
\par 他死了,因为这样的死是他整个一生必然和自然的结局。他应该这样死去,因为他的生命赖以支持的一切一下子崩溃了,象一个幻影,象子虚乌有的空想那样烟消云散了。他死了,因为他最后的希望已告幻灭,他借以欺骗自己和支撑全部生活的一切,顷刻间在他本人面前豁然迸裂,清清楚楚地现出了本相。真相以刺目的光芒照得他难以逼视,假象在他自己眼里也成了假象。在最后的时刻,他听到了一位旷世奇才向他述说了他本人的命运,使他遭到了万劫不复的谴责。随着最后一个音符从天才的C——茨的琴弦上飞入他的耳朵,艺术的奥秘也在他心目中全部揭晓,那位永葆青春、经久不衰的真正天才以其真诚压垮了他。他一生只在神秘的、无形的苦痛中感受到某种沉重的压力,真相过去只在梦境中出现并且不可接触、难以捉摸地折磨着他,尽管偶尔也向他表露,但他总是仓皇逃避,用自己一生的假象作盾牌;对于这一切他有所预感,但在这以前一直不敢正视,忽然间,这一切一下子灿灿然昭示在他那双迄今顽固地不承认光明是光明、不承认黑暗是黑暗的眼前。但是,真相是他那双第一次看到往昔、现状和前景的眼睛所无法忍受的;真相震垮了、烧尽了他的理智。真相象闪电一般锐不可当地击中了他。他一生提心吊胆、战战兢兢唯恐发生的事情一下子发生了。仿佛一柄斧钺一生悬在他头上,他一生的每时每刻都在无法形容的痛苦中准备着斧钺落到他头上,而斧钺终于落下了!这一击是致命的。他想逃避对自己的审判,可是无处可逃:最后的希望已化为泡影,最后的口实已不能成立。多少年来,他一直把那个女人视为累赘,觉得她活着自己就没法过;他盲目地相信,一旦那个女人死去,他必定一下子得庆重生。那个女人死了,他终于一身无牵挂,终于自由了!他抱着孤注一掷的心情,想最后一次象一个铁面无私的法官那样毫不留情地自己对自己作出评判;但是,他那松弛的琴弓只能虚软地重复那位天才的最后一个乐句······,就在这一瞬间,对他虎视眈眈已有十年之久的癫狂症,终于不可避免地把他彻底打垮。
\newpage
\section*{四}
\par 我复元得很慢;等到我已经完全能下床了,我的头脑仍没有彻底摆脱麻痹状态,在很长一段时间内,我无法理解自己究竟出了什么事。有时我以为自己做了一个梦,我记得自己但愿所发生的一切果真能变成一场梦!夜里入睡前,我希望一觉醒来一下子又回到我们贫寒的住所,又能见到父亲和母亲······。然而,我的处境毕竟清清楚楚地摆在我的面前,我渐渐明白自己只剩下孑然一身,寄人篱下。于是我第一次意识到我成了一个孤儿。
\par 我开始贪婪地观察我如此突然地来到其间的新环境。起初,我觉得什么都新鲜,什么都稀奇,陌生的面孔、陌生的生活方式,无不使我感到困惑;古老的公爵府第里一间间屋子至今历历在目,那里的房间高大宽敞,陈设豪华,但都是那样幽暗、阴森,我记得当时极其害怕穿过某一座老长老长的厅堂,我觉得进去了会压根儿出不来。我的病尚未痊愈,我的情绪也是阴暗、苦闷的,跟这宅子庄严沉郁的气氛完全合拍。何况,某种我自己还不清楚的哀伤在我幼小的心中正日益增长。我常常在一幅画、一面镜子、一座工艺精巧的壁炉或一尊仿佛故意藏在深深的凹壁中以便偷看我、吓唬我的雕像前莫名其妙地止步,一下子忘掉自己为什么站住,打算干什么,开始想什么;等到从出神状态中猛然醒来,我往往感到恐慌得厉害,我的心怦怦直跳。
\par 在我卧病期间,除了那位小个子老医生外,间或来探望我的人中间给我印象最深的是个年纪已经相当大的男人;他样子很严肃,可是心地十分善良,他望着我的时候眼睛里总是流露出那么深切的同情!跟其他所有的人相比,我最喜欢看到他的脸。我很想同他交谈,可是我不敢:他表面上老是愁眉苦脸,说话不多,口气生硬,从来不见他唇边浮起一丝笑意。他就是发现我昏倒并把我收养在自己家中的X公爵本人。到我渐渐复元的时候,他来探望的次数便愈来愈少。最后一次他给我带来了糖果、一本有图画的儿童书;他吻了我,给我画十字表示祝福,要我显得高兴一些。他安慰我,还说不久我将会有一个小朋友,跟我一样也是个女孩子,就是他的女儿卡嘉,眼下她在莫斯科。接着,他跟一个上了年纪的法国女人、他的孩子的保姆以及照料我的一个使女谈了几句,向他们指指我,然后走了出去,从此我有整整三个星期没见到他。公爵在自己家里过着十分孤僻的生活。宅第的一大半由公爵夫人占用;她跟公爵也往往一连几个星期不见面。后来我注意到,甚至全家人都很少谈及公爵,仿佛他根本不在宅内。大家都尊敬他,甚至看得出都很喜欢他,然而却把他当作一个奇特的怪人看待。他本人大概也知道自己非常古怪,与众不同,因此尽量少在人前露面······。底下我还有很多机会谈到他,而且要详细得多。
\par 一天早晨,有人给我穿上洁净细密的衬衣,外罩带有白色孝徽的黑绸连衣裙,还给我梳了头;我瞧着这件孝服感到一种困惑的惆怅。我从楼上被领到楼下公爵夫人起居的房间里去。当我被带到她那里的时候,我竟站住了直发愣,因为我还从未置身于如此富丽堂皇的环境。但这印象只有短暂的一瞬,我一听到公爵夫人的声音(她吩咐把我带近些),顿时脸色变白。我在穿衣服的时候就准备忍受某种折磨,尽管天知道我怎么会产生这样的想法。反正我进入如今的新生活对周围的一切都抱着一种奇怪的不信任态度。但公爵夫人对我非常和蔼,还吻了我。我把胆子稍许放大,朝她看了一眼。这就是我昏倒后苏醒时所看见的那位美丽的女士。但我在吻她的手时,全身颤抖不已,怎么也鼓不起勇气来回答她的问话。她命我坐在她身旁的一张矮凳上。这个位子象是事先就指定给我的。看来,公爵夫人除了把整个心贴在我身上,给我爱抚,充分代替我的母亲之外,没有其他的愿望。可我怎么也不明白自己交上了好运,在她心目中丝毫没有增添好感。人家给了我一本精美的图画书,叫我看。公爵夫人自己正在给某人写信,偶尔放下笔来跟我谈谈;偏偏我颠三倒四,语无伦次,一句得体的话也没有说好。总而言之,虽然我的身世极不平凡,其中大半由命运以及各种不妨说是神秘的安排在起作用,反正不乏饶有兴味、不可思议甚至近乎荒诞的情节,然而这出赚人热泪的文明戏却好象被我故意杀了风景,因为我本人却表现出是个最平凡不过的孩子,胆小、怕生,甚至有点儿笨头笨脑。特别是末了这一点丝毫不合公爵夫人的口 味,看来她很快就对我完全失去了兴趣,这当然只能怪我自己。下午两点过后,开始有客来访,公爵夫人对我的态度一下子又变得体贴、亲切起来。当客人们问起我时,她回答说,这是一个十分精彩的故事,接着使开始用法语叙述。在她讲故事的过程中,人们不住地望着我,频频摇头,感叹连连。一个青年男子举起长柄眼镜来打量我,一个满身香水味的白发小老头儿想吻我,我脸上红一阵、白一阵.低首垂目,浑身发抖,坐着不敢动弹。我的心隐隐作痛,重又飞向那逝去的日子,重又来到我们的顶楼上。我想起了父亲,想起我们默默度过的漫长的夜晚,想起了妈妈;当我回忆妈妈的时候,禁不住热泪盈眶,咽喉梗阻,我真想逃跑,真想溜走,一个人躲起来······。后来,会客结束,公爵夫人的脸明显地绷紧了些。她看我的目光已比较阴沉,说话的语调也比较生硬,特别使我害怕的是她那双咄咄逼人的黑眼睛有时会盯住我看上一刻钟,还有两片紧闭的薄嘴唇。傍晚,我被带回到楼上。我带着几分寒热蒙眬入睡,夜里醒来,又给颠三倒四的梦境触动了愁怀,不觉悲从中来;第二天早晨,昨天的一切又从头再演,我又被带去见公爵夫人。最后,她向客人讲我的离奇故事大概自己都感到腻味了,客人们对我表示同情也已经不耐烦。何况,我又是一个极普通的孩子。一位上了年纪的女客曾问起:她跟我相处不觉得无聊吗?我记得公爵夫人亲口用说私房话的语调回答:“没有半点天真烂漫的味道。”一天傍晚,我被带走后没有再到那里去。对我的恩宠也就至此告终,不过,我可以去我要去的任何地方。由于刺激太深,我也无法终日枯坐一处,有机会离开周围所有的人,到楼下大房间里去,就高兴得什么似的。据我的记忆,我倒是很想跟家人们聊聊,可我是那么害怕惹他们生气,所以宁愿不跟别人待在一起。我喜欢溜到人家不注意的某个角落里.躲在某件家具背后,当即开始在那里回想和思考我所经历的一切,以此消磨时间。但说也奇怪,我好象把自己在父母身边一番经历的结尾给忘了,也就是把那段悲惨的遭遇通通给忘了。我脑海中闪现的只是一幅幅景象、一桩桩事实。诚然,那一夜、提琴、爸爸——一切我都记得,我也记得自己怎样设法给他弄钱;但要从所有这些事件中悟出道理来,理出头绪来,我却做不到······只觉得心头愈来愈沉重。每当我回忆到跪在死去的妈妈旁边作祈祷的那一刻,总是有一股冷气直砭我的肌骨;我哆嗦着,发出轻微的叫喊,随后呼吸变得困难,整个胸部受到重压,心突突乱跳,结果我总是从角落里仓皇逃出。不过,也许我说得不对,人家没有把我一个人撇下不管:对我的照看是毫不懈怠、尽心尽意的,公爵的嘱托无不一一照办,他吩咐给我充分的自由,不要施加任何限制,但是一分钟也不能让我失踪。我发现不时有家人或仆役往我所在的屋子里探头张望,一句话也不对我说又走了,这样的关切使我十分惊讶,乃至有些不安。我不明白这是为什么。我总觉得有人为了某种目的看管着我,打算以后怎么样处置我。我记得自己总想走得远一些,必要时知道往那儿躲。有一次,我闯到府第正面的楼梯上。它全由大理石砌成,宽阔的梯阶铺着地毯,到处装饰着鲜花;连花盆也精美异常。楼梯的每一片平台上都有两个身材颇高的人默默地坐着,衣着色彩鲜艳,领结白得耀眼,两人都戴手套。我对他们看看,心中直纳闷,怎么也猜不透他们为什么坐在这里一语不发,除了互相对视,什么也不做。
\par 我愈来愈喜欢只身作这样的漫游。此外,我常常从楼上逃下去还有一个原因。楼上住着公爵的一位老姑母,她几乎足不出户。这位老太太在我记忆中留下的印象相当突出。她差不多是宅内最显要的人物。所有的人在与她交往中都恪守一套严谨的礼仪,连一向那么高傲专断的公爵夫人也得每周两次在规定的日子亲自上楼去向姑母请安。她一般上午来;双方开始作枯燥的对话,中间往往出现静穆的冷场,那时老太太不是喃喃地背诵祈祷文,就是数念珠。直要到老太太自己离座起身,吻过公爵夫人的嘴唇,以此表示会晤结束,否则请安者不会先走。过去,公爵夫人每天得向这位长辈问好,但后来,根据老太太的意愿,规矩才有所减免,在一周的其余五天,公爵夫人只须每天早上派人问候她的健康。总的说来,年迈的郡主过的几乎是修女的生活。她从未出嫁.三十五岁那年进了修道院,在那里度过十七年,但没有削发;之后,她离开修道院到莫斯科去,一来和身体一年不如一年的姐姐、Л伯爵的遗孀同住,二来跟另一个姐姐、也没有结过婚的X郡主和解(她们闹翻已有二十余年)。不过据说,这三位老太太没有和睦相处过一天,虽曾上千次表示要分道扬镳,却又没能这样做,因为她们最终发现,她们之中每一个都是其余两个排遣晚年寂寞和防止老人痼疾猝发所少不了的。但是,尽管她们的生活没有多少乐趣,尽管在她们莫斯科的府第里笼罩着肃穆寂寥的气氛,全城名流还是认为自己有义务不间断地向三位深居简出的老太太问安。大家把她们视为保存着全套贵族祖训传统的司库,看作阀阅世家的活的年鉴。伯爵夫人给人们留下许多美好的回忆,她是一位出类拔萃的女性。从彼得堡来的人总是最先拜访这几位。凡是受到过她们接见的,别处的门也就为他们敞开着。但是伯爵夫人去世了,姐妹也分了手:姐姐X郡主留在莫斯科继承没有子女的伯爵夫人遗产中归她的那部分,修女妹妹移居到彼得堡她侄儿X公爵那里去。而公爵的两个孩子卡嘉和萨沙,却留在莫斯科的姑婆身边,盘桓尊前,聊慰寂寞。公爵夫人虽然深爱自己的子女,在规定的举丧期间始终和孩子们分开也不敢出一声怨言。我忘了说,我住到公爵府中去时,宅内还在继续志哀;不过,悼亡期不久即将结束。
\par 老郡主全身孝服,总是穿一件普通丝绸料子的黑色连衣裙,戴浆硬的细裥白衬领,这使她的模样有点儿象养老院里收容的老太婆。她从不放下念珠,坐车出门去做礼拜总是郑重其事,逢到斋期肉乳不进,接见的不外乎各级神职人员和老成持重的来客,读的书都是圣经教义,整个生活方式无异于出家人。楼上静得可怕,连房门也不能咿咿呀呀,因为老太太的耳朵象十五岁的姑娘一样灵敏。倘若有什么东西发出咯噔甚或只是嘎吱一声响,老太太马上会派人来询问原因。大家说话都压低嗓门,走路都蹑手蹑脚,可怜那个年纪也已经很大的法国女人最后不得不放弃平素爱穿的响跟鞋。鞋跟只能割爱。我来到宅内两个星期以后,老郡主派人来了解我是谁,怎么会到宅内来的,等等。有关情况立即向她禀报。于是又有第二名专差奉命向法国女人质询:为什么郡主至今没见到过我?这下顿时忙得不亦乐乎:开始替我梳头,洗脸洗手(其实本来就非常干净),教我见了老郡主怎样趋前行礼,怎样显得和颜悦色,怎样说话,——总之,把我闹得晕头转向。然后由我们这方面派出一名女仆去请示:郡主是不是愿意见孤女了?得到的答复是暂时不见,但指定次日做完礼拜以后进谒。我一宿没有睡好,事后有人谈起,我说了一整夜的胡话,我在幻觉中老是走到郡主跟前为什么事情求她宽宥。我进谒的时间终于到了。我见到一位瘦小干瘪的老太太坐在奇大的安乐椅里。她向我频频点头,还戴上眼镜把我仔细打量。据我记忆所及,她对我一点没有好感。当时曾指出我完全不懂规矩,屈膝、吻手一概不合礼数。垂询开始后,我勉强应答;但当问到父母的时候,我哭了。老太太见我这样控制不住自己的感情,大为不悦;不过,她开始安慰我,叫我把希望寄托给上帝;其后又问我最后一次进教堂已有多久。由于我不大明白她问话的意思,因为我在教育方面给忽视得很厉害,致使郡主大惊失色。她派人去把公爵夫人叫来。经过商议,决定本星期日就带我到教堂里去。在这以前郡主表示要为我祈祷,但吩咐把我带出房间,因为据她说,我给她留下了十分不愉快的印象。这没有什么可奇怪的,事情本该如此。但有一点是很明显的:我完全没有赢得好感。当天就有人奉命来说我太顽皮,说我发出的声响整个宅第都听得见,其实我整天坐着一动也不动;显然这是老太太的错觉。可是第二天又下达同样的指责。偏偏那时我失手打破了一只茶杯,法国女人和使女们都陷于绝望,我立刻给搬到最远的一间屋子去住。那些人也都惶惶不可终日地跟我一起前往。
\par 但我不知道这一切后来是怎样告终的。反正因为这个缘故,我乐于下楼去在一间间大屋子里独自漫步,知道在那里我不会惊动任何人。
\par 记得有一次我坐在楼下的一间厅堂里。我双手捂住面孔,脑袋前倾,这样坐了不知几个小时。我老是在思量,在考虑;我的未成熟的智力排解不了我的全部悲哀,只觉得心头愈来愈沉重,愈来愈郁闷。忽然,有人站在我身旁轻声问道;
\par “你怎么啦,可怜的孩子?”
\par 我抬头一看,原来是公爵;他脸上的表情流露出深切的怜悯和同情;但我却以无限沮丧和凄苦的神态望着他,以致他那双碧蓝的大眼睛里闪起了泪花。
\par “苦命的孤儿!”他轻轻抚摩我的脑袋说道。
\par “不,不,不是孤儿!不!”我说,这时从我胸膛里迸出一声悲切的呻吟,接着,我心中的一切都翻腾起来了。我站起来抓住他的一只手吻着,涕泪交流地用恳求的语调一个劲儿地说:
\par “不,不,不是孤儿!不!”
\par “我的孩子你怎么啦,我可爱而又可怜的涅朵琦卡?你怎么啦?”
\par “我的妈妈呢?我妈妈在哪儿?”我嚎啕痛哭放声大叫,再也掩藏不住一腔悲苦,心力交瘁地跪倒在他面前。“我的妈妈呢?亲人哪,告诉我,我妈妈在哪儿?”
\par “原谅我,我的孩子! ······唉,可怜的孩子,是我触痛了她的创伤······。我真不应该!来,跟我来,涅朵琦卡,跟我来。”
\par 他拉着我的手快步走去。他直到心坎深处都给震动了。后来,我们走进我还没有看见过的一间大屋子。
\par 这是一间供着神像的祈祷室。时近黄昏,长明灯的亮光与神像的金叶、宝石交相辉映。亮闪闪的叶片仅仅露出圣徒们暗淡无光的面容。这里的一切跟别的屋子迥然不同,这里的气氛是那么神秘、森严,给我的印象过于强烈,我的心竟为一种恐怖的感觉所控制。除此以外,当时我的神经又是那样脆弱!公爵急忙让我跪在圣母像前,他自己跪在我旁边。
\par “祈祷吧,孩子,祈祷吧;让我们一起祷告!”他用轻微、急促的音调说。
\par 但我无法祈祷;我愣住了,甚至可以说吓慌了,我想起最后那一夜父亲在我母亲的尸体前所说的话,于是我的神经过敏症又告发作。我再次病倒,并且险些乎死在我整个病程的这一继发期内;下面我就要谈到此事的经过。
\par 一天上午,我耳朵里响起一个熟悉的名字。我听到了C——茨的名字。有人在我床畔提到他。我打了个寒噤;往事涌上心头,我在痛苦的回忆和遐想中躺了不知几个钟点,完全陷入谵妄状态。我醒来已经很晚,四周一片黑暗,过夜的小灯熄灭了,坐在我房间里的一个使女也不在。忽然,远处有音乐飘入我耳中。乐声时而完全沉寂下来,时而渐趋清晰可闻,好象愈来愈近。我不记得一种什么样的感受把我抓住,也不记得一种什么样的意图在我紊乱的头脑里产生。反正我下了床,也不知哪来的力气,胡乱穿上我的孝服,摸索着走出房间。在第二间、第三间屋子里我都没有遇见一个人影。后来,我摸到甬道里。乐声愈来愈听得清楚。甬道中央有扶梯通到楼下;我总是走这条路到大房间里去。扶梯上灯光明亮;楼下有人走动;我躲在角落里不让人看见,直到可以了,我才下去进入第二条甬道。响亮的乐声是从隔壁大厅里发出来的;那里喧哗嘈杂,热闹非凡,好象有几千人汇聚一堂。大厅直通甬道的一扇门上挂着大红丝绒的双层巨幅帘幔。我撩起外面的一层,站到两重帘幔之间。我的心跳得几乎叫我站也站不稳。但几分钟以后,我抑止住激动的心情,终于鼓起勇气把第二重帘幔从边上揭开一点点······。我的上帝啊!我一直不敢走进去的这座阴森森的大厅此刻银烛满堂,灯火辉煌,如同光的海洋朝着我涌过来;我已习惯于黑暗的眼睛在最初的一刹那竟被刺得生疼,什么也看不见。香气如一阵热风直扑我的面庞。走来走去的人多得数不清,似乎个个笑容满面,喜气洋洋。女人的服饰都是那样华贵、那样色泽鲜艳;我看到处处都是神采飞扬的眼光。我站在那里,简直跟着了魔一样。我觉得这一切好象在某时某地的梦中见到过······。我回想起昔日的黄昏时分,回想起我们的顶楼、高高的小窗、楼下深处路灯明亮的街道、对面那幢房屋张挂红色帘幔的窗户、大门口车水马龙的景象、趾高气扬的马匹的蹄声和响鼻、嘈杂的人声、窗上的人影、隐约的乐声······。对了,这不正是那个天堂吗? ——这个想法在我头脑里刷地一闪,——这不正是我一心想和可怜的父亲一同去的地方吗?······敢情这不是空想!······是的,过去我在幻想中、在梦境中看到的跟眼前的一模一样!病中发烧的头脑里又点燃起想象的烈火,无法解释的狂喜的热泪夺眶而出。我用目光搜索父亲,我在想:“他一定在此地,一定在。”紧张的期待使我心跳气急。但是音乐停了下来,只觉得耳际嗡然作响,整个大厅里掠过一片窃窃私议之声。我贪婪地注视着在我前边浮晃的一张张面孔,竭力想认出某一个人来。忽然,大厅里起了一阵异样的骚动。我看到台上出现一个高高瘦瘦的老头儿。他那苍白的脸微微含笑,他向左右前后鞠躬致意,弯腰的动作似乎不太灵活;他手中拿着一把小提琴。接着,大厅里鸦雀无声,好象这些人一齐屏住了呼吸。所有的视线同时集中到老头儿身上,大家都在等待。他拿起琴来,用弓触到弦上。音乐开始了,我觉得我的心忽然被什么东西紧紧攥住。我屏住气,忍住无限的悲怆,凝神谛听这声音,只感到有些耳熟,好象我在什么地方听见过;这声音唤起了某种预感,某种非常可怕的事情发生之前的预感,而我的心中也在酝酿着一场凶猛的风暴。随后,琴声转趋激越,音乐变得急速、尖利。接着仿佛有人发出绝望的号叫、凄惨的恸哭,仿佛有谁在这一大群人中间哀哀求告,纵使声泪俱下也是枉然,只得在绝望中吞声。音乐在向我的心诉说一个愈来愈熟悉的故事。但是心不愿相信。我咬紧牙关不打哼哼,我抓住帘幔以免摔倒······。我曾数次闭上眼睛又突然睁开,但愿这是一场梦,但愿在我熟悉的一个可怕的时刻猛然醒来,发现自己梦见了那最后的一夜,听到了同样的声音。我睁开眼睛,希望证实自己的想法,急切地向人群中望去,——不,这是另一些人,这是另一些面孔······。我觉得大家都和我一样期待着什么,和我一样不胜心底隐痛的折磨;他们似乎都想冲着这可怕的呻吟和哀鸣大喝一声,喝令它们静下来,不要作践他们的心灵,似呻吟和哀鸣的音流愈淌愈凄厉、愈悲切、愈拖长。突然间响起最后一声很长的惨叫,把我的五脏六腑都震动了······。毫无疑问!这正是那一声惨叫!我可以断定,我已经听到过,这声惨叫跟当时一样,跟那天夜里一样刺透了我的心。“父亲!父亲!”我头脑里闪电般冒出一个想法。“他在此地,这是他,他在叫我,这是他的琴声!”仿佛一声浩叹发自全体人群,接着,暴雨般的掌声震撼了整个大厅。就在这个当儿,哭声随着绝望的尖叫冲出我的胸膛。我再也忍耐不住,竟掀开帘幔闯进大厅。
\par “爸爸,爸爸!这一定是你!你在哪儿?”我高声叫喊,几乎完全忘其所以。
\par 我不知道我有没有跑到那高个儿老头跟前,只记得人们纷纷闪开,给我让路。我拚命哀叫着向他那边扑过去;我以为马上就要和父亲拥抱······。忽然,我看到有人用细长嶙峋的双手把我抓住了举起来。一双乌黑的眼睛直盯着我,简直想要喷出火来把我烧毁。我瞪着那老头儿。“不!他不是我父亲;他是杀死我父亲的人!”这念头在我脑中一闪。我被某种极度愤激的心情所控制.骤然觉得他冲我发出狂笑,觉得这狂笑在大厅里哄然引起一片喧嚷;我失去了知觉。
\newpage
\section*{五}
\par
\par
\par
\par
\par
\newpage
\section*{六}
\newpage
\section*{七}

\end{document}