\documentclass[12pt, UTF8]{ctexbook}
\usepackage{geometry}
\usepackage[all, PDF]{xy}
\usepackage{indentfirst}
\usepackage{graphicx}
\usepackage{subfigure}
\usepackage{xcolor}
\usepackage{soulutf8}

% subtitle
% Неточка Незванова
\newcommand\subtitle[1]{{\normalsize #1}}

% use russian for author name
\usepackage[OT2,T1]{fontenc}

\setlength{\parindent}{2em}

\geometry{a4paper,centering,scale==0.75, left=2.5cm,right=2cm,top=2.54cm,bottom=2.54cm}

\begin{document}
\title{涅朵奇卡·涅茨瓦诺娃 \\ \subtitle{\fontencoding{OT2}\selectfont Netoqka Nezvanova}}
\author{陀思妥耶夫斯基 \\ \small\fontencoding{OT2}\selectfont F\"edor Miha\u{i}luvhq Dostoevski\u{i}}
\maketitle
\newpage

\section*{一}
\par 我记忆中没有我的生父的印象。他死的时候我才两岁。我母亲又嫁了别人。这次再醮给她带来了很多痛苦,尽管她改嫁是出于爱情。我的继父是个乐师。他的命运很不寻常:这是我认识的人中间最古怪、最奇特的一个。在我童年时代最初的印象中,他留下的痕迹太深刻了,这对我一生都有影响。为了便于理解我要讲的故事,我先在此概述一下他的履历。下面我要讲的一切,都是后来我从大名鼎鼎的小提琴家B那里知道的,他是我继父年轻时的伙伴和密友。
\par 我的继父姓叶非莫夫。他出生于一位非常有钱的地主的村庄,继父的父亲是个穷乐师,度过漫长的漂泊生涯之后在这位地主的庄上落了户,受雇假如他的乐队。这位地主生活极其阔绰,平生最爱音乐,而且爱的成瘾。据说,他从来不离开自己的村庄,连莫斯科也不去,可是有一回突然出国到一处矿泉疗养地去了,而且只去了几个星期,唯一的目的就是去听一位赫赫有名的小提琴家的演奏,因为报上说他要在那个疗养地举行三场音乐会。这位地主拥有一个相当不坏的私人乐队,他几乎把全部收入都花在这上头。我的继父刚进这个乐队时吹单簧管。他在二十二岁那年结识了一个奇怪的人物。在他们那个县份,住着一位富有的伯爵,可是他为了养一个私人戏班子不惜于倾家荡产。这位伯爵因为自己的意大利出生的乐队长行为不端而把他辞退了。乐队长的品性确实不好。他被解雇以后,更是潦倒不堪,老是在乡下小酒店里喝的酩酊大醉,有时候索性祈求施舍,全省谁也不愿意给他一个职位。我的继父竟跟这样一个人交了朋友。这种奇怪的交往实在难以解释,因为谁也看不出我的继父由于学朋友的样在行为方面有什么变化,甚至起初不准他跟那个意大利人厮混的地主,后来对他们的友谊睁一只眼闭一只眼。最后,乐队长突然死了。他是清晨被农民在堤坝旁边的水沟里面发现的。经过验尸,确定他死于中风。他的遗物存放在我的继父那里,我继父当即出示文件,证明他有充分的权力继承这些遗物,因为死者留下一张亲笔所写的字条,指定叶菲莫夫为自己遗产的继承人。遗产包括一件黑色燕尾服和一把小提琴:燕尾服由死者保存的很仔细,因为他始终抱有觅得一席之位的希望;小提琴看上去却很平常。没有人对这比遗产提出什么争议。可是过了若干时日,伯爵乐队里的首席小提琴手带着铂爵的信来见地主。伯爵在信上于叶菲莫夫情商,却他让出意大利人身后留下的那把提琴,因为伯爵很想把它买下来给自己的乐队使用。伯爵愿意出三千卢布,还说已派人去请过叶果尔·叶菲莫夫多次,以便当面了结这笔交易,但他执意不肯。伯爵最后写道,提琴货真,但他的价钱也实足不假,绝不会少一个子儿,并认为叶菲莫夫的顽固是一种多疑的表现,生怕成交是欺负他老实和外行,所以伯爵动了气,请地主开导开导叶菲莫夫。
\par 地主派人把我继父叫去。
\par “你为什么不肯出让提琴?”他问道。“你又用不着它。人家出你三千卢布,这是十足的价钱,要是你以为人家会出更高的价钱,可就错了,伯爵不会欺骗你的。”
\par 叶菲莫夫回答着说,他自己不想去见伯爵,如果要他去,那就只能按主人的意志办;提琴他不愿意卖给伯爵,如果硬要从他这里把提琴抢走,那也只能按住人的意志办。
\par 很明显,他还祥的回答触到了地主性格中最敏感的一根弦。事情是这样的. 地主一向自豪地说他懂得怎样对待他的乐师, 因为他们个个都是真正的艺术家,因此他的乐队并不但笔伯爵的高明,甚至同京城里的乐队相比也不逊色。
\par“好!” ,地主说。“我通知伯爵, 说你不愿卖琴就是不愿, 因为卖与不卖的权利完全在你,懂呜?不过我要问你:你要提琴干嘛? 你的乐器是单簧管. 虽然你的单簧管吹得相当蹩脚。把琴让给我吧。我出三千卢布。(谁知道此琴是这样的一件乐器!)”
\par 叶菲莫夫莞尔一笑。
\par “不,老爷,我不卖给您,”他答道,“当然,您可以······”
\par “难道我在逼你,难道我在强迫你?!”地主终于沉不住气叫了起来,偏偏事情是当着伯爵的乐师之面发生的,来者据此情景可能推断:地主对他的乐队的全体乐师都不给好看待。“滚开,没良心的东西!从今以后别让我再看见你!要是没有我,冲你那支吹得这样糟糕的单簧管,你能上哪儿混饭吃?你在我这里有吃有穿,还领薪俸;你过的是上等人的日子,把你当艺术家看待,可你根本不想明白这一点,简直无知无觉。滚开,别待在此地招我生气!”
\par 地主总是把他生气的对象从自己身边赶走,因为对自己不放心,怕他的火爆性子发作。而他是说什么也不愿意对他的“艺术家”过于严历的(他管自己的乐师们都叫“艺术家”)。
\par 买卖没有成交,事情似乎到此为止了,不料一个月以后,伯爵的小提琴手忽然大启讼端:他本人出首告发我继父应对意大利人之死负责,说我继父怀着自私的目的把他致于死地,为的是占有价值可观的遗产。伯爵的乐师声称遗嘱是在硬逼软骗之下写出来的,并表示能为这项指控提供人证。伯爵劝之再三,地主也为我的继父说情,但什么也不能动摇告发者的主意。人家把情况摊在他面前:法医对乐队长尸体所做的检验是正确的,硬要告发岂非违背明摆着的事实,也许是因为得不到曾经为他洽购的那件名贵乐器而怀恨在心,想泄私愤。伯爵的乐师一意孤行,还赌神罚咒地说自己是对的,说中风并非酗酒引起,而是中毒造成的,要求复查。乍看起来,他的论点似乎颇有道理。不用说,事情还是闹开了。叶菲莫夫被抓起来关进城里的监狱。这桩官司打起来以后,引起了全省的注意。案件进展很快,结果查明伯爵的乐师犯有诬告罪。判决给了他应得的惩罚,但他始终坚持自己的看法是对的。最后,他承认自己没有任何证据,他提出的论点是他自己臆造的,但他捏造所有这些事实是根据一种假设、一种猜想行事的,因为直到复查结束、正式确认叶菲莫夫先生无罪为止,他还坚信不幸的乐队长之死是叶菲莫夫造成的。不过,对此人的判决没有来得及执行,因为他突然患脑炎发了疯,接着就死在监狱医院里。
\par 整个这桩公案的始末,地主的行为是光明磊落的。他为我继父出力,仿佛我继父是他的亲生儿子似的。他曾几次到监狱里去探望我的继父,安慰他,给他钱;得悉叶菲莫夫喜欢抽烟,就给他带去最好的雪茄;宣告我继父无罪释放时,他让乐队全体成员大事庆贺。地主把叶菲莫夫这个案子看作是关系到整个乐队的事情,因为他对自己的乐师的品行即使不比他们的才能看的更重,至少是不相上下。过了整整一年,忽然有消息在省里传开,说是一位著名的法国小提琴家途径省城,打算举行几场音乐会。地主马上开始设法请他来做客。事情进行的很顺,法国人答应前来。对他的来临已经做好一切准备,还邀请了县里几乎所有的知名人士,不料情况陡起变化。
\par 一天早晨,有人来报告说,叶菲莫夫不知去向。开始到各处去寻找,可是杳无踪影。乐队缺了一只单簧管岂不急死人?在叶菲莫夫失踪三天之后,地主忽然接到法国人写来的信,那位小提琴家在信中傲慢地拒绝了地主的邀请,并且表示(当然是拐弯抹角地),今后在跟那些私人拥有乐队的老爷们打交道时要格外谨慎,看到真正的天才处在根本不知道其价值地人监督之下实在太杀风景,临了还说,叶菲莫夫的例子足以证实他所言不虚,此人是位真正地艺术家,是他在俄国遇到的最好的小提琴演奏家。
\par 地主读了这封信大为惊愕。他这一气直气的他发昏章第十一。什么?他对叶菲莫夫如此关怀备至,如此厚爱有加,而就是这个叶菲莫夫竟昧着良心向一位欧洲的艺术家,向他高度敬重其见解的人如此恶毒地诋毁他!此外,这封信在另一点上也是令人不解的;信中说,叶菲莫夫是个有真正天才的艺术家,他是位小提琴家,可是人们竟不能发现他的才华,强迫他演奏另一种乐器。这一切使地主惊讶万分,他当下准备进城去面晤法国人,忽然伯爵派人送来一封便简,请地主立刻到他那里去,并说他了解全部情况,那位路过的演奏家此时在他家里,叶菲莫夫也在,伯爵闻说叶菲莫夫的诽谤后大为震惊,已下令不准他离开,信上最后说,之所以必须请地主前去,还因为叶菲莫夫的指责甚至涉及伯爵本人;兹事体大,必须尽快加以澄清云云。
\par 地主马上赶往伯爵家中,随即同法国人见面,向他介绍了我继父的全部身世,并说他没料想到叶菲莫夫竟有这般了不起的才华,相反,叶菲莫夫在他那里只是一个很蹩脚的单簧管手,现在他头一回听说,这个离他而去的乐师竟是位小提琴家。地主还说,叶菲莫夫并不是农奴,他享有完全的自由,如果他确实受到束缚,任何时候都可以离开他家。法国人感到很奇怪。他们把叶菲莫夫叫来,他简直跟过去判若两人:态度傲慢,答话带着讥笑,并坚持自己向法国人所说的情况属实。这一切使伯爵恼怒到了极点,他当面骂我的继父是混蛋、造谣中伤的小人,应该得到最可耻的下场。
\par “请放心,伯爵大人,敝人跟阁下并非初次相交,对阁下颇有了解,”我的继父说,“多蒙阁下恩典,在下几乎受到刑事处分。敝人知道,阿列克塞·尼基福雷奇——府上过去的乐师——告发在下是受了何人的嗾使。”
\par 听到这样骇人听闻的责难,伯爵肺都快气炸了。他好不容易才控制住自己;但此时厅堂里一位有事来见的伯爵的官员宣称他不能听任这一切不了了之,说叶菲莫夫这种侮辱性的物理态度包含着恶毒的、不公正的指控和污蔑,所以他谨请允许他在伯爵府第里立即逮捕叶菲莫夫。法国人也表示极度的慷慨,说他无法理解这种丧尽天良的负义行为。于是我的继父暴跳如雷地回答,他宁可接受处分、审判,哪怕再来一次刑事侦讯,也强似迄今为止他在地主乐队里过的那种日子,由于他极度贫困,不能更早离开那里;说完,他就跟逮捕他的官员一起走出厅堂。他被锁在一间偏僻的屋子里,说是明天把他押送进城。
\par 将近午夜时分,拘押我继父物的屋子门打开了。进来的是地主。他穿着睡袍,趿着拖鞋,双手打着一盏点亮的灯笼。看来他睡不着,痛心的焦虑迫使他在这般时分离开床衾。叶果尔也没睡,他惊讶地望着进来的地主。地主把灯笼放好,怀着十分激动的心情坐在他对面的一把椅子上。
\par “叶果尔,”他对叶菲莫夫说,“你为什么要这样伤我的心?”
\par 叶菲莫夫不答。地主又问了一遍,他的话流露出某种深刻的感情,一种奇怪的忧伤。
\par “天知道我为什么要这样伤您的心,老爷!”我继父终于一甩手答道。“想必是鬼迷住了我的心窍!我自己也不知道是谁推动着我这样干!反正我不能在您那儿再待下去,不能再待下去······。魔鬼把我缠住了!”
\par “叶果尔!”地主又开言道。“回到我那儿去吧,我把一切都忘掉,什么都原谅你。听着:你可以当我的首席乐师;我给你一份跟旁人不一样的薪金······”
\par “不,老爷,不,您别说了:我不能在您那儿待下去!我告诉您,魔鬼缠上了我。我要是再待下去,会放火烧掉您的房子;有时候我苦闷得只恨爹娘不该把我生下来!现在我自己也不能担保,老爷,您还是别管我吧。这都是从那个魔鬼跟我结交后开始的······”
\par “谁?”地主问。
\par “就是那个像狗一样咽了气的意大利人,这条狗导出都不受欢迎。”
\par “这么说,叶果鲁什卡,是他教你拉琴的喽?”
\par “是的!他教会了我许多东西,把我引向毁灭。我还是从来没有见到他的好。”
\par “难道他是个小提琴高手,叶果鲁什卡?”
\par “不,他自己知道的不多,可是教的挺好。我是自己学会的;他只不过是示范而已,——这比正规的办法容易,要是我撒谎,就让我这支胳臂烂掉。现在我自己也不知道要什么。老爷,您问我:‘叶果尔卡!你要什么?我什么都能给你,’——可是,老爷,我一句话也没法回答您,一位我自己不知道要什么。老爷,您还是别管我吧,下次再说。我要对自己干一件惊人的事情,好让我被远远地打发走,事情才能了结!”
\par “叶果尔!”地主在沉默片刻后说。“我不能这样撇下你不管。既然你不愿意在我那里干下去,你可以走;你是自由人,我不能强留;但我现在不肯就这样离开你。你用你的琴拉一首曲子给我听听,叶果尔,拉吧!看在上帝的份上,拉一首!我不是命令你,你要明白我的意思,我不强迫你;我含着眼泪请求你:叶果鲁什卡,看在上帝的份上,把你为法国人演奏的曲子也拉给我听听!吐一吐你的心曲!你很固执,我也很固执;要知道我也有自己的脾气,叶果鲁什卡!我心中是有你的,你心中也该像我一样才对。除非你自愿且乐意地把为法国人演奏的曲子拉给我听,否则我日子没法过。”
\par “行,那就照办!”叶菲莫夫说。“老爷,我发过誓永远不在您面前演奏,单单不为您演奏,可现在我心上的束缚解除了。我可以为您演奏不过这是第一次,也是最后一次,老爷,以后您在任何时候、任何地方再也听不到我的演奏,哪怕向我许一千卢布的愿也没用。”
\par 于是他拿起琴来,开始演奏他自己所作的俄罗斯民歌变奏曲。据B说,这套变奏曲是他的第一部、也是写的最好的小提琴作品,此后他任何曲子从来都没有演奏的这样好,这样充满灵感。这位地主听音乐本来就没法不动心,这回索性放声痛哭。演奏完毕时,他从椅子上站起来,掏出三百卢布交给我的继父,说:
\par “现在你走吧,叶果尔。我把你从这里放出去,伯爵那一头我去对付;不过你听着:往后你可别跟我遇上。你面前的道路宽又广,万一咱俩在路上碰头,对我对你都不好受。好啦,那就分手吧!······等一下!我对你还有一句临别赠言,很简单:别喝酒,要用功,莫骄傲!我是像你的亲生父亲一样对你说话。注意,我重复一下:要用过,别喝酒,一旦你开始借酒浇愁(叫人愁苦的事将来多的很!)——那就前功尽弃,非完蛋不可,也许你自己会像那个意大利人一样在不定什么地方的水沟里咽气。好啦,现在分手吧!······等一等,吻我一下!”
\par 他们相互吻了一下,随后我的继父就被放出去。
\par 他刚获得自由,先是立即在附近一个县城里把三百卢布胡乱花光,同时跟一帮最堕落、最下流的无赖交上朋友,后来一个人落得穷愁潦倒、孤苦无依,不得不加入一个小地方的流动戏班子,在可怜巴巴的乐队里拉第一个小提琴,或许是唯一小提琴。这一切跟他原先的设想不太吻合,他本来打算尽快到彼得堡去深造,谋到一个好位子,不折不扣地把自己造成一个艺术家。但是,小乐队里的生活很不如意。我继父不久就跟江湖戏班子的班主闹翻,并且离开了那里。那时他彻底泄了气,甚至不顾一切走出了深深刺痛他的自尊心的一步。他写了一封信给前面提到的那位地主,向它描述了自己的境况,请他资助。信的语气还相当要面子,但是没有回音。于是他再写一封,措辞上极尽卑躬屈膝之能事,称地主为自己的恩人,把他尊为真正的艺术鉴赏家,目的还是求他帮助。回音总算来了。地主寄来一百卢布和由他贴身侍从代笔的寥寥数行复信,叫叶菲莫夫今后再也不要向他提出任何请求。继父得到了这点钱,当即想动身去彼得堡,可是付账还债以后,只剩下那么一点点钱,彼得堡之行根本无法考虑。他仍旧留在小地方,重新加入一个小地方乐队,后来在那里又待不下去,如此不断地换来换去,心中念念不忘能快一点去彼得堡,其实却在小地方跑了整整六年。后来,他忽然大起恐慌。他绝望地发现,在不规则的、贫困的生活不断折磨下,他的才华遭到了不知多大的损失,于是在一个早上抛下班主,拿起提琴,几乎靠乞讨走到彼得堡。他在某处的一个楼顶上住下,在那里第一次遇到了B,彼时B刚从德国来,也想为自己开辟前程。他们很快就交了朋友,B至今还满怀深情回忆他们相交一场。他俩都是青年,怀抱一样的希望,有着相同的目标。但B的青春还刚刚开始,他经历的贫困和苦痛尚少;撇开这些不谈,他首先是一个日耳曼人,在奔向目标的道路上能坚持不懈、持之以恒,充分意识到自己的力量,并且对于自己能有多大的作为几乎早有成竹在胸。可是他的伙伴叶菲莫夫已经三十岁;他已经疲倦、困乏,整整七年不得不东飘西泊,在小地方的戏班子和地主的私人乐队里混口饭吃,耐心既完全丧失,最初旺盛的精力也消耗殆尽。过去支撑着他的只有一个永远不变的固定观念——好歹得摆脱窘境,积一笔钱到彼得堡去。但这个观念事模糊的、朦胧的;这是一种不可违拗的内心的召唤,随着岁月的流逝,这呼声在叶菲莫夫心中也不象最初那样清晰了,当他来到彼得堡时,几乎已经处于无意识状态,只是按照夙愿和反复思量这次进京的老习惯行事而,几乎连自己也不知道要在京城里干什么。他的热情近似歇斯底里,带有怄气和阵发的性质,似乎他想用这种热情欺骗自己,借以使自己相信,他身上最初的精力、最初的热情、最初的灵感尚未枯竭。这股不断迸发的劲头个不冷不热、有条有理的B震动很大;他感到目眩神迷,把我继父当作未来伟大的天才音乐家看待。他不能想象这位伙伴将来的命运会是什么别的样子。但不久B就睁开眼睛把他看透了。他清楚地看到,所有这些阵发地狂热、焦躁的情绪无非是想到自己怀才不遇而不自觉地表现出来的绝望的挣扎;说到底,甚至他的才也许一开始就没有什么了不起,而多是盲目和不切实际的自信、浅薄的自负以及不断幻想自己是盖世奇才的白日梦。“但是,”B如此说,“我对我这位伙伴奇异的天性不能不表示惊讶。我眼看着在病态的强烈欲望与内心的软弱无力之间不断进行凶猛的生死搏斗。这个不幸的人整整七年光靠幻想将来成名聊以自慰,甚至在不知不觉中丢掉了我们这些技艺中最起码的东起,甚至丧失了最基本的业务能力。偏偏在他乱糟糟的想象中无时无刻不在为未来构思气吞山河的宏大计划。他不唯要成为第一流的天才,成为世界上数一数二的小提琴家;他不唯已经把自己当作这样的天才,——他还想成为作曲家,事实上他根本不懂得对位法。但最使我惊讶的是,”B又说,“这个人尽管绝对无能,尽管在技艺方面只是极其贫乏,然而他对艺术却有非常深刻、非常明晰、可以说是出于本能的立即。他的艺术感和鉴赏力是那样高超,无怪乎会失去自知之明,看不到本能决定他是个深刻的艺术批评家,却把自己当作艺术大师、天才演奏家。有时候他用毫无学术味道的粗言俗语能对我说出极其深刻的道理,使我大惑不解:他从来不看书报,什么也不学,可是这一切他是通过什么方式领悟的呢?我在自修过程中,”B接着说,“有许多地方得益于他和他的指点。至于我本人,”B继续说,“我对自己的看法是稳定的。我也热爱本行艺术,虽然从我一开始走上这条道路就有自知之明,知道在某种意义上我只能当一名艺术的苦力;但我感到自豪的是没有象一个懒惰的奴隶那样把自己的拿点天赋埋没掉,相反把它扩到了一百倍;如果说,现在人们夸奖我演奏时干净利落,惊叹功夫到家,那末,这一切都得归功于坚持不懈的努力,归功于有自知之明,宁可把自己看的渺小,永远力戒骄傲,力戒过早的自满,力戒懒惰,因为懒惰是这种自满情绪的必然结果。”
\par B也曾尝试向自己最初非常佩服的伙伴进几句忠言,但结果只是白白惹他生气。他们之间的关系开始疏远了。不久B注意到,淡漠、苦闷和无聊开始愈来愈频繁地控制叶菲莫夫,而他热情冲动的次数则愈来愈少,过后出现的是一种阴沉凄凉、灰心丧气的状态。再后来,叶菲莫夫干脆把琴放下,有时一连几个星期不去碰它。这与彻底的堕落已相去不远,很快,这个不幸的人染上了所有的恶习。当初地主告诫他的事情果然发生了,他开始纵酒无度。B瞧着他无法不感到震惊,他的忠告不起作用,他甚至不敢开口。渐渐地,叶菲莫夫落到恬不知耻的地步:他竟心安理得地花B的钱过日子,这样做甚至好像有充分权力似的。其时维持生活的钱行将告罄;B靠教课竭力苦撑,或者受雇在商人、日耳曼人、小官吏家的晚会上演奏,尽管报酬很少,但他们总能给几个钱。叶菲莫夫根本不愿看到伙伴的难处,对他疾言厉色,有时几个星期不跟他说一句话。一次,B用及其婉转的语气劝他不要过于轻视那把琴,以免对此乐器完全荒废;不料叶菲莫夫大发脾气,声言他将故意永远不去碰自己的琴,那副架势好像会有人跪下来求他似的。另一次,B在一个晚会上演奏需要有个搭档,就请叶菲莫夫跟他合作。这一邀竟使叶菲莫夫暴跳如雷。他怒气冲冲地宣称自己不是街头提琴师,不会像B那样卑鄙,在完全不能赏识他的功夫和才华地臭商人面前拉琴,贬低崇高的艺术。B听罢没回答一句话,出门演奏去了,但叶菲莫夫在伙伴走后对这次邀请反复思考,认为这一切无非暗示他在花B的钱过日子,想让他知道,叫他尝试挣点钱。等B回来以后,叶菲莫夫忽然斥责他行为卑鄙,并表示一分钟也不能跟他待在一起。他确实有两天不知去向,但第三天回来了,仿佛什么也没有发生似的,又继续过原来地那种生活。
\par B本来想结束这种不像话的生活,跟他的伙伴一刀两断,仅仅由于旧的习惯和友谊,加上B瞧着那个堕落的人心中老大不忍,才没有这样做。后来,他们终于分道扬镳。命运向B作了微笑:他找到一座强有力的靠山,成功地举行了一场出色的音乐会。那时他已经是个优秀的艺术家,他那蒸蒸日上的名气旋即给他带来歌剧院乐队里面的一个席位,在那里他很快就取得了完全应该取得的成功。分手的时候,他给了叶菲莫夫一些钱,含着眼泪恳求他回到正路上来。直到现在,B想起他来还是抑制不住一种特殊的感情。跟叶菲莫夫相交一场是他青年时代最深刻的印象之一。他们曾一起开始向自己的目标进军,彼此曾有过非常热烈的好感,叶菲莫夫的乖张性格和十分显著的缺点本身甚至使B对他产生更加强烈的感情。B了解他;他把叶菲莫夫看得透亮,事先就知道这一切将以上面告终。分袂之际,他们相互拥抱,两人都哭了。当时叶菲莫夫流着眼泪呜呜咽咽地说,他是个彻底完蛋的,最最不幸的人,这一点他早就知道了,但现在才看清自己的末路。
\par “我没有才华!”他的除了结论,说时面无人色。
\par B大大地为之心动。
\par “听着,叶果尔·彼得罗维奇,”他对我继父说,“你何苦自暴自弃呢?你的绝望只能毁了你自己;你既没有耐性,又没有勇气。刚才你在灰心丧气的情绪控制下说自己没有才华。不对!你有才华,你可以相信我的话。你有才华。单凭你对艺术的感受和理解,我就看出这一点。我可以用你的全部生活向你证明这一点。你不是把你过去的经历告诉过我吗?当初你也曾不自觉地陷于同样绝望地境地。那时,你的第一位老师——你曾经对我讲过他很多故事的那个怪人——最先在你身上激发起对艺术的爱,最先察觉到你的才华。当时你也产生了强烈而痛心的感觉,就跟现在一样。但你自己不知道你是怎么搞的。你在地主家里待不下去,你自己也不知道你到底要什么。你的老师死得太早。他撇下你的时候你只有一些模模糊糊的志向,主要的是没有使你认清自己。你觉得你需要走另一条更宽广的路,应该向另外的目标进军,可是你不懂得该如何去达到目的,于是在苦闷中痛恨当时你周围的一切。你六年贫困的岁月没有虚度;你在学,在想,在认识自己和自己的能力,现在你对艺术,对自己的使命有了理解。我的朋友,需要耐性和勇气。等待着你的命运要比我的更值得羡慕:你的艺术家气质超过我一百倍,只要上帝把我的耐性的十分之一赐给你就够了。要用功,别喝酒,正像那位好心的地主对你说过的那样,主要的是你得重新起步,从头开始。什么东西使你苦恼?贫困?但贫困能造就艺术家。事业的起点总是和穷字分不开的。现在还没有人把你放在眼里,谁都不愿意认识你;人世间就是这样。你等着,一旦人们知道你有才能,可气的事情还不止这些呢。妒忌、卑劣的小心眼儿、特别是种种荒唐的蠢事将比贫穷更加变本加厉地往你身上压下来。才华需要同情,需要有人理解,可是你知道取得一点点成就,你会看到,包围着你的将是些什么样的面孔。他们会把你靠艰苦的劳动、咬紧牙关、挨饿熬夜练出来的功夫说的一文不值,对你嗤之以鼻。你未来的伙伴们不会鼓励你、安慰你;他们不会向你指出你的真善美,但会幸灾乐祸地挑你的每一处毛病,偏偏把你不好和不对的地方指向你,表面上冷冰冰地瞧不起你,心里却象过节一样庆祝你犯的每一个错误(好像有人能不犯错误似的!)。你生性傲慢,往往在不适当的场合逞骄,可能会得罪自尊心很强的小人,那就糟了——你只有一个人,而他们人多;他们会像针刺那样折磨你。甚至我也已经开始有此感受。你得立刻振作精神!你还不算太穷,你的日子能过下去,不要嫌弃粗活,有柴就劈,象我在不足道地生意人晚会上劈柴一样。但你缺乏耐性,你有急躁的毛病;你不够朴实,耍小聪明太多,想得太多,脑筋动得太多;你口头上大言不惭,临到需要拿起琴弓地时候又胆怯了。你自尊心太强,胆量又太小。勇敢一些,耐性一些,耐心等待,学着点儿,如果你不寄望于自己的力量,那就碰碰运气;你身上有激情,有感觉。也许碰运气能达到目的,即使达不到,也不妨碰碰运气,反正不会失去什么,因为奖赏实在太大了。这个意义上说,老兄,咱们的运气是了不起的大事。”
\par 叶菲莫夫怀着很深的感情听过去伙伴这番话。但在B往下说的过程中,叶菲莫夫苍白的脸色逐渐泛红,两腮出现了血色;他的眼睛里燃起了少有的勇气和希望。不久,这种高尚的勇气转换为自负,接着又变成平时那份狂妄,最后,当B快要结束这番规劝的时候,叶菲莫夫已经心不在焉,听得不耐烦了。不过他还是热烈地和B握手,向他表示感谢,并且拿出从自暴自弃和灰心丧气迅速跃向极端傲慢狂妄那种老脾气,用过于自信的口气叫他的朋友不必为他的命运操心,说他知道该如何安排自己的未来,他希望不久也能找到一座靠山,举行音乐会,那时便可一下子名利双收。B耸耸肩膀,但没有给过去的伙伴泼冷水他们就此分手,虽然分别的时间不长——这是不言而喻的。叶菲莫夫把给他的钱立刻花的精光,又去要了第二次,然后是第三次,然后是第四次······第十次,最后B实在忍不住了,便推说不在家。从此叶菲莫夫下落不明。
\par 几年过去了。有一次B排练归来,在一条陋巷里肮脏的小酒店门口碰见一个衣衫褴褛的醉汉再叫他的名字。那人是叶菲莫夫。他大大变了样,脸上浮肿、发黄;显然,放荡的生活在他身上打下了不可磨灭的烙印。B非常高兴,没说上两句话,就被他拖进小酒店。到了那意见偏僻、乌黑的小屋子里,B才看清楚他的摸样。叶菲莫夫的衣着几乎全是破烂,一双靴子坏得不像话。衬衣的前胸沾满酒污。他的头发开始斑白、脱落。
\par “你怎么啦?你现在哪里?”B问。
\par 叶菲莫夫窘态毕露,起初甚至有些发慌,语无伦次,答非所问,以致B以为他神经错乱了呢。后来,叶菲莫夫承认,不喝一点伏特加他就没办法说话,可是小酒店里对他早就不相信了。他这样说时脸是红的,尽管竭力做一些豪放的手势给自己壮胆;但造成的印象却是厚颜无耻、娇柔做作,简直令人不忍卒睹;善良的B看到自己原先的忧虑果然完全成为事实,又动了恻隐之心。他先吩咐把伏特加端上来。叶菲莫夫由于感激脸上顿时变样,他激动得含着眼泪准备吻他恩人的手。进餐的时候、B惊讶万分地得悉这个不幸的人结了婚。但更使他惊讶的是了解到,妻子竟然构成了叶菲莫夫的全部灾难和悲哀,结婚彻底摧残了他的才华。
\par “怎么会的呢?”B问。
\par “老弟,我已经两年没拿起提琴了,”叶菲莫夫答道。“简直是个村妇、厨娘、毫无教养的粗俗女人。别提她了!······我们一天到晚打架,旁的什么也不干。”
\par “既然这样,你为什么要结婚?”
\par “当时没东西吃啊。我结识了她:她有千把卢布,我就不管三七二十一结了婚。是她爱上我的。她自己缠住我不放。又没人撺掇她!钱花完了、喝光了,老弟,哪儿还有什么才华!全都落了空!”
\par B看得出,叶菲莫夫似乎急于向他表白自己没有过错。
\par “我把一切扔下了,”叶菲莫夫又说。他向B表示,在这以前他的琴艺几乎达到了完美的境界;虽然从我一开始走上这条道路就有自知之明B是全城数一数二的小提琴家,可是跟他比起来还差得远呢,如果叶菲莫夫愿意的话。
\par “那你为什么不干呢?”B诧异地问。“你该找份事情做啊!”
\par “没意思!”叶菲莫夫一甩手说。“你们那里有谁懂得一点点皮毛?你们懂得什么?懂个屁!你们只会从什么芭蕾音乐中抽一段五去来胡闹一通。优秀的小提琴家你们既没有见过,也没有听过。何必去碰你们呢;你们爱怎么干,就怎么干吧!”
\par 说到这里,叶菲莫夫又把手一甩,身体在椅子上一晃,因为他已经醉得相当厉害。后来他邀请B上他家去;但是B谢绝了,只问了他的住址,答应明天就去看他。叶菲莫夫此刻酒醉饭饱,已经用嘲弄的目光瞧他过去的伙伴,千方百计用话刺他。他们离座起身时,叶菲莫夫连忙把B的贵重皮裘递给他,做出以卑事尊的样子。经过第一间屋子,他就停下来向酒店里的掌柜、伙计和客人介绍B是全城首屈一指和独一无二的小提琴家。总而言之,此时此刻他的表演十分令人恶心。
\par 第二天上午,B在顶楼上找到了他,当时我们全家就住在那么一间屋子里,过着极度穷困的生活。我那时只有四岁,可是我母亲改嫁叶菲莫夫已有两年。我母亲是个不幸的女人。从前她当过家庭教师,受过很好的教育,长得也漂亮,可是由于穷,嫁给了一个老公务员,也就是我的生父。他们在一起只过了一年。我的生父突然去世以后,为数无多的遗产由他的各个继承人瓜分,留下母亲和我,还有就是她分得的一点点钱。带着一个还不会走路的婴儿再去当家庭教师谈何容易。这时,她在一个偶然的机缘下遇到了叶菲莫夫,的确爱上了他。我母亲富于热情和幻想,把叶菲莫夫当作了不起的天才,相信了他前程似锦的那些自命不凡的话,想到有幸成为一个天才的精神支柱和生活指导,我母亲得意非凡,就嫁给了他。不出一个月,她的理想和希望统统化作泡影,她面前留下的只是寒碜的现实。叶菲莫夫跟我母亲结婚也许确实看在她有千把卢布这一点上,等到钱一花完,便叉起两支胳膊,似乎欣然找到一个借口,立即向所有的人宣称,结婚毁了他的才华,说他在窄闷的屋子里面对饥饿的一家子没法工作,说在这样的条件下头脑里不可能产生歌曲和音乐,看来他命中注定得受这份罪云云。后来,好像他自己也确信自己的牢骚发得有理,似乎还为有了新的口实感到高兴。看样子,这位不幸的、毁掉的天才自己在寻找表面的理由。以便把所有的挫折、所有的灾难一古脑儿推到那上头。至于接受这样一个可怕的念头:他在艺术上早已经完蛋,而且是永远地完蛋了——他做不到。他象抵抗恶梦似地向这个可怕的结论作拼死的挣扎,及至现实把他压倒、使他偶尔有儿分钟睁开眼睛的时候,他觉得自己马上就会吓得发疯。对于长期以来构成他全部生活的信念,他不能这样轻易地放弃,直到最后一分钟依然认为那一分钟还没有过去。在彷徨的时刻,他就借熏人的酒味浇胸中之块垒。可能,他自己也不知道这时候妻子对于他是多么需要。这是一个活的借口,的确,我继父对这样一个设想几乎着了魔,认为一旦他把坑了他的妻子葬入坟墓,一切都将走上正轨。可怜的妈妈不理解他。作为一个十足的幻想家,在冷酷无情的现实中刚迈出第一步,她就受不了,变得暴躁易怒,动辄骂人,时刻跟存心折磨她取乐的丈夫吵架,不断逼着他工作。但是,我继父的盲目自大、固定观念和荒谬见解使他变得几乎毫无心肝和麻木不仁。他只是笑着赌咒决不拿起提琴,除非妻子死去,并以狠心的直率态度向她宣布这一决心。不管出现什么情况,妈妈对他的炽热的爱至死不渝,可是这样的生活她忍受不了。她变得老是生病、多灾多难,身心处在没完没了的痛苦之中,陈了这一切不幸之外,一家人的吃饭问题全要她一个人操心。她开始做菜,承接包饭。但是丈夫偷偷地从她那儿把钱都弄走,逼得她常常开不出人家包的饭。当B来访的时候,她正在洗衣服,并且把一件旧的连衫裙重新染色。我们就一直这样在我们的顶楼上凑合着过。
\par 我们家的贫穷使B吃惊。
\par “你呀,完全是胡说八道,”他对我继父说,“这儿哪有什么摧残才华的事情?她明明在养活你,你在干什么?”
\par “什么也不干!”继父回答说。
\par 但B还不知道妈妈所有的灾难。丈夫经常把形形色色捣乱撒野的浪荡鬼一大帮一大帮地带到家里来,那时真可以说是无所不为!
\par B对他过去的伙伴劝说了半天,最后表示,如果叶菲莫夫再不改过,他决不提供任何帮助,还直截了当地说不给他钱,因为他又会把钱喝光,临了他要叶菲莫夫用提琴拉点儿什么给他听听,看看能为叶菲莫夫想些什么办法。等我继父去取琴的时候,B悄悄地要把钱给我母亲,可是她不拿。她这是第一次落到接受施舍的地步!于是B就把钱交给我,可怜的妈妈哭了。继父取了琴来,但要求先给他一点伏特加,说否则没法演奏。于是就派人去买伏特加。酒喝下去以后,劲头就上来了。
\par “看在朋友交情上,我给你拉一首我自己的作品。”他对B说,并从柜子底下拖出厚厚一本尘封的练习簿。
\par “这些都是我自己写的,”他指着本子说。“你不妨瞧瞧!老弟,这可不是你们的那些芭蕾音乐!”
\par B默默地翻阅了几页,然后打开他自己带着的乐谱,要我继父把自己的作品放在一边,从他带来的乐谱中拉几段给他听听。
\par 继父有些动气,不过由于担心失去新的靠山,还是照B的吩咐做了。于是B发现,在他们分手期间,他从前的伙伴确实下了很多功夫,进步不小,虽然他吹说打从结婚以后就没有拿过琴。我那可怜的母亲高兴得什么似的。她瞧着丈夫,重又为他感到自豪。善良的B也衷心欣喜,决定设法安置我继父。那时B已经认识许多有地位的人物,他当即开始去托人情,向人推荐这位可怜的伙伴,但事前要他保证好自为之。B先掏钱把他的衣着搞得象样些,带着他去见几位名人,因为B打算给他谋的位子关键在那些人物身上。叶菲莫夫只是口头上神气十足,其实恐怕他万分乐意地接受了老朋友的建议。据B说,我继父拚命讨好他,生怕失去他的照应,那种阿谀奉迎、卑躬屈膝的丑态使B为他害臊。叶菲莫夫明白别人想把他引上正途,甚至酒也不喝了。后来,在一个剧团的乐队里总算给他谋到一个位子。他以良好的成绩通过了考试,因为一个月刻苦努力他就把一年半荒废的东西补了回来,并保证要继续练功和一丝不苟地对待新的职责。但我们家的境况毫无改善。继父的薪金他一个子儿也不给妈妈,全都自己花掉,跟他很快又交上的一批新朋友一起喝光、吃光。他结交的大都是剧团职员、合唱队员、龙套演员,总之都是些他可以在其中高踞首席的人,对于真正有才能的却避不交往。他已使那些新朋友对他产生某种特殊的敬意,刚认识就向他们宣传,他是个被埋没的人,他有伟大的天才,是妻子断送了他,并说他们的乐队指挥根本不懂音乐。他嘲笑乐队里所有的演奏员,嘲笑排演的剧目,还嘲笑上演的几部歌剧的作者。后来,他开始侈谈一种新的音乐理论,总之,使整个乐队都感到腻烦!他跟同事、指挥一一闹翻,对上司傲慢无礼,成了出名的一个最不安分、最爱争吵、同时又最卑琐的人,弄得人人讨厌。
\par 这样一个不足道的人,这样一个成事不足、败事有余的演奏员,偏偏这样踌躇满志,这样自命不凡、目空一切,确实是非常奇特的怪现象。
\par 最后,继父跟B也翻了脸;他编造了最拙劣的谣言当作明明白白的事实抛出去,对B进行极其恶毒的诽谤。他在乐队里鬼混了半年之后,终于因玩忽职守、行为失检被撵走。但他没有随即离开那个地方。不久,有人看到他身上又跟过去一样破破烂烂,因为比较像样的衣服又都变卖和抵押了。他去找过去的同事,不管人家欢迎不欢迎这样的客人,在他们面前散播流言,搬弄是非,哭诉日子难过,叫所有的人去看看他那恶魔般的妻子。当然,有人爱听,有人乐于给这位遭逐的同伴灌上几杯,让他大放厥词。何况他的嘴皮子总是那样尖刻,往往击中要害,而夹杂在话中的那股冲天的怨气和各种放肆的怪论,自有一部分人欣赏。他被当作一个神经失常的小丑,闲得无聊时不妨叫他胡言乱语扯上一通。人们喜欢当着他的面谈论某一位新来的小提琴家,故意逗他。叶菲莫夫听到这话,脸上就会变色,怯生生地打听是谁来了,谁是崭露头角的天才,并立即开始妒忌他的声誉。大概从这个时候开始,他才真正陷入经常性的精神错乱,也就是形成了一个固定观念:他是首屈一指的小提琴家,至少在彼得堡是无双的,但他命运多舛,怀才不遇,由于种种妒贤忌能的阴谋,他始终不为人所了解,至今没没无闻。他对末了这一点甚至颇为得意,因为有这样一些人就是喜欢把自己看作受欺压者,喜欢把抱怨挂在嘴上,或在暗中崇拜自己得不到承认的伟大聊以自慰。彼得堡所有的小提琴手他个个了如指掌,在他看来,其中没有一人能与他匹敌。凡是知道这个疯疯癫癫的可怜虫的,不管是行家还是逢场作戏的票友,都喜欢在他面前谈论某某才华出众的著名小提琴家,好让他也发表自己的看法。他们没有人能如此巧妙地用如此夸张的漫画手法描绘当代著名的音乐家。甚至遭到他这般挖苦的那些艺术家也有些怕他,因为知道他那张嘴够损的,承认他的指摘有根据,他的见解有道理,如果确实该骂的话。人们惯常在剧场的走廊里和后台看到他。职员们对他不加拦阻,因为少不得这样一个人,于是他成了个国产的忒耳西忒斯【荷马的史诗《伊里亚特》中人物,他别有用心地劝兵临特洛伊城下的希腊人不要打完仗就回去,通常用来比喻喋喋不休地喜欢跟大家抬扛的人。莎士比亚的历史剧《特洛伊罗斯与克瑞西达》也写了这个丑陋和好谩骂的希腊人。】。这样的生活持续了两三年,最后,他连扮演这样一个角色也惹得人人厌烦。接下来我继父遭到毫不含糊的驱逐,在他生前的最后两年似乎销声匿迹了,哪儿也看不到他。不过,B遇见过他两次,可是瞧着他那副狼狈相,在B的心中同情又压倒了厌恶。B招呼他,但我继父动了气,假装没听见,把一顶破旧不堪的帽子拉得遮住眼睛,打旁边走过去。后来,在一个不知什么大节日,一清早有人向B通报,说他以前的伙伴叶菲莫夫来向他拜节。B出去见他。叶菲莫夫喝得醉醺醺地站在那儿,开始行大幅度鞠躬礼,几乎碰到地上,嘴唇不住牵动,固执地不肯走进房间。他的举动的意思是说:我们这种没有才能的人,怎么能同您这样的名流交往,我们这种小人物能到此听差的所在,来向您拜个节,行个礼就走,已经心满意足。总之,这一切恶劣、无聊之至,令人作呕。此后B很久没有见到他,直至发生这幕惨剧结束他可悲的、病态的、堕落的一生为止。他的生命是以可怕的方式结束的。这幕惨剧不仅与我童年时代最初的印象紧紧连在一起,甚至和我的一生也有密切关系。事情是这样发生的······。但我首先必须说明,我的童年时代是怎么一回事,在我最初的印象中留下如此痛苦的痕迹、把我那可怜的妈妈逼死的这个人对于我又意味着什么。

\newpage
\section*{二}
\par 我自己记得起来的事情开始得很晚,大约在八岁以后。我不知道八岁以前的事情怎么没有给我留下一点清晰的印象,否则现在我也不至于回忆不起来。但从八岁半开始的一切,我都记得很清楚,一天又一天接连不断,仿佛这以后的事情顶多发生在昨天。诚然,这以前的事情我也能朦朦胧胧想起一些来:在幽暗的角落里,古老的圣像旁老是点着一灯如豆;此外,有一次我在街上给马撞倒了,据后来人家告诉我,我因此躺在床上病了三个月;还有,在这次卧病期间,我夜里在和我同睡的妈妈身旁醒来,对于我梦见的异象、深夜的寂寥和在屋角制造声响的耗子忽然怕得要命,我躲在被窝里吓得哆嗦了一宿,但不敢叫醒妈妈,——现在我根据这一点推断,我对她比什么都怕。但从我一下子开始意识到自己的时刻起,我的头脑发育得很快、很突然,许多完全不是孩子应该有的印象对我却可怕地易于接受。我觉得一切在自己面前豁然开朗,一切很快都变得明白易懂。我对自己开始记得很清楚的那个时期,在我头脑里留下了强烈而痛苦的印象。这印象后来天天再现,并且一天比一天深刻;它给我生活在父母身边的那段时期,从而也给我的整个童年时代抹上一层阴暗而奇怪的色彩。
\par 现在我觉得.我好象从酣睡中骤然醒来(不过,当时我的感受自然没有这样明确)。我来到一间很大的屋子里,天花板很低,空气很坏,又不干净。墙壁的颜色灰不溜丢,屋角是一只老大的俄式炉子;窗子临街,或者毋宁说是向着对面一幢房子的屋顶,短而且宽,犹如几道裂缝。窗台离地板相当高,我记得,我必须搁一把椅子、一张板凳才勉强够得到窗口,家里没人的时候我喜欢坐在那儿。从我们的住处看得见半个城;我们就住在一幢极大的六层楼房的屋顶下面。我们的全部家具陈设只有一张满是败絮和灰尘的漆皮破沙发、一张白木桌子、两把椅子、妈妈的床褥、角落里一口放东西的小橱、一只老是歪向一边的抽屉柜和几扇纸糊的破屏风。
\par 记得是在黄昏时分,一切都乱七八糟,东西扔满一地:刷子、抹布、我们的木质器皿、一只碎瓶子,还有不知道什么别的物件。我记得妈妈恼火得厉害,不知为什么在哭。继父坐在屋角,照例穿着那件很破的常礼服。他用嘲笑的口气回答妈妈,使她更加恼怒,于是刷子和器皿又纷纷摔到地上。我哭叫起来,向他俩身边扑过去。我害怕极了,紧紧搂住爸爸,用我的身体保护他。天知道为什么,反正我觉得妈妈在对他发无名之火,他没有过错;我想代替他请求原谅,代替他承受无论什么样的惩罚。我对妈妈怕得要死,并以为大家都这样怕她。妈妈起先愣了一下,接着抓住我的一条胳膊,把我拉到屏风后面去。我的胳膊在床上撞得生疼,但是恐惧比疼痛更厉害,所以我连眉头也没皱一下。还记得,妈妈开始指着我用痛苦和激动的语调对父亲不知说些什么(我在下面的叙述中将称他父亲.因为我在很久以后才知道他不是我的生父)。这场戏历时有两个钟点,我战战兢兢地等待着,竭力揣测这一切将以什么告终。后来,争吵总算平静下来,妈妈不知到什么地方去了。这时爸爸把我叫去,吻我,在我头上抚摩,把我抱到大腿上,我紧紧地、甜蜜地偎在他胸前。这也许是我第一次领略到亲人的疼爱;可能真因为如此,从那个时期开始,我才对一切都记得那么清楚。我也看得出,我是因为卫护父亲而赢得他的亲热,恐怕也是在那个时候,我第一次惊异地想到,是妈妈害他吃了很多苦。从此这个念头就永远留在我头脑里,而且一天比一天更使我感到气愤。
\par 从这一刻起,我开始无限地爱我的父亲,但这是一种奇特的爱,完全不象小孩的感情。我可以说,这更象是母亲的怜惜孩子的感情,如果这样来形容我的爱在一个小孩身上不是十分可笑的话。我觉得父亲总是那样可怜,那样受欺凌、遭践踏、吃苦头,以致在我看来,如果不发疯似地爱他,不安慰他,不跟他亲热,不千方百计为他着想,那简直是可怕的、不近人情的事情。但我至今仍不明白,为什么偏偏是我会产生父亲在世上受苦倒霉这样的想法!这是谁向我灌输的?我小小年纪,对于他个人的蹇滞怎么可能有些许的理解?然而我却能理解,尽管在想象中把一切都按我自己的方式重新加以解释、改制;但直到现在我还是想不出,我头脑里怎么会形成这样的印象。也许妈妈对我太严厉了,所以我对父亲抱有好感,以为他和我一样受苦,同病相怜。
\par 我已经谈了最初从婴儿的梦境中惊醒的情形,谈了我一生中最初的举动。我的心从最初的一刹那起就受到伤害,我的头脑也就以不可思议和十分消耗精力的速度开始发展。我已经不能满足于浮面的印象。我开始思考、推论、观察;但这种观察发生得太早、太反常了,故所我的想象不能不把一切都按自己的方式加以改制,于是我一下子进入了一个特殊的世界。我周围的一切变得象父亲经常对我讲的神奇的童话故事,而在那个时候我不可能不把它当作百分之百的真实。奇怪的概念产生了。我很清楚地了解,——但我不知道怎样会了解的,——我生活在一个奇怪的家庭,我父母跟当时我见到过的那些人完全不一样。“为什么,”我心想,“为什么在我看来别人连外表也跟我的父母不一样?为什么我发现别人脸上有笑容,而在我们这个角落里从来不笑,从来不快活,为什么这一点立刻使我吃惊?”是什么力量、什么原因促使我这个才九岁的小孩如此用心地看和用心地听?傍晚,我用妈妈的掩襟旧棉袄往自己的破衣衫外面一裹,拿着铜子儿到小店里买几戈比的食糖、茶叶和面包,在我们的楼梯上或街上会遇到一些人,我总是仔细听他们说的每一句话。我懂得了,但记不得是怎样懂得的,反正在我们那个角落里永远是无法忍受的悲哀。我绞尽脑汁,竭力想猜透为什么会这样的原因,也不知道是什么人帮我对这一切按自己的方式作出了解答;总之,我责怪妈妈,认为她是我父亲的冤家。这里我又要说明一下:我不明白,如此荒唐的概念在我的想象中是怎样形成的。我在多大的程度上爱父亲,也就在多大的程度上恨我那可怜的母亲。有关这一切的回忆,直到现在还深深地、痛苦地折磨着我。但另外有件事情比前面那一件更加促进我奇怪地靠拢父亲。有一次,晚上九点多钟,妈妈差我到小店里去买酵母,爸爸不在家。我回家的时候摔倒在街上,把一碗酵母全洒了。我首先想到的是妈妈会大发脾气。同时,我的左胳膊疼得非常厉害,使我站也站不起来。行人纷纷止步站在我周围;一位老婆婆开始把我扶起,可是在旁边跑过的一个男孩却用钥匙敲我的脑袋。后来,人家把我扶了起来,我捡起碗的碎片,勉强拖着两条腿,摇摇晃晃回家去。忽然,我看见了爸爸。他站在我们对面那幢豪华的楼房前的人群里。这幢房屋为一些身价颇高的人所有,装饰得富丽堂皇,台阶旁停着许多马车,乐声从窗户里边飘到街上。我抓住爸爸的衣襟,给他看打破的碗,开始哭哭啼啼地说,我不敢去见妈妈。我好象确信他会卫护我的。但是,为什么我会确信,是谁向我暗示,是谁教我认定他比妈妈更喜欢我呢?为什么我走到他跟前时并不害怕?他拉住我的手开始安慰我,然后说他要让我看什么事情,并把我抱起来。我什么也看不见,因为他抓住了我摔伤的一支胳膊,使我疼得要命;但我没有叫喊,怕扫他的兴。他一再问我看见了什么。我竭力想迎合他,就回答说我看到红色的帷幕。当他要把我抱到更靠近那幢房子的街道另一边去时,我不知为什么突然哭了起来,搂住他,要求赶快回到楼上妈妈那里去。我记得,当时爸爸的怜爱开始叫我感到有些沉重,因为我那么喜欢的两个人中的一个疼我、爱我,而对另一个人我甚至不敢走近她,这种状况我受不了。但妈妈几乎完全没有生气,就打发我去睡觉。我记得,胳膊疼得愈来愈凶,使我发了寒热。不过我为事情这样顺利结束特别感到高兴。这一夜我梦见的始终是对面那幢挂着红色帷幕的房子。
\par 第二天醒来,我首先想到的、首先关心的是挂着红色帷幕的房子。妈妈刚走出院子,我爬到窗台上,开始望着那幢房屋。它早已激起了我作为一个孩子的好奇心。我特别喜欢在华灯初上的傍晚看它,那时这幢房子灯烛辉煌,整块大玻璃后面的紫红色帘幔开始射出一种特别的、血一般的闪光。台阶旁几乎不断有豪华的马车来到,拉车的都是雄赳赳的骏马,大门口的吆喝、忙乱、马车的彩灯、坐车前来的盛装妇女——一切都吸引着我的好奇心。这一切在我幼小的心灵中本来就具有某种帝王气派和神话色彩。如今,我在那里见到父亲以后,这幢富丽的房子在我心目中变得加倍神奇美妙。如今,在我受到震惊的想象中开始产生一些古怪的念头和猜测。在父亲和母亲这样的怪人中间,我自己会变成这样稀奇古怪的一个孩子,我倒觉得是很自然的。他俩性格的对比特别使我吃惊。比方说,我诧异于妈妈老是为我们的穷家业操心忙碌,老是责备父亲,说只有她一个人为一家子劳累;我不禁向自己提出一个问题:究竟为什么爸爸一点不帮帮她,究竟为什么他象个外人住在我们家?妈妈的某些话使我对此有了点儿概念,我惊讶地了解到:爸爸是位艺术家(这个词儿一直留在我记忆中),爸爸是个有才华的人。于是在我的想象中立即形成一个概念:艺术家是跟别人不一样的特殊的人。也许是父亲的举止本身使我产生这个想法;也许我听到了一些现在已经忘怀的什么话;反正有一次父亲怀着特殊的感情当我的面说了一些话的时候,我觉得他的话出奇地可以理解。他说,总有一天他将不再过穷日子,他自己将成为富豪;还说,只有等妈妈死了以后,他才能复活。我记得,听了这些话我最初害怕极了。我不能再待在房间里,一个人跑到寒冷的过道里去,用胳膊肘抵着窗子,双手掩面放声大哭。但后来,我对这件事经常反复地思考过后,我对父亲这个骇人听闻的愿望习惯下来过后,想象力忽然帮了我的忙。再说,我自己也不能长期为不明真相而苦恼,我非得作出某种假设不可。于是,——我不知道这一切是怎样开的头,——但最后我认为,等妈妈死了以后,爸爸将离开这个凄凉的住所,带着我到别的地方去。但是去哪儿?——我直到最后还是没能想清楚,反正我认定我们将一道走。只记得,凡是我能用来点缀我将跟他前往的那个地方的一切,凡是我的想象力所能创造的一切辉煌、华丽、瑰奇的东西,在这些幻想中全部都用上了。我觉得,那时我们将立刻成为富人;我不再被差遣到小店里去买东西,这是我非常不愿意干的,因为我走出家门,邻近一幢房屋的孩子老是欺负我,而这是我极其害怕的,特别当我拿着牛奶或黄油的时候,知道万一洒了要受到严厉的惩罚。后来我在幻想中确定,爸爸将马上给自己做些好衣服,我们将住进一幢漂亮的房子,而此刻,这幢挂红色帷幕的豪华楼房,和爸爸在房子门前的相遇,他要让我看里边的什么事情——这一切都为我的想象提供帮助。在我的头脑里立即形成一种设想:我们正是要住进这幢房子,并将在里边永远生活得同过节一样,永远幸福顺遂。从此,每天晚上我怀着紧张的好奇心,隔窗展望这幢对我来说具有魔力的房屋,加上车水马龙的盛况和我从未见过的服饰华丽的宾客,我仿佛听到从窗户里边飘出美妙的乐声;我凝望着窗帘上时隐时现的人影,努力想猜透那里在做什么,——我总觉得那里是天堂和一年到头的节日。我憎恨我们贫寒的住所,憎恨我自己所穿的破衣衫。有一次,我照例爬到窗台上,妈妈向我叱喝,命我从窗台上下来,我当即想到:她就是不让我看那幢房子,不让我想它,她妒忌我们的幸福,这一回她要从中作梗······。整个晚上,我一直用怀疑的目光留神注意着妈妈。
\par 对于妈妈这样一个永远受苦受难的人,在我身上怎么会产生如此狠心的态度呢?如今我才了解她悲惨的一生,回想起这个苦命人我痛心难禁。即使在当时,在我那阴暗而古怪的童年时代,在我生命初期发育如此反常的时代,我的心也常常被痛苦和怜悯攥紧,接着,忧虑、惶惑和怀疑便落入我的心田。当时良心即已在我身上起来反抗,我常常怀着痛苦和内疚的心情感觉到自己对妈妈是不公平的。但我们母女俩好象彼此挺疏远的,我连一次也不记得自己曾和她表示亲昵。现在,哪怕是最琐屑的回忆也往往刺痛和震撼我的心灵。记得有一天(现在我要讲的事情当然是琐碎、庸俗、不足道的,但正是这样的回忆使我特别难过,留在我脑海中的印象也最为痛苦),——有一天傍晚,父亲不在家,妈妈要差我到小店里去给她买茶叶和食糖。但她反复思量,老是拿不定主意,一边出声地数铜币——她只能调度这点可怜巴巴的钱。我想她数了有半个钟点,可还是数不好。有时候,妈妈甚至会陷入一种无意识的状态,想必是悲哀的缘故。现在我回忆起来,她一边数,一边老是念念有词,声音不大,疾徐有节,仿佛是无意间漏出话来;她的嘴唇和面颊没有血色,手哆嗦个不停,在她一个人自言自语的时候,又老是摇头。
\par “不,不要了,”她看看我以后说,“我还是躺下睡觉吧。啊?涅朵琦卡,你想睡吗?”
\par 我不做声;于是她把我的头抬起一点,用那么安详、那么亲切的目光望着我,她的脸绽开了那样充满母爱的笑容,使我的心顿时隐隐作痛,怦怦直跳。加上她叫我涅朵琦卡,这意味着此刻她特别喜欢我。这个叫法是她自己发明的,把我的教名安娜深情地改成涅朵琦卡这样一个小名;当她这样叫我的时候,就意味着她要和我亲热一番。我深受感动,我想搂住她,依偎在她身边,和她一起哭一场。她面色苍白,接着在我头上抚摩了很久,——也许已经是做着机械的动作,忘了在和我亲热,一边不断地说着:“我的孩子,小安娜,涅朵琦卡!”我的眼泪夺眶欲出,但我竭力忍住,坚持不哭。我似乎顽固地不愿在她面前宣泄自己的感情,虽然自己十分难过。这不可能是我的心肠在自然而然地趋于冷酷。单凭对我严厉这一点,她不可能使我跟她对立得这么厉害。不!是我对父亲的那种不可思议的、异乎寻常的爱在那里作祟。有时夜里我在屋角短短的床垫上冰冷的被窝里醒来,老是感到一种莫名的恐惧。我迷迷糊糊地回想起,不久前我更小一些的时候,我和妈妈睡在一起,不象现在那样害怕夜里醒来;只要挨到她身边,把眼睛眯起来,紧紧搂住她——马上又睡着了。我毕竟感觉到不能不暗自偷偷地爱她。后来我注意到,许多孩子往往畸形地缺乏感情;如果他们爱谁的话,就爱得异乎寻常。我的情况亦然如此。
\par 我们家里偶尔也会接连几个星期保持死一般的沉寂。父亲和母亲吵腻了,我照旧生活在他们之间,老是沉默、思量、忧伤,老是在我的幻想中追求着什么。我瞧着他俩,完全理解他们的相互关系。我理解他们这种深刻、永久的敌意,理解笼罩在我们乱糟糟的家中的愁云惨雾,——当然,前因后果我未加探究,我能理解多少就理解多少。在冬季漫长的晚上,我往往缩在某个角落里,连续几个小时贪婪地观察他们,注视父亲的脸,力图猜透他在想些什么,是什么事情这样吸引着他的心思。后来我又对妈妈感到惊异、害怕。她会不知疲倦地在房间里来回走上几个钟头,甚至夜里失眠症折磨着她的时候也常常起来一边走,一边自言自语,仿佛屋子里只有她一个人,时而把一双手摊开,时而交叉在胸前,时而又在可怕的、无穷尽的郁悒中拚命地扭绞。有时眼泪在她脸上潸潸地流,也许她常常自己也不明白这眼泪是哪儿来的,因为她不时陷入无知觉的状态。她患有一种痼疾,但她完全不予理睬。
\par 我记得,我的孤独和我不敢打破的缄默使我愈来愈感到沉重难挨。我开始懂事到那时已有整整一年,我老是在思忖,在憧憬,在暗中用一下子萌生的迷离恍惚的空想折磨自己。我像幽居在密林中孤僻成性。后来,爸爸最先注意到了,把我叫到自己跟前,问我为什么这样凝视着他。我记不得向他回答了些什么,只记得他听了以后若有所思,最后瞧着我说,明天他要带识字课本回来开始教我认字。我焦急地期待着这识字课本,足足想了一夜,对于这识字课本是怎么回事还不甚了了。第二天终于来临,父亲真的开始教我。我根据三言两语就明白了对我提出的要求,所以学得很快,因为我知道这样能讨他喜欢。这在我当时的生活中是最幸福的一段时间。当他夸奖我悟性好,在我头上抚摩、吻我的时候,我立刻会高兴得流泪。父亲渐渐地喜欢我了;我已经有勇气和他交谈,我们常常会连续谈上几个小时而不觉得厌烦,虽则他对我说的话有时候我一句也不懂。但我不知怎的有些怕他,唯恐他以为我和他在一起感到无聊,所以竭力向他表示我全都明白。久而久之,晚上和我坐在一起在他已成为一种习惯。只要暮色渐浓,他回到家里,我马上拿着识字课本走到他跟前去。他让我坐在他对面的小凳上,教完了课,就开始读一本不知什么书。我什么也不懂,但是不停地哈哈大笑,估计这样他会得到极大的欣慰。的确,他对我很感兴趣,他瞧我笑得这么欢自己也乐了。就在那个时期,有一天教课完毕,他开始给我讲童话故事。这是我听到的第一个童话。我象着了迷似地坐着,听得心急如焚,随着情节的发展飞到遥远的地方,到故事终了大喜若狂。不仅是童话本身如此吸引着我,——不,我把一切都信以为真,同时纵任自己丰富的想象力驰骋天际,并随即把臆想和现实混杂在一起。我想象中马上会出现挂红色帘幔的房子,不知怎的就象剧中人登场似地,那里有自己给我讲这个故事的父亲,有阻挠我们俩不知到哪里去的妈妈,最后——或许毋宁说首先——是我,连带着我的胡思乱想和满脑瓜荒诞离奇的幻影,——所有这一切在我头脑里彻底搅混,不久便形成最最纷乱的一团糟,以致我有若干时间完全失去了分寸,完全失去了现实感,天知道生活在什么地方。这时我迫不及待地想跟父亲谈等待着我们的未来,谈他自己的期望,谈有朝一日我们离开这间顶楼以后他将带我一同前去的地方。从我这方面说,我确信这一切很快就能实现,但是如何实现,这一切又将是怎么样的——我不知道,并为此苦苦思索,伤透脑筋。有时——尤其是在晚上——我会觉得爸爸马上就要偷偷地向我丢个眼色,把我叫到过道里去,那时我准备瞒过妈妈,顺手拿起我的识字课本,还有从不知何时就没配框子贴在我家墙上的一张蹩脚的平版石印画(那是我决意一定要拿走的),然后我和爸爸悄悄地逃往某一个地方,再也不回家,再也不到妈妈这里来。有一次妈妈不在家,我选择父亲情绪特别好的机会,——那是在他稍许喝了点酒的时候,——走到他身边跟他闲聊,目的是想旋即把谈话转到我心爱的题目上去。我终于设法使他笑了起来,于是我紧紧搂着他,心吓得直发颤,象是准备谈一件神秘而可怕的事似的,开始没头没脑、语无伦次地问他,我们要去什么地方,还要等多久,我们该带些什么,我们将如何生话,还有,我们是不是到挂红色帷幕的房子里去?
\par “房子?红色的帷幕?怎么回事?傻丫头,你在说什么胡话?”
\par 于是我比先前更加害怕地开始向他解释,等妈妈死了,我们再也不住顶楼,他将把我带到一个地方去,我们俩将变得富有而幸福,临了还要他相信,这一切是他自己向我许诺的。我在说服他的时候,自己完全相信,我父亲先前确实说过这句话,至少我认为如此。
\par “妈妈?死了?妈妈什么时候死?”他愕然望着我连连问道,两道杂有些许霜华的浓眉顿时皱紧,脸色也有些变了。“你在说些什么,可怜的傻丫头······”
\par 接着他开始骂我,对我讲上很久,说我是个糊涂的孩子,什么也不懂······我不记得还说些什么,反正他非常扫兴。
\par 他的责备我一句也不明白,不明白他是多么痛苦,因为我把他在气头上饱含酸辛对妈妈说的话都听进去,记牢了,甚至在心中已经想得很多。不管他当时是怎么个人,不管他本人的行为荒唐到什么程度,但是这一切必然使他吃惊。尽管我完全不明白他为什么生气,不过我感到伤心万分,我哭了,在我看来,等待着我们的一切太重要了,叫我一个糊涂孩子既不敢说,也不敢想这件事。此外,虽然我没有一下子明白他的意思,但我隐隐约约感觉到自己对不起妈妈。恐惧随即向我袭来,疑惑乘虚潜入心坎。他看到我痛哭流涕、苦恼不堪的情状,便开始安慰我,用衣袖给我擦去眼泪,叫我别哭。我们俩默默地坐了一段时间;他颦蹙愁眉,似乎在考虑什么事情,后来又开始对我说话,但是,无论我怎样集中注意,他说的一切我都感到极其费解。根据这番话至今还留在我记忆中的只言片语推测,他当时向我解释,他是何许样人,他是多么伟大的艺术家,可是谁也不了解,他是一个才华出众的人。我还记得,他问我懂不懂,我当然作出肯定的回答,然后他要我应对,他有没有才华?我答道:“有才华”,他听了莞尔一笑,大概到末了他自己也觉得可笑:竟然跟我谈这样一个对他说来十分严肃的题目。卡尔·菲奥多雷奇的来到打断了我们的谈话,爸爸指着他对我说:
\par “可是卡尔·菲奥多雷奇连半点才华也没有。”
\par 于是我笑了,情绪完全好转。
\par 这位卡尔·菲奥多雷奇是个饶有兴味的人物。在我一生的那个时期,我见到的人少得可怜,所以怎么也忘不了他。我现在还记得很清楚:他是个日耳曼人,姓迈耶尔,出生在德国,到俄国来一心一意想进彼得堡的芭蕾舞团。但他的舞技实在不行,因此他甚至没有被录用当群舞演员,只在话剧团跑跑龙套。他在福丁布拉斯【 莎士比亚的悲剧《哈姆雷特》中的挪威王子。】的扈从中间担任没有台词的角色,或者扮演维罗那骑士之一,跟二十来人一起举着硬纸板做的短剑齐声高呼:“誓为国王而死!”不过,世上恐怕没有一个演员象这位卡尔·菲奥多雷奇那样一片丹心忠于自己所演的角色。他一生最大的不幸和悲哀是他没有进入芭蕾舞团。他把芭蕾艺术看得高于世上一切艺术,并在某种意义上对之一往情深,正象爸爸对小提琴一往情深那样。他和爸爸还在话剧团同事的时候就结识了,此后这位退休的龙套一直不离开他。他俩经常见面,两个人都悲叹自己命乖运舛,怀才不遇。卡尔·菲奥多雷奇可算是世上最重感情的人,对我继父怀有最热烈、无私的友情,但爸爸对他似乎没有什么特别的好感,无非当他一般的熟人而已,因为也没有别人可以相与。除此之外,爸爸凭着自己目空一切的性格,怎么也无法理解芭蕾也算一门艺术,从而每每把那个可怜的日耳曼人气得直哭。爸爸知道他有这根脆弱的心弦,老是去触动它,当不幸的卡尔·菲奥多雷奇气冲冲地提出反驳的时候,爸爸就笑他。后来我从B那里听到许多有关这个卡尔·菲奥多雷奇的事情,B把他叫做纽伦堡饭桶。关于他跟父亲的友谊,B讲了很多;反正他们不时碰头,在一起喝上几杯,便一起开始怨命,哀叹他们的才能被埋没。我记得这种会面的情景,还记得,我瞧着这一对畸人,有时也会抽抽搭搭地哭起来,自己却不知道为什么哭。这总是发生在妈妈不在家的时候,因为那个日耳曼人怕得她要命,他到我们家来照例先在过道里站一会,看看有没有人出来,万一了解到妈妈在家,立刻从楼梯上跑下去。他每次都带来一些德文诗歌,热情洋溢地念给我们俩听,然后一边朗诵,一边译成不通的俄语,想让我们领会诗意。这会逗得爸爸乐不可支,我也往往把眼泪都笑出来。但是有一回他们俩弄到一部俄国人写的作品,它点燃了这一对朋友身上如火如荼的激情,以后他们聚在一起,几乎老是读这部作品。我记得那是一位负有盛名的俄国作家所写的诗剧【指库柯尔尼克1834年所作的描写一个画家的戏剧幻想曲《贾科博·桑纳扎雷》。库柯尔尼克的“盛名”来自他于同年发表的忠君剧本《上帝的手拯救了祖国》。】。这本书开头的几行我背得很熟,在事隔数年之后,我偶然看到这本书,并不费力气就把它认出来了。这部诗剧写的是一个伟大的画家的不幸遭遇,名字叫杰纳罗还不知是贾科博,他在某一页上大叫:“我得不到公认!”在另一页上喊道:“我得到了公认!”一会儿说:“我没有才华!”可是才相隔几行又说:“我有才华!”结尾是非常凄惨的。这个剧当然是庸俗透顶的作品;奇怪的是它对这两个读者却产生十分幼稚和可悲的影响,他们从主人公身上发现和自己有许多相似之处。记得卡尔·菲奥多雷奇有时会激动得从座位上跳起来,跑到房间的另一角去,情恳辞切、声泪俱下地请求爸爸和我(他总是用法文的“小姐”称呼我)就他和命运、世人究竟孰是孰非作出即决裁判。他还当场跳起舞来,一边表演各种各样的舞姿,一边喊着要我们立刻告诉他,他是不是一个艺术家,能不能说他不是,也就是说他没有才华?爸爸也会一下子高兴起来,悄悄地向我眼睛,大概是预先通知我,他这就要拿日耳曼人大大地开开心。我觉得太可乐了,但爸爸向我扬一扬手,我只好勉强忍住不笑,差点儿没把自己憋死。即使到了现在,只要一想起来,我还是不能不笑。这个可怜的卡尔·菲奥多雷奇此刻好象在我眼前。他个儿长得奇小,又非常瘦,头发已经花白,一个红通通的鹰钩鼻沾着烟末子,两条罗圈腿怪难看的,尽管如此,他对自己这两条腿的构造似乎颇为得意,还穿着紧身裤。他做完最后一个跳跃动作,摆好姿势向我们伸出双手含笑致意,就象舞蹈演员在舞台上做完一套动作面带笑容亮相那样,这时爸爸有几秒钟默不作声,仿佛拿不定主意发表评论,故意让得不到公认的舞蹈家保持原来的姿势,可怜他仅靠一条腿竭力维持平衡,弄得左右摇晃。最后,爸爸现出一本正经的表情望着我,好象邀请我充当不偏不倚的证人听他的评论;与此同时,舞蹈家也向我投来胆怯、哀求的目光。
\par “不,卡尔·菲奥多雷奇,你怎么也不行!”爸爸终于说,一边还装得他自己也不愿道破这痛苦的真理。于是从卡尔·菲奥多雷奇胸中迸出一声真正的喟叹;但他倏忽之间又打起精神来,做着加快的手势重新要求予以重视,说他跳的舞别有章法,恳请我们再评判一次。然后他跑到另一个角落又跳起来,有时跳得卖力极了,脑袋甚至碰到天花板撞得生疼,但他拿出斯巴达人的精神英勇地熬住疼痛,重新停下来亮相,重新含笑向我们伸出颤巍巍的双手,重新请求决定他的命运。但爸爸不为所动,依旧绷着脸回答:
\par “不,卡尔·菲奥多雷奇,看来你命该如此:你怎么也不行!”
\par  这时我再也忍耐不住,开始笑得前俯后仰,爸爸也跟着我笑起来。卡尔·菲奥多雷奇这才发现人家在拿他开心,气得面红耳赤,眼睛里噙着泪水对爸爸说:
\par “你是个背心(信)起(弃)义的碰(朋)友!”
\par 他怀着深深受到伤害的感情说这句话,当时虽然可笑,但后来却使我为这个可怜的人觉得难受。
\par 说完,他拿起帽子从我们家跑出去,指着皇天后土发誓永不再来。但这样的决裂并不持久;过几天他又到我们家里来,又开始读那个享有盛名的诗剧,又是涕泗滂沱,然后天真的卡尔·菲奥多雷奇又请我们裁判他和世人、命运的是非,不过这回一定要本着真正的朋友情谊认真裁判,不再拿他开心。
\par 有一次,妈妈差我到小店里去买什么东西,我小心翼翼地拿着找给我的一个银币回家。登上梯阶时,我遇见父亲正要走出家门。我冲他笑了起来,因为我在他面前总抑制不住自己的感情,他俯身吻了我一下.发现我手里拿着一个银币······。我忘了交代一点:我对他的面部表情太熟悉了,他心里想要什么,我几乎都能一目了然。如果他闷闷不乐.我也为之心碎。当他身无分文,因而已经成了瘾的酒连一滴也喝不成的时候,他最容易犯愁,也愁得最厉害。但我在梯阶上遇见他的那个时刻,我觉得他有点异样。他的眼珠子变浑浊了,目光游荡飘忽;起初他没有注意到我;可是当他看见我手里有一个银币的时候,脸突然涨红,随后又泛白,本想伸手把钱拿走,可马上又缩了回去。显然,他内心在进行斗争。最后,他大概战胜了自己,命我上楼去,自己往下走了几级,但一下子又站住,急忙把我叫回去。
\par 他显得非常不好意思的样子。
\par “听着,涅朵琦卡。”他说,“把这钱给我,回头我还给你。好吗?你一定肯给爸爸的,对不对?你不是心肠挺好的吗,涅朵琦卡?”
\par 我好象预感到了这一着。但在最初的一刹那,我想到妈妈会非常生气,不禁有些胆怯,而更重要的是我本能地为自己和父亲感到羞耻,所以没有把钱交给他。他在瞬息之问看出了这一点,赶紧说:
\par “那就不要了,不要了!······”
\par “不,不,爸爸,你拿去;我就说钱丢了,说是邻居的孩子把钱抢走了。”
\par “好,好;我知道你是个聪明的小姑娘,”他说着,哆嗦的嘴唇现出微笑;一旦他感觉到手里有钱,并不掩饰自己的欣喜。“你的心地真好,你是我的小天使!让我吻一下你的小手。”
\par 他抓住我的一只手想亲吻,但我很快把手抽了回来。我被一种怜悯的感觉控制住了,羞愧愈来愈叫我难受。我撇下父亲.慌慌张张跑上楼去,甚至没有和他告别。我走进房间的时候,一种过去我不知道的苦恼使我两腮灼热,心突突地跳。不过我还是放大胆子对妈妈说,我把钱掉在雪地里了,怎么也找不到。我估计至少要挨一顿打,但这倒没有发生。妈妈起先确乎气得要命,因为我们实在太穷。她冲我大叫大嚷,但紧接着就好象改变了主意,不再骂我,只说我一点也不麻利,又不尽心,说我这样不爱惜她的钱,可见我不那么爱她。这句话最使我痛心,即使我挨了一顿打,也不至于如此。但妈妈对我已经有所了解。她已经注意到我过于敏感,动不动便会进入亢奋状态,所以沉痛地指责我不爱她,想用这样的办法给我较大的震动,促使我今后多留点儿神。
\par 傍晚,到了爸爸应当回来的时候,我跟往常一样在过道里等他。这一次我处在极大的不安之中。我的情绪被痛苦地折磨着我的良心的感觉搅乱了。最后,我看见父亲回来,高兴极了,以为我的心情会因此而轻松一些。他带着几分醉意,但一见到我,马上现出神秘、尴尬的表情,接着把我拉到角落里,胆怯地望着我们的房门,从兜里掏出他买的一个糖酱饼,开始悄声告诫我往后切莫自己拿钱和瞒着妈妈把钱藏起来,这是恶劣的、可耻的、极坏的;还说刚才这样做是因为爸爸非常需要钱用,但他会归还的,以后我可以说钱找到了,不过私自拿妈妈的钱是可耻的,叫我今后决不可存这个念头,要是我听他的话,他还要给我买糖酱饼;末了,他甚至还补充几句,要我心疼妈妈,因为妈妈身体很不好,怪可怜的,她一个人为我们大家干活。我听着心里非常害怕,浑身发抖,眼泪忍不住要掉下来。我震惊得话也说不出,动也动不了。最后,他走进房间里去,叫我别哭,这件事不要对妈妈说。我注意到他自己也窘得厉害。整个晚上我心情惶恐,破题儿第一遭不敢朝父亲看,不敢走到他身边去。他大概也在避免和我目光交接。妈妈在房间里走来走去,一边按她的老习惯神志不清似地自言自语。这一天她感到特别不舒服,大概病又发作了。内心的痛苦终于使我发起寒热来了。到了夜里,我睡不安稳。病态的幻象老缠着我。后来我实在受不了,伤心地哭了起来。我的哭声把妈妈吵醒;她叫了我一声,问我怎么啦。我不回答,可是哭得更加伤心。于是她点了蜡烛,走到我床边,开始安慰我,以为我做梦吓着了。“唉,你呀,真是个蠢丫头!”她说。“直到现在梦见了什么还哭鼻子。好啦,别哭了!······”她吻了我一下,叫我去跟她睡在一起。但我不要,我不敢抱住她,不敢到她那里去。我在难以想象的苦恼中折腾。我想把什么都告诉她。我已经打算开口,但想到爸爸和他的禁令,欲言又止。“你也真可怜,涅朵琦卡!”妈妈说着安置我睡下,用她的旧披风把我裹起来,因为她发现我浑身直打寒颤,“八成你将来也象我一样多病!”这时她充满忧伤地望着我,我架不住她的目光,只得把眼睛眯起来,转过头去。我不记得自己是怎样睡着的,但朦胧中听到可怜的妈妈还说了很久催我入眠的话。我还从来没有忍受过比这更难熬的折磨。我的心简直给挤得感到疼痛。第二天早晨,我觉得好些了。我开始跟爸爸说话,故意不提昨天的事,因为我料想这样他一定会非常愉快。他当即高兴起来,而在这以前,他自己也老是皱眉蹙额望着我。现在看到我挺快活的样子,他满心欢喜,几乎象小孩子一样感到满足。不久,妈妈从家里出去,他便再也不克制自己。他开始热烈地吻我,使我沉浸在一种歇斯底里的狂喜之中,又哭又笑。后来他说要给我看一件了不起的好东西,说我见了这件东西一定非常喜欢,算是给我的奖励,因为我是个聪明而又善良的小姑娘。于是他解开背心的扣子,取下用黑色的细绳套在他脖子上的一把钥匙。接着,他神秘地对我瞧瞧,似乎想从我的眼睛里看出他认为我一定会感觉到的全部喜悦;他打开箱子,备加小心地从里边取出一只形状很奇怪的黑匣子,以前我从来没有看见过他有这件东西。他战战兢兢拿起这只匣子,顿时象换了一个人;他的笑容不见了,脸上骤然出现一种庄严的表情。随后,他用钥匙打开那只神秘的匣子,从中取出一件我从未见过的东西,它的形状非常奇特。他诚惶诚恐地把它拿到手里,说这是他的小提琴,他演奏的乐器。于是他开始郑重其事地低声对我说好多好多话,但我不懂他说些什么,只记得有我已经知道的一些词语,——说他是个艺术家,他有才华,将来他要演奏小提琴,总有一天我们都将成为富人,得到很大很大的幸福。热泪挤满了他的眼眶,顺着腮帮子直淌。我感动极了。最后他吻了吻提琴,让我也吻它一下。他见我很想仔细看看那把琴,便把我带到妈妈床前,把琴放在我手里,但我看到他紧张得全身发抖,生怕我把它摔坏了。我接过提琴,碰了一下上面的弦,琴弦发出轻微的声响。
\par “这是音乐!”我朝爸爸望了望说。
\par “对,对,是音乐!”他高兴地搓搓手加以肯定。“你是个聪明的孩子,你是个善良的孩子!”但我看得出,他在夸奖和欣喜的同时,并没有忘记为他的琴担心,弄得我也紧张起来,赶忙把琴还给他。提琴仍旧和刚才一样谨慎唯恐不周地放进匣子,匣子锁上后再放回到箱子里,爸爸重又抚摩着我的脑袋,答应以后每次都给我看小提琴,只要我同现在一样懂事、善良、听话。就这样,小提琴驱散了我们共同的忧伤。不过到了晚上,爸爸在离家外出的时候,悄悄地叮嘱我,要我记住昨天他对我说的那些话。
\par 我就这样在我们的存身之所渐渐成长,我的爱——不,应该说我的痴情,因为我不知道还有什么确切的字眼可以表达我对父亲的这种遏制不住的、我自己也颇以为苦的感情,——逐步发展到了丝毫经不起刺激的过敏状态。我只有一种乐趣——想他;只有一个心愿——做一切会带给他哪怕是一点点欢欣的事情。我不知有多少次在楼梯上等候他回来,往往冻得瑟瑟抖,脸色发青,目的只是为了早一眨眼的工夫知道他已到家,只是为了尽快看他一眼。逢到他对我稍加怜爱的时候.我就大喜欲狂。与此同时,我又常常为自己如此固执地冷淡可怜的妈妈而感到痛心;有时我望着她,哀伤和怜悯使我肝肠寸断。他们长期处于敌对状态,我不能视若无睹,必须在他们两人之间作出抉择,必须站在某人一边;结果我站到了这个半疯子的一边,原因仅仅在于:他在我心目中是那么可怜,那么卑微,而且最初曾那么不可思议地刺激我的幻想。不过,谁能作出判断呢?我眷恋他,也许正因为他怪得很,连外表也如此,而且不象妈妈那样老是绷着脸,他差不多是个疯子,他身上往往会表现出类乎杂耍丑角的姿态,做出一些孩子气的举动,说到底,也许正因为我不那么怕他,甚至不那么尊敬他,不象对妈妈那样。他更象是我的平辈。渐渐地.我感到甚至主动权在我这一边,我逐步使他听命于我,他已经少不了我。我为此暗暗感到自豪,心中洋洋得意,由于明白他少不了我,有时我甚至向他撒娇。的确,我这种异态的好感有点儿象罗曼司······。但这部罗曼司是注定不能持久的,不久我就失去了父母。他们的生活最后以一幕可怕的惨剧告终,它在我的回忆中创巨痛深。事情是这样发生的——
\newpage
\section*{三}
\par 当时有一条非同小可的新闻轰动了整个彼得堡。据传,大名鼎鼎的小提琴家C——茨途经此地。彼得堡的音乐界纷纷大起忙头。歌唱家、优伶、诗人、画家、音乐爱好者乃至那些从来不喜欢音乐而且一向既谦逊又自豪地声称一个音符也不懂的人,无不如饥似渴地竞相设法弄到入场券。有热情而且花得起二十五卢布买票的大有人在,然而举行音乐会的场子连这些人的十分之一也容纳不下;C——茨在欧洲素享盛誉,到老来赢得无数桂冠,他的才华象永不凋落的鲜花,据说近来他已难得公开演奏,还有人声言他这是最后一次周游全国,以后将告别乐坛——所有这些因素都产生一定的作用。总而言之,这个消息引起了强烈而深刻的反响。
\par 我已经说过,每一位初次来访或者多少有点名气的小提琴家到彼得堡,都会在我继父身上造成极不愉快的影响。他总是急于赶在头里去听访问演出,以便尽早了解人家的艺术造诣。他听到人们对来访者的赞扬,甚至往往感到痛苦,直要到他能够挑出这位新来的提琴手技艺上的瑕疵,用刻薄的口吻到处发表自己的评论,才得宽心。可怜这个疯疯颠颠的人认为全世界只有一个天才、一位艺术家,此人当然就是他自己。但是,乐坛奇才C——茨来访的传闻对他产生的震动很大。必须说明一点,最近十年来,彼得堡没有听到过一位才华出众的名家,更不用说能与C——茨匹敌的好手了;故所我父亲对欧洲第一流艺术家的演奏水平毫无概念。
\par 后来别人告诉我,C——茨要来访问演出的消息刚一传开,人们又在剧场的后台看见我父亲。据说,他异常激动地来到那里,急煎煎地打听C——茨和即将举行的音乐会的情况。大家已好久没见他到后台来,他的出现甚至引起了一点骚动。有人故意用挑逗的口气对他说;“叶果尔·彼得罗维奇,这回您老将听到的可不是什么芭蕾舞曲,而是管保您坐立不安的音乐!”据说,他听了这番揶揄,面色刷地变白,不过仍然带着歇斯底里的笑容答道:“那还得走着瞧;远来的和尚未必都会讲好经;C——茨一向在巴黎,法国人把他捧上了天,可是大家知道法国人的话究竟有几分可信!”如此等等。当时在场者发出哄堂大笑,可怜的父亲自尊心受到了伤害,但他沉住气又添上几句,说他并没有发表什么具体的意见,还是走着瞧,反正后天快到了,那时,种种神乎其神的说法究竟是否属实便可分晓。
\par 据B所述,这天傍晚他遇见了有名的票友X公爵——这是一位深通音律、酷爱艺术的人。他们在一起边走边谈小提琴家访问演出的事,忽然B在一条街的拐角上看见我父亲站在商店的橱窗前凝视那里用大号铅字印着C——茨举行独奏音乐会消息的海报。
\par “您看到那个人没有?”B指着我父亲向公爵说。
\par “他是谁?”公爵问。
\par “他的事您已经听说了。这就是我对您提到过不止一回的那个叶菲莫夫,以前您甚至帮过他的忙。”
\par “哦,有意思!”公爵说。“您谈过许许多多有关他的事情。据说这个人很有意思。我倒想听听他的演奏。”
\par “这不值得,”B答道,“而且怪难受的。我不知道您觉得如何,反正我听了总感到揪心的沉痛。他的一生是一部可怕而荒谬的悲剧.我深深地同情他,不管此人多么下流,在我心中对他的好感还没有泯灭。公爵,您说这个人大概挺有意思。不错,但他给人的印象太痛苦了。首先,他是个疯子;其次,这个疯子犯有三项罪行,因为,除了自己以外,他还毁了另外两个人:他的妻子和女儿。我了解他,如果他确实相信自己犯了罪,会立地倒毙。可怕就可怕在他已经有八年几乎相信了这一点,而他和自己的良心也斗争了八年,直要到在这一点上完全确信——不是几乎相信——为止。”
\par “您说他境况很不好?”公爵问。
\par “是的,但贫穷目前对他差不多是一种幸福,因为贫穷可以做他的口实。现在他可以向任何人声称,正是贫穷阻碍着他,如果他有钱的话,他就有时间,就无须乎操心,人们一定马上会发现他是位了不起的艺术家。他结婚时抱有一个奇怪的希望,以为他妻子的一千卢布能帮助他站稳脚跟。他这一举动象个空想家、诗人,其实他一生都如此行事。您可知道,整整八年他不住口地说的是什么?他硬说他的不幸都是妻子造成的,说妻子妨碍他。他两手一叉,什么也不干。可要是他没有这个妻子,他将是世上天字第一号的可怜虫。他已有好多年没有拿起琴来,——您知道为什么?因为他每次拿起琴弓的时候,自己内心不得不承认,他是个废物,是个零,不是什么艺术家。如今琴弓搁在一边,他至少还保存着一线渺茫的希望——也许并非如此。他沉湎于幻想之中,总以为将来会出现奇迹,他可以一下子变成世界上最最出名的人。他的座右铭是:aut Caesar,aut nihil【拉丁语:要末做恺撒大帝,要末被人瞧不起。】,仿佛恺撒是可以在转眼间突然变出来似的。他念念不忘的是一举成名。如果这样的心情成了一个艺术工作者的主要和唯一的动力,那他便不是一位艺术家,因为他已经丧失了主要的艺术本能,也就是丧失了对艺术的爱,不再因为艺术并非其他、并非名气、仅仅因为它是艺术而爱它。但是C——茨则相反:他一拿起琴弓,除了他的音乐之外,对他来说世上什么都不存在。继琴弓之后,他首先关心的事情是钱,大概第三位才是名气。不过他很少考虑名气······。您可知道,这个不幸的人此刻在想些什么?”B指着叶菲莫夫又说。“现在占据他头脑的是世界上最愚蠢、最无聊、最可怜而又最可笑的一个问题,那就是:究竟他比C——茨高明,还是C——茨比他高明,除此以外什么都不在他心上,因为他还以为自己是全世界首屈一指的音乐家。谁要是能使他信服他根本不是艺术家,我可以对您说,他会象遭到雷击一般立地倒毙,要知道,他把整个一生都奉献给了一个根深蒂固的固定观念,因为最初他确曾表现出真正的天赋,一旦要他放弃这样的固定观念,那太可怕了。”
\par “他听了C——茨的演奏以后不知会怎么样,这倒是耐人寻味的,”公爵道。
\par “是啊。”B若有所思地说。“不过,他震动过后立刻故态复萌;他的疯狂能盖过事实真相,他马上会想出某种理由来解嘲。”
\par “您认为是这样吗?”公爵道。
\par 这时,他们快要走到叶菲莫夫近旁。我父亲想悄悄地溜之大吉,但B叫住了他,同他交谈起来。B问他去不去听C——茨的音乐会。父亲淡漠地表示不一定,说他有比听外国演奏家的音乐会更重要的事情,不过,到时候要是有空的话,去听听倒也不妨。说到这里,他情虚地向B和公爵瞥了很快的一眼,并且怀着戒心莞尔一笑,然后举帽点头,推说无暇多谈,打旁边走了过去。
\par 但我在前一天便已知道父亲的心事。尽管他究竟为什么事情苦恼我不知道,但我看得出他心神不宁得可怕;甚至妈妈也注意到这情形。当时她正病得厉害,几乎迈不开腿脚。父亲一忽儿回家来,一忽儿又出去。早上有三四位客人来找他,都是他过去的同事,这使我大为诧异,因为自从爸爸完全离开剧院以后,大家都跟我们断绝了往来,除了卡尔·菲奥多雷奇,我几乎从未见过旁人到我们家来。末了,卡尔·菲奥多雷奇气急败坏地跑来,并带来一份海报。我留神听,仔细看,对于这一切感到坐立不安,仿佛造成这派惶惑气氛以及我从爸爸脸上看到的忐忑情状都是我一人之过。我很想弄清楚他们谈些什么,当时我第一次听到C——茨的名字。后来我明白了,要见到这位C——茨,至少得花十五卢布。我还记得,爸爸大概沉不住气了,便把手一甩,说他知道这些洋玩意儿和吹得神乎其神的奇才是什么货色,C——茨他也知道,无非都是些犹太佬,存心来诓俄国人的钱.因为俄国人会凭空相信任何无稽之谈,更何况是法国人大吹大擂的事情。我已经懂得什么叫做没有才华。客人们笑了起来,不久都走了,撇下爸爸如坐针毡。我明白他在为某一件事情生那个C——茨的气,为了讨他的好,驱散他的愁闷,我走到桌子旁边,拿起那份海报来拼读,把C——茨的名字念出声来。尔后,我笑了起来,看看坐在椅子上若有所思的爸爸,说,“这人想必跟卡尔·菲奥多雷奇一个样:他大概也是怎么也不行的。”爸爸打了个寒颤,象是吓了一跳,他从我手中把海报抢去,又是叫嚷,又是跺脚,拿起帽子走出房门,但旋又回来,把我叫到过道里去,吻了我一下,开始以一种紧张和隐藏着恐惧的神态对我说,我是个懂事而善良的孩子,想必不愿意扫他的兴,说他指望我帮他一次大忙,但究竟帮什么忙他没说。再者,我听着他这样说只觉得怪不好受;我看得出,他的言语和疼爱并非出于真心,这一切都使我震惊。我开始为他苦恼,为他忧虑。
\par 次日吃饭的时候——那已经是音乐会的前夕,爸爸沮丧到了极点。他变得面目全非,不住地瞧瞧我,又瞧瞧妈妈。后来,他竟跟妈妈谈起什么事情来了,这使我大为纳罕,因为他几乎从来不跟妈妈交谈。饭后他对我表示异样的亲热:他不时用各种借口把我叫到过道里去,一边四顾张望,似乎生怕被人撞见,一边总是抚摩着我的脑袋,吻我,对我说,我是个好心的孩子,我是个听话的孩子,说我一定爱爸爸,一定能做到他要我去做的事情。凡此种种,无不引起我难以忍受的惆怅。直到他第十次把我叫到楼梯那儿去,我才明白是怎么回事。他带着无可奈何的痛苦表情不安地东张西望,问我知道不知道,昨天早晨妈妈拿回来的二十五个卢布放在什么地方?我听他问起此事,吓得发了呆。但正在这个当儿,楼梯上有人发出声响,爸爸惊恐地撇下我跑了出去。他到晚上才回家,神态尴尬,忧心忡忡,在椅子上默默地坐下来,频频用不好意思的目光看我。我感到一阵莫名的恐惧,有意避开他的视线。后来,在床上躺了一整天的妈妈把我叫去,给了我几个铜币,差我到小店里去给她买点儿茶叶和白糖。我们家喝茶是非常难得的事,除非妈妈身体很不好,有寒热,否则她舍不得把钱花在这种按我们的境况来说是奢侈的项目上。我拿了钱走到过道里拔腿就跑,仿佛唯恐有人追上来。但我担心的事情还是发生了:爸爸在街上追上了我,并把我带回到楼梯边。
\par “涅朵琦卡!“他用发颤的声音开口道。“我的小宝贝!你听着,把这点钱给我,我明天就······”
\par “爸爸!爸爸!”我叫着跪下来求他。“爸爸!我不能!这不行!得给妈妈买茶叶······。不能把妈妈的钱拿走,怎么也不行!下次我去拿······”
\par “那末你是不肯喽?你不肯?”他恶狠狠地向着我悄声说。“这么说,你是不愿意爱我喽?那好吧!现在我不管你了。你跟妈妈待在一起,我要离开你们,我不带你走。听见没有,你这个坏心的小丫头!你听见没有?”
\par “爸爸!”我满怀恐惧嚷道。“把钱拿去,给!现在我该怎么办呢?”我扭绞着双手,揪住他的衣襟说。“妈妈会哭的,妈妈又会骂我的!”
\par 他大概没料到会遇上这样的阻力,可钱还是拿去了,最后,由于受不了我的怨言和哭声,他把我撂在楼梯上,自己跑了下去。我只得上楼,可是到了我们家的房门口,再也支撑不住;我不敢进去,也不能进去;我的心受到剧烈的震荡,象一潭被搅浑的水。我用双手掩面扑到窗前,就象第一次从父亲口中听到他但愿妈妈早死时那样。我神思恍惚,不敢动弹,哆嗦着谛听楼梯上的任何些微声响。后来我听到有人匆匆上楼来。这是他;我能分辨出他的脚步声。
\par “你在这儿?”他压低嗓门问。
\par 我急忙向他跑去。
\par “拿去!”他喝道,一边把钱往我手中塞。“拿去!把钱拿回去!如今我不是你父亲了,你听见没有?如今我不愿做你的父亲了,你听见没有?如今我不愿做你的父亲了!你爱妈妈胜过爱我,你就到妈妈那儿去吧!我不愿意认识你!”说完,他把我推开,又跑下楼去了。我哭着去追他。
\par “爸爸!好爸爸!我听话就是了!”我喊着。“我爱你胜过爱妈妈!把钱拿去,给!”
\par 但他已经听不见我的叫喊;我连他的影儿也看不到。整个晚上我失魂落魄,不住打着冷战。我记得妈妈似乎在向我说些什么,几次把我叫去;我好象处在昏迷状态,什么也听不见,什么也看不清。最后,一切以歇斯底里发作而告终:我又是哭,又是叫;妈妈吓得不知如何是好。她让我躺在她被窝里,我搂住她的脖子,哆嗦着每分钟都担心发生什么事情,也不记得是怎样入睡的。这样过了一宿。第二天上午,我很晚才醒来,妈妈已经不在家里。这个时候她总是出去干自己的事。爸爸那儿好象来了个外人,他们俩在高谈阔论。我勉强等到客人走后,屋里只剩下我们俩了,就哭着扑到父亲跟前,为昨天的事求他原谅我。
\par “你答应我仍跟以前一样做一个聪明的孩子吗?”他声色俱厉地问我。
\par “我答应,爸爸,一定!”我回答说。“我告诉你,妈妈的钱放在什么地方。她的钱就在这抽屉里的一只匣子里,昨天还放在那里。”
\par “昨天还在?在哪儿?”他嚷着全身猛地一震,从椅子上直立起来。“放在什么地方?”
\par “钱锁起来了,爸爸!”我说。“你等着:晚上妈妈会差我去兑破的,因为我看到零钱都花完了。”
\par “我需要十五个卢布,涅朵琦卡!你听见没有?只要十五卢布!今天你给我弄来;明天我就全还给你。现在我去给你买果汁糖,买榛子······还要给你买一个洋娃娃······明天也买······我每天都带好东西回来,只要你做一个聪明的小姑娘!”
\par “不要,爸爸,不要!我不要你送东西;我也不要吃好东西;你买了,我也要还给你的!”我哭着喊道,因为顷刻间我的心整个儿疼痛如绞。在这一刹那,我感觉到了他并不怜惜我,并不疼爱我,因为他看不到我是多么爱他,以为我给他效劳图的是果汁糖或洋娃娃。在这一刹那,我尽管只是个小孩,却把他彻底看透了,并且已经觉得,这种想法对我造成了不可平复的伤害,我已经不可能爱他,我失去了我原来的那个爸爸。我的许诺使他欢欣雀跃;他看到我为了他一切都能豁出去,一切都愿意干,至于这“一切”当时对于我究竟包含多少内容,只有上帝知道。我明白,这点钱对于可怜的妈妈意味着什么,我知道,丢了这点钱她会懊丧得病倒,所以我的心灵在发出追悔的惨叫。但他什么也看不出来;他把我当做三岁的婴孩,其实我全明白。他的欢欣超越了一切界限,他吻我,劝我别哭,向我许愿,说我们今天就离开妈妈远走高飞,——无疑是迎合我一贯的梦想,最后他从兜里掏出一份海报,开始说服我相信,今天他要去见的那个人是他的冤家,是他的死对头,但他的冤家对头决不会得逞。他跟我谈起自己的敌人来,他本人倒十足象个小孩子。他注意到我不象往常听他说话时那样面露笑容,只是默默地听着,他便拿起帽子走出房间,因为他急于到什么地方去;但临走的时候,他又吻了我一次,讪讪地笑着向我点点头,似乎对我不大放心,又象是竭力不让我改变主意。
\par 我已经说过,他的精神有些失常;但这在前一天就看得出来。他需要钱买音乐会的入场券,而这场音乐会对他来说将是决定一切的。他仿佛预感到这场音乐会将要决定他的整个命运,但他急昏了,所以昨天竟要把几个铜币从我手里夺走,好象这点钱够买票似的。他在吃饭的时候神态更加反常。他在位子上完全坐不定,什么也不吃,一刻不停地站起来,随即又坐下,象是改变了主意,忽而拿起帽子,似乎准备出去,忽而又一下子变得出奇地心不在焉,老是喃喃自语,接着突然看看我,对我眼睛,打打手势,迫不及待地希望尽快把钱弄到手,对于我直到现在还没有从妈妈那儿拿到钱有些生气。甚至妈妈也注意到了这些反常的表现,望着他直纳闷儿。我简直象个被判处死刑的囚犯。饭后,我躲到角落里,象害疟疾似地全身发抖,一分钟一分钟直数到通常妈妈差我去买东西的时候。我一辈子也没有度过更难捱的时刻;这段时间将永远留在我的回忆中。当时我说得上是百感交集!在某几分钟内,一个人的意识经历的感受比几年还多。我觉得自己的行为非常要不得;其实,正是他自己启发了我的善良本性——当初他第一次懦怯地把我推向恶的一边,自己吓了一跳,便向我说明我的行为非常要不得。难道他不明白,要蒙蔽一颗渴望自觉地体味印象的心灵是多么困难?何况这颗心灵对于善与恶已经感觉得很多,思索得也很多。我明明懂得,显然有极端不得已的难处迫使他又一次驱使我去做坏事,从而牺牲我的可怜而得不到保护的童年,即便再度动摇我那立足未稳的良心也在所不惜。此刻,我缩在角落里独自寻思;他为什么要许诺奖赏我已经自愿决定去做的事情?过去所不知道的种种新的感受、新的意向和新的问题成堆地在我头脑中浮起,我给这些新问题折腾得苦不堪言。后来,我忽然开始为妈妈着想;我想象着她丢失这最后一点辛苦挣来的钱会伤心到什么程度。这样等着,等着,妈妈终于放下勉力支撑着在干的活,把我叫去。我哆嗦着往她那儿走。她从柜子里取出钱来交给我,说:“去吧,涅朵琦卡;不过,看在上帝份上,别再象前些日子那样让人少绐了找头,也别稀里糊涂地丢了。”我用哀求的目光看看父亲,但他点点头,向我露出赞许的笑容,急煎煎地搓着手。钟敲六下,而音乐会定于七点开始。这一番等待也够他受的。
\par 我走到楼梯上停下来等他。激动和焦急使他置一切必要的谨慎于不顾,紧跟在我后面跑了出来。我把钱交给他,楼梯上暗得很,我看不清他的脸,但我感觉到他接过钱时浑身在发抖。我发了呆一般站在那里一动也不动;直到他差我上楼去把他的帽子拿来,我才如梦初醒。他自己甚至不愿进去。
\par “爸爸!难道······你不跟我一起去吗?”我用断断续续的声音问,心中还惦记着我最后的一线希望,希望他会保护我。
\par “不······你先一个人待着······知道吗?等一下,等一下!”他忽然想起了什么,急忙说。“等一下,我这就去买好东西给你;你先去把我的帽子拿来。”
\par 我的心象给一只冰冷的手蓦地揪住。我发出一声惊呼,把他推开,急忙跑上楼去。我走进屋子的时候脸无人色,此刻我如果说钱被人抢走了,妈妈会相信我的.但这时节我什么也说不出来。一阵绝望引起的歇斯底里使我扑倒在妈妈的被窝上,双手捂住面孔。一分钟以后,房门不好意思地发出呀的一声,爸爸走了进来。他是来取帽子的。
\par “钱呢?”妈妈骤然叫嚷起来,她一下子就猜到发生了不寻常的事情。“钱在哪儿?说呀!你说呀!”她把我从床上拖起来,让我站在屋子中央。
\par 我默不作声,低头看着地上,我几乎闹不清自己究竟是怎么一回事,也闹不清别人在对我做什么。
\par “钱呢?”妈妈扔下了我,突然转向正要拿起帽子的爸爸,又高声问道。“钱在哪儿?”她再问一遍。“啊!她把钱给了你?你这个目无神明的害人精!你这个恶魔!你要把她也毁掉!连她这样一个孩子你也不放过?!不成!你不能走!”
\par 转眼间,她跑到门口,把房门从里边锁上,钥匙藏在身边。
\par “你说!老老实实承认!”她开始向我问罪,由于愤激过度,几乎嗓子也哑了。“老老实实承认一切!说,你说呀!要不······我简直不知道会把你怎么处!”
\par 她抓住我的两只手扭绞着拷问我。她气得快发疯了。在这一刹那,我发誓保持沉默,一句话也不提到爸爸,但还抱着万一的希望最后一次举目看他······。他只要对我瞥上一眼,只要听他说一句我巴巴地盼着、在心中祈求他说的话,无论忍受怎样的痛楚,无论遭到怎样的拷问,我也是幸福的······。可是,我的天哪!他却用无情的威胁性手势禁止我开口,殊不知此时此刻来自任何人的其他威吓对我都不起作用。我觉得咽喉好象被堵塞了,气喘不过来,两腿一软,就倒在地板上人事不省······。我又象昨天那样发生了一次神经性的休克。
\par 我是在我家房门上忽然响起叩门声时惊醒的。妈妈用钥匙开了门,我看见一个穿号衣的人进来惊讶地向我们一一环顾,问哪位是叶菲莫夫乐师。继父说他就是。于是那个听差递交了一封便简,说他是此刻正在公爵宅第的B派来的。信封内有一张C——茨音乐会的请柬。
\par 一名身穿华丽号衣的听差奉主人——一位公爵——之命来找穷乐师叶菲莫夫,——这件事在瞬息间给妈妈留下了强烈的印象。在这个故事的一开始我谈到过她的性格,这个可怜的女人始终爱着父亲。现在,尽管经受了整整八年接连不断的忧患和困苦,她的心仍然未变:她仍然可以爱他!天知道,也许现在她突然看到丈夫时来运转了。即使是某种希望的一点点影子对她也会产生影响。可能,她也多少感染到她那宝贝丈夫毫不动摇的自信亦未可知!再说,这种自信也不可能对她这样一个脆弱的女人毫无影响,根据公爵的垂青,她在倏忽之间可以为丈夫设想出上千种前程。才一眨眼的工夫,她已经准备跟他重归于好,她可以原宥他把一家子的生活弄到这般光景,即使考虑到他刚刚犯下的又一罪恶——牺牲她唯一的孩子——这件事情,在重新燃起的热情冲动下,在新希望的鼓舞下,她也可以把这桩罪恶缩小为普通的失检行为,看成是穷极无聊的生活、走投无路的境况所造成的懦怯表现。痴情在她身上还没有泯灭,此时此刻她又愿意为她那堕落的丈夫提供宽恕和同情。
\par 父亲一时手忙脚乱起来;公爵和B的关注也使他震悚。他直接走到妈妈跟前,对她悄悄地说了些什么,妈妈便从屋子里走了出去。两分钟以后,她带了兑破的钱回来,爸爸立即给了信差一个银卢布,来人颇有礼貌地鞠一躬后走了。妈妈出去了不多一会,拿来一只熨斗,取出丈夫最体面的一个胸衬动手把它烫平。她亲手把一条白色的麻纱领带在丈夫脖子上系好,这条领带连同还是父亲刚进剧院任职时做的一件黑色燕尾服(虽然穿得很旧了)保存着备而不用已不知有几许时日。装束停当后,父亲拿起帽子,但临走时要了一杯水喝;他面色苍白,精神疲惫地坐到椅子上。水是我递给他的,也许,憎恶的感觉重又潜入妈妈的心房,使她最初的热情冲动冷却了下来。
\par 爸爸走了出去;屋子里剩下我们俩。我退到角落里,一声不吭,久久地望着妈妈。我从未见过她这样愤激:她的嘴唇发颤,苍白的面颊一下子烧得通红,每隔一会儿就全身哆嗦。尔后,她的悲苦开始通过怨诉、啜泣和哀叹发泄出来。
\par “这都怪我,都怪我这个苦命的人!”她自言自语道。“她将来怎么办呢?我死了以后.她怎么办呢?”她站住了继续说,这个念头象闪电一般把她击中在房间中央。“涅朵琦卡!我的孩子!我可怜的涅朵琦卡!苦命的孩子!”她把我抱起来,神经质地搂着我说。“我活着尚且不能把你抚养照看好,将来能把你托给谁呢?哦,你不懂我的意思!你懂吗?我刚才说的话你能记住吗,涅朵琦卡?往后你不会忘记吧?”
\par “不会,不会,好妈妈!”我把十指交叠在一起求她放心。
\par 她长久地、紧紧地把我搂在怀里,想起一旦要跟我分离就颤栗不已。我的心在破裂。
\par “妈妈!妈妈!”我呜咽着说。“你为什么······为什么不喜欢爸爸?”别的话都哽住了说不出来。
\par 但听得从她胸中迸出一声呻吟。接着,又一阵揪心的悲伤驱使着她在屋子里来回走动。
\par “可怜哪,我可怜的孩子!我没注意到她已经长大了!她知道,全都知道!我的天!她得到的是什么印象,看到的是什么榜样!”她在绝望之余,又拼命绞自己的双手。
\par 后来她走到我跟前,怀着疯狂的爱吻我,吻我的手,在上面洒下许多泪水,求我原谅······。象这样的痛苦我从来没有看到过······。最后,她似已心力交瘁,陷入昏昏沉沉的状态。如此过了足有一个钟点。然后她疲乏不堪地站起来,叫我去睡。我走到自己的角落里,钻进被窝,可是没法入睡。我为妈妈深感苦恼,我也为爸爸深感苦恼。我焦急地等着他回来。一想起爸爸,我就被一种恐怖攫住。过了半个小时,妈妈拿起烛台向我走过来,看看我睡着了没有。为了安她的心,我眯上眼睛,假装已经入睡。她对我察看了一番,轻手轻脚走到食橱前,打开橱门给自己倒了一杯酒。她把酒喝下去以后,自己上床睡觉,让蜡烛点着留在桌上,门也不上锁,逢到爸爸晚归时照例如此。
\par 我迷迷糊糊躺着,但是不能成眠。我刚合上眼睛,立即被一可怕的幻象惊醒过来,吓得直哆嗦。我的忧伤愈来愈加剧。我想叫喊,但喊声在我胸中给堵住了。直至夜已深了,我听得我家的房门被推开。我不记得过了多久,但到我忽然完全睁开眼睛的时候,我看见了爸爸。我似乎觉得他的脸色白得可怕。他坐在紧挨房门的一把椅子上,似乎在沉思。死一般的岑寂笼罩着屋子。行将泪尽的残烛用惨淡的微光照着我们的住所。
\par 我看了好大一会工夫,可爸爸还在老地方没有离开,他一动不动地坐着,姿势始终未变,耷拉着脑袋,胳膊直僵僵地支在膝盖上。我几次想叫他,可是没能出声。我的麻痹状态还没有结束。最后,他猛然醒来,把头一抬,离座起身。他在屋子中央站了有好几分钟,象是在下决心做一件事情!随后突然走到妈妈床边凝神听了一会,等到确信她睡着了以后,便走向放着他的小提琴的那只箱子。他开了箱子的锁,取出黑色的琴匣,把它放在桌上;接着他又四顾张望,他的眼神迷茫,飘忽不定,我还从来没有发现过他这般模样。
\par 他刚拿起琴来,旋又把它放下,转身锁上房门。过后,他发现食橱打开着,便悄悄地走到橱前,看到一只杯子和酒,就倒出来喝了。于是他第三次去拿提琴,但第三次把它放下,并走到妈妈床边。我吓得不敢动弹,只好静观其变。
\par 他倾听了很久很久,然后一下子掀开被子,开始用一只手触摸妈妈的面孔。我打了个寒战。他再次俯下身去,几乎把脑袋贴着她,但当他最后一次竖起上身的时候,他那惨白的脸上似乎掠过一丝微笑。他轻轻地、小心地给睡着的妈妈盖好被子,盖住她的头、脚······我感到一种前所未知的恐怖而开始发抖,妈妈的状态叫我害怕,她的酣睡叫我害怕,我不安地望着那毫无动静的轮廓,她的肢体隔着被子显示着棱角毕现的线条······。一个可怕的念头象一道电光在我头脑里闪过。
\par 一切都安排好以后,父亲又走到橱前,把剩下的酒统统喝光。他周身颤栗着向桌子那边走去。他脸上一丝血色都没有,旁人简直认不出他来。他重又拿起提琴。我看见过这把琴,知道它是什么东西,但这时我预料会发生吓人的、可怕的、怪异的情况······因而最初的几个琴音使我猛吃一惊。爸爸开始拉琴了。但琴声焦躁遽促;他不时停下来,状似在搜索记忆;终于,他痛苦地颓然放下琴弓,用异样的目光看看床上。那里有什么东西总叫他放不下心来。他又走到床边······。他的一举一动都落在我眼里,我怀着恐怖的心情屏息凝神注视着他。
\par 忽然,他匆匆忙忙地开始在手边寻找什么东西,于是,刚才那个可怕的念头又象闪电似地烫了我一下。我想起来了:妈妈为什么睡得这样熟?为什么爸爸用手触摸她的面孔她也不醒来?末了,我看到爸爸将所有我们的衣裳凡是能找到的一股脑儿拖去,包括妈妈的棉袄、他自己的旧外套、睡袍乃至我脱下的衣杉,把妈妈盖在一大堆衣服下面完全看不出来;她始终一动不动地躺着,身体的任何部分都没有半点反应。
\par 她睡得好熟啊!
\par 干完了以后,他好象松了口气。这下再也没有什么妨碍他了,但有件事情还是叫他放不下心来。他把蜡烛搬了个地方,脸朝门站着,这样床上的景象可以看也不去看一眼。最后,他拿起提琴,心一横,用弓猛击琴弦······。音乐开始了。
\par 然而这不是音乐······。我一切都记得清清楚楚,直到最后的一刹那,当时震动我注意力的一切我都记得。不,这不是我后来曾有机会听到的音乐!这不是小提琴的声音,而象是某人极其可怕的嗓门在我们幽暗的住所里第一次发出巨响。要末我得的印象是不正确的、病态的,要末我亲眼目睹的一切使我的感官受到震荡,容易产生可怕的、无比痛苦的反应,但我坚信我听到了呻吟,是活人的哀叫、号哭;通过这些声音宣泄出来的是不折不扣的绝望。临了,凡是恸哭中包含的全部惨痛、苦楚中包含的全部凄怆、灰心中包含的全部哀伤似乎一下子聚合在一起,当凝集着这一切的凄厉的结尾和弦终于拉响的时候······我再也支持不住了,——我开始颤抖,泪水从我的眼眶里迸涌,随着一声绝望的惨叫,我扑到爸爸身边,双手抱住他。他惊呼之余,放下了提琴。
\par 他茫然若失站着有一分钟光景。后来,他的眼珠子开始转动,目光向左右两侧游荡,他象是在寻找什么,突然抓住那把提琴,在我头上高高举起······再过一会儿,他也许会把我当场打死。
\par “爸爸!”我冲他高声叫唤。“爸爸!”
\par 听到我的声音,他象一张树叶瑟瑟发抖,并且倒退了两步。
\par “啊!原来还有你呢!我以为全都完了!原来还有你也和我一起留下!”他狂叫着夹住我的臂膀把我举到空中。
\par “爸爸!"我又发出呼唤。“看在上帝份上,别吓我!我害怕!哇——畦!”
\par 我的哭声使他一愣。他把我轻轻放到地板上,默默地对我看了半晌,似乎在辨认和追忆什么印象。骤然间,他象是倒了个过儿,仿佛被一个可怕的念头吓了一大跳,——从他变得模糊的眼睛里溅出了泪花;他俯身向着我,开始端详我的面庞。
\par “爸爸!”我处在恐怖的折磨下对他说。“别这样看我,爸爸!咱们离开这儿!快走!咱们逃走吧!”
\par “对,逃走,咱们逃走!是时候了!咱们走,涅朵琦卡!快,快!”于是他着了忙,好象这才刚刚想到他该怎么办。他匆忙四顾,见地板上有妈妈的一方巾帕,便捡起来放进兜里,又看见一只系带的软帽,也把它拾起来藏在身边,状似在准备出远门的行装,把他用得着的东西全都带走。
\par 我转眼就穿好自己的衣服,也急急忙忙开始把我觉得路上大概用得着的东西通通带走。
\par “好了没有?好了没有?”父亲问。“都准备好了吗?快!快!”
\par 我胡乱打好一个包裹,系上一方头巾,我们俩已经准备走出去了,这时我忽然想起,应该把墙上的一张画也带走。爸爸当即表示同意。此时他挺安分,说话声音很轻,只是催我快走。画挂得很高,我们两人掇来一把椅子,再往上面放一张板凳爬上去,费了不少工夫总算把画拿下来。现在,我们做好了远行的一切准备。他携起我的手,我们已经迈开脚步,但忽然爸爸又叫我站住。他把自己的脑门揉了好久,似乎在回忆还有什么没做。后来,他大概想起了他应当做的事情,便取出放在妈妈枕头底下的一串钥匙,开始在柜子里匆匆寻找什么东西。最后,他回到我身旁,拿来从抽屉里找到的一些钱。
\par “给,你把这点钱拿去,藏好了,”他压低了嗓门对我说,“别丢了,记住,记住!”
\par 他把钱先放在我手中,接着又拿回去塞在我怀里。我记得,当这些银币触到我的身体时,我打了个寒颤,我仿佛直到那时才明白钱是什么东西。现在我们又作好了准备,可是他忽然又把我叫住。
\par “涅朵琦卡!”他对我说,一边似在费力地思索。“我的孩子,我忘了······那是一件什么事情?······反正是必须做的······我记不得了······对,对!我想起来了,是这么回事!······你过来,涅朵琦卡!”
\par 他把我带到供神像的一个角落里,叫我跪下。
\par “祈祷吧,我的孩子,祈祷吧!这对你有好处!······是的,会有好处的,”他指着神像,奇怪地望着我,向我喃喃低语。“祈祷吧,祈祷吧!”他用一种恳请、央求的语调说。
\par 我双膝跪下,两手合握,满怀已经完全把我抓住的恐怖和绝望仆倒在地,如此俯卧有好几分钟,仿佛呼吸已经停止。我努力把自己的全部思想、全部感情集中起来作祈祷,但恐惧还是一再把我压倒。我架不住忧伤的困扰,微微抬起身子。我已经不想跟爸爸走了,我怕他,我想留下。结果,憋在心中折磨着我的那个问题还是从我胸膛里冲了出来。
\par “爸爸,”我泪汪汪地说,“那末妈妈呢?······妈妈怎么啦?她在哪儿?我的妈妈在哪儿?······”
\par 我再也说不下去,便放声大哭。
\par 他也含泪望着我。于是,他拉着我的手,带我走到床边,扒开胡乱扔着的衣服堆,把被子掀去。我的上帝!她僵卧在那儿,已经冰凉发青。我仆到她身上,抱住她的尸体,自己几乎失去了知觉。父亲让我跪下。
\par “向她行个礼,孩子!”他说。“跟她告别······”
\par 我鞠了一躬。父亲和我一起鞠了躬······。他面容惨白,频频翕动嘴唇念念有词地嘟囔着些什么。
\par “这不能怪我.涅朵琦卡,不能怪我。”他颤颤巍巍地指着尸体对我说。“你听着,这不能怪我;这不是我的错。记住,涅朵琦卡!”
\par “爸爸,咱们走吧,”我惶惶然低声道。“该走了!”
\par “是的,现在该走了,早该走了!”说完,他紧紧抓住我的手,慌忙走出房间。“走,这就出发!谢天谢地,谢天谢地,现在一切都结束了!”
\par 我们下了楼梯;睡眼惺忪的扫院人为我们开了大门,一边用怀疑的眼光瞧着我们;爸爸似乎生怕他问长问短,率先逃出大门,我差点儿追他不上。我们走完屋前的那条街,来到运河的堤岸上。石块铺就的路面上夜来下过一场雪,此时还飘着零星小雪。天很冷,我连骨头架子都一起打着哆嗦,只顾死命地抓住爸爸的燕尾服衣襟跟在他后面跑。他腋下夹着小提琴,不时停下来扶一扶胳肢窝里的琴匣。
\par 我们走了约莫有一刻钟,最后,他顺着便道的斜坡折向一条沟渠,在末尾一座石墩上坐下。旁边是一个冰窟窿,与我们相去仅在咫尺之间。四周一个人也没有。天哪!当时一下子占据我心怀的那种可怕的感觉,直到现在我还记得。整整一年我梦寐以求的理想终于实现了。我们离开了我们那个凄凉的住所······。但这和我所盼望的、梦想的难道是一回事?我对那个人的爱大不同于儿童的感情,我也为他的幸福作过设想,可是在我儿童的想象中构思的难道是这样的图景?此刻最使我感到内疚的是妈妈。“我们为什么孤零零把她撇下?”我心想。“为什么象抛弃废物一样把她的躯体扔下不管?”我记得,这一点最叫我坐立不安,心神不宁。
\par “爸爸!”我实在忍受不了心中这个疙瘩的折磨,还是开了口。“爸爸!”
\par “什么事?”他生硬地问。
\par “爸爸,我们为什么把妈妈丢在那里?我们为什么把她抛弃?”我哭了起来。“爸爸!我们回去吧!我们去叫人看看她。”
\par “对,对!”他猛然一震,嚷着从石墩上站起来,象是想出了一个新的主意可以把他的犹豫扫除干净。“对,涅朵琦卡,不应该这样;应该去看妈妈,她在那边冷得很!你到她那儿去,涅朵琦卡,去吧,那里并不暗,那里有蜡烛;别怕,你去叫个人看看她,然后再到我这儿来,你一个人去,我在此地等你······。我不走开,一步也不离开。”
\par 我转身就走,可是刚踏上便道,我的心就好象给什么东西刺了一下······。我回过头去,见他已经从另一边跑了,在这个时刻扔下我一个人,自己逃跑!我没命地叫起来,怀着极度的恐惧向他追去。我追得上气不接下气;他愈跑愈快······眼看着快要从我的视野中消失。路上我看到他的一顶帽子,这是他逃跑时失落的,我把帽子捡起来,继续追赶。我奔得行即气闭,两腿发软。我觉得自己正落入狼狈不堪的境地,我总以为这是一场梦,有时内心会产生和我在梦中逃避别人追赶时完全相同的感受;我两腿发抖,结果被人追上,我倒下去不省人事。痛苦的心情简直要把我撕裂;我可怜他,一想到他不穿大氅,不戴帽子,撇开我,撇开他心爱的孩子逃跑······我的心一阵阵作痛。我要追上他,只是为了再一次热烈地吻他,叫他不要怕我,让他放心,既然他不愿意,我可以不跟随他,我可以一个人回到妈妈那儿去。后来,我看见他拐向某一条街。我跑到那条街的转角上,也跟着他拐弯,并且还能分辨在我前边的那个身影······。这时,我的体力已经不支,又是哭,又是喊。我记得自己在奔跑中曾跟两个路人相撞,他们在便道中央站住脚,莫名其妙地望着我们俩。
\par “爸爸!爸爸!”我发出最后一次呼喊,接着突然在便道上滑了一下,跌倒在一幢房子的大门旁。我感到自己这一跤摔得满脸都是血。再过一刹那,我便失去了知觉。
\par 我苏醒过来时,身在又暖又软的被窝里,只见旁边好些和蔼可亲的面容都在欢迎我恢复知觉。我看到一位鼻梁上架着眼镜的老太太,一位个子很高的先生深表同情地望着我,还有一位年轻美丽的女士,末了是一位白发苍苍的老者,他扼住我的手腕,一边在看表。这次醒来是我新生活的开始。我在奔跑时曾遇上其中的一位,他是X公爵,我就跌倒在他的宅第大门旁。给我父亲送C——茨的音乐会请柬去的正是这位公爵。公爵经过一再查访,才弄清楚我是谁。他了解这一离奇的事件后颇为所动,决定把我收留在自己家里,和他自己的孩子一起抚养。他们四出寻找爸爸的下落,打听到他在城外大发癫狂的时候被人拦住了。他被送进一所医院,两天以后在那里死去。
\par 他死了,因为这样的死是他整个一生必然和自然的结局。他应该这样死去,因为他的生命赖以支持的一切一下子崩溃了,象一个幻影,象子虚乌有的空想那样烟消云散了。他死了,因为他最后的希望已告幻灭,他借以欺骗自己和支撑全部生活的一切,顷刻间在他本人面前豁然迸裂,清清楚楚地现出了本相。真相以刺目的光芒照得他难以逼视,假象在他自己眼里也成了假象。在最后的时刻,他听到了一位旷世奇才向他述说了他本人的命运,使他遭到了万劫不复的谴责。随着最后一个音符从天才的C——茨的琴弦上飞入他的耳朵,艺术的奥秘也在他心目中全部揭晓,那位永葆青春、经久不衰的真正天才以其真诚压垮了他。他一生只在神秘的、无形的苦痛中感受到某种沉重的压力,真相过去只在梦境中出现并且不可接触、难以捉摸地折磨着他,尽管偶尔也向他表露,但他总是仓皇逃避,用自己一生的假象作盾牌;对于这一切他有所预感,但在这以前一直不敢正视,忽然间,这一切一下子灿灿然昭示在他那双迄今顽固地不承认光明是光明、不承认黑暗是黑暗的眼前。但是,真相是他那双第一次看到往昔、现状和前景的眼睛所无法忍受的;真相震垮了、烧尽了他的理智。真相象闪电一般锐不可当地击中了他。他一生提心吊胆、战战兢兢唯恐发生的事情一下子发生了。仿佛一柄斧钺一生悬在他头上,他一生的每时每刻都在无法形容的痛苦中准备着斧钺落到他头上,而斧钺终于落下了!这一击是致命的。他想逃避对自己的审判,可是无处可逃:最后的希望已化为泡影,最后的口实已不能成立。多少年来,他一直把那个女人视为累赘,觉得她活着自己就没法过;他盲目地相信,一旦那个女人死去,他必定一下子得庆重生。那个女人死了,他终于一身无牵挂,终于自由了!他抱着孤注一掷的心情,想最后一次象一个铁面无私的法官那样毫不留情地自己对自己作出评判;但是,他那松弛的琴弓只能虚软地重复那位天才的最后一个乐句······,就在这一瞬间,对他虎视眈眈已有十年之久的癫狂症,终于不可避免地把他彻底打垮。
\newpage
\section*{四}
\par 我复元得很慢;等到我已经完全能下床了,我的头脑仍没有彻底摆脱麻痹状态,在很长一段时间内,我无法理解自己究竟出了什么事。有时我以为自己做了一个梦,我记得自己但愿所发生的一切果真能变成一场梦!夜里入睡前,我希望一觉醒来一下子又回到我们贫寒的住所,又能见到父亲和母亲······。然而,我的处境毕竟清清楚楚地摆在我的面前,我渐渐明白自己只剩下孑然一身,寄人篱下。于是我第一次意识到我成了一个孤儿。
\par 我开始贪婪地观察我如此突然地来到其间的新环境。起初,我觉得什么都新鲜,什么都稀奇,陌生的面孔、陌生的生活方式,无不使我感到困惑;古老的公爵府第里一间间屋子至今历历在目,那里的房间高大宽敞,陈设豪华,但都是那样幽暗、阴森,我记得当时极其害怕穿过某一座老长老长的厅堂,我觉得进去了会压根儿出不来。我的病尚未痊愈,我的情绪也是阴暗、苦闷的,跟这宅子庄严沉郁的气氛完全合拍。何况,某种我自己还不清楚的哀伤在我幼小的心中正日益增长。我常常在一幅画、一面镜子、一座工艺精巧的壁炉或一尊仿佛故意藏在深深的凹壁中以便偷看我、吓唬我的雕像前莫名其妙地止步,一下子忘掉自己为什么站住,打算干什么,开始想什么;等到从出神状态中猛然醒来,我往往感到恐慌得厉害,我的心怦怦直跳。
\par 在我卧病期间,除了那位小个子老医生外,间或来探望我的人中间给我印象最深的是个年纪已经相当大的男人;他样子很严肃,可是心地十分善良,他望着我的时候眼睛里总是流露出那么深切的同情!跟其他所有的人相比,我最喜欢看到他的脸。我很想同他交谈,可是我不敢:他表面上老是愁眉苦脸,说话不多,口气生硬,从来不见他唇边浮起一丝笑意。他就是发现我昏倒并把我收养在自己家中的X公爵本人。到我渐渐复元的时候,他来探望的次数便愈来愈少。最后一次他给我带来了糖果、一本有图画的儿童书;他吻了我,给我画十字表示祝福,要我显得高兴一些。他安慰我,还说不久我将会有一个小朋友,跟我一样也是个女孩子,就是他的女儿卡嘉,眼下她在莫斯科。接着,他跟一个上了年纪的法国女人、他的孩子的保姆以及照料我的一个使女谈了几句,向他们指指我,然后走了出去,从此我有整整三个星期没见到他。公爵在自己家里过着十分孤僻的生活。宅第的一大半由公爵夫人占用;她跟公爵也往往一连几个星期不见面。后来我注意到,甚至全家人都很少谈及公爵,仿佛他根本不在宅内。大家都尊敬他,甚至看得出都很喜欢他,然而却把他当作一个奇特的怪人看待。他本人大概也知道自己非常古怪,与众不同,因此尽量少在人前露面······。底下我还有很多机会谈到他,而且要详细得多。
\par 一天早晨,有人给我穿上洁净细密的衬衣,外罩带有白色孝徽的黑绸连衣裙,还给我梳了头;我瞧着这件孝服感到一种困惑的惆怅。我从楼上被领到楼下公爵夫人起居的房间里去。当我被带到她那里的时候,我竟站住了直发愣,因为我还从未置身于如此富丽堂皇的环境。但这印象只有短暂的一瞬,我一听到公爵夫人的声音(她吩咐把我带近些),顿时脸色变白。我在穿衣服的时候就准备忍受某种折磨,尽管天知道我怎么会产生这样的想法。反正我进入如今的新生活对周围的一切都抱着一种奇怪的不信任态度。但公爵夫人对我非常和蔼,还吻了我。我把胆子稍许放大,朝她看了一眼。这就是我昏倒后苏醒时所看见的那位美丽的女士。但我在吻她的手时,全身颤抖不已,怎么也鼓不起勇气来回答她的问话。她命我坐在她身旁的一张矮凳上。这个位子象是事先就指定给我的。看来,公爵夫人除了把整个心贴在我身上,给我爱抚,充分代替我的母亲之外,没有其他的愿望。可我怎么也不明白自己交上了好运,在她心目中丝毫没有增添好感。人家给了我一本精美的图画书,叫我看。公爵夫人自己正在给某人写信,偶尔放下笔来跟我谈谈;偏偏我颠三倒四,语无伦次,一句得体的话也没有说好。总而言之,虽然我的身世极不平凡,其中大半由命运以及各种不妨说是神秘的安排在起作用,反正不乏饶有兴味、不可思议甚至近乎荒诞的情节,然而这出赚人热泪的文明戏却好象被我故意杀了风景,因为我本人却表现出是个最平凡不过的孩子,胆小、怕生,甚至有点儿笨头笨脑。特别是末了这一点丝毫不合公爵夫人的口 味,看来她很快就对我完全失去了兴趣,这当然只能怪我自己。下午两点过后,开始有客来访,公爵夫人对我的态度一下子又变得体贴、亲切起来。当客人们问起我时,她回答说,这是一个十分精彩的故事,接着使开始用法语叙述。在她讲故事的过程中,人们不住地望着我,频频摇头,感叹连连。一个青年男子举起长柄眼镜来打量我,一个满身香水味的白发小老头儿想吻我,我脸上红一阵、白一阵.低首垂目,浑身发抖,坐着不敢动弹。我的心隐隐作痛,重又飞向那逝去的日子,重又来到我们的顶楼上。我想起了父亲,想起我们默默度过的漫长的夜晚,想起了妈妈;当我回忆妈妈的时候,禁不住热泪盈眶,咽喉梗阻,我真想逃跑,真想溜走,一个人躲起来······。后来,会客结束,公爵夫人的脸明显地绷紧了些。她看我的目光已比较阴沉,说话的语调也比较生硬,特别使我害怕的是她那双咄咄逼人的黑眼睛有时会盯住我看上一刻钟,还有两片紧闭的薄嘴唇。傍晚,我被带回到楼上。我带着几分寒热蒙眬入睡,夜里醒来,又给颠三倒四的梦境触动了愁怀,不觉悲从中来;第二天早晨,昨天的一切又从头再演,我又被带去见公爵夫人。最后,她向客人讲我的离奇故事大概自己都感到腻味了,客人们对我表示同情也已经不耐烦。何况,我又是一个极普通的孩子。一位上了年纪的女客曾问起:她跟我相处不觉得无聊吗?我记得公爵夫人亲口用说私房话的语调回答:“没有半点天真烂漫的味道。”一天傍晚,我被带走后没有再到那里去。对我的恩宠也就至此告终,不过,我可以去我要去的任何地方。由于刺激太深,我也无法终日枯坐一处,有机会离开周围所有的人,到楼下大房间里去,就高兴得什么似的。据我的记忆,我倒是很想跟家人们聊聊,可我是那么害怕惹他们生气,所以宁愿不跟别人待在一起。我喜欢溜到人家不注意的某个角落里.躲在某件家具背后,当即开始在那里回想和思考我所经历的一切,以此消磨时间。但说也奇怪,我好象把自己在父母身边一番经历的结尾给忘了,也就是把那段悲惨的遭遇通通给忘了。我脑海中闪现的只是一幅幅景象、一桩桩事实。诚然,那一夜、提琴、爸爸——一切我都记得,我也记得自己怎样设法给他弄钱;但要从所有这些事件中悟出道理来,理出头绪来,我却做不到······只觉得心头愈来愈沉重。每当我回忆到跪在死去的妈妈旁边作祈祷的那一刻,总是有一股冷气直砭我的肌骨;我哆嗦着,发出轻微的叫喊,随后呼吸变得困难,整个胸部受到重压,心突突乱跳,结果我总是从角落里仓皇逃出。不过,也许我说得不对,人家没有把我一个人撇下不管:对我的照看是毫不懈怠、尽心尽意的,公爵的嘱托无不一一照办,他吩咐给我充分的自由,不要施加任何限制,但是一分钟也不能让我失踪。我发现不时有家人或仆役往我所在的屋子里探头张望,一句话也不对我说又走了,这样的关切使我十分惊讶,乃至有些不安。我不明白这是为什么。我总觉得有人为了某种目的看管着我,打算以后怎么样处置我。我记得自己总想走得远一些,必要时知道往那儿躲。有一次,我闯到府第正面的楼梯上。它全由大理石砌成,宽阔的梯阶铺着地毯,到处装饰着鲜花;连花盆也精美异常。楼梯的每一片平台上都有两个身材颇高的人默默地坐着,衣着色彩鲜艳,领结白得耀眼,两人都戴手套。我对他们看看,心中直纳闷,怎么也猜不透他们为什么坐在这里一语不发,除了互相对视,什么也不做。
\par 我愈来愈喜欢只身作这样的漫游。此外,我常常从楼上逃下去还有一个原因。楼上住着公爵的一位老姑母,她几乎足不出户。这位老太太在我记忆中留下的印象相当突出。她差不多是宅内最显要的人物。所有的人在与她交往中都恪守一套严谨的礼仪,连一向那么高傲专断的公爵夫人也得每周两次在规定的日子亲自上楼去向姑母请安。她一般上午来;双方开始作枯燥的对话,中间往往出现静穆的冷场,那时老太太不是喃喃地背诵祈祷文,就是数念珠。直要到老太太自己离座起身,吻过公爵夫人的嘴唇,以此表示会晤结束,否则请安者不会先走。过去,公爵夫人每天得向这位长辈问好,但后来,根据老太太的意愿,规矩才有所减免,在一周的其余五天,公爵夫人只须每天早上派人问候她的健康。总的说来,年迈的郡主过的几乎是修女的生活。她从未出嫁.三十五岁那年进了修道院,在那里度过十七年,但没有削发;之后,她离开修道院到莫斯科去,一来和身体一年不如一年的姐姐、Л伯爵的遗孀同住,二来跟另一个姐姐、也没有结过婚的X郡主和解(她们闹翻已有二十余年)。不过据说,这三位老太太没有和睦相处过一天,虽曾上千次表示要分道扬镳,却又没能这样做,因为她们最终发现,她们之中每一个都是其余两个排遣晚年寂寞和防止老人痼疾猝发所少不了的。但是,尽管她们的生活没有多少乐趣,尽管在她们莫斯科的府第里笼罩着肃穆寂寥的气氛,全城名流还是认为自己有义务不间断地向三位深居简出的老太太问安。大家把她们视为保存着全套贵族祖训传统的司库,看作阀阅世家的活的年鉴。伯爵夫人给人们留下许多美好的回忆,她是一位出类拔萃的女性。从彼得堡来的人总是最先拜访这几位。凡是受到过她们接见的,别处的门也就为他们敞开着。但是伯爵夫人去世了,姐妹也分了手:姐姐X郡主留在莫斯科继承没有子女的伯爵夫人遗产中归她的那部分,修女妹妹移居到彼得堡她侄儿X公爵那里去。而公爵的两个孩子卡嘉和萨沙,却留在莫斯科的姑婆身边,盘桓尊前,聊慰寂寞。公爵夫人虽然深爱自己的子女,在规定的举丧期间始终和孩子们分开也不敢出一声怨言。我忘了说,我住到公爵府中去时,宅内还在继续志哀;不过,悼亡期不久即将结束。
\par 老郡主全身孝服,总是穿一件普通丝绸料子的黑色连衣裙,戴浆硬的细裥白衬领,这使她的模样有点儿象养老院里收容的老太婆。她从不放下念珠,坐车出门去做礼拜总是郑重其事,逢到斋期肉乳不进,接见的不外乎各级神职人员和老成持重的来客,读的书都是圣经教义,整个生活方式无异于出家人。楼上静得可怕,连房门也不能咿咿呀呀,因为老太太的耳朵象十五岁的姑娘一样灵敏。倘若有什么东西发出咯噔甚或只是嘎吱一声响,老太太马上会派人来询问原因。大家说话都压低嗓门,走路都蹑手蹑脚,可怜那个年纪也已经很大的法国女人最后不得不放弃平素爱穿的响跟鞋。鞋跟只能割爱。我来到宅内两个星期以后,老郡主派人来了解我是谁,怎么会到宅内来的,等等。有关情况立即向她禀报。于是又有第二名专差奉命向法国女人质询:为什么郡主至今没见到过我?这下顿时忙得不亦乐乎:开始替我梳头,洗脸洗手(其实本来就非常干净),教我见了老郡主怎样趋前行礼,怎样显得和颜悦色,怎样说话,——总之,把我闹得晕头转向。然后由我们这方面派出一名女仆去请示:郡主是不是愿意见孤女了?得到的答复是暂时不见,但指定次日做完礼拜以后进谒。我一宿没有睡好,事后有人谈起,我说了一整夜的胡话,我在幻觉中老是走到郡主跟前为什么事情求她宽宥。我进谒的时间终于到了。我见到一位瘦小干瘪的老太太坐在奇大的安乐椅里。她向我频频点头,还戴上眼镜把我仔细打量。据我记忆所及,她对我一点没有好感。当时曾指出我完全不懂规矩,屈膝、吻手一概不合礼数。垂询开始后,我勉强应答;但当问到父母的时候,我哭了。老太太见我这样控制不住自己的感情,大为不悦;不过,她开始安慰我,叫我把希望寄托给上帝;其后又问我最后一次进教堂已有多久。由于我不大明白她问话的意思,因为我在教育方面给忽视得很厉害,致使郡主大惊失色。她派人去把公爵夫人叫来。经过商议,决定本星期日就带我到教堂里去。在这以前郡主表示要为我祈祷,但吩咐把我带出房间,因为据她说,我给她留下了十分不愉快的印象。这没有什么可奇怪的,事情本该如此。但有一点是很明显的:我完全没有赢得好感。当天就有人奉命来说我太顽皮,说我发出的声响整个宅第都听得见,其实我整天坐着一动也不动;显然这是老太太的错觉。可是第二天又下达同样的指责。偏偏那时我失手打破了一只茶杯,法国女人和使女们都陷于绝望,我立刻给搬到最远的一间屋子去住。那些人也都惶惶不可终日地跟我一起前往。
\par 但我不知道这一切后来是怎样告终的。反正因为这个缘故,我乐于下楼去在一间间大屋子里独自漫步,知道在那里我不会惊动任何人。
\par 记得有一次我坐在楼下的一间厅堂里。我双手捂住面孔,脑袋前倾,这样坐了不知几个小时。我老是在思量,在考虑;我的未成熟的智力排解不了我的全部悲哀,只觉得心头愈来愈沉重,愈来愈郁闷。忽然,有人站在我身旁轻声问道;
\par “你怎么啦,可怜的孩子?”
\par 我抬头一看,原来是公爵;他脸上的表情流露出深切的怜悯和同情;但我却以无限沮丧和凄苦的神态望着他,以致他那双碧蓝的大眼睛里闪起了泪花。
\par “苦命的孤儿!”他轻轻抚摩我的脑袋说道。
\par “不,不,不是孤儿!不!”我说,这时从我胸膛里迸出一声悲切的呻吟,接着,我心中的一切都翻腾起来了。我站起来抓住他的一只手吻着,涕泪交流地用恳求的语调一个劲儿地说:
\par “不,不,不是孤儿!不!”
\par “我的孩子你怎么啦,我可爱而又可怜的涅朵琦卡?你怎么啦?”
\par “我的妈妈呢?我妈妈在哪儿?”我嚎啕痛哭放声大叫,再也掩藏不住一腔悲苦,心力交瘁地跪倒在他面前。“我的妈妈呢?亲人哪,告诉我,我妈妈在哪儿?”
\par “原谅我,我的孩子! ······唉,可怜的孩子,是我触痛了她的创伤······。我真不应该!来,跟我来,涅朵琦卡,跟我来。”
\par 他拉着我的手快步走去。他直到心坎深处都给震动了。后来,我们走进我还没有看见过的一间大屋子。
\par 这是一间供着神像的祈祷室。时近黄昏,长明灯的亮光与神像的金叶、宝石交相辉映。亮闪闪的叶片仅仅露出圣徒们暗淡无光的面容。这里的一切跟别的屋子迥然不同,这里的气氛是那么神秘、森严,给我的印象过于强烈,我的心竟为一种恐怖的感觉所控制。除此以外,当时我的神经又是那样脆弱!公爵急忙让我跪在圣母像前,他自己跪在我旁边。
\par “祈祷吧,孩子,祈祷吧;让我们一起祷告!”他用轻微、急促的音调说。
\par 但我无法祈祷;我愣住了,甚至可以说吓慌了,我想起最后那一夜父亲在我母亲的尸体前所说的话,于是我的神经过敏症又告发作。我再次病倒,并且险些乎死在我整个病程的这一继发期内;下面我就要谈到此事的经过。
\par 一天上午,我耳朵里响起一个熟悉的名字。我听到了C——茨的名字。有人在我床畔提到他。我打了个寒噤;往事涌上心头,我在痛苦的回忆和遐想中躺了不知几个钟点,完全陷入谵妄状态。我醒来已经很晚,四周一片黑暗,过夜的小灯熄灭了,坐在我房间里的一个使女也不在。忽然,远处有音乐飘入我耳中。乐声时而完全沉寂下来,时而渐趋清晰可闻,好象愈来愈近。我不记得一种什么样的感受把我抓住,也不记得一种什么样的意图在我紊乱的头脑里产生。反正我下了床,也不知哪来的力气,胡乱穿上我的孝服,摸索着走出房间。在第二间、第三间屋子里我都没有遇见一个人影。后来,我摸到甬道里。乐声愈来愈听得清楚。甬道中央有扶梯通到楼下;我总是走这条路到大房间里去。扶梯上灯光明亮;楼下有人走动;我躲在角落里不让人看见,直到可以了,我才下去进入第二条甬道。响亮的乐声是从隔壁大厅里发出来的;那里喧哗嘈杂,热闹非凡,好象有几千人汇聚一堂。大厅直通甬道的一扇门上挂着大红丝绒的双层巨幅帘幔。我撩起外面的一层,站到两重帘幔之间。我的心跳得几乎叫我站也站不稳。但几分钟以后,我抑止住激动的心情,终于鼓起勇气把第二重帘幔从边上揭开一点点······。我的上帝啊!我一直不敢走进去的这座阴森森的大厅此刻银烛满堂,灯火辉煌,如同光的海洋朝着我涌过来;我已习惯于黑暗的眼睛在最初的一刹那竟被刺得生疼,什么也看不见。香气如一阵热风直扑我的面庞。走来走去的人多得数不清,似乎个个笑容满面,喜气洋洋。女人的服饰都是那样华贵、那样色泽鲜艳;我看到处处都是神采飞扬的眼光。我站在那里,简直跟着了魔一样。我觉得这一切好象在某时某地的梦中见到过······。我回想起昔日的黄昏时分,回想起我们的顶楼、高高的小窗、楼下深处路灯明亮的街道、对面那幢房屋张挂红色帘幔的窗户、大门口车水马龙的景象、趾高气扬的马匹的蹄声和响鼻、嘈杂的人声、窗上的人影、隐约的乐声······。对了,这不正是那个天堂吗? ——这个想法在我头脑里刷地一闪,——这不正是我一心想和可怜的父亲一同去的地方吗?······敢情这不是空想!······是的,过去我在幻想中、在梦境中看到的跟眼前的一模一样!病中发烧的头脑里又点燃起想象的烈火,无法解释的狂喜的热泪夺眶而出。我用目光搜索父亲,我在想:“他一定在此地,一定在。”紧张的期待使我心跳气急。但是音乐停了下来,只觉得耳际嗡然作响,整个大厅里掠过一片窃窃私议之声。我贪婪地注视着在我前边浮晃的一张张面孔,竭力想认出某一个人来。忽然,大厅里起了一阵异样的骚动。我看到台上出现一个高高瘦瘦的老头儿。他那苍白的脸微微含笑,他向左右前后鞠躬致意,弯腰的动作似乎不太灵活;他手中拿着一把小提琴。接着,大厅里鸦雀无声,好象这些人一齐屏住了呼吸。所有的视线同时集中到老头儿身上,大家都在等待。他拿起琴来,用弓触到弦上。音乐开始了,我觉得我的心忽然被什么东西紧紧攥住。我屏住气,忍住无限的悲怆,凝神谛听这声音,只感到有些耳熟,好象我在什么地方听见过;这声音唤起了某种预感,某种非常可怕的事情发生之前的预感,而我的心中也在酝酿着一场凶猛的风暴。随后,琴声转趋激越,音乐变得急速、尖利。接着仿佛有人发出绝望的号叫、凄惨的恸哭,仿佛有谁在这一大群人中间哀哀求告,纵使声泪俱下也是枉然,只得在绝望中吞声。音乐在向我的心诉说一个愈来愈熟悉的故事。但是心不愿相信。我咬紧牙关不打哼哼,我抓住帘幔以免摔倒······。我曾数次闭上眼睛又突然睁开,但愿这是一场梦,但愿在我熟悉的一个可怕的时刻猛然醒来,发现自己梦见了那最后的一夜,听到了同样的声音。我睁开眼睛,希望证实自己的想法,急切地向人群中望去,——不,这是另一些人,这是另一些面孔······。我觉得大家都和我一样期待着什么,和我一样不胜心底隐痛的折磨;他们似乎都想冲着这可怕的呻吟和哀鸣大喝一声,喝令它们静下来,不要作践他们的心灵,似呻吟和哀鸣的音流愈淌愈凄厉、愈悲切、愈拖长。突然间响起最后一声很长的惨叫,把我的五脏六腑都震动了······。毫无疑问!这正是那一声惨叫!我可以断定,我已经听到过,这声惨叫跟当时一样,跟那天夜里一样刺透了我的心。“父亲!父亲!”我头脑里闪电般冒出一个想法。“他在此地,这是他,他在叫我,这是他的琴声!”仿佛一声浩叹发自全体人群,接着,暴雨般的掌声震撼了整个大厅。就在这个当儿,哭声随着绝望的尖叫冲出我的胸膛。我再也忍耐不住,竟掀开帘幔闯进大厅。
\par “爸爸,爸爸!这一定是你!你在哪儿?”我高声叫喊,几乎完全忘其所以。
\par 我不知道我有没有跑到那高个儿老头跟前,只记得人们纷纷闪开,给我让路。我拚命哀叫着向他那边扑过去;我以为马上就要和父亲拥抱······。忽然,我看到有人用细长嶙峋的双手把我抓住了举起来。一双乌黑的眼睛直盯着我,简直想要喷出火来把我烧毁。我瞪着那老头儿。“不!他不是我父亲;他是杀死我父亲的人!”这念头在我脑中一闪。我被某种极度愤激的心情所控制.骤然觉得他冲我发出狂笑,觉得这狂笑在大厅里哄然引起一片喧嚷;我失去了知觉。
\newpage
\section*{五}
\par 这是我的病程的第二阶段,也是最后阶段。
\par 当我重新睁开眼睛的时候,看见她一个孩子——跟我年龄相仿的女孩子——俯视着我,我的第一个动作就是向她伸出双手。从第一次看到她的脸开始,我的心里就充满幸福,充满一种甜滋滋的预感。请设想一张十全十美的面庞,那是—种令人惊异的、光彩照人的美,在这样的俏脸庞前面你会突然止步,简直跟中了一箭差不多,会产生一种甜蜜的不好意思的感觉,会欣喜得颤栗,会在心中感谢她的存在,感谢她打你身旁经过,让你的视线落到她的脸上。这女孩子是公爵的女儿卡嘉,她刚从莫斯科回来。她见我向她伸出双手,立即面露笑容,使我脆弱的神经浸沉在说不出是酸是甜的欣悦之中。
\par 小郡主把在一旁跟医生谈话的父亲叫来。
\par “哦,感谢上帝!谢天谢地!”公爵拿起我的一只手说,并非做作的感情使他脸上光彩焕发。“我很高兴,很高兴,非常高兴,”他按照老习惯说得很快。“这就是卡嘉,我的孩子,你们认识一下,——如今你有伴儿了。你快点儿好起来,涅朵琦卡。这鬼丫头,真把我吓坏了!······”
\par 我恢复健康的过程进展很快。几天后,我已经下床走动。卡嘉每天早上到我床前来,总是笑容满面,而且笑声不绝。我象盼望幸福那样盼着她来;我多么想吻吻她啊!可是这顽皮的小姑娘顶多只来几分钟;她没法安分老实地坐一会。不停地活动、奔跑、跳跃、喧嚷和发出合宅都能听见的响声,是她不可或缺的需要。因此,她从一开始就向我声明,她坐在我床边实在闷得慌,所以她将难得来看我,而且是她瞧我可怜才来,——也是无可奈何,不能不来;不过,等我复元了以后,我们会好起来的。从此,她每天早上见了我第一句话就问:
\par “怎么样,病好了没有?”
\par 由于我还面黄肌瘦,在我忧郁的脸上浮现的笑容也是怯生生的,小郡主立刻皱眉摇头,懊恼地跺脚。
\par “昨天我明明对你说过,希望你好一些!怎么搞的?大概不给你吃东西,是不是?”
\par “是的,不多,”我羞怯地回答,因为我一看见她就感到羞怯。我千方百计想赢得她的好感,故而我对自己的每句话、每个动作都特别留神。她的来到愈来愈叫我喜欢。我一眼不眨地望着她,她走了以后,我仍然象着了魔似地望着她刚才所站的那一边。她开始出现在我的梦中。我醒着的时候,她不在身旁,我就在想象中跟她长谈,做她的朋友,和她一起顽皮、淘气;如果我们为什么事情挨了骂,我就和她一起哭鼻子,——总之,我象掉进了爱河一般眷恋着她。我渴望着恢复健康,按照她给我的忠告尽快胖起来。
\par 早晨,卡嘉跑到我屋子里来,一开口就嚷道:“还没有好?还是这样瘦!”这时,我照例象犯了什么过失似地感到胆怯情虚。但是,卡嘉对于我不能在一昼夜间恢复健康所表示的惊讶却是丝毫不假的;因此到后来她真的生气了。
\par “那么,要不要我今天给你拿一块大蛋糕来?”有一天她问我。“你吃了很快会胖起来的。”
\par “拿来吧, ”我欣然答道,这样我可以再一次见到她。
\par 小郡主问过我的健康情况后,通常在我对面的一把椅子上坐下,开始用一双黑眼睛把我仔细察看。在她和我相识之初,她怀着极其天真的诧异心情不时这样从头到脚对我打量。但我们之间的话谈不起来。我在卡嘉面前总是害羞,她的大胆奔放益发使我局促拘谨,其实我万分希望跟她谈谈。
\par “你怎么不说话?”在一阵冷场之后卡嘉问。
\par “爸爸在干什么?”我问道,心里每次为找到一句开场白而高兴。
\par “不干什么。爸爸觉得挺好。今天我喝了两杯茶,多喝了一杯。你呢?”
\par “一杯。”
\par 又是冷场。
\par “今天福斯塔夫想咬我。”
\par “那是狗吗?”
\par “是的,是一条狗。你难道没看见过?”
\par “不,我看见过。”
\par “那你干吗还问?”
\par 由于我不知怎么回答才好,小郡主又诧异地望着我。
\par “怎么?我跟你说话你觉得快活吗?”
\par “是的,非常快活,希望你常来。”
\par “人家也这样说:要是我常到你这儿来,你会觉得快活的。你快下床吧,我今天给你拿蛋糕来······。你怎么老不说话呀?”
\par “呣。”
\par “你一定老是在想吧?”
\par “是的,我想得很多。”
\par “可是人家告诉我,我说得多、想得少。难道说话不好吗?”
\par “不。我喜欢听你说话。”
\par “呣,回头我问问廖塔尔太太,她什么都知道。那末你想些什么?”
\par “我在想你,”我顿了一下才回答。
\par “你觉得这样快活吗?”
\par “是的。”
\par “这么说,你喜欢我喽?”
\par “是的。”
\par “可是我还没有喜欢你。你太瘦了!我要给你拿蛋糕来。回头见!”
\par 小郡主从屋子里飘然离去,她吻我的时候几乎已经起飞。
\par 不过,午后果然有蛋糕。她无比兴奋地跑进来,一边哈哈大笑,得意非凡,因为她毕竟给我拿来了别人不准我吃的东西。
\par “你多吃一点,好好吃,这是我的蛋糕,我自己没吃。好了,再见!”话音未落,人影已杳。
\par 另一次她突然飞到我屋子里来,也是在午后一个出乎意料的时刻,她的黑色鬈发仿佛遇上了龙卷风,两腮红得发紫,眼睛分外明亮;这表明她跑来跑去、蹦蹦跳跳已有一两个小时。
\par “你会玩板羽球吗?”她喘吁吁地大声问,话说得很快,正急于到什么地方去。
\par “不会,”我答道,并为自己不能说“会!”而惋惜万分。
\par “你呀!算了,等你病好了,我教你。我来就为这件事。现在我跟廖塔尔太太玩去。再见;人家在等我。”
\par 我终于从病床上下了地,尽管身体还虚弱。我第一个想法就是再也不离开卡嘉。她身上有一股力量不可抗拒地吸引着我。我对她怎么也看不够,这使卡嘉大惑不解。我对她的好感之深、热情之高,使她不可能不注意到这一点;最初,她觉得这是闻所未闻的怪事。记得有一次在做什么游戏的时候,我忍不住搂住她的脖子开始吻她。卡嘉从我的拥抱下挣脱出来,她抓住我的两只手,好象我欺负了她似地沉下脸来问我:
\par “你怎么啦?你干吗吻我?”
\par 我窘得要命,仿佛做错了什么事,被她急口令一般的问话猛地一震,一个字也答不上来。小郡主把肩膀一挺,表示疑惑没有得到解答(这是她的一个习惯动作),非常认真地闭紧肥软的嘴唇,中断了游戏,坐在角落里的沙发上,从那边对我观察了很久很久,一边默默思考,大概想解答她头脑里突然冒出来的新问题。这也是她在遇到各种疑难情况时习惯的做法。反过来说,我也有很长时间适应不了她性格上这种大起大落的表现。
\par 起先我责怪自己.认为我身上确实有很多怪脾气。虽然这是事实,可我还是感到困惑和苦恼:为什么我不能从第一次见面就和卡嘉成为好朋友,为什么不能一下子赢得她永远的好感。我的挫折伤透了我的心,每次遇上卡嘉的疾言厉色和不信任的眼光,我总想哭。而且,我的悲哀不是与日俱增,而是每小时都在加剧,因为卡嘉的任何事情都发展得很快。几天后,我发现她完全不喜欢我了,甚至开始讨厌我。在这个小姑娘身上,一切都表现得迅速、干脆,但从她天真无邪的直爽性格反映出来的这些疾如闪电的行动,包含着真正高尚的气质,否则,有人也许会说她粗野。她对我先是产生怀疑,继而甚至变为轻蔑,其原因最初好象在于我不会做任何游戏。小郡主爱跑爱跳,她强健、好动、麻利;我则相反。我病后还很虚弱,不好动,爱沉思;游戏并不使我快活;反正我完全缺乏能为卡嘉所喜欢的本领。此外,我还受不了别人对我有所不满,我会立刻变得忧郁、沮丧,哪里还谈得上弥补自己的过错,扭转于我不利的印象,——总之可以说毫无希望。这是卡嘉百思不得其解的。例如,她在我身上足足花了一个钟点,教我怎样玩板羽球,可是毫无结果,起初她甚至有些怕我,照例用惊讶的眼光端详着我。由于我马上会变得忧郁起来,由于眼泪即将夺眶而出,她对我想之再三,从我身上既得不到结果,自己又思考不出一个所以然来,最后完全撇下我不管,自己一个人玩儿,再也不来邀我,甚至一连几天不跟我说一句话。这对我的刺激很大,我简直无法忍受她的蔑视。新的孤独对于我几乎比以前的更加难受,我又开始闷闷不乐,愁绪重重,凄凉的哀思又压在我心上。
\par 负责照看我们的廖塔尔太太终于发觉了我们关系中的这一变化。由于我首先引起她的注意,我被迫忍受孤独的现象使她吃惊,她直接向小郡主质问,责备她不懂得好好对待我。小郡主眉头一皱,肩膀一挺,说她跟我在一起没有意思,我什么也不会玩儿,我老是在想心事,她宁可等弟弟萨沙从莫斯科回来,那时他们俩会快活得多。
\par 但是,廖塔尔太太并不满足于这样的回答,她向小郡主指出,她不应该撇下我不管,要知道我的身体还不好,我不能象卡嘉一样跳跳蹦蹦,这反倒更好,因为卡嘉过于顽皮,还说她干了哪些蠢事,前天差点儿给一条叭喇狗咬死,——总之,廖塔尔太太狠狠地训了她一顿,临了还责令她来找我立即和解。
\par 卡嘉十分注意听廖塔尔太太的教训,似乎从她的这番道理中确实领会到一些新的有说服力的内容。她扔下正在大厅里滚着玩儿的铁环,走到我跟前,郑重其事地望着我,用不理解的口气问:
\par “您难道要玩儿?”
\par “不,”我答道,卡嘉挨廖塔尔太太骂的时候,我害怕了,我为自己、也为卡嘉害怕。
\par “那您到底要怎么样?”
\par “我想坐一会;我跑不动;不过您别生我的气,卡嘉,因为我非常爱您。”
\par “好吧,那我就一个人玩儿,”卡嘉语气平稳、一字一顿地说,似乎觉得挺奇怪,因为照这样看事情不能怪她。“再见,我不生您的气了。”
\par “再见,”我说着也站起来向她伸出一只手。
\par “也许,您愿意接个吻?”她略加考虑后问,谅必记起了不久前我们之间发生的龃龉,想尽量使我高兴,以便快些跟我和解。
\par “随您的便,”我心存万一的希望答道。
\par 她走到我面前,一本正经地吻了我一下,连笑也不笑。就这样完成了向她提出的要求,甚至做得比必要的更多,为的是让她奉命去安慰的一个穷苦女孩子心满意足,于是她高高兴兴地从我身边跑开,很快在各处屋子里又响起她的笑声和喊声,直至她累得上气不接下气地倒在沙发上休息,重新积聚精力。整个晚上她都用疑惑的目光望着我,谅必我在她眼里显得古怪透顶。看来她想跟我谈谈,想弄清楚她对我感到不解的地方;但这一次我不知为什么她没有问。平时,廖塔尔太太每天上午给卡嘉上课,教她法语。课程的全部内容就是复习语法和读拉封丹的寓言。给卡嘉授课的分量不多,因为她好不容易才答应每天坐下来读两小时书。她是应父亲之请、母亲之命接受这项条件的,履行起来倒是诚心诚意的,因为她自己作了保证。她具有非凡的资质,理解力很强。但她也有一些小小的怪僻:如果她对什么事情不理解,马上开始自己思考这件事,而决不肯找别人给她解释,——她似乎羞于这样做。据说,有时她为了一个问题得不到解决会连续几天大伤脑筋,只恨自己不靠别人帮忙对付不了,直要到山穷水尽的时候才去请廖塔尔太太帮她解决她啃不动的问题。她的一举一动也是这样。其实她想得很多,尽管乍看起来并非如此。但与此同时,她又天真得和她的年龄不相称,有时会提出极其荒唐可笑的问题来;在别的场合,她的回答又表现出细致和巧妙的远见卓识。
\par 由于我也可以学点儿什么,廖塔尔太太对我进行了考试,发现我读的能力很好,写的能力却很差,她便认为非立即教我法语不可。
\par 我没有意见,于是,在某一天早晨,我和卡嘉一起在课桌旁坐下。偏偏这一回卡嘉笨得出奇,并且极度心不在焉,廖塔尔太太简直不相信这是卡嘉。而我差不多在一堂课的时间内就已识得所有的法文字母,因为我竭力想以勤奋来讨好廖塔尔太太。一堂课下来,卡嘉把廖塔尔太太气得七窍生烟。
\par “您【廖塔尔太太要求学生严格按“上流社会”的规矩说法语,所以自己说话第二人称也总是用敬称。】瞧瞧她.”廖塔尔太太指着我对卡嘉说,“人家是个有病的孩子,第一次上课成绩就比您强十倍。您不害臊吗?”
\par “她知道得比我多?”卡嘉惊讶地问。“她还刚开始认字母呢!”
\par “您花了多少时间才把字母认下来?”
\par “三堂课。”
\par “可她一堂课就认下来了。可见,她的理解力比您强两倍,很快就会超过您的。难道不是吗?”
\par 卡嘉考虑了一会儿,认为廖塔尔太太的话有理,顿时脸红得象一团火。她每次遇到挫折,或者觉得懊恼,或者感到自豪,或者调皮捣蛋的事情败露时,——总之几乎在一切情况下,她的第一个反应就是羞得面红耳赤。这一回,她眼眶里几乎闪出了泪花,但她没有则声,只对我看了简直想要把我烧成灰烬的一眼。我马上就猜到是怎么回事。原来这小姑娘的自尊心强到极点。当我们从廖塔尔太太那里出来的时候,为了快些消除她的懊恼,我主动表示,法国女人说这番话完全不能怪我;但卡嘉一声不吭,仿佛没有听见我的话。
\par 一小时以后,我坐在屋子里看书,头脑里一直想着卡嘉的事,正在为她又不愿跟我说话纳闷和害怕;这时她闯了进来。她双眉颦蹙朝我一看,照例在沙发上坐下,目不转睛地注视着我有半个小时。最后,我实在忍不住了,就用眼神向她提问。
\par “您会跳舞吗?”卡嘉问我。
\par “不,不会。”
\par “可是我会。”
\par 冷场。
\par “您会弹钢琴吗?”
\par “也不会。”
\par “可是我会。这是很难学会的。”
\par 我不做声。
\par “廖塔尔太太说,您比我聪明。”
\par “廖塔尔太太生您的气了,”我答道。
\par “那末,难道爸爸也会生气?”
\par “我不知道。”
\par 又是冷场;小郡主用一只小小的脚焦躁地踏着地板。
\par “光凭您的头脑比我的灵,往后您就可以笑我?”她终于懊丧得沉不住气问道。
\par “哦,不,不!”我嚷着霍地从座位上站起来,想扑过去和她拥抱。
\par “郡主,您这样想、这样问不害羞吗?”这时冷不防响起了廖塔尔太太的声音,其实她在观察我们、听我们谈话已有五分钟。“您应当感到惭愧!您竟会妒忌一个可怜的孩子,向她夸耀您会跳舞、弹钢琴。真难为情;我要把一切都告诉公爵。”
\par 小郡主的两颊烧得通红。
\par “这是不健康的感情。您这样问已经伤害了她。她的父母是穷人,雇不起教师教她;她是自己学的,因为她有一颗纯洁、善良的心。您应当喜欢她才对,可是您却要跟她吵架。真难为情,真难为情!要知道,她是个孤儿。她没有亲人。您还大可向她夸耀您是公爵小姐,而她不是!现在我离开您,您把我对您说的话好好想想,然后改正。”
\par 小郡主整整想了两天!两天听不到她的笑声和喊声。我夜里醒来,听见她梦中也在继续同廖塔尔太太评理。这两天下来,她甚至消瘦了些,她那光彩焕发的小脸蛋儿也不是那么红喷喷的了。到了第三天,我们俩才在楼下大房间里相遇。小郡主刚从她母亲那里来,见了我以后,她收住脚步,在对面不远处坐下。我提心吊胆、全身哆嗦着等候下文。
\par “涅朵琦卡,为什么我得为您挨骂?”她终于问。
\par “这不是为我,好卡嘉,”我急于为自己辩解。
\par “可是廖塔尔太太说我伤害了您。”
\par “不,好卡嘉,不,您没有伤害我。”
\par 小郡主把肩膀一挺,表示不解。
\par “您为什么老是哭?”在沉默片刻后她又问。
\par “如果您讨厌,我就不哭,”我噙着泪水回答。
\par 她又耸耸肩膀。
\par “以前您也老是哭吗?”
\par 我不回答。
\par “您为什么住在我们家?”过了一会儿,郡主突然问道。
\par 我愕然望着她,我的心好象被什么东西扎了一下。
\par “因为我是个孤儿,”我终于鼓起勇气来回答。
\par “您有没有爸爸妈妈?”
\par “有过。”
\par “他们不喜欢您吗?”
\par “不······他们喜欢我,”我硬着头皮回答。
\par “他们是穷人?”
\par “是的。”
\par “很穷?”
\par “是的。”
\par “他们什么也没有教过您?”
\par “教过我认字。”
\par “那时您有玩具吗?”
\par “没有。”
\par “有蛋糕吗?”
\par “没有。”
\par “你们有几间屋子?”
\par “一间。”
\par “一间屋子?”
\par “一间。”
\par “有佣人吗?”
\par “没有,没有佣人。”
\par “那末谁给你们做事呢?”
\par “我自己去买东西。”
\par 小郡主问的话愈来愈刺痛我的心。往事的回忆、孤女的身份、郡主的惊诧——这一切无不伤害我的感情,使我的心不断沁血。我激动得浑身颤栗,眼泪堵塞了咽喉。
\par “这么说,您住在我们家很高兴喽? "
\par 我不吭声。
\par “您有好衣裳吗?”
\par “没有。”
\par “坏的呢?”
\par “有。”
\par “我看见过您的衣裳,人家给我看过。”
\par “那您为什么要问我?”我说着站起身来,一种前所未知的新的感受引起我手脚发抖。“您为什么要问我?”我气得涨红了脸往下说。“您为什么嘲笑我?”
\par 小郡主也红了脸站起来,但旋即克制了自己的激动。
\par “不······我没有笑,”她答道。“我只不过想知道,您的爸爸妈妈很穷,这是不是真的?”
\par “您为什么向我问起爸爸妈妈?”我忍不住心灵的创痛哭了起来。“为什么您这样问他们的事?他们怎么了您啦,卡嘉?”
\par 卡嘉站着有些发窘,不知如何回答。正在这个当儿,公爵走了进来。
\par “你怎么啦?涅朵琦卡?”他朝我脸上一看,见我在流泪,就问。“你怎么啦?”他又看看卡嘉,发现她脸上火辣辣的。“你们在谈什么?你们为什么吵架?涅朵琦卡,你们为什么吵架?”
\par 但我不能回答。我抓住公爵的一只手吻着,眼泪扑簌簌掉下来。
\par “卡嘉,不要撒谎。到底怎么回事?”
\par 卡嘉不会撒谎。
\par “我说我看见过她有很坏的衣裳,那还是她和爸爸妈妈住在一起的时候穿的。”
\par “谁给你看了?谁竟敢这样做?”
\par “是我自己看到的,”卡嘉断然答道。
\par “好!你不愿推到别人身上的,我了解你。底下呢?”
\par “可是她哭了起来说,我为什么要嘲笑她的爸爸妈妈。”
\par “这么说.是你嘲笑了他们?”
\par 尽管卡嘉没有笑,但当我第一次认为她在嘲笑时,她显然有这样的意图。她一句话也没有回答,这表明她自己也承认此事。
\par “马上去向她道歉,”公爵指着我对卡嘉说。
\par 小郡主站在原地一动不动,脸色煞白。
\par “去!”公爵说。
\par “我不愿意,”卡嘉终于低声说,但态度非常坚决。
\par “卡嘉!”
\par “不,我不愿意,不愿意!”她忽然两眼闪光、双脚跺地嚷了起来。“爸爸,我不愿去道歉。我不喜欢她。我不愿跟她住在一起······。她整天哭哭啼啼,这怪不了我。我不愿意,不愿意!”
\par “你跟我走,”公爵说着抓住她的一只手,拉着她往自己书斋里走。“涅朵琦卡,你到楼上去。”
\par 我想跑到公爵跟前去为卡嘉讨情,但公爵厉声重申他的命令,我吓得手脚冰凉,只得上楼去。到了我们的屋子里,我倒在沙发上,双手捧住脑袋。我一分钟一分钟地数着时间,焦急地等候卡嘉,恨不得扑到她脚下。最后,她终于回来了,一句话也不说,打我身旁经过走到角落里。她哭得双眼通红,两颊浮肿。我的决心顿时消失。我惊恐地望着她,吓得不能动弹。
\par 我竭力责备自己,竭力向自己证明一切都是我的过错。我无数次想走到卡嘉跟前去,可又无数次停下来,不知道她会怎样对待我。如此过了一天、两天。到第二天傍晚,卡嘉情绪有所好转,甚至滚动铁环在各间屋子里跑了一阵,但旋即放下游戏,一个人坐到角落里。临睡前,她忽然面朝我转过身来,甚至向我跨了两步,她张开嘴想对我说些什么,但半道上又止步转身,躺下睡了。接着又过了一天,诧异的廖塔尔太太终于开始盘问卡嘉:她究竟怎么啦?她一下子变得这样安分,是不是病了?卡嘉回答了几句,还拿起拍子准备打板羽球.但廖塔尔太太刚转过身去,她就涨红脸哭了。她从屋子里跑出去,不让我看到她。末了,事情总算得到解决:在我们发生争吵三天以后的下午,她突然走进我的房间,不好意思地挨到我跟前。
\par “爸爸命我向您道歉,”她说,“您原谅我吗?”
\par 我急忙抓住卡嘉的双手,激动得气急心慌地说。
\par “当然!当然!”
\par “爸爸命我跟您亲吻,——您愿意吻我吗?”
\par 作为回答,我开始吻她的双手,在上面洒了不少眼泪。我朝卡嘉一看,见她神态有些异样。她的小嘴微微翕动,下巴发颤,眼睛湿润,但她倏然间克制了内心的激动,唇边曾有一瞬浮起笑意。
\par “我去告诉爸爸,说我跟您亲吻过了,我向您道了歉,”她低声说,象在独自沉吟。“我已经三天没见到他;他不准我去见他,除非向您道歉,”她沉默片刻后又找补了几句。
\par 说完,她带着羞怯和沉思的表情下楼去,好象还拿不准父亲是否愿意见她。
\par 但一小时过后,楼上响起了喧嚷声、笑声和福斯塔夫的吠声,有什么东西打翻并摔破了,有几本书跌落在地板上,铁环格啷啷从这间屋子滚到那间屋子,——总之一句话,我知道卡嘉又跟她父亲和好如初,我的心也因欣悦而颤动。
\par 但卡嘉不来找我,显然想避免和我交谈。然而我却有幸引起她极大的好奇心。她愈来愈经常地在我对面坐下,这样便于仔细看我。她对我进行观察的方式比较天真,这个被合宅上下当作宝贝娇纵惯了的任性的小姑娘,不明白我怎么会几次在她根本不想遇见我的时候和我狭路相逢。但这是一颗美好、善良的童心,它总是能凭直觉找到一条正路。她所崇拜的父亲对她拥有最大的影响。母亲简直爱得她发狂,但又对她严得要命;卡嘉从她那里继承了顽强、好胜和坚定的性格,可也全盘继承了母亲的怪僻,直至精神上的唯我独尊。公爵夫人的教育观比较奇特,卡嘉所受的教育也是娇纵无度和严厉无情的奇怪混合物。昨天允许的事情今天忽然无缘无故地遭禁,使孩子心中的是非观念受到损害······。以后还会发生这样的故事······。我只想指出一点:孩子已经有能力确定自己对父母的态度。和父亲在一起时,卡嘉的本性充分外露,毫无隐蔽;和母亲在一起时则相反,显得内向、多疑、依头顺脑。但她这种柔顺并非真正出于心悦诚服,而是为一套体系所格。底下我会详加说明。不过,应当为我的卡嘉说句公道话,她到底能理解母亲的心,当她服从母亲意志的时候,已经充分意识到母爱的无限强烈有时会达到病态的狂热程度,——于是小郡主慷慨地把这一点也考虑在内。惜乎这种考虑后来对她容易亢奋的头脑没有什么好处!
\par 但我几乎不明白自己正在发生什么变化。一种新的、莫名其妙的感觉激荡着我的身心,我可以毫不夸大地说,我在为这一新的感情而苦闷,而烦恼。简单地说——请原谅我用这样的语汇——我爱上了我的卡嘉。是的,这是爱,真正的爱,伴有眼泪和欢欣的爱,热情洋溢的爱。她有哪一点吸引着我?怎么会产生这样的爱?这是从我第一次看见她就开始的,当时我的全部感官都被这个天使般可爱的孩子的模样所震动,那是一种舒适的震动。她身上的一切都美;她的每一种坏脾气都不是和她一起诞生的,都是后来染上的,而且始终处在斗争状态。处处都可以看出美好的本性暂时被赋予不正确的形态,但她身上的一切,从这一斗争开始,都闪耀着可喜的希望,一切都预示着美好的未来。人人都欣赏她,人人都爱她,不光我一个人如此。比如,在下午三点钟,有人带领我们出去散步,所有的行人一看见她,都会愕然止步,往往有惊讶的叹赏从这位天之骄子背后传来。她生下来就幸运,她应当为幸福而生——这便是我和她见面得到的第一个印象。也许,这是在我身上第一次被唤起美的感觉,也是我对她产生爱的全部原因。
\par 小郡主主要的坏脾气,或者说她性格上的主要素质是好胜。这种素质不可阻挡地力求在其自然形态中得到体现,同时必然会处于有偏向的状态即斗争状态。她的好胜及于十分幼稚的琐事,自尊心往往强到这样的程度:如果遇到什么阻力,不管属于何种性质,都不会使她伤心、生气,而只会感到惊奇。她想不通,怎么可能出现与她的愿望不一致的情况.然而,正义感在她心里总是会占上风的。如果她确信自己错了,会立即服从裁判而毫无怨言,决不翻悔。至于在这以前她对待我的态度与她的性格不符,我认为都可以归结为她对我抱有令人费解的反感,这种反感一时搅乱了她的整个身心的协调与和谐;这也是势所必然,因为她过分耽于自己的爱好,全仗榜样和经验把她引上正途。她出的点子都能取得圆满的结果,但那是通过不断发生的偏差和谬误付出了代价的。
\par 卡嘉很快就完成了对我的观察,最后决定不再管我。她采取的办法是好象宅内根本没有我这个人。对我不说一句多余的话,甚至几乎必要的话也不说;我被排除在各项游戏之外,但不是强行排除,而是巧妙地做得仿佛是我自己不愿参加。课还是照常上,虽然廖塔尔太太把我作为悟性和安分的榜样要她学习,但我已经伤害不了她那极端敏感的、连我们的叭喇狗约翰·福斯塔夫爵士也可能伤害的自尊心。福斯塔夫平时阴阳怪气,但要是把它惹急了,可就凶得象只老虎,连主人的权威也不承认。还有一个特点;它不喜欢任何人;但它真正恨得最甚的敌人毫无疑问是老郡主······。不过,下文还要讲到这段故事。高度自尊的卡嘉千方百计想战胜福斯塔夫的冷淡,哪怕家里只有这么一只畜生不承认她的权威和力量,不向她俯首帖耳,不喜欢她,她也不痛快。于是小郡主决心主动向福斯塔夫进攻。她要役使一切,君临一切,岂容福斯塔夫逃脱注定的命运?但是,不屈的叭喇狗愣是不肯就范。
\par 一天下午,我们俩坐在楼下大厅里,福斯塔夫趴在屋子中央懒洋洋地享受午后的休息。这当儿小郡主忽发奇想要降伏它。她扔下自己的游戏,踮着脚开始小心翼翼地接近那条叭喇狗,一边用各种肉麻的名字亲切地称呼它,和颜悦色地向它招手。但福斯塔夫老远就龇着可怕的牙齿,小郡主只得止步。她本想走到福斯塔夫跟前把它抚摩一番,叫它跟自己走,而福斯塔夫向来不许任何人抚摩它,只有把它当作宠儿的公爵夫人除外。卡嘉的设想称得上一项艰巨的壮举,内含着非同小可的危险,因为福斯塔夫可以毫不费力地咬断她的手或者把她撕裂,如果它认为有此必要的话。它的力气大得象熊,我提心吊胆地从远处注意卡嘉的花招。但要一下子劝阻她是不容易的,甚至福斯塔夫极不客气地露出的牙也远远不足以吓退她。小郡主明白直接走近它肯定不可能,便在困惑中开始绕着她的对手打转。福斯塔夫毫无动静。卡嘉绕了两圈,已把直径大大缩短;接着又绕了第三圈。可是当她走到福斯塔夫视为禁区的地方时,那条狗又龇牙咧嘴,凶相毕露。卡嘉一跺脚扫兴地退回来,坐在沙发上沉思默想。
\par 过了十来分钟,她想出新的招数诱惑对手,当即走了出去,然后带着些甜面包、馅儿饼回来,——不妨一言以蔽之曰:她更换了武器。可是福斯塔夫不为所动,因为它可能吃得太饱。它对扔给它的一块面包连肴也不看,当小郡主又来到福斯塔夫认为属于自己疆界的禁区前面时,双方形成了较第一次更为严重的对峙状态。福斯塔夫昂首露牙,发出不太响的嘟囔之声,做了一个幅度不大的动作,仿佛准备一跃而起。小郡主气得面红耳赤,扔下馅儿饼回到原地方坐下。
\par 她处在强烈的激愤之中。她一只脚频频踏着地毯,腮帮子红似火焰,眼眶里甚至闪起懊丧的泪花。偏偏她朝我看了一眼,周身的热血顿时往她头脑里直涌。她断然站起来,踏着最坚定的步子径向猛犬走去。
\par 也许,这一次福斯塔夫实在太惊愕了。它竟让敌手进入禁区,直到双方相去仅及咫尺,它才以一声最凶险的咆哮迎接不顾一切的卡嘉。卡嘉站住一会儿,但也只是一会儿工夫,接着又果断地迈步向前。我吓得发了呆。我还从未见过小郡主如此热情冲动;她眼睛里闪耀着胜利的光芒。这时按她的模样可以画一幅绝妙的图像。她勇敢地顶住了暴怒的叭喇狗的威胁性目光,在它的血盆大口前面没有发抖;福斯塔夫半个身体已站起来。从它毛茸茸的胸中发出一声惊心动魄的怒吼;再过一分钟,它也许已经把卡嘉撕裂。但是小郡主把一只小手骄傲地按到它身上,以胜利的姿态在狗背上抚摩三次。有一刹那工夫叭喇狗拿不定主意。这是最恐怖的一刹那;但它突然费劲地站起来伸了个懒腰,谅必认为不值得跟小孩子打交道,悠悠然从屋子里走了出去。小郡主踌躇满志地站在被她征服的那个地方,向我投了难以言传的一瞥,陶醉在胜利喜悦中的一瞥。可是我面如土色;她注意到了,并且微微一笑。不过,她的两颊也已蒙上一层死灰色。她勉强走到沙发旁,几乎象昏厥似地倒在上面。
\par 但我对她已经着了迷,而且迷得晕头转向。打从我替她捏了一大把冷汗的那天起,我已不能自持。我陷入了相思的痛苦,有千百次直想扑过去搂住她的脖子,可又被恐惧束缚在原地不能动弹。我记得自己当时竭力避开她,怕给她看到我激动的情状,但要是她蓦地闯进我躲在那里的某间屋子,我会手脚发抖,心跳得天旋地转。我觉得淘气的小郡主看出了这一点,有两天连她也不知如何是好。但不久她就对此安之若素。这样过了一个月,我在暗中足足受了一个月的折磨。我的感受具有一种不可思议的韧性(如果可以这样形容的话):我的本性耐力无与伦比,若非万不得已,不会发生感情的爆炸或突然宣泄。必须指出,在整个这段时间内,我和卡嘉没说上两三句话,但我根据某些很难捉摸的迹象发现,她这样做并非出于对我的遗忘或漠视,而是出于有意的规避,好象她发誓要和我保持一定的距离。但我夜里已睡不着觉,而白天甚至在廖塔尔太太面前也无法掩饰自己的窘态。我对卡嘉的爱到了怪诞的程度。有一次我偷偷地拿了她的一方手绢,另一次拿了她扎辫子的一条缎带,夜里泪流满面地吻着这些东西。起初,卡嘉的冷淡使我深感委屈,但后来我的心思都搅乱了,自己也没法把自己的感觉理出个头绪来。就这样,新的印象渐渐排挤了旧的,想起自己凄凉的往昔也不再引起我的剧痛,回忆已被新的生活取而代之。
\par 记得我有时夜里醒来,下了床蹑着脚走到小郡主床前。我可以借我们一盏过夜小灯的微光一连几个小时凝视睡着的卡嘉;或者坐在她床边,俯身到她面前,让她的呼吸向我送来阵阵热流。我不顾吓得瑟瑟打颤,悄悄地吻她的小手、臂膀、头发,甚至吻她的脚,如果有一只脚露在被外的话。由于我已有整整一个月目不转睛地注意她,慢慢地发觉卡嘉一天比一天显得若有所思,她的性格正在失去固有的均衡:有时整天听不见她的动静,有时却会掀起前所未有的喧闹。她变得比较急躁、挑剔,动不动就脸红生气,对待我甚至不惜在小事情上采取狠心的做法。忽而不愿和我一起进餐,不愿挨着我坐,似乎对我表示厌恶;忽而到她母亲身边去,一连几天待在那里,可能知道我离开了她要害相思;忽而开始对我连续瞧上几个钟点,窘得我无地自容,脸上红一阵、白一阵,可又不敢从屋子里走出去。卡嘉已有两次抱怨头痛发热,而以前我从未发现她有任何疾病。直到某一天早晨,终于下达了一道特别指示:根据小郡主的迫切愿望,她搬到楼下妈妈屋里去了,因为公爵夫人得悉卡嘉抱怨头痛发热吓得半死。必须说明一下,公爵夫人对我极为不满,她也注意到了卡嘉身上所起的变化,并把这种变化归咎于我,用她的话来说,这都是我的阴郁的性格影响到她女儿的性格。她早就想把我们分开,然而一直推迟采取行动,知道在这个问题上不得不同公爵发生重大的争论,因为公爵虽然处处对她让步,但有时却毫不通融,固执到近乎刚愎。公爵夫人对他有充分的了解。
\par 小郡主搬走使我大为震惊,我足有一星期处于极不正常的紧张状态。我忍受着渴念的煎熬,苦苦思索卡嘉讨厌我的原因。忧伤撕扯着我的胸臆,义愤开始在我受了伤害的心中翻腾。我身上忽然产生一种不甘屈辱的心情,到了有人带领我们出去散步的钟点,我在和卡嘉见面时,一反过去的常态,摆出卓然独立、煞有介事的架势望着她,这甚至使她感到意外。当然,这样的变化在我身上只是阵发性的表现,过后心又开始痛得更厉害,我也变得比以前更羸弱、更胆怯。一天早晨,小郡主终于又回到楼上,这使我莫名其妙,又喜不自胜。她先是狂笑着跑去跟廖塔尔太太拥抱,宣布她又搬回来了,接着也向我点点头;她请求这天上午不要上课,获准后就奔跑跳跃痛痛快快玩了半天。我从未见过她这样活泼快乐。但向晚她却转趋沉静,若有所思,她那俏脸蛋儿又蒙上一层忧郁的阴影。晚上,公爵夫人来看她时,我见卡嘉不自然地努力想装出挺快活的样子。但是,母亲走后,她一个人忽然掉下了眼泪。我大吃一惊。小郡主发现我在注意她,便走出房间。反正她身上在酝酿着一场意想不到的危机。公爵夫人向好几位大夫征求过意见,每天把廖塔尔太太叫去询问有关卡嘉的情况,连最琐屑的细节也不忽略,还吩咐对她的一举一动仔细观察。只有我一个人对真相有所预感,我的心在希望的鼓舞下开始加速跳动。
\par 简言之,一段小小的罗曼司正在演出并且临近尾声。卡嘉回到我们楼上来的第三天,我注意到,她整整一个上午老是用异样的目光长久地望着我······。我有好几次遇见这样的目光,每一次我们俩都涨红了脸,低首垂目,仿佛彼此害羞。临了,小郡主笑出声来离我而去。等到钟敲三下,人家给我们穿戴起来,准备出去散步。卡嘉忽然走到我跟前。
\par “您一只鞋的鞋带松开了,”她对我说,“我来帮您系。”
\par 我因为卡嘉终于开始跟我交谈而脸红得象樱桃,正想自己俯下身去。
\par “让我来!”她不耐烦地说着笑了起来。她当即弯下腰,硬把我的脚搬起来搁在她的膝盖上,动手系鞋带,我感到呼吸急迫,由于惊喜交加而不知如何是好。她系好鞋带站起来,把我从头到脚打量一番。
\par “脖子也敞着,”她说时用一个指头碰了一下我脖子上裸露的皮肤。“我来重新裹一下。”
\par 我不加抗拒。她把我的围脖解开,按她的习惯重新结扎。
\par “要不然着了凉会咳嗽的,”她十分调皮地笑道,但见一双水汪汪的黑眼睛向我一闪。
\par 我简直忘其所以;我不知道自己是怎么回事,也不知道卡嘉是怎么回事。感谢上帝,我们的散步很快结束了,否则我会忍不住在街上搂住她亲吻。不过,上楼梯的时候,我还是偷偷地在她肩膀上吻了一下。她发现了这个动作,身体一震,但一句话也没说。傍晚,她被打扮得漂漂亮亮领到楼下去。公爵夫人那里有客。但这天晚上一场风波闹得阖宅不宁。
\par 卡嘉发了一次神经性的惊厥。公爵夫人吓得魂飞魄散。医生说不出究竟,只得归结为儿科疾病,归因于卡嘉的年龄,可是我另有想法。翌晨,卡嘉来到我们面前时跟平时一样红润、快活、精力充沛,但也带来一些从未有过的怪僻和奇想。
\par 首先,整个上午她完全不听廖塔尔太太的话。尔后,她忽然要到老郡主那儿去。老太太一向讨厌这个侄孙女,经常跟她不睦,不愿见她,这回却一反常态,允许她进见。最初一切顺利,一老一小在第一个钟点内相处得挺好。狡黠的卡嘉居然请求宽恕,说自己老是吵吵嚷嚷,调皮捣蛋,使老郡主不得安宁。老太太含着眼泪庄严地宽恕了她。但是这个鬼丫头决意走得更远。她甚至煞有介事地历数自己的种种淘气行为,其实这些还仅仅是念头和设想。卡嘉扮演了一个顺从、虔敬的真诚悔过者的角色,总而言之,用迷汤把道学气十足的老郡主灌得飘飘然:因为卡嘉是全家的宝贝,她甚至能迫使母亲也依从她的古怪想法,老郡主眼看这样一个天之骄子向她俯首服输,自尊心当然得到很大的满足。
\par 淘气的小姑娘承认,她曾打算把一张名片粘在老郡主的衣服上,打算让福斯塔夫埋伏在她床下,打算砸碎她的眼镜,把她的书统统搬走,用妈妈那里的法文小说掉包,还打算弄些爆竹来洒在地板上,还打算把一副纸牌藏在她兜里,等等,等等,不一而足。总之,调皮捣蛋的花样可谓层出不穷。老太太愈听愈不象话,气得脸色红里泛青;最后,卡嘉忍不住纵声大笑,从姑婆身边逃之夭夭。老太太立即吩咐把公爵夫人叫去。于是乎掀起一场轩然大波。公爵夫人花了两个小时泪汪汪地央求老太太饶恕卡嘉,姑念她在发病,不要施加惩罚。起先,老郡主水泼不进,她扬言明天就搬出宅第;直至公爵夫人保证把惩罚推迟到女儿病好为止,以后一定执行,以平老郡主的义愤,她的态度才和缓下来。卡嘉经受住了一顿严厉的呵责。她被关到楼下公爵夫人房间里去。
\par 但午后这小淘气还是设法脱了身。我悄悄下楼去的时候,恰好在扶梯上碰见她。她把扶梯门推开一条缝招呼福斯塔夫。我瞬即猜到她在打主意进行可怕的报复。事情是这样的——老郡主深恶痛绝的对头莫过于福斯塔夫。这条狗不向任何人讨好,也不喜欢任何人,傲慢、自大、骄矜达于极点。它不喜欢别人,但显然要所有的人都对它表示一定的尊敬。大家也确乎以这样的态度对待它,不过在尊敬中杂有相当的畏惧。然而,自从老郡主驾临此地,一切都发生了变化:福斯塔夫遭到骇人听闻的侮慢,具体说来,它竟被明令禁止到楼上去。
\par 最初,受辱的福斯塔夫简直气疯了,整整一个星期老是用爪子抓由楼上通下面一间屋子的扶梯门;但它不久就猜透了遭放逐的原因,在老郡主出门上教堂的第一个星期日便尖叫狂吠着向她扑去。当时幸得有人相救,她才免遭此犬疯狂的报复,因为福斯塔夫受辱被逐都是老郡主的命令,她声称见不得那条狗。从此,上楼的禁令便以最严厉的方式对福斯塔夫执行;逢到老郡主要下楼,总是先把福斯塔夫赶到最远的屋子里去。仆人们担当着无比重大的责任。但那只记仇的畜生曾先后三次找到机会闯上楼去。它一冲上扶梯,立刻穿越所有的房间直奔老郡主的寝室。什么也拦阻不住它。幸而老太太的房门经常上锁,福斯塔夫只得在门外恶狠狠地咆哮一通,直到有人赶来把它轰下楼去为止。在桀骜不驯的叭喇狗来访期间,老郡主自始至终没命地呼叫,简直象已经被吃掉了似的,而且每次都要受惊致病。她曾一再向公爵夫人发出ultimatum【拉丁文:最后通牒。】,有一次甚至走了嘴,竟说她和福斯塔夫两者必须有一个离开宅子;然而公爵夫人不愿同福斯塔夫分离。
\par 公爵夫人喜欢的人不多,但除了自己的孩子,她最爱的便是福斯塔夫。原因何在?大约在六年前,公爵一次散步归家,带回来一只肮脏而有病的小狗:模样寒伧得可怜,品种倒是一头纯粹的叭喇狗。是公爵救了它的命。由于这位新来的居住者举止粗野,不识礼貌,在公爵夫人坚持下被打发到后院去用绳子拴起来。公爵没有表示异议。过了两年,公爵全家住在别墅里,卡嘉的弟弟小萨沙掉进了涅瓦河。公爵夫人失声惊呼,她第一个动作便想紧跟在儿子后面投入河中。当时若非有人拉住,她必死无疑。其时孩子正被水流很快地冲走,只有他的衣服浮在水面上。人们急急忙忙把一条船解去缆绳,但除非出现奇迹,否则孩子得救看来是无望的了。忽然,一头硕大无朋的叭喇狗纵身蹿入水中截住漂流的小孩,衔着他胜利游到岸上。公爵夫人跑过去和又脏又湿的狗亲吻。但福斯塔夫(当时它的名字还是毫无诗意和十足平民色彩的弗里克萨)讨厌任何人跟它表示亲昵,竟在公爵夫人肩膀上尽牙齿所及的深度狠狠咬了一口,作为对她的拥抱和亲吻的回答。公爵夫人终生摆脱不了这个创痛,然而她的感激也是无限的。福斯塔夫被洗刷得干干净净,还得到一只工艺精美的银项圈,并从此获准进入内宅。它在公爵夫人起坐室里一张华贵的熊皮上定居下来,不久,公爵夫人已能够抚摩它而无须担心迅即遭到惩罚。及至闻悉她的爱犬名叫弗里克萨,公爵夫人大为骇然;大家马上给它另起一个尽可能古色古香的名字。但诸如赫克托、塞伯拉斯之类的名字已过于俗气;要求这个名字能充分配得上这位得到全家宠爱的英雄。后来,公爵考虑到弗里克萨无与伦比的饕餮本领,建议给叭喇狗取名福斯塔夫【莎士比亚的喜剧《温莎的风流娘儿们》中一个肥胖、贪吃、好色的人物。】。这个诨号受到热烈欢迎,从此永远成为那条狗的名字。福斯塔夫表现得很好,象个地道的英国人那样不声不响、阴阳怪气,从不先向任何人跑过去,只要求别人保持相当距离绕过它在熊皮上的地盘,处处向它表示一定的尊敬。它偶尔显出垂头丧气、百无聊赖的样子,在这样的时刻,福斯塔夫会痛心地想起,它的不共戴天的仇敌、胆敢侵犯它权利的冤家对头,尚未受到惩罚。于是它悄悄地来到通往楼上的扶梯前,发现那里的门照例锁着,便在附近找个地方躺下,躲在角落里居心叵测地窥伺,看有没有人粗心大意开了门不锁上。这记仇的畜生往往有耐心等上三天。但主人有命令严密监视扶梯楼门,所以福斯塔夫迄今已有两个月不得上楼。
\par “福斯塔夫!福斯塔夫!”小郡主叫着,一边打开扶梯门,招手欢迎福斯塔夫到我们楼上去。
\par 这时,福斯塔夫发觉有人开门,已经准备跃过对它设置的警戒线。但小郡主的召唤在它听来太不可思议了,以致它一时完全不敢相信自己的耳朵。它象猫一般狡猾,为了佯装没有发觉开门人的疏忽,故意走到窗前.把强壮的前爪搁在窗台上,开始观察对面的一幢房屋,——总之,它摆出一副十足的旁观者姿态,在散步时路经此地逗留片刻,欣赏一下邻近那幢房屋出色的建筑风格。其实,它的心在欣悦的期待中跳得正欢。及至门在它面前开得笔直,不唯如此,还有人招呼它.邀请它,恳求它上楼去立刻报仇雪恨,不难想象这时它惊讶、狂喜和激愤到何等程度!福斯塔夫高兴得尖叫一声,露出牙齿,威风凛凛、杀气腾腾地象一支离弦的箭冲上楼去。
\par 它的冲刺势头之猛,半道上有一把椅子给碰了一下,竟弹出丈把远以后翻倒在地。福斯塔夫象一发射出的炮弹向目标飞去。廖塔尔太太发出恐怖的惊呼,但福斯塔夫已冲到那扇禁门前,举起两只前爪使劲撞了一下,然而没有把门撞开,于是它因功败垂成而发出悲愤的嗥叫。紧接着可以听到老郡主惊心动魄的呼喊。此时,千军万马已从四面八方纷纷赶来,合宅人等会师楼上,有人把皮笼口巧妙地套到狗嘴上,福斯塔夫四条腿都给绊住。结果,狂暴的福斯塔夫被用套索牵着下楼,从战场上铩羽而归。
\par 一名使臣奉命去见公爵夫人。
\par 这一回,公爵夫人并不倾向于宽恕,可是该处罚谁呢?她一下子就猜到了事情的原委,她的视线落在卡嘉身上······。果然不出所料;卡嘉站在那里吓得面色煞白,全身颤抖。可怜的小郡主这才意识到自己的恶作剧后果严重。这件事可能怀疑到仆人、无辜者身上,所以卡嘉准备把真相和盘托出。
\par “是你干的好事?”公爵夫人声色俱厉地问。
\par 我见卡嘉面无人色,便挺身而出,用坚定的语气说:
\par “是我放过了福斯塔夫······怪我不小心,”可我添上一句,因为我的全部勇气在公爵夫人逼视下已化为乌有。
\par “廖塔尔太太,请儆戒一下!”公爵夫人说完就走出房间。
\par 我朝卡嘉看了一眼:她站着呆若木鸡,两手垂在身旁,苍白的脸望着地上。
\par 处罚公爵的孩子所采用的唯一方式是空室禁闭。在空屋子里待上一两个钟头本来是无所谓的。但如果违背本人的意愿硬把一个孩子关起来,并对之宣布不得自由行动,那可是相当严厉的惩罚。对卡嘉或她的弟弟一般关两个小时。考虑到我这次罪行的性质实在骇人听闻,决定把我关四个小时。我怀着病态的喜悦进入禁闭室。我心里想着小郡主。我知道自己胜利了。不过,我在空屋子里不止待四个小时,而是直待到清晨四点钟。下面是事情的经过。
\par 我被禁闭起来以后过了两小时,廖塔尔太太得悉她的女儿从莫斯科来到彼得堡,刻下突然病了,想和她见面。廖塔尔太太走的时候把我给忘了。照看我们的一名使女大概以为我已经得到释放。卡嘉给叫到楼下去,被迫在母亲身边直待到夜晚十一点。回来时发现我不在床上,她极为惊讶。使女给她脱去衣服,安置她睡下;小郡主并不问起我,她有她的理由。她上床后准备等我,因为她肯定我被禁闭四个小时,估计我们的保姆会把我送回来。但娜斯佳把我忘得一干二净,何况我是一贯自己脱衣服的。我就这样在禁闭中过夜。
\par 凌晨四点钟,我听见有人在大敲大擂禁闭室的门。我是躺在地板上凑合着睡的,惊醒后吓得叫了起来,但当即分辨出卡嘉比谁都响的声音,其次是廖塔尔太太,再次是娜斯佳、管家妇的声音。门终于被打开,廖塔尔太太含着眼泪把我搂住,为她把我忘了这件事深表歉意。我也泪流满面地和她拥抱。我冻得直打哆嗦,全身的骨头因躺在光地上酸痛得厉害。我用目光寻找卡嘉,可是她跑到我们寝室里,一纵身钻进被窝去了,我进去时,她已经睡着,或者假装睡着了。其实,她从晚上便开始等候我,等着,等着,不知不觉入了梦乡,一直睡到清晨四点钟。她醒过来以后,立即闹翻了天,把已经回来的廖塔尔太太以及保姆、使女通通吵醒,于是我才得救。
\par 次日早晨,合宅上下都知道了这件事,连公爵夫人也说对我过于严厉了。至于公爵,这一天我生平第一次看见他大发雷霆。上午十点钟,他情绪无比激动地来到楼上。
\par “请问,您怎么能这样做呢?”他开始质问廖塔尔太太。“您是怎样对待一个病孩的?这是野蛮行为,十足的野蛮行为,太不文明了!一个虚弱的病孩,受不起惊吓的小姑娘,生来就容易想入非非,居然在黑屋子里整整关了她一夜!这不是要毁了她吗?难道您不了解她的身世?这太野蛮了,简直不讲人道,这就是我要对您说的,廖塔尔太太!怎么可以这样处罚孩子?是谁想出来的,是谁发明这样的处罚办法?”
\par 可怜的廖塔尔太太窘得要命,她泪汪汪地开始向公爵说明原委,说她把我给忘了,因为她女儿来了;并说这个处罚办法本身并不坏,如果时间不是太长的话;还说连让·雅克·卢梭【让·雅克·卢梭(1712—1778)——法国作家、哲学家、教育家,他写过一本论儿童教育的书《爱弥儿》。】也说过这样的话。
\par “您居然抬出了让·雅克·卢梭!让·雅克不可能说这话。让·雅克不是权威,让·雅克·卢梭不敢妄谈教育,他没有权利这样做。让·雅克·卢梭连自己的孩子也不要,廖塔尔太太!让·雅克是个坏人,廖塔尔太太!”
\par “让·雅克·卢梭!让·雅克是坏人?!公爵!公爵!您在说什么呀?”
\par 廖塔尔太太全身的血液都沸腾了。
\par 廖塔尔太太是个不可多得的好人,她最讨厌动不动就生气;但要是触及她崇敬的人物,渎犯高乃依【高乃依(1606-1684)——法国古典主义戏剧奠基人,代表作为悲剧《熙德》。】、拉辛【拉辛(1639-1699)——法国古典主义戏剧家,代表作为悲剧《费德尔》。】的古典主义巨影,侮慢伏尔泰,说让·雅克·卢梭是坏人,——那可不得了!廖塔尔太太的眼泪夺眶而出,这位法国老太太气得浑身发抖。
\par “您这话太没有分寸了,公爵!”她终于抑止不住愤怒说道。
\par 公爵立刻发觉自己失言,并表示歉意,然后走到我面前,深情地吻了我一下,给我画了个十字,从房间里走出去。
\par “Pauvre prince!【法语:可怜的公爵!】”廖塔尔太太也深受感动地说了一句。于是我们坐下来上课。
\par 但小郡主上课时心不在焉得厉害。在去吃午饭之前,她走到我对面站住,满脸通红,嘴角带笑,夹住我的肩膀,匆匆忙忙,似乎很不好意思地说:
\par “怎么样?昨天替我受苦了吧?饭后我们到大厅里玩儿去。”
\par 这时有人打我们身边走过,小郡主倏即扭过脸去不看我。
\par 黄昏时分,我们俩吃过饭以后手拉着手下楼到大厅里去。小郡主情绪激动,呼吸急促。我则感到从来没有过的喜悦和幸福。
\par “你愿意抛球玩儿吗?”她问我。“你站到这儿来!”
\par 她叫我站在大厅的一角,但她自己没有退开去把球抛给我,而是在离我仅三步的地方站住,朝我看看,双手捂住骤然泛红的脸倒在沙发上。我向着她跨出一步;她以为我可能走开。
\par “不要走,涅朵琦卡,跟我在一起待一会,”她说,“我马上就会好的。”
\par 她一骨碌从沙发上爬起来,满面通红、热泪纵横地扑过来和我拥抱。她的腮颊湿漉漉的,嘴唇象两颗熟透的樱桃,鬈发蓬乱披散。她疯狂地吻着我,吻我面孔、眼睛、嘴唇、脖子、胳臂;她歇斯底里地哭着,我紧紧贴在她怀里,我们象两个好朋友,象一对久别重逢的恋人一般甜甜蜜蜜,亲亲热热地互相拥抱。卡嘉的心跳得厉害,它的每一次搏动我都能听到。
\par 但这时隔壁房间里有人在叫唤卡嘉到公爵夫人那儿去。
\par “啊,涅朵琦卡!我走了!晚上见,夜里见!你先到楼上去等我。”
\par 她吻了我最后的一次,那是温柔、无声而又热烈的一吻,然后离开我赶往娜斯佳发出叫唤的方向。我好象重新获得了生命,急忙跑上楼去扑倒在沙发上,脑袋埋在枕头里,由于喜出望外而嚎啕大哭。心中象有砧杵在捣个不停,简直要把胸膛捅穿。我不记得我是怎样捱到深夜的。直至钟敲十一点,我上床睡觉。小郡主到十二点才回来;她老远向我微笑,但没有说话。娜斯佳开始给她脱衣服,仿佛故意磨磨蹭蹭个没完。
\par “快一点,快一点,娜斯佳!”卡嘉嘀咕道。
\par “您怎么啦,郡主?您的心扑腾腾直跳,八成是上楼的时候跑得太快了吧?······”娜斯佳问。
\par “哎呀,我的上帝!你真噜苏,娜斯佳!快,快!”小郡主恼怒地在地板上跺了一下脚。
\par “嚯,好大的脾气!”娜斯佳说,她正在给小郡主脱鞋,就把这只脚吻了—下。
\par 好容易一切都结束了,小郡主躺下睡觉,娜斯佳从屋子里走出去。一转眼卡嘉便从床上跳起来跑到我跟前。
\par “到我床上去跟我一起睡!”她说着把我从被窝里拖起来。顷刻间.我已到了她床上,我们俩抱做一团,贪婪地互相贴紧。小郡主把我全身都吻遍了。
\par “我记得以前你夜里是怎样吻我的!”她说时脸红得象朵罂粟花。
\par 我哭了。
\par “涅朵琦卡!”卡嘉也含泪向我耳语。“我可爱的天使,我很早很早就爱你了!你可知道是打什么时候开始的?”
\par “什么时候?”
\par “打从爸爸命我向你道歉开始的,也就是你卫护你爸爸的那一次,涅朵琦卡······。可——怜——的——孤——儿!”她拉长了调子说着又把我吻遍。她又是哭,又是笑。
\par “卡嘉!”
\par “什么事?说呀,什么事?”
\par “为什么我们俩这么多时间一直······一直······”我没把这句话说完。我们互相拥抱,足有三分钟一句话也不说。
\par “你倒说说看,你对我是怎么想的?”小郡主问。
\par “哦,我想得可太多了,卡嘉!我老是在想,白天夜里都想。”
\par “夜里你还说我来着,我听见过。”
\par “真的?”
\par “还哭了不知多少次。”
\par “你瞧!你到底为什么老是这样傲慢?”
\par “我实在太蠢了,涅朵琦卡。不知怎么的,我往往犯这样的毛病,完全没有旁的原因。我老是生你的气。”
\par “为什么?”
\par “就为了我自己是个坏东西呗。先是因为你比我好;后来因为爸爸更喜欢你。爸爸的心地善良,涅朵琦卡!你说对吗?”
\par “对,当然!”一想起公爵,我禁不住含着眼泪答道。
\par “他是个好人,”卡嘉郑重其事地说,“可是我对他有什么办法呢?他老是那样······。后来,我向你道歉,差点儿哭了起来,为了这个缘故,我又生气得要命。”
\par “我看到了,我看到了当时你想要哭。”
\par “你给我住口,傻丫头,你自己最爱哭鼻子!”卡嘉用手捂住我的口向我喝道。“听着,我很想爱你,后来一下子又想恨你,而且恨得咬牙切齿!······”
\par “究竟为什么?”
\par “反正我就是生你的气。我也不知道为什么!可是后来我看到你离开了我日子没法过,我就寻思:这可恶的丫头,我得好好给她吃点儿苦头!”
\par “啊,卡嘉!”
\par “我的心肝!”卡嘉吻着我的手说。“后来我不愿跟你说话,怎么也不愿意。我抚摩福斯塔夫的那一回,你可记得?”
\par “哦,你的胆量真大!”
\par “其实,我怕——得——要——死,”小郡主拖长音调说。“你知道我为什么要走到它跟前去?”
\par “为什么?”
\par “因为当时你在场。我看到你在瞧着······懂吗?!所以我不顾一切地走过去。我把你吓坏了吧,啊?你为我捏了一把汗?”
\par “真把我吓死了!”
\par “我看见的。等到福斯塔夫走开的时候,我高兴得不得了!我的天,这凶猛的家伙走了以后,我才愈想愈怕!”
\par 小郡主神经质地哈哈大笑,然后一下子抬起她发热的脑袋来,开始注视着我。眼泪象一颗颗珍珠在她长长的睫毛上颤动。
\par “你身上究竟有哪一点能使我这样爱你?瞧你的模样,苍白的皮肤,头发的颜色又那么淡,傻里傻气,爱哭鼻子,一双浅蓝色的眼睛,可怜的孤儿!!!”
\par 说罢,卡嘉又凑过来吻了我无数次。她有好几颗眼泪掉在我的面颊上。这是深受感动流下的眼泪。
\par “其实,我非常非常爱你,可硬是不承认!心想:我决不告诉她!我的脾气就有那么犟!现在想来,我何必怕你呢?我何必不好意思告诉你呢?瞧,现在我们俩多好!”
\par “卡嘉!我觉得太好了!”我大喜欲狂地说。“高兴得心都疼了!”
\par “是的,涅朵琦卡!你听我说下去······我问你,是谁把你的名字叫成涅朵琦卡的?”
\par “妈妈!”
\par “你把妈妈的事都告诉我好吗?”
\par “好,一定都告诉你,”我欣然答道。
\par “你把我的两条花边手绢放到哪儿去了?还有,你为什么把我的缎带拿走?真不害臊!这些我都知道。”
\par 我笑了起来,羞得面红耳赤,差点儿掉下眼泪。
\par “我心想:我要捉弄捉弄她,不能一下子告诉她。有时候我又想:我一点也不喜欢她,我讨厌她。可你老是那样柔顺,象一只绵羊!其实,我非常担心你认为我愚蠢!你挺聪明,涅朵琦卡,你确实聪明得很。难道不是吗?”
\par “你说到哪儿去了,卡嘉!”我几乎动了气。
\par “不,你确实挺聪明,”卡嘉坚定而认真地说,“这我是知道的。一天早晨我从床上起来,心里只觉得说不出有多么爱你!我整整一夜老是梦见你。我心想;我要搬到妈妈那儿去住。我不愿意爱你,不愿意!可是第二天夜里将要入睡的时候,我又盼望你跟隔天夜里一样走到我床前来,而你果然来了!当时我假装睡着了······。啊,我们俩真是不识羞的一对儿,涅朵琦卡!”
\par “你究竟为什么不愿意爱我?”
\par “我没有把意思说清楚。其实,我一直爱你!一直爱你。后来简直耐不住了,我寻思:总有一天我要把她吻一个痛快,或者狠狠地拧她,拧得她半死不活。我这就让你尝尝滋味,你这个蠢丫头!”
\par 说着,小郡主拧了我一把。
\par “还记得我给你系鞋带的事吗?”
\par “记得。”
\par “‘记得’;你乐意不?我瞧着你,心想;多可爱的小姐儿,让我给她把鞋带系好,不知她会有什么想法?当时我自己也挺乐意。说实话,我真想跟你亲吻······可是没有这样做。后来心里愈想愈可乐,太可乐了!在我们一起散步的时候,一路上也是这样,我一下子直想放声大笑。我瞧着你就想笑出米。你代替我坐班房,我非常高兴。”
\par “班房”指的是那间空屋子。
\par “你在里边怕不怕?”
\par “怕得要死。”
\par “我高兴还不光是因为你把责任揽到自己身上,而是因为你要代替我坐班房!我心想:这时候她在哭,可我是这样爱她!明天我一定要好好地吻她,吻一个痛快!真的,我并不可怜你,确实不可怜你,虽然我哭了。”
\par “我倒挺高兴,我偏不哭!”
\par “你没哭?啊,你好狠的心肠!”小郡主嚷着把嘴唇向我凑过来。
\par “卡嘉,卡嘉!我的天哪,你多美啊!”
\par “难道不是吗?现在,你爱把我怎么样就怎么样!折磨我,拧我吧!请你拧我一把!我的宝贝,拧我一把!”
\par “淘气鬼!”
\par “还有吗?”
\par “傻丫头······”
\par “还有吗?”
\par “还要你吻我一下。”
\par 于是我们一起接吻,一起哭,一起笑;我们的嘴唇都吻肿了。
\par “涅朵琦卡!首先,以后你每天到我榻上来睡。你喜欢ベーゼ不?我们可以在一起ベーゼ。其次,我不喜欢你愁眉苦脸的样子。你为什么老是闷闷不乐?你愿意告诉我吗?”
\par “我要把一切都告诉你;不过现在我不愁闷,我挺快活!”
\par “我一定要让你的腮帮子变得和我一样红喷喷的!啊,但愿明天快来吧!你困不困,涅朵琦卡?”
\par “不。”
\par “那我们就谈吧。”
\par 于是我们又扯了近两个钟头。天知道还有什么我们没谈到。先是小郡主向我介绍她对未来的全部设想和目前的情况。我了解到,她爱爸爸超过所有的人,几乎超过我。接下来我们一致认为,廖塔尔太太是个好人,她一点也不严厉。随后我们马上想好明天、后天做些什么,简直把今后二十年的生活都筹划好了。按照卡嘉出的点子,我们将这样过日子:一天由她向我发号施令,我样样照办,第二天倒过来——我发号施令,她百依百从,以后我们俩平分秋色指挥对方;将来会有人故意不服从命令,那时我们先吵一架做做姿态,然后赶快和解。总括为一句话:我们的未来无限幸福。最后,我们扯得疲倦了,我的眼皮儿渐渐撑不开来。卡嘉笑我贪睡,可她自己比我先睡着。次日早晨,我们同时醒来,匆匆接了个吻,因为有人要进我们屋里了,我赶紧回到自己床上。
\par 整整一天,我们高兴得不知如何相处是好。我们老是躲开所有的人,生怕被人发现。后来,我向她叙述自己的身世。卡嘉听了我的故事,震惊得流下了眼泪。
\par “你的心肠真够硬的!为什么你早不告诉我?我一定会非常非常爱你!那些男孩在街上把你打得疼不疼?”
\par “疼,我对他们怕极了!”
\par “哼,真可恶!告诉你,涅朵琦卡,我亲眼看到过一个男孩在街上打另一个男孩。明天我悄悄地带上对付福斯塔夫的短鞭子,要是遇上这样的坏孩子,我就狠狠地揍他,狠狠地揍他!”
\par 她的眼睛里射出愤怒的光芒。
\par 我们最怕有人走进来。我们唯恐在キス的时候被人撞见。而这一天我们ベーゼ至少有一百次。如此过了一天、两天,我担心会乐极而死,只感到幸福得喘不过气来。但是,我们的幸福没有持续多久。
\par 小郡主的一举一动廖塔尔太太都必须向公爵夫人报告。她对我们观察了整整三天,这三天里边她收集到许多值得汇报的情况。她终于把注意到的现象去向公爵夫人和盘托出:我们俩都处在极度亢奋的状态,已经有三天老是形影不离,不停地接吻,象疯子似地又哭又笑,象疯子似地谈个没完,而这是过去所没有的;廖塔尔太太不知道该把这一切归因于什么,但她觉得小郡主正经历着某种病态的危机,最后她认为减少我们见面的机会比较好。
\par “我早就这样想过,”公爵夫人答道,“我知道这个古怪的孤女会给我们招来麻烦。我所听说的有关她的情况和她过去的生活,简直不堪设想,太可怕了!她对卡嘉有明显的影响。您说,卡嘉很喜欢她?”
\par “可说爱得发疯。”
\par 公爵夫人懊恼得脸都红了。她已经在妒忌我,把我视为争夺她女儿的对手。
\par “这是反常的,”她说。“过去她们彼此合不来,坦白说,对此我却感到欣慰。尽管这孤女年纪还小,我还是极不放心。您明白我的意思吗?她从吃奶的时候起便接受了她的那一套教育、习惯乃至准则。我不懂,公爵认为她有什么可爱之处?我曾无数次建议把她送进寄宿学校。”
\par 廖塔尔太太本想为我辩护,但公爵夫人已决定让我们分离。当即派人去把卡嘉叫来,并且到了楼下才向她宣布,在下星期日以前我们不得见面,也就是整整一个星期不在一起。
\par 夜晚,我才得悉究竟,这简直是晴天霹雳;我在想卡嘉,我觉得她将无法忍受我们的分离。我忧伤过度,到夜里成了病;第二天上午,公爵前来探望,并悄悄地安慰我,说事情还有希望。公爵尽了最大的努力,但是没有结果,公爵夫人不肯收回成命。我渐渐陷于绝望,感到悲不自胜。
\par 第三天早晨,娜斯佳带给我一张卡嘉的字条。卡嘉用铅笔写得非常潦草,内容如下:
\par 我非常爱你。我坐在妈妈身边,老是想逃出去看你。不过,我一定能逃出去——我向你保证,所以你别哭。写信告诉我,你是多么爱我。我整夜都在梦中拥抱你,涅朵琦卡,我痛苦极了。给你捎去一颗糖。再见。
\par 我的回信也大同小异。我对着卡嘉的字条哭了一整天。廖塔尔太太竭力向我施加爱抚。晚上我得悉她去找了公爵,说要是不让我和卡嘉见面,我一定会第三次病倒,并为她向公爵夫人汇报了此事表示后悔。我向娜斯佳打听;卡嘉怎样了?她告诉我说,卡嘉不哭,但脸色惨白。
\par 早晨,娜斯佳悄悄地对我说:
\par “您到公爵书斋里去。从右边的扶梯下去。”
\par 一种预感使我周身的血液加速流动。我紧张地期待着跑下楼去,把书斋门打开。她不在里边。突然,卡嘉从背后把我搂住,热烈地吻了我一下。笑声、眼泪······。卡嘉一下子挣脱了我的怀抱,象一只松鼠爬上父亲的肩膀,但没有稳住,又从那里纵身跳到沙发上。公爵也跟着摔倒。小郡主快乐得哭了。
\par “爸爸,你真是个好人,爸爸!”
\par “你们这一对淘气鬼!你们是怎么搞的?这算什么友谊?这算什么爱情?”
\par “住口,爸爸,你不知道我们的事儿。”
\par 于是我们又互相拥抱。
\par 我开始从近处把她端详。三天来她瘦了。红喷喷的小脸蛋儿渐渐失去光泽,转为苍白。我一阵心酸,哭了起来。
\par 娜斯佳来敲门了。这是已经发现卡嘉不见并开始询问的暗号。卡嘉顿时脸色煞白。
\par “够了,孩子们。我们每天都可以碰头。再见吧,愿上帝赐福给你们!”公爵说。
\par 他瞧着我们也受到感动,但他的设想完全落了空。傍晚,从莫斯科传来信息,说小萨沙突然患病,业已奄奄一息。公爵夫人决定第二天就动身前往。由于事出仓猝,直到与小郡主告别之前我一无所知。告别一节是公爵坚持安排的,公爵夫人勉强答应了。小郡主肝肠欲断。我失魂落魄地跑到楼下,冲过去和她拥抱。远行的马车已等在大门口。卡嘉望着我大叫一声,顿时昏倒。我急忙吻她。公爵夫人设法使她恢复知觉。后来卡嘉总算苏醒过来,重又把我抱住。
\par “再见了,涅朵琦卡!”她忽然对我说,脸上莫名其妙地牵动了一下笑出声来。“你别在意;这不要紧,我没有病,我过一个月回来。那时我们再也不分离。”
\par “够了,”公爵夫人用平稳的语气说,“出发吧!”
\par 但小郡主再次回到我身边。她歇斯底里地把我搂在怀里。
\par “你是我的生命!”她抱着我匆匆低语道。“再见!”
\par 我们最后一次互相拥抱,随后小郡主便消失了——她这一次去了很久很久。直到八年以后我们才得重逢!
\par ··················
\par 我故意如此详细地叙述我的童年时代的这一插曲,即卡嘉在我生活中第一次出现的始末。我们的故事有着不可分割的联系。她的罗曼司就是我的罗曼司。正象我注定要遇见她一样,她也注定要找到我。再者,我不能放弃又一次神游我的童年时代的乐趣······。下面我要讲得快一些。我的生活忽然陷入一潭死水,直到我已经满了十六岁,我才仿佛重又醒来······
\par 先简单地交代一下公爵举家去莫斯科以后我的境遇。
\par 我和廖塔尔太太留在彼得堡。
\par 过了两个星期,专差来通知说,公爵一家回彼得堡要无限期推迟。廖塔尔太太由于家庭原因不能到莫斯科去,她在公爵家任职便告结束;但她仍留在这个家庭里,转隶公爵夫人的大女儿亚历山德拉·米海洛夫娜。
\par 我还没有提起过亚历山德拉·米海洛夫娜,其实我只见过她一回。她是公爵夫人与前夫所生的女儿。公爵夫人的出身和来历有些不明不白;她的前夫是个包税人。公爵夫人再嫁时,不知该如何安置她的大女儿。乘龙快婿非她所能企及。她可望得到的陪嫁并不丰厚;直到四年前,她总算嫁得一个富有而且官衔不小的人。亚历山德拉·米海洛夫娜进入了另一个社会,她在自己周围看到的是另一圈子的人。公爵夫人每年去看她两次,公爵——她的继父——每星期都带卡嘉去看她。但近来公爵夫人不大喜欢让卡嘉去姐姐那儿,公爵是偷偷带她去的。卡嘉十分爱她的异父姐姐。但她们俩在性格上彼此形成鲜明的对照。亚历山德拉·米海洛夫娜当时大约二十二岁,她娴静、温柔、多情;她的娟秀的容貌似乎被某种深藏的郁悒、内心的隐痛投下了严峻的阴影。严肃和阴郁同她天使般眉清目秀的容貌不大相称,犹如丧服穿在小孩身上。你望着她,不可能不对她产生深切的同情。她形容憔悴,我第一次见到她的时候,据说她有成痨的趋势。她过着十分孤僻的生活,既不喜欢宾客盈门,也不喜欢外出应酬,简直象个修女。当时她没有孩子。我记得,
\par 有一天她来找廖塔尔太太,见了我特地走过来深情地吻了我一下。与她同来的是一位清癯瘦削、上了年纪的男人。他瞧着我,禁不住潸然泪下。他就是小提琴家B。亚历山德拉·米海洛夫娜抱住我问,我愿不愿意住到她家去,做她的女儿?我朝她脸上一看,认出她是我的卡嘉的姐姐,便和她拥抱,心中感到一阵说不出的苦楚,致使我的整个胸膛都隐隐作痛······仿佛有人又一次在我身旁叹道;“苦命的孤儿!”这时,亚历山德拉·米海洛夫娜取出公爵的一封信给我看。信中有几行是写给我的,我读了以后泣不成声。公爵祝愿我幸福长寿.并要求我爱他的另一个女儿。卡嘉也给我写了几行,说她现在不能和母亲分开!
\par 当天晚上,我就开始另一种生活,来到另一户人家,进入另一个圈子,再次从心上割去已经使我感到如此可爱、如此亲近的一切。我来到她家已心力交瘁,精神上创巨痛深······。下面则是另一段故事的开端。
\newpage
\section*{六}
\par 我的生活进入一个风平浪静的新时期,仿佛我来到了世外桃源······。我在收养我的人家住了八年多,在整个这段时间内,除了屈指可数的几次以外,我不记得宅内举行过什么晚会、宴会或者至爱亲朋的聚会。这里只有两三个人偶尔来访,音乐家B则是这一家的老朋友,来找亚历山德拉·米海洛夫娜的丈夫的几乎都有事务,此外没有旁人到我们家来。亚历山德拉·米海洛夫娜的丈夫经常忙于事业和公务,难得闲暇便平均分配给家庭和社交生活。与显贵之间的不容忽视的纽带使他不得不经常在社会上出头露面。几乎到处都传播着有关他权欲熏心的流言;但出于他素来以一丝不苟著称,由于他据有十分显要的地位,而幸运好象老是自己在路上守候他,故所舆论远远谈不上使他失去人心。更有甚者,大家对他经常怀有一种特殊的感情,相反对他的妻子则毫无好感。亚历山德拉·米海洛夫娜生活在完全的孤独之中;但这对她却好象正中下怀。她的娴静的性格似乎生来就是为了过与世隔绝的生活。
\par 她把整个心都贴在我身上,象对亲生孩子一样爱我;虽则和卡嘉惜别的泪痕未干,心头的痛楚未已,我却如饥似渴地投入我的恩人慈爱的怀抱。此后,我对她炽热的爱从未间断。她充当着我的母亲、姐姐、朋友,填补着我所失去的一切,滋养着我的少女时代。加上我凭直觉、凭预感不久就发现,她的命运完全不象乍看起来那样美妙,尽管根据她表面闲适宁静的生活,根据表面的自由,根据常挂在她脸上的安详明朗的笑容可能产生这样的错觉,因此我在成长过程中对我的恩人的命运每天都有新的了解,我的心痛苦而缓慢地揣测的情况每天都有所澄清。随着惆怅的加深,我对她的眷恋也愈益增强。
\par 她的性格怯懦、软弱。望着她眉清目秀、从容自若的相貌,一时料想不到某种忧惧可能扰乱她磊落的胸襟。无法想象她可能不喜欢什么人;在她心中同情始终占据上风,甚至压倒憎恶;然而事实上她只有少数知己,生活十分孤寂······。她的本性热情而且善感,但同时又仿佛自己害怕自己获得的印象,好象每分钟都在监视自己的心,唯恐它忘其所以,甚至不让它耽于遐想。偶尔在最出神的时刻,我会意外地发现她眼睛里含着泪花,仿佛沉痛的回忆在她脑际倏地一闪,苦苦地折磨着她的良心,又仿佛有什么东西虎视眈眈地窥伺着她的幸福。似乎她愈是幸福,生活愈是处在无忧无虑的静谧时刻,哀伤就愈是迫近,悒郁和眼泪骤然出现的可能就愈大,这象是她经常发作的毛病。整整八年间,我记不起哪一个月她完全不发病。看来丈夫非常爱她,她对丈夫则简直近乎崇拜。然而乍看起来,他们之间好象有什么言未尽意之处。她的命运包含着什么秘密;至少我最初就开始怀疑······
\par 亚历山德拉·米海洛夫娜的丈夫第一次就给我留下阴郁的印象。这个印象形成于童年时代,以后再也磨灭不了。从外表看,他的身材瘦长,好象有意用一副绿色的大眼镜掩藏自己的目光。他性格内向,待人接物缺乏热情,甚至单独跟妻子在一起也找不到话说。他显然不喜欢与人交往。对我他毫不在意,每当晚上我们三人在亚历山德拉·米海洛夫娜客厅里聚首喝茶时,我总是因为有他在场而感到很不自在。我时而偷偷地对亚历山德拉·米海洛夫娜觑上一眼,往往忧伤地发现,她似乎也在仔细斟酌自己的一举一动,如果注意到丈夫的神态格外严厉阴沉,她就脸色煞白,或者突然间满面通红,仿佛从丈夫的某一句话里听到或揣度到责备的意味。我觉得她跟丈夫很难相处,可是看起来她又好象一刻也离不开丈夫。我惊诧于她对丈夫的异常尊敬,丈夫的一言一行她都十分重视,好象她要千方百计使丈夫满意,可又感觉到老是不能实现自己的愿望。她似乎在竭力博取丈夫的赞赏:看到丈夫脸上露出一丝笑意,听到半句亲切的话语——她立刻喜不自胜,简直象处在腼腆的、还不敢存奢望的爱情的最初阶段。她侍候丈夫犹同照料一个别扭的病人。我觉得,他总是带着某种怜悯的眼光看待亚历山德拉·米海洛夫娜,这使后者相当难堪。当他握过妻子的手回到自己书斋里去时,亚历山德拉·米海洛夫娜便整个儿变样。她的动作、谈话立刻趋于活跃、舒展。但每次同丈夫见面过后,一种惶惑的感受还会在她心中滞留很久。她当即开始回想刚才丈夫说了哪些话,逐字逐句加以玩味。她往往向我询问:彼得·亚历山德罗维奇是不是这样说的,她有没有听错?——仿佛在研究他说的话有无言外之意,直要到一小时以后,大概断定丈夫对她十分满意,她的忧虑纯属多余,这才完全振作起来。那时她一下子变得和善、快活、高兴,又是吻我,又是和我一起笑,或者坐到钢琴前面去即兴弹奏一两个钟点。但她的快乐常常会戛然而止,一下子转为哭泣,当我满怀忧虑、惶惑和恐惧望着她的时候,她立刻压低了嗓门,好象生怕被第三者听见似地向我解释,她的眼泪算不了什么,说她挺快活,叫我不用为她而苦恼。有时丈夫不在,她会一下子紧张起来,阢陧不安地询问丈夫的情况,派人去了解他在做什么,向使女打听为什么吩咐备车,他要到哪儿去,他是不是病了,他心情愉快还是闷闷不乐,他说了些什么,等等,等等。关于他的事业和公务,亚历山德拉·米海洛夫娜好象不敢主动跟他谈起。当丈夫劝她或要她做什么的时候,她总是恭听从命,那种诚惶诚恐之状活象是丈夫的奴隶。她十分喜欢丈夫称赞她,比如对某一件东西、一本书或者她的一件手工活表示赞赏。亚历山德拉·米海洛夫娜听了似乎颇为得意,旋即喜形于色。如果彼得·亚历山德罗维奇无意间想到跟两个婴孩亲热亲热(这是极其难得的),那时做妻子的喜悦更是无法估量。她的面容豁然开朗,喜上眉梢,逢到这样的时刻她甚至会在丈夫面前过分陶醉。例如,她甚至会大胆到这种程度:未经对方要求,她自己向彼得·亚历山德罗维奇提出(当然是畏畏缩缩、声音发颤地建议),请丈夫听一下她刚得到谱子的乐曲;或请他就某一本书谈谈自己的看法;甚或请允许把当天给她留下特殊印象的一部作品读两页给他听听。丈夫通常不愿拒绝满足她的愿望,甚至会向她现出俯就的浅笑,犹如在含笑姑息一个宠儿的怪癖,唯恐过早挫伤孩子的天真。然而,不知为什么,这淡淡的微笑、这居高临下的宽厚态度、这存在于他们之间的不平等现象,却深深地刺痛我的心;我保持沉默,克制自己,只是带着孩子气的好奇心和难免偏激的成见孜孜不息地观察他们。有时我注意到,彼得·亚历山德罗维奇不自觉地骤然一震,若有所悟,象是既不自然、又非自愿地蓦地想起一桩不堪回首而又无法忘却的事情;倏忽间,俯就的浅笑从他脸上消失,他的眼睛一下子凝视着惊慌失措的妻子,看到那种充满怜悯的目光,我禁不住打起寒战来;我现在意识到,如果这怜悯的对象是我,我一定深感痛苦。在这同一瞬间,喜悦也从亚历山德拉·米海洛夫娜脸上消失。音乐或朗读顿告中断。她脸色变白,但勉力支撑自己,默不作声。那是极不愉快的、难捱的时刻,这时刻也可能持续相当长久。最后由丈夫打破僵局。他好象勉强遏制内心的懊恼和激动离座起身,保持阴郁的沉默在室内走几个来回,握一下妻子的手,喟叹一声,显然出于无奈匆匆说几句似乎意在安慰妻子的话,然后走出房间,而亚历山德拉·米海洛夫娜不是泪如泉涌,就是长时间陷于凄惨的郁悒。丈夫晚上和她分手时,往往象对小孩子一般给她画十字祝福,亚历山德拉·米海洛夫娜总是含着眼泪、虔诚恭谨地接受他的祝福。但我忘不了有几个晚上亚历山德拉·米海洛夫娜一反常态的情形(八年中间在我们这个家里只发生过两三回,没有更多)。在她通常娴静的脸上,一贯的自卑和对丈夫的虔敬竟被愤怒所取代。风暴的酝酿几乎要一个小时;丈夫显得比平时更加沉默、更加严厉、更加阴郁。可怜的女人那颗伤痛的心仿佛终于忍无可忍。她用激动得断断续续的声音开始说话,先是结结巴巴,语无伦次,全是些影影绰绰的暗示和欲言又止的哀诉;尔后,她似乎不堪满腔悲苦的折磨,一下子涕泪迸流,放声大哭,接着则是愤怒、责备、抱怨、绝望的总爆发,——仿佛她陷入了一场病态的精神危机。那时值得一看的倒是她丈夫以惊人的耐心忍受这一切,以体贴的态度劝说她平静下来,吻她的双手,末了甚至陪着她也哭起来,于是亚历山德拉·米海洛夫娜会猛醒过来,好象良心对着她大喝一声,指控她罪无可逭。丈夫的眼泪震撼了她,她绝望地扭绞着双手,歇斯底里地哽咽着伏在丈夫脚边祈求宽恕并马上得到了宽恕。但她忍受良心的折磨、流着眼泪祈恕还会持续很久,有好几个月她将在丈夫面前显得更加胆怯,更加可怜。这些责难和埋怨的底细我莫名其妙,因为彼时我从房间里被支开了,而且方式总是极不巧妙的。但要完全瞒过我也不可能。通过观察我有所发现,有所料及,从一开始我心中便产生一个模糊的疑团,觉得有某种秘密笼罩着这些现象,觉得这颗受到伤害的心灵象火山一样突然爆发不单单是神经性的危机;丈夫老是面有愠色必有缘故,他同情可怜的神经衰弱的妻子象是假惺惺的姿态;妻子在丈夫面前老是畏畏缩缩、战战兢兢同样事出有因,她那种温良柔顺的爱甚至不敢向丈夫表示,岂不怪哉?这孤寂的环境,修道院式的生活,丈夫在场时她忽而通红、忽而死一般惨白的脸色——都不是无缘无故的。
\par 但他们夫妇之间爆发类似的危机次数极少,我们的生活十分单调,我对这种生活已经看得太真切了,再者,我在各方面成长得很快,许多新的感受在我身上已开始觉醒,尽管还不自觉,但分散了我进行观察的注意力,因而我最终也就习惯了这样的生活,习惯了这些现象,习惯了我周围的人物。当然,有时我看着亚历山德拉·米海洛夫娜,不能不沉入深思,但我的思索暂时还没有结果。我深深地爱着她,我尊重她的忧伤,故而不敢让我的好奇扰乱她善感的心。她理解我的用意,几次想要感谢我对她的眷恋之情。她注意到我的关切,往往含着眼泪强作微笑,自己取笑自己老是哭哭啼啼,聊以解嘲;或者没头没脑地开始向我述说,她十分满足,十分幸福,大家对她都很好,她迄今为止所知道的人个个都非常喜欢她,她深感苦恼的是彼得·亚历山德罗维奇终年为她忧戚,担心她精神上得不到安宁,其实正相反,她是那么幸福,那么幸福!······说到这里,她满怀深情把我搂住,脸上洋溢着慈爱,我的心只觉得一阵酸痛,这可以说是对她表同情的酸痛。
\par 她的形象在我的记忆中永远不会磨灭。她相貌端正,而憔悴和苍白似乎益发衬托出她静穆的美。郁郁勃勃的黑发朝下梳得光滑齐整,在颊腮的边缘投下浓重的阴影,给人一种庄重严肃的印象;但唯其如此,她那双孩子般明净碧蓝的大眼睛、柔和的目光、腼腆的微笑以及整个温顺的脸庞与之形成的对比才更加动人,她那苍白的容颜有时显得如此天真、胆怯、一无遮蔽,似乎对自己的每一种感受、心灵的每一阵冲动都害怕,既怕瞬息的欢乐,也怕经常的幽怨。不过,间或有幸福静谧的时刻,她那洞察心灵的眼神如同朗朗乾坤一般光明磊落;碧天如洗的双眸闪耀着挚爱的光辉,饱孕着仁慈的甘醇,洋溢着深切的同情并将这种同情倾注于禀性高尚、惹人怜爱、祈求哀悯的一切,——那时,你的整个心灵都会被她征服,情不自禁地向往着她,并且从她那里接过这份磊落、这份娴静、这份温顺和这份爱。人们有时也会这样看着碧空出神,愿意在甜蜜的遥望遐想中度过好多钟点,自己的心情在这样的时刻会觉得比较舒畅、平静,仿佛在灵魂的一泓清水中映出了庄严伟大的穹苍。如果强烈的感情染红了她的面颊,她的胸部因激动而起伏不停(这是经常发生的),那时她的眼睛会象闪电般发光,简直会迸溅出火花来,似乎她的整个心灵都移到眼睛里去了,而正是这颗心象保存圣洁的神火一般保存了真善美的感觉,也恰恰是真善美的感觉给了她精神的力量。在这样的时刻,她显得意气高昂。在这种突如其来的情感冲动下,文静、怕事的性情豁然开朗,一变而为高度的振奋、纯正的热忱,同时又蕴含着如许率真的稚气,如许无邪的信念,恐怕一位画家愿意付出半生的代价捕捉到这样热情奔放的时刻,把这张令人感佩的面庞搬上画布。
\par 我来到这户人家的最初几天就看出,亚历山德拉·米海洛夫娜在孤寂中对于我甚至相当欢迎。那时她还只有一个孩子,她才做了一年母亲。但我等于是她的女儿,她无法把我和她自己的孩子严格区分开来。她着手抚养我真是本着一副火热的心肠!开初她是那样急于求成,致使廖塔尔太太望着她忍俊不禁。的确,我们一下子什么都干起来,简直弄得彼此无法理解。比如,她亲自着手给我授课,可是一下子想教许许多多东西,结果在她这方面表现的狂热、激情和好心的急躁比我实际的得益更多。起先她为自己教导无方而苦恼,但我们大笑一通过后又重新开始,而且亚历山德拉·米海洛夫娜不顾最初受到的挫折,大胆声称她反对廖塔尔太太的教育方法。她们笑着进行辩论,但我的新老师斩钉截铁地宣布她反对任何教育体系,坚称她和我通过摸索定能找到正确的道路,说没有必要硬往我头脑里灌一些干巴巴的知识,认为成败的关键全在于能不能启我之蒙,善不善于唤醒我的良知,——她这话有道理,因为她正在取得全面的胜利。首先,从一开始就完全取消了师生名位之分。我们象一对朋友在一起学习,有时仿佛我在教亚历山德拉·米海洛夫娜而不觉其用心良苦。我们之间常常发生分歧,我为要证明自己的理解是对的,不遗余力地争得脸红脖子粗,亚历山德拉·米海洛夫娜则循循善诱地把我引上正途。最后,当我们得出正确的结论时,我马上恍然大悟,当即揭穿亚历山德拉·米海洛夫娜的花招,再考虑到她在我身上花费这么多精力以及为了使我得益往往连续几小时因势利导,我在每一堂课后忍不住搂住她的脖子和她紧紧拥抱。我的过敏的性格使她惊讶而又感动,甚至近乎不可理解。她好奇地开始询问,想从我口中了解我的过去,而每次听了我的故事,她对待我总是变得更体贴、更认真——我说认真,是因为我的悲惨的童年除了引起她的同情以外,好象还赢得某种尊敬。在我倾诉衷曲之后,我们照例要谈上很久,通过这样的长谈,她又向我解释我的过去,因此我实际上好象在重新经历往事,重新学到许多东西。廖塔尔太太经常认为这种谈话过于严肃,看到我情不自禁流下的眼泪,更觉得完全不合时宜。我的看法则相反,因为上了这样的课以后,我就觉得轻松愉快,好象我压根儿没有什么不幸的遭遇。此外还有一件事我对亚历山德拉·米海洛夫娜简直太感激了,那就是:她使我一天比一天愈来愈懂得自爱。从前,我幼小的心灵带着累累伤痕、阵阵痛楚,只会诉苦怨命而不懂得打击之所由来,以后就不公正地心肠变硬;廖塔尔太太何尝理解,正是在亚历山德拉·米海洛夫娜的不断启发下,先前反常而过早地从我心田里蓬勃发芽的感情,才得慢慢地趋向正常,达到和谐的平衡。
\par 每天第一桩事情总是我们俩一起来到育儿室里,把她的孩子弄醒,给他穿衣、收拾、喂奶,逗他,教他说话。我们把孩子的事料理妥帖以后,才坐下来上课。我们的课程内容很多,但这到底是什么科目,只有上帝知道。其中什么都有,可又什么都说不上。我们一起阅读,互道各自获得的印象,或者放下书本弄弄音乐,几个钟点就这样在不知不觉中飞逝。晚上B常常来,他是亚历山德拉·米海洛夫娜的好友,廖塔尔太太也来;那时往往开始十分热烈的谈话,议论艺术,议论在我们的圈子里只限于耳闻的生活,议论现实、理想、过去和未来,我们每每坐到午夜以后。我全神贯注地听着,和别人一起喜怒哀乐;正是从这样的夜谈中,我详细了解到有关我父亲和我孩提时期的种种情况。我逐渐长大了;尽管为我延师施教,但要是离开了亚历山德拉·米海洛夫娜,我一定什么也学不到。地理老师要我在地图上寻找城市河流,我简直象睁眼瞎子。和亚历山德拉·米海洛夫娜一起学地理,我们等于在周游列国,接触到无数珍闻奇观,度过无数光怪陆离、令人神往的时刻;我们双方都劲头十足,终于觉得她读过的那些书实在太少,不得不另外找一些书。不久,我可以自己把山川湖泊的位置指给我的地理老师看,尽管必须为他说句公道话,他在一点上始终保持住对我的优势,那就是:他能够精确无误地说出某个市镇的经纬度,说出那里的居民有几千几百乃至几十人。历史老师的束脩也总是按时奉送,从不拖欠;但他走后,我和亚历山德拉·米海洛夫娜却按我们自己的方式学习历史:我们自己找书本,有时直读到深夜,或者说得确切些是亚历山德拉·米海洛夫娜读给我听,因为她还负责检选内容。最使我感奋的事情莫过于这样的读书。我们俩意气风发,好象自己就是书中的英雄。当然,我们读夹缝文章比读白纸黑字的兴趣更浓;何况,亚历山德拉·米海洛夫娜讲得又生动,仿佛我们从书上读到的事情都是她亲身经历过的。我是一个孩子,她有着一颗受了伤害的心,忍受着生活的苦楚;我们这样如醉如狂地夜半共读而且乐此不疲,也许很可笑,但我不在乎!我知道,她等于在我身旁休息。记得我望着她,有时会奇怪地陷入沉思,默默猜想;在我真正开始生活之前,我已经能理解生活中许多现象。
\par 我终于满了十三岁。彼时,亚历山德拉·米海洛夫娜的健康则每况愈下。她变得更经不起刺激,她那种无处宣泄的阵发性忧郁愈来愈加剧,丈夫来看她的次数开始增多,他陪妻子坐的时间也愈来愈长,当然照例几乎一言不发,态度严峻,面色阴沉。她的命运更引起我的关切。我的童年时代已近尾声,在我的头脑里已形成许多新的印象、看法、爱好、猜想;不言而喻,存在于这个家庭之中的谜愈来愈使我苦恼。有时候我觉得自己对这个谜略知一二。有时候我又陷于冷淡、麻木甚至懊丧,由于任何一个疑问都找不到答案而忘却了自己的好奇心。也有一些时候——这种情况愈来愈频繁——我奇怪地感到需要独自一人静静地思索,不断地思索。我目前的状况有些象我还住在父母身边的时候,当时,在和父亲相遇之前,我有一整年老是在想,老是在思考,从自己的角落观察大千世界,结果,在自己心造的幻影中间变得十分古怪孤僻。差别在于目前我的心情更焦躁、更急切,更多新的不自觉的冲动,更想望活动和感受,因而我不能象过去那样把注意力集中在某一点上。就亚历山德拉·米海洛夫娜这方面说,她也好象在主动与我保持距离。我到了这样的年龄,几乎已经不可能做她的朋友。我不是小孩子,我问得太多,有时候竟看得她在我面前抬不起头来。往往有一些奇怪的时刻。我不忍看见她哭,我望着她,常常自己的眼泪即欲夺眶而出。我搂住她的脖子,和她热烈拥抱。可她又能怎样回答我呢?我感觉到自己只能加重她的精神负担。但另有一些时候——这是痛苦、郁悒的时刻——她自己会神经质地拚命把我搂紧,象是在向我寻求同情,因为她不堪忍受孤独的凄清,而我是了解她的,我和她同是天涯沦落人。然而,我们之间毕竟隔着一重秘密,这是明摆着的,所以我在这样的时候往往自己退到别处去。我在她身边觉得无所措手足。再者,我们之间已很少有连接双方的纽带,只有音乐。但医生近来告诫她不要弄音乐。至于同坐共读更成了最大的难题。我们肯定读不到第二页就会停止,因为每一个词也许都有所暗示,每一句无关紧要的话可能都是谜。如今两个人在没有第三者的情况下兴致勃勃地倾心交谈——已经是我们双方都竭力加以回避的事情。
\par 就在这个当儿,命运出人意料地把我的生活转了一个奇而又奇的弯子。我的注意力、我的感情、心灵、头脑以近乎狂热的劲头一下子转向另一个始料所未及的方面,我本人也在不知不觉中飞往一个新的世界;我无暇回头,无暇环顾,来不及深思;我可能走向毁灭,甚至感觉到这种危险;但诱惑压倒了恐惧,于是我闭着眼睛向前瞎闯,去碰碰运气。接下来我有很长一段时间要撇开我曾热中于寻找出路而不得、近来已开始感到厌烦的那个领域。下面我要说明这是怎么一回事,事情又是怎样发生的。
\par 饭厅共有三个出口:一个通大房间,第二个通我的房间和育儿室,第三个通藏书室。藏书室又有另一个通道,它与我的房间只隔一间文书室,通常在那里工作的是彼得·亚历山德罗维奇一名业务上的助手、录事、帮办、秘书兼factor【拉丁文:居间人。】。藏书室和书柜的钥匙由他保管。一天午后,他不在家,我在地板上发现一把钥匙。我在好奇心的驱使下,用这钥匙开了门走进藏书室。这是一间相当大的屋子,光线充足,四周八只大柜子摆满了书。书的数量很多,其中很大一部分是彼得·亚历山德罗维奇继承的遗产。另一部分是亚历山德拉·米海洛夫娜陆续购置的。在这以前,给我阅读的书都经过慎重挑选,所以我不难料到,有许多东西是不准我接触的,有许多东西对我来说是个秘密。因此,我怀着难以抑制的好奇心,伴随着一阵惊慌、高兴以及某种不可名状的特殊感觉打开第一只柜子,取出第一本书。这一柜全是小说。我拿了其中的一部,锁上柜子,把书带走时的心情异乎寻常,心一阵乱跳,一阵几乎停止,仿佛预感到我的生活面临着重大的转折。回到自己屋里,我把房间锁上,翻开那部小说。但我顾不上看书,我的心惦着另一件事:我先得设法牢固而稳妥地把藏书室控制在自己手中,既不让任何人知道,又始终不放弃取得任何一本书的可能。于是我把书放回原处,宁可推迟到适当的时刻再享受个中的乐趣,钥匙则藏了起来。这是我平生所做的第一件坏事。我静观事态的发展,结果非常顺利:彼得·亚历山德罗维奇的秘书兼帮办点了蜡烛在地板上寻找那把钥匙,找了整整一个晚上和上半夜,决定第二天上午叫一名锁匠来;锁匠从他带来的一串钥匙中间给另配了一把。事情就此了结,再也没有人提及丢失钥匙一节;我也做得特别谨慎,颇见心计,直到一个星期以后,确信绝对没有引起任何怀疑,我才到藏书室里去。起初我挑选秘书不在家的时候,后来就从饭厅进去,因为彼得·亚历山德罗维奇的录事仅仅把钥匙保管在自己口袋里,他本人和书籍从来不发生更进一步的接触,连藏书室里也不走进去。
\par 我如饥似渴地开始读书,不久便完全入了迷。我所有新的需求,所有前不久的意向,我这般年纪还相当模糊的欲望——这一切都是我的早熟所引起,本来在我心中不耐烦地蠢蠢欲动,如今一下子折往另一个始料所未及的方向,而且久不回头,仿佛获得了极其充足的新的食粮,仿佛找到了一条正确的道路。不久,我的心和头脑就给迷住了,我的想象有了纵情驰骋的广阔天地,以致我把迄今为止置身其中的整个世界统统丢在脑后。象是命运女神在我朝思暮想、心驰神往的新生活的门坎上把我拦住,先把我带到高山之巅,向我展示未来瑰奇的全景,指出灿烂诱人的前程,然后放我登上探索神秘世界的征途。我是注定了先从书上看到这幅前景,然后在幻想中,在希望中,在热情的迸发中,在一颗少女的心的甜蜜搏动中重新体会咀嚼。我不加选择地开始读书,拿到什么就读什么,但命运在保护着我:到那时为止我所知道和经历的事情都是光明正大、中规中矩的,我现在已经不会被书上若干不道德、不干净的章节所迷惑。以前,我受到儿童直觉、小小年纪和全部往事的保护。现在,意识似乎一下子为我照亮了我过去的全部生活。的确,我读的书几乎每一页都似曾相识,象是自己早就经历过的;通过千奇百怪的形态、瑰丽多姿的画面呈现在我眼前的生活以及书中人物的喜怒哀乐,好象我都感受过了。我怎么能够不给深深地吸引住以至于忘却目前的一切,几乎到了脱离现实的程度?我读过的每一本书都体现着同样的命运法则,同样的主宰人生的休咎精神,但又发端于某一主要的人生法则,它是吉凶、得失、祸福之所维系。凭着几乎是被某种自我保全的感觉激发起来的全部本能,我千方百计试图猜透的正是这一隐约存于我心的法则。我仿佛事先得到通知,仿佛有人告诫过我。仿佛有一股未卜先知的力量挤进我的心灵,使我一天比一天感到更有希望,虽则与此同时我也愈来愈急于冲进这未来,奔向这每天从书上读到的生活,因为它每天都以巨大的艺术魅力和迷人的盎然诗意使我眼花缭乱。但是,我已经说过,我的幻想远远走在我的焦躁的心情前头,说实话,我的大胆仅仅存在于想象之中,实际面对未来却裹足不前。因此,就象事先同自己达成了协议似的,我不自觉地决定暂时仅仅满足于幻想世界,因为在这个世界里唯我独尊,那里只有诗意,只有欢欣,至于悲哀和不幸即便有,也不占主要地位,只起必要的过渡作用,以加强先苦后甜的对比,好让命运来个急转弯,把我头脑里如火如荼的罗曼司引向幸福的结局。按照我现在的理解,我当时的情绪便是如此。
\par 这样的生活,充满幻想的生活,与我周围的现实迥然异趣的生活,居然持续达三年之久!
\par 这样的生活是我的秘密,而且过了整整三年我还是拿不准是否应该担心这个秘密一旦被揭穿。这三年间我得到的体会,对我说来实在太亲切、太贴心了。我自己在所有这些幻想中也印得太鲜明了,到后来,不管什么人的目光,倘若无意间射入我的心灵,都会引起我的惶惑和惊恐。再者,我们这一家人生活在修道院一般孤寂的状态,与社会几乎隔绝,因此我们每个人都不由自主地养成一种注意力集中于自身的习惯,一种自我封闭的需要。我的情况也是如此。这三年中,我周围的一切毫无变化,什么都还是老样子。我们依旧被笼罩在凄凉、单调的氛围中,如今想来,要不是我背地里热中于自己的秘密活动,这种气氛势必把我的心灵蠹蚀一空,最后迫使我不顾一切地冲出这死气沉沉的圈子,反叛的后果当在未定之天,也许是走向毁灭。廖塔尔太太年事已高,几乎不出自己的房门;两个孩子还太小,B又过于一成不变,而亚历山德拉·米海洛夫娜的丈夫仍然和以前一样道貌岸然,一样不假辞色。他们夫妇之间的关系依旧是那样神秘,只是在我心目中山雨欲来之势日益凶险,我愈来愈为亚历山德拉·米海洛夫娜忧虑。我眼看她的生命之火在悒悒寡欢、毫无特色的日子里行将熄灭。她的健康情况几乎一天比一天恶化。她的精神状态似乎终于陷入绝境,某种不可知的、可怕的力量压得她喘不过气来,她自己也说不上这究竟是怎么回事,反正把它当作命中注定的十字架驮到背上。在这样的隐痛折磨下,她的心逐步硬化;甚至她的神智也改变方向,渐趋阴郁。特别令我吃惊的是:我觉得,随着我的年岁的增长,她好象愈来愈跟我疏远,在对我的态度上甚至由躲躲闪闪转为不耐烦的恼恨。某些时候她简直不喜欢我,好象我妨碍着她。我说过自己开始有意识地离开她退到别处去,在这样做过一次以后,仿佛也染上了她这种隐秘乖僻的性情。因此,这三年来我所过的日子,这三年来在我的心灵、幻想、认识、希望和狂喜中所形成的东西,我都向她瞒得紧紧的。一旦彼此有所隐瞒,我们再也没有推诚相见,虽则我觉得自己一天比一天更加爱她。现在,我每次想起她过去多么疼我,多么慷慨地担起责任把蕴藏在自己心中的爱无保留地倾注在我身上,而且始终信守充当我母亲的诺言,总禁不住潸然泪下。诚然,自身的悲哀有时使她顾不上我,她会有好一阵子象是把我忘了,何况我也竭力不去干扰她,所以我简直是在无人觉察的情况下长到十六岁的。但在头脑清楚、能够比较清醒地环顾周遭的时刻,亚历山德拉·米海洛夫娜会突然为我着急起来,她会不耐烦地把我从我房间里叫出去,不管我在上课或在做什么事情,向我提出一大堆问题,好象在进行试探、盘诘,接着就整天不离开我,设法了解我的心思、愿望,显然在关心我的成长,关心我的现在和将来,本着无限深情和一片赤忱准备为我提供帮助。但她对我的了解已经大大脱节,故而有时候做得未免过于天真,在我看来也过于一目了然。例如在我快满十六岁的时候,她有一回把我的书一一翻遍,问我平时读些什么,及至发现我还没有跳出给十二岁的孩子阅读的书籍范围,她似乎大吃一惊。我料到是怎么回事,便留神注意她。接连两个星期,她似乎在对我进行预试,了解我的智力水平和精神需求。后来她下了决心,于是我桌子上出现了沃尔特·司各脱的《艾凡赫》,其实这本书我看了至少三遍。起初,她紧张地期待着,看我有何反应,并仔细推敲我的印象要得要不得,生怕越轨出格;过后,我们之间这种在我看来过于触目的别扭现象终告消除,我们两人的心都热乎起来,我为自己不必再瞒她而高兴得不得了!我们把这部小说读完以后,我简直把她乐坏了。在我们共读过程中,我发表的每一条意见都在点子上,每一点感想都不离谱。在她看来,我的智力已经发展得太快。她在惊喜之余,又欣然着手关心我的教育,——她再也不愿和我分开,但这不是她的意志所能左右。命运不久又把我们拆散,阻碍了我们的接近。其原因无非是她的旧病复发,她陷于悲苦的深渊,接着又是疏远、隐瞒、猜疑,也许甚至怀恨。
\par 但即使在热乎的时候,也有我们控制不住的短暂片刻。通过共读、交谈几句体己话、弄弄音乐——我们会忘其所以,间或甚至有些说过了头,事后我们彼此都有些尴尬。及至发现失了分寸,我们惊恐地面面相觑,目光带着多疑的好奇和怵惕。我们各自都规定了与对方接近时不可逾越的界限,哪怕想越过也不敢。
\par 一天傍晚,暮色苍茫,我在亚历山德拉·米海洛夫娜起坐室里心不在焉地读书。她坐在钢琴前根据她心爱的一支意大利音乐主题作即兴变奏。当她终于回到该咏叹调的原型旋律时,我被沁入我心坎的音乐吸引住了,不好意思地开始低声吟唱这支曲调。我哼着哼着,不久完全出了神,竟站起来走到钢琴旁边;亚历山德拉·米海洛夫娜似乎猜透了我的意思,开始由独奏改为伴奏,深情地注意让钢琴与我的歌声的每一个音符合拍。她似乎难以想象我的嗓音竟有如此丰富的色彩。在这以前,我从未当着她的面唱过,而且自己也不知道我究竟有没有才能。现在我们俩一下子都上了劲。我愈唱愈响;我从亚历山德拉·米海洛夫娜弹出的每一小节伴奏音型体会到她的惊喜心情在不断增强,这使我越发感到精力充沛,激情洋溢。这一曲唱得非常成功,真是声情激越,她兴冲冲抓住我的双手,感奋地看着我。
\par “安涅塔【涅朵琦卡的教名是安娜,安涅塔是安娜的法国化叫法。】!你有一副绝妙的嗓子,”她说。“我的上帝!我怎么一直没有发现!”
\par “我自己也是刚刚发现,”我得意忘形地回答。
\par “愿上帝赐福给你,我亲爱的孩子,我的无价之宝!快感谢上帝给了你这份天赋。谁知道······啊,我的天,我的上帝啊!”
\par 这意外的发现使她感奋异常,简直不知道对我说什么好,不知道该如何对我施加爱抚。我们之间久矣乎没有这样推心置腹、相互亲近了。一小时以后,家里的气氛简直跟过节差不多。当即派了人去请B。在等候他来的时候,我们信手打开另一本我较为熟悉的乐谱,开始唱另一首咏叹调。这一回我有些怯场。我不愿遭到挫折而破坏刚才的印象。但很快我的嗓音本身给了我鼓励和支持。我自己愈来愈惊诧于我的歌喉的力量,通过这第二次试唱,一切怀疑均告消除。喜不自胜的亚历山德拉·米海洛夫娜急忙派人去把自己的孩子叫来,甚至把孩子的保姆也叫来,最后,她的头脑完全发热了,竟自己去把丈夫从书斋里请来——在别的时候,她是连想也不敢这样想的。彼得·亚历山德罗维奇以体谅的态度听完这一新闻,向我表示祝贺,并主动宣布应该让我得到学习的机会。亚历山德拉·米海洛夫娜连忙吻丈夫的手,感同身受天大地大的恩典。后来B也到了。这位老音乐家高兴非凡。他很喜欢我,先是回忆昔日与我的父亲交往一场,在听我唱了两三次以后,他郑重其事地、甚至带着一些神秘的表情宣称,我无疑有相当的禀赋,甚至可能有才华,对我不加培养是不应该的。紧接着,大概经过一番思考之后,他和亚历山德拉·米海洛夫娜一致认为,一开始就过于夸奖我有危险,我注意到他们当即交换了一个眼色,暗中达成默契,其实他们瞒着我所玩的这套把戏十分幼稚,极不高明。后来,我又唱了一通,发现他们竭力作矜持状,甚至故意指出我的一些不足之处,我在心中暗暗笑了一个晚上。但他们这种姿态没有保持多久,B第一个沉不住气,又高兴得不亦乐乎。我万万想不到他竟如此爱我。晚上的谈话始终洋溢着友情的温暖。B介绍了几位著名歌唱家和演奏家的成名史,那种惺惺惜惺惺的景仰之情随时溢于言表。稍后,由于提起了我的父亲,话题便转到我身上,接着又谈到我的童年,谈到公爵和公爵的全家;自从和他们分别以后,我很少听到这一家人的消息。亚历山德拉·米海洛夫娜自己也知道得不多。最了解情况的是B,因为他曾去过莫斯科多次。但是谈话到此开始折往一个神秘的、我所不理解的方向,有两三个地方,尤其是涉及公爵的,我听得莫名其妙。亚历山德拉·米海洛夫娜提起了卡嘉,但是有关她的情况B却语焉不详,象是故意避而不谈。这使我大惑不解。我没有忘掉卡嘉,过去我对她的眷恋在我身上并未湮灭,不但如此,甚至相反,我从来没有想过卡嘉会起什么变化。迄今为止,我忽略了我们分手以来天各一方的岁月悠悠,忽略了我们没有互通任何音信,忽略了我们的性格和所受教育的不同。说到底,在我思想上卡嘉从未离开过我,她好象始终和我在一起,尤其在我的遐想中,在我向壁虚构的小说情节和荒诞奇遇中,我总是和她携手同行。我每读完一部小说,往往把自己想象成其中的女主人公,同时让小郡主充当我的好朋友,从而把这部小说一分为二,其中一部分当然是我的创造,虽则我常常无情地剽窃我心爱的几位作家。最后,我们的家庭会议决定为我延师教授声乐。B推荐了最好、最有名的一位。第二天就有一个意大利人Д来到我们家里;他听了我的唱以后发表的意见和他的朋友B不谋而合,但随即表示,如果我到他那里去同他的另外几个女学生一起上课,这对我来说好处会大得多,因为通过竞争可以取长补短,而且在那里我所需要的一切应有尽有,这些条件都有利于提高我的声乐素养。亚历山德拉·米海洛夫娜表示同意;从此,每周三次,上午八点,我由一名女仆陪同去音乐院上课。
\par 现在我要叙述一桩奇遇,它对我的影响实在太大了,并以急转弯的形式标志着我的年龄进入了一个新时期。当时我刚满十六岁,与此同时,我心中骤然出现一种不可理解的麻木状态;我自己也不知道为什么,一种难熬的、苦闷的寂寥占据了我的心房。我所有的美梦、所有的欲望顿时销声匿迹,连幻想的本领也仿佛因缺乏动力而告衰竭。冷漠取代了原先缺乏经验的热情。凡是我爱之极深的人一致公认的我的才具,也失去了我的好感,遭到我无情的鄙薄。我对什么都不感兴趣,甚至对亚历山德拉·米海洛夫娜也漠不关心,并为此而责备自己,因为我不能不承认这一点。间或插入我这种麻木状态的不是莫名其妙的伤感,便是突如其来的眼泪。我只想找机会一人独处。就在这个奇怪的时刻,一桩奇怪的事情震撼了我的整个心灵,并把这一潭死水化作狂风暴雨。我的心受到了伤害······事情是这样发生的——
\newpage
\section*{七}
\par 我走迸藏书室(这将是我终生难忘的时刻),拿了沃尔特·司各脱的小说《圣罗南之泉》,这是我还没有读过的唯一的小说。我记得,揪心的、无端的苦闷象某种预感折磨着我。我直想哭。夕阳的斜晖从高窗涌入,倾泻在闪闪发光的拼花地板上,把室内照得辉煌明亮。静悄悄。周围和邻近的房间里也阒无一人。彼得·亚历山德罗维奇不在家,亚历山德拉·米海洛夫娜则卧病在床。我真的哭了。我打开小说的第二部,一页页心不在焉地翻着,试图把在我眼前浮晃的片断字句连贯起来。我象在向书问卜,如同某些人随手翻开书页占吉凶那样。人生有时全部智慧力量和精神力量都处于反常的紧张状态,似乎即将点燃起自觉的熊熊烈焰;就在这瞬息之间,被震撼的心也许不堪对未来的预感的困扰,急于体味未来的酸甜苦辣,竟在梦中看到了先兆。您的整个躯体如饥似渴地向往着生活,心中燃烧着狂热和盲目的希望之火,仿佛在召唤未来,连带着召来它的秘密和未知数,哪怕连带着召来狂风暴雨也无妨,但一定要召来生活。我便是在这样的时刻走进藏书室的。
\par 记得我合上了那本书,正是为了随后闭着眼睛把它翻到无论哪一页,先在心中就自己的未来确定要占卜的事情,再读那一页书上写着些什么。可是,当我把书打开的时候,却看到一张写满了字的信笺。信笺折成本身四分之一那么大,压得扁而又扁,大概夹在书中被遗忘了已有好多年。我怀着极大的好奇心开始察看。这是一封没有上款的信,下款只有C.O两个起首字母,这引起了我加倍的注意。我把这张几乎粘在一起的信笺展开;由于年深月久,它在书页之间留下了与所折大小相等的一块颜色较淡的痕迹。纸张的折皱处磨损得很厉害;看来,当初这封信不知被读过多少遍,并当作宝贝珍藏起来。墨迹褪了色,由黑变青,——可见写信的日子在很久很久以前!若干字句偶然映入我的眼帘,我的心紧张地期待着。我把信拿在手里转来转去举棋不定,故意迟迟不开始读。我偶然把它放到向光处:果然!洒在字里行间的一滴滴眼泪干后还留有痕迹,有几处整个字母因泪浸而漫漶。这是谁的眼泪?我终于紧张地屏住气读了正面的上半页,一声惊呼从我胸中冲口而出。我把书放回原处,锁上柜子,把信藏在头巾下面,跑到自己屋里,锁上房门,重新开始读信。但我心里的鼓点打得实在急,只觉得一个个语词和字母在我眼前乱晃乱跳。有很长一段时间我什么也没有看懂。信的内容非比寻常,一开头就包含着秘密;它象闪电一般把我击中,因为我已知道这封信是写给谁的。我知道,如果读完这封信,我几乎等于犯了罪;但在这个时刻我已难以自持。信是写给亚历山德拉·米海洛夫娜的。
\par 下面我要援引信的本文。我模模糊糊弄懂了其中的涵义,事后久久摆脱不了谜底带来的忧思。从这一刻起,我的生活犹同发生了一大转折。我的心受到很大的震荡,很久很久、几乎永远不能恢复平静,因为这封信引起了一连串后果。这次我就未来问卜果然灵验。
\par 这是一封告别信,此后乃成绝响,令人不忍卒读;我看完以后,觉得心被死死地揪紧,仿佛我自己失去了一切,一切都从我身边永远被夺走了,除了再也不需要的生命,什么也没剩下。写这封信的人究竟是谁?亚历山德拉·米海洛夫娜以后是怎么过的?信中既有许多暗示和迹象足以作出正确的判断,又有许多费人疑猜的哑谜。但我所料几乎没有错误;再者,信的措辞也提供不少线索,透露了两颗心为之破碎的这一场相交的全部性质。写信人的思想感情溢于言表。这些思想感情太不寻常,而且,我已经说过,为解开谜底提供的线索也太多了。现在我把这封信逐字逐句抄录如下:
\par 你说过你忘不了我——我相信,今后我就全仗你这句话维持生命。我们必须分手,我们离别的时刻到了!我早知道会有今日,我的娴静美人,我的幽怨仙子,但我直到此刻才明白。在我们的这段时间内,在你爱我的这段时间内,我的心始终为我们的爱情隐隐作痛,而现在我却松了一口气,你能相信吗?我早知道会有这样的结局,这在我们之前就已注定!这是命运!亚历山德拉,听我把话说完:我们是不相称的;我一直有这个感觉,时刻不忘!我配不上你,我应该为我们尝到的幸福受罚,应该由我一人承当!你说:直到你垂青于我之前,我在你面前是个什么?天哪!已经两年过去了,可我至今还有点飘飘然;我至今不能理解,你竟会爱上我!我不明白我们怎么会接近到那种程度,不明白这是怎样开始的。你可记得,我和你相比算是个什么?我配得上你吗?我有哪一点可取,我有什么出类拔萃之处?在遇见你以前,我是个头脑简单的粗人,老是垂头丧气,愁眉不展。我并不向往另一种生活,思想上并不考虑,也不向它招手呼唤,甚至不愿这样做。我的一切都受到某种压抑,除了我的限期完成的日常工作,我不知道世上还有什么更重要的。我唯一需要关心的只是明天,其实对此我也相当淡然。过去,那是很久以前的事了,我也曾做过类似的梦,我也曾象个傻子似地想入非非。但从那以后已过了不知多少时间,我一直过着孤独、刻板、平静的生活,甚至不觉得我的心在严寒中变冷冻僵。它睡着了。我知道而且认定,对我来说永远不会升起另一个太阳;我相信这一点而且毫无怨言,因为我知道命该如此。当你打我身旁走过的时候,我不懂得我可以斗胆举目向你观看。在你面前我象是一名奴隶。在你的身旁我的心既不颤动,也不酸痛,它没有向我预报你的芳踪。我的心是平静的,它没有认出你的心来,虽则它在自己美好的姐妹近旁感到挺亮堂。我知道这一点,我隐约感觉到这一点。我之所以有此感觉,因为即使是最卑微的一颗小草也能分得苍天的霞光,同这颗依头顺脑、可怜巴巴的小草旁边艳丽的花朵一样承受晨曦的温暖和爱抚。在我了解一切之后,你该记得也就是在那天晚上以后,在那些话从根本上撼动了我的心灵以后,我被震惊得两眼发黑,晕头转向,竟至于不相信自己,也不理解你!这事我从未对你说过。你什么也不知道;过去我不是你遇见我时所看到的那个样子。倘若我能这样做,倘若我敢说的话,我早就向你和盘托出了。但我一直保持沉默,现在却要把一切都告诉你,好让你了解将要和你分手的是怎样一个人!你可知道,最初我是怎样理解你的?欲念象烈火烧遍我的全身,象毒药注入我的血液;它扰乱了我的思想和感觉,我如醉如痴,如堕五里雾中,没有以平等的身分,没有以无愧于你纯洁的爱情的姿态,而是不自觉地、无所用心地对待你光明磊落的怜爱。我没有理解你。我把你当作在我心目中忘了身分与我为伍的人,殊不知你是想把我提高到与你相齐。你可知道,我对你产生过什么样的怀疑?你可知道,忘了身分与我为伍意味着什么?不过,我不想通过我的自供来伤你的心,我只说一句:你把我大大估计错了!我决计不可能上升到与你相齐。我只能怀着无限的爱可望而不可即地凝视着你,那是在我了解你以后,然而这没有赎偿我的过失。我的经你抬举的欲念并不是爱情,——我害怕爱情;我不敢爱你;爱情应该是相互的、平等的,可是我不配······我简直不知道自己是怎么回事!哦!我该怎样向你说明这一点,怎样使你明白我的意思呢!······起先我根本不相信······哦!你可记得,当我最初的激动平静下来,我的目光恢复清明,心中只剩下一种最纯洁、最无邪的感情时,我的第一个反应是惊讶、尴尬、恐慌?你可记得,我突然放声大哭扑倒在你的脚下?你可记得,你又窘又怕地含着眼泪问过我怎么啦?我不则声,我无法回答你;但我的心却在裂成碎片;幸福象不堪承受的重负压迫着我,我的号哭仿佛在说:“凭什么?我凭什么赢得这份幸福?我凭什么得到这酬报?”我的姐妹,我的姐妹!哦!我曾经多次(这是你不知道的)偷偷地吻你的衣衫,我只能偷偷地吻,因为我自知配不上你;当时我屏住了气,我的心跳得慢而且沉,好象要停下来永远静止似的。当我握住你的手时,我浑身打颤,面如土色;你的光明磊落使我慌乱。啊,可惜我笨口拙舌没法把积在我心中急切向你诉说的一切尽情倾吐。你可知道,你对我始终如一的怜爱有时使我感到沉重、痛苦?当你吻我的时候(这有过一次,是我永志不忘的),——泪雾蒙住了我的眼睛,我的整个灵魂霎时间发出一阵呻吟。当时我为什么不死在你脚下呵?!此刻我在信上第一次称“你”,尽管你早就命我这样称呼。你可明白我要说什么?我要向你说出一切,而且现在就说出来:是的,你对我爱之甚深,你曾象姐妹爱兄弟一般爱我,你曾象爱自己的创作一般爱我,因为你复活了我的心,唤醒了我的智慧,把甜蜜的希望注入我的胸中;我却不能,也不敢;迄今为止,我从未把你叫做我的姐妹,因为我不可能做你的兄弟,因为我们是不相称的,因为你把我看错了!
\par 可是,你瞧,我写的都是你,即便在眼前这个可怕的灾难时刻,我想的也只是你,尽管我知道你在为我难受。哦,不要为我难受吧,我亲爱的朋友!你可知道,此时我是多么自己看不起我自己!这一切都公开了,议论已经沸沸扬扬!为了我,你将为人所不齿,你将遭到鄙视、嘲笑,因为在他们眼里我实在太卑微了!啊,一切都怨我配不上你!如果我在他们心目中多少有些价值,如果我能得到他们较多的尊敬,他们是会原谅你的!然而我卑微、轻贱、可笑,而卑贱之甚莫过于招人耻笑。不是有人叫嚷了吗?正因为这些人已叫嚷起来,我才泄了气;我素来软弱。你也许不知道我目前的处境:我自己在笑话自己,我觉得他们说的是事实,因为我连自己也觉得自己既可笑、又可恨。我有这样的感觉;我甚至憎恨自己的面貌、自己的体态,憎恨自己的习惯、自己的种种不良作风;这一切一贯为我所憎恨!哦,原谅我在穷途末路中如此野性毕露!你自己教我要把一切都告诉你。我毁了你,给你招来了嫉恨和讥笑,因为我配不上你。
\par 这个想法折磨着我;它在我头脑里敲个不停,撕扯着、伤害着我的心。我总觉得自己不是你爱的那个人,不是你以为从我身上找到的人,你把我看错了。这就是我感到痛苦的原因,也正是这一点此刻折磨着我,并将把我折磨致死或致疯!
\par 别了,别了!现在一切都已公开,我已经听到他们的喧嚷和议论;我在自己心目中也变得渺小、轻贱,为自己,甚至为你、为你的抉择感到羞愧,我诅咒我自己;现在,为了你的安宁,我必须销声匿迹。这是向我提出的要求,你将永远见不到我!这是必要的,是命中注定的!我得到的太多了;命运作了错误的安排;现在它要改正错误,索还一切。我们由相遇而相识,如今各奔东西,也许后会有期!然而在何处再会,在何时再见?哦,亲爱的,告诉我,我们将在哪里重逢,我怎样能找到你,怎样能认出你来?到那时你还认得我吗?你占据了我的整个心房。啊,为什么,究竟为什么如此对待我们?我们为什么要分离?请指点我,因为我不理解,我无法理解,百思不得其解;请你教我如何把生命撕成两半,如何把心从胸膛里挖出来,如何过没有心的日子?否则,叫我如何敢想起再也见不到你,永远见不到,永远!······
\par 天哪,他们嚷得多凶啊!我现在真为你担忧,太可怕了!我刚遇见你的丈夫;我和你都没法跟他比,虽则我们在他面前是无罪的。他全知道;他把我们看得很清楚;他全明白,他以前就洞若观火。他英勇地站到你一边;他能拯救你;他能卫护你抵御那些流言蜚语和大叫大嚷;他爱你,并且无限地尊重你;他是你的救星,而我却要逃跑!······我去找了他,我想吻他的手!······他叫我立即动身。事情已经决定!据说,为了你,他跟那些人全闹翻了;那里大家都反对你!人们指责他姑息纵容、优柔寡断。我的上帝啊!那里的人还说你什么?他们不知道,他们无法理解,不可能懂得!原谅吧,可怜的人,你就原谅他们吧。正象我原谅他们一样;他们从我这里拿走的比从你那里拿走的更多!
\par 我神思恍惚,不知道此刻在给你写些什么。昨天分手时我对你说了什么?我全忘了。当时我失魂落魄,而你在哭······宽恕我害得你流了这些眼泪!我太软弱、太懦怯了!
\par 我还想对你说几句话······哦呵!但愿能再一次把眼泪洒在你的手掌上,就象此刻洒在这封信上一样!但愿能再一次伏在你的脚下!你的感情是多么美好,他们哪里知道呵!然而他们都是瞎子;他们骄矜、傲慢;他们看不到,而且永远不会看到这一点。他们没长这样的眼睛!即便按他们的法典来衡量你也是清白无罪的,但他们不会相信,哪怕全世界发誓向他们担保都不会相信。他们怎能理解呵!难道他们真的会谴责你?哪一个将首先发难?啊,他们决不会手软的,他们必将群起而攻!他们之所以敢这样做,因为他们懂得该怎样做。他们将同声谴责,并说他们自己毫无过错,结果却是造孽!咳,其实他们对自己的所作所为并不自觉!可惜不能把真情无保留地告诉他们,让他们看一看,听一听,明白过来,相信事实!应该说,他们并不那样恶毒······此刻我走投无路,也许把他们说得太坏了!也许我以自己的恐惧在吓唬你!别害怕,不要怕他们,亲爱的!会有人谅解你的;说到底,有一个人已经对你谅解,你可以放心,那就是你的丈夫!
\par 别了,别了!我不感谢你!永别了!
\par  C.O.
\par 我惶惑之余,半天不明白自己发生了什么事情。我既震动,又惊慌。我已经过了三年轻松的空想生活,却在幻梦方酣之际跟现实撞了个满怀而措手不及。我惊恐地感觉到自己掌握着一大秘密,感觉到这个秘密牵制着我的整个存在······但怎么个牵制法呢?我自己还不了解。现在我不由自主地深深地卷入了迄今为止构成我周围整个世界的那些人的生活和关系,我为自己担心。作为一个与他们非亲非故的不速之客,我凭什么介入他们的生活?我将给他们带来什么?如此意外地把我同他人的秘密拴在一起的羁绊将如何解脱?谁知道?也许,我要扮演的这个新的角色对于我自己、对于他们都是痛苦的。可我又不能装聋作哑,不能拒演这个角色,不能把我知道了的事情永远埋在我的心底。我会发生什么情况?我将做些什么?说到底,我又知道了些什么呢?成千个还不清楚的、朦朦胧胧的疑团在我面前纷纷浮起,已经不耐烦地挤压着我的心。我茫然不知所措。
\par 记得随后另有若干时刻我产生了一些新的、奇怪的、迄今从未体验过的印象。我觉得自己胸中好象发生了某种变化,原先的郁悒顿时从心上消释,开始取而代之的是一种新的感受,对之我还不知道该如何作出反应——不知应该伤心还是高兴。此时此刻我象一个人行将永远离开自己的家,离开迄今为止一直安稳、静谧的生活,准备登程前往未经探索的远方,临行向自己周遭作最后一次环顾,思想上告别自己的过去,心中却被怅惘的预感平添几分悲苦,未知前途休咎如何,也许难关重重、险象环生。终于,神经质的号哭从我胸中迸发出来,通过一阵歇斯底里的发作解除了我心头的重压。我需要见人,需要听到别人的声音,需要跟人拥抱,抱得紧些,再紧些。现在我已不能、也不愿一人独处;我跑去找亚历山德拉·米海洛夫娜,整个晚上都和她在一起。我请求她不要弹琴,自己也不愿唱歌,虽然她再三要求。我忽然觉得一切都不胜负担,注意力在任何事情上都集中不起来。我恐怕和她一起哭了。我只记得我把她吓得非同小可。她劝我定一定神,不要惊慌。她忧心忡忡地注视着我,一再说我病了,说我不知道保养身体。最后,我心力交瘁地离开了她;我的神志迷离恍惚,上床时只觉得热一阵冷一阵。
\par 过了好几天,我才定下神来,对自己的处境有了比较清醒的认识。这时,我和亚历山德拉·米海洛夫娜过着无比孤寂的生活。彼得·亚历山德罗维奇不在彼得堡。他有事去莫斯科,在那里待了三个星期。虽然分别不久,亚历山德拉·米海洛夫娜却陷于可怕的忧伤。她间或显得较为平静,但一个人关在屋子里,可见我也是她的累赘。何况我自己也不愿见人。我的头脑以一种病态的紧张节奏工作着;我好象处在一团烟雾之中。有时我会长久陷入排遣不开的愁思;那时我仿佛梦见有人在暗中讥笑我,好象有什么东西潜入我的头脑,扰乱和毒化我的每一个想法。我摆脱不了无时无刻不在我眼前浮晃、使我不得安宁的痛苦幻象。在我想象中出现的是长时期无法解脱的痛楚、苦恼,是逆来顺受然而白白作出的牺牲。我觉得,这种牺牲所祭祀的对象在蔑视她、嘲弄她。我觉得,我看到了恶人宽恕好人罪孽的怪现象,这使我痛彻心肺!在这同时,我又竭力想打消自己的猜疑;我诅咒这种疑心,痛恨自己不该把并无确证而仅仅是一些猜想的预感当作定见,痛恨我自己向自己也不能证实我所得到的印象。
\par 尔后,我在头脑中反复玩味信上的字句、这一声声诀别的绝叫。我想象着那个不相称的人;我力图猜透“不相称”这个词儿的全部辛酸意味。这惨别的哀号使我心痛如绞。“我自己觉得可笑,为你的抉择感到羞愧。”这是怎么回事?这是怎样两个人?他们为什么忧伤,为什么苦恼,他们失去了什么?我勉强抑制紧张的情绪重又读那封信,其中充满了揪心的绝望,字句的涵义对我说来是那么奇怪、那么费解。信一再从我手中跌落,我的心情愈来愈惶惑,愈来愈激动······归根到底,这一切总得有个结局,可是我看不到出路何在,或者不敢正视它!
\par 我差不多要病倒了,就在这个当儿,某一天我们院子里响起了彼得·亚历山德罗维奇从莫斯科回来的马车声。亚历山德拉·米海洛夫娜欢呼着跑去迎接丈夫,我却站在原地动也不动。记得当时我自己也被自己的激动搅得心慌意乱。我忍不住跑到自己房间里去。我不明白为什么突然如此害怕,但我为这种害怕担忧。一刻钟以后我被叫去,要向我转交公爵的一封信。我在客厅里遇见与彼得·亚历山德罗维奇从莫斯科同来的一位陌生人,我从听到的只言片语得悉,他打算在我们这里住很久。他受公爵委托到彼得堡来料理公爵家族的一些要事(他家的事务很久以来一直是由彼得·亚历山德罗维奇经管的)。来客向我递交了公爵的信,并说公爵夫人也想给我写信,直到最后一分钟还声称信一定要写,但结果什么也没有交给他,只是请他转告,说她实在不知道给我写什么好,反正信上什么也写不清楚,说她浪费了五张信笺,后来统统撕成碎片,还说要互相通信非重新做朋友不可。公爵夫人托他转告,要我相信不久定将同她见面。那位陌生的先生回答了我焦急的提问,说即将会面的消息是真实的,公爵全家确乎不久就要启程来彼得堡。听到这个消息,我高兴得不知如何是好,急忙回到自己屋子里去,锁好房门,流着眼泪拆阅公爵的信。公爵向我许诺,不久我就可以见到他和卡嘉;他感情深厚地为我的音乐才华向我表示祝贺,最后祝福我前途光明,并答应妥善安排我的未来。我边读边哭,但在我欢欣的眼泪里掺和着难忍的哀愁,以致我记得当时为自己感到心惊胆战;我自己也不知道自己是怎么回事。
\par 过了几天。我隔壁那间屋子过去是彼得·亚历山德罗维奇的文书室,如今新来的那位客人每天上午在那里工作,晚上也往往工作到半夜。他和彼得·亚历山德罗维奇经常在后者的书斋里锁上房门一起工作。一天午后,亚历山德拉·米海洛夫娜要我到她丈夫书斋去走一趟,问他要不要和我们一起喝茶。我在书斋里没找到彼得·亚历山德罗维奇,估计他很快就会进来,我就站在那里等候。墙上挂着他的画像。记得我看到这幅画像,突然打了个寒噤,并开始怀着我自己也莫名其妙的激动心情仔细加以端详。画像挂得相当高,屋子里又相当暗,我便搬过一把椅子来站在上面,以便看得清楚些。我想寻找什么,仿佛指望发现澄清我的疑问的关键。记得当时首先使我震惊的是画像的一双眼睛。我当即一愣,因为我几乎从未见过此人的双目:他一直把双目隐藏在眼镜后面。
\par 我还在童年时代就不喜欢他的眼光,大概出于一种不可理解的、奇怪的偏见,但这种偏见现在似乎被证实了。我的想象犹如乐器定好了调。我忽然觉得:画像上的眼睛在我的逼视下尴尬地侧向一边,力图躲避我的目光;那双眼睛包藏着虚伪和欺骗。我觉得自己所料不差,一时竟闹不清这在我心中引起的是怎样一种隐秘的喜悦。一声轻轻的呼喊从我胸中迸发出来。这时我听到背后有窸窣之声。我回头一看:彼得·亚历山德罗维奇站在我的椅子跟前注视着我。我觉得他骤然红了脸。我窘得要命,赶紧从椅子上跳下来。
\par “您在此地做什么?”他厉声问道。“您到此地来有什么事?”
\par 我不知如何作答。稍定了定神,我勉强向他转达了亚历山德拉·米海洛夫娜的邀请。我不记得他向我回答了什么话,也不记得自己是怎样走出书斋的,反正到了亚历山德拉·米海洛夫娜那边,我把她等着的回音忘得一干二净,就胡乱说了一声他会来的。
\par “你怎么啦,涅朵琦卡?”她问。“你瞧瞧自己:脸涨得通红。你怎么啦?”
\par “我不知道······我走得太快了······”
\par “彼得·亚历山德罗维奇究竟对你说了些什么?”她惶惑地打断我的话。
\par 我不做声。这时响起了彼得·亚历山德罗维奇的脚步声,我立即走出屋子。我在极度的哀伤中足足等了两个小时。后来终于有人叫我去见亚历山德拉·米海洛夫娜。亚历山德拉·米海洛夫娜沉默不语,心事重重。我走进屋子,她很快地凝神瞧了我一眼,但随即低头垂目。我觉得她脸上现出一种窘迫的表情。不久我就发现她心境不佳,很少说话,对我看也不看;B关切地问她身体可好,她抱怨说头痛。彼得·亚历山德罗维奇却比往常健谈,但他只跟B交谈。
\par 亚历山德拉·米海洛夫娜心不在焉地走到钢琴旁边。
\par “给我们唱点儿什么吧,”B向我说。
\par “对,安涅塔,把你新练的一首咏叹调唱给我们听听吧,”亚历山德拉·米海洛夫娜附和道,她好象欢迎这个打破僵局的机会。
\par 我朝她一看;她望着我,不安地期待着。
\par 但我不善于克制自己。我没有走到钢琴边去唱点儿什么敷衍了事,只觉得心乱似麻,不知道如何推托过去;临了,我恼恨起来,干脆表示拒绝。
\par “你为什么不愿唱?”亚历山德拉·米海洛夫娜问,她意味深长地看了我一眼,同时又向丈夫投了一瞥。
\par 这两道视线使我忍无可忍。我在极度的懊丧中从桌旁站起来,但已经不加掩饰,一边烦躁、恼火得发抖,一边气呼呼地重申我不想唱,不能唱,身体不好。我这样说着横眉面对着大家,但上帝可以作证,这时我恨不能回到自己房间里去躲开所有的人。
\par B感到诧异,亚历山德拉·米海洛夫娜则现出明显的不悦,一句话也不说。但是彼得·亚历山德罗维奇突然离座起身,说他忘了办一件事,看来还因错过了原定的时间很不高兴,所以匆匆走了出去,临行表示待会儿他也许再来,但还是握了一下B的手作别。
\par “您究竟怎么啦?”B问。“看您的脸色确实象有病。”
\par “是的,我不大舒服,很不舒服,”我不耐烦地答道。
\par “的确,你脸色苍白,可刚才又是通红通红的,”亚历山德拉·米海洛夫娜说到这里突然顿住。
\par “够了!”说罢,我径向她走去,一边凝视着她的眼睛。可怜的亚历山德拉·米海洛夫娜顶不住我的目光,象犯了什么过错似地低下头来,苍白的两颊泛起一阵淡淡的红晕。我握住她的一只手吻了一下。亚历山德拉·米海洛夫娜望着我,立即现出并非做作的、天真的喜悦。“原谅我,今天我的行为十足是一个可恶的坏孩子,”我怀着深情对她说,“不过,我确实不舒服。请不要见怪,放我走吧······”
\par “我们大家都是孩子,”亚历山德拉·米海洛夫娜面露羞涩的微笑说,“我也是孩子,而且比你更坏,坏得多,”她向我俯耳添加了一句。“去吧,祝你健康。不过,看在上帝份上,不要生我的气。”
\par “为什么?”我问。如此天真的自供使我惊讶。
\par “为什么?”她惶恐地跟着我重复道,甚至象是为自己捏了一把汗。“为什么?你瞧,涅朵琦卡,我这个人也真是······我对你说什么来着?去吧!······你比我聪明······我还不如孩子。”
\par “别说了,”我深受感动,不知对她说什么好。我再一次吻了她以后,匆匆走出房间。
\par 我的懊丧和郁悒简直难以形容。此外,我还恨我自己,感觉到我太不谨慎,太不善于处事。我羞愧得直想哭,结果在深沉的忧思中入睡。早晨醒来,我立即产生一个念头:昨晚发生的事纯属海市蜃楼,我们无非在彼此故弄玄虚,捡到一些鸡毛蒜皮就急急忙忙当作惊天动地的大事,这一切都起因于我们在接受外来印象方面缺乏经验,不习惯。我觉得一切都应归咎于那封信,它太叫我心烦了,害得我头昏脑胀,于是我拿定主意,往后最好还是什么也不想,我以如此不寻常的轻易方式斩断了我的全部忧思,确信我能够同样轻易地照自己拿定的主意去做,心中平静了些,到动身去上声乐课时情绪已大大好转。早晨的空气使我的头脑彻底恢复清醒。我非常喜欢早晨向老师就教之行。八点以后,市内又重新活跃起来,煞有介事地开始一天的生活,这时候在市内行走是一大乐事。我们通常总是走最热闹、最繁忙的街道,我十分欣赏我的艺术生涯在这样的环境中开始,十分欣赏这样的对比:一方面是这种无足轻重、但活跃忙碌的日常生活,另一方面是近在咫尺的一幢大房子三楼上等待着我的艺术世界——不过,其他所有从上到下住满这幢楼房的人在我看来却又跟艺术毫不相干。我腋下夹着五线谱走在这些行色匆匆的路人之间;陪送我的老仆娜塔莉亚每次都在她自己不知道的情况下出一道题让我解答:她想得最多的是什么事情?还有,我那位一半意大利、一半法兰西血统的老师是个怪人,有时候热情很高,但绝大部分时间内古板得很,尤其啬刻异常,——这一切都挺有意思,能使我大笑,也能令我深思。此外,我虽则怯生生地、但怀着强烈的希望热爱我的艺术,大造空中楼阁,为自己设计最美妙的未来,在回家的路上往往被自己如火如荼的幻想烤得热烘烘的。总之一句话,在这些时间内我几乎是幸福的。
\par 这一回,我十点钟下课回家,也是处在这样的时刻。我记得当时兴冲冲地想一件什么事情出了神,把其余的一切都忘了。但我正要登上扶梯的时候,突然象有人烫了我似地吓得我直跳起来。从我的上方响起了彼得·亚历山德罗维奇的声音,这当儿他正好从扶梯上下来。一种强烈的不愉快的感觉攫住了我,昨天的事给我留下的回忆实在太可憎了,我竟怎么也无法掩饰自己的愤懑之情。我向他微微弯腰致意,但此刻我的面部表情想必直抒胸臆,致使他讶异地在我跟前停住脚步。我注意到他这一动作,自己顿时涨红了脸,急忙上楼。他目送着我嘀咕了几句,然后走他自己的路。
\par 我懊丧得快要哭了,不明白自己心中究竟是怎么一回事。上午余下的时间我坐立不安,不知道该下什么决心尽快了结和摆脱这一切。我曾千百次发誓要理智从事,千百次为自身感到恐慌。我觉得自己憎恨亚历山德拉·米海洛夫娜的丈夫,同时又为自己感到绝望。这一次,由于情绪不断剧烈波动,我真的恹恹成病,再也无法控制自己。我开始恼恨所有的人,上午一直坐在自己屋子里,甚至没有去见亚历山德拉·米海洛夫娜。后来她自己来了。她对我一看,几乎失声惊呼。我的脸色难看至极,一照镜子连我自己也吓坏了。亚历山德拉·米海洛夫娜陪我坐了整整一小时,象照料小孩一样地照看我。
\par 但是,她的关怀使我伤感,她的疼爱使我难堪,我实在不忍看她,最后只得请她让我安静一会。她走的时候对我很是放心不下。我的愤懑终于化为眼泪和一次晕厥。向晚,我稍感缓解······
\par 我稍感缓解是因为我下了决心去找她。我决定去跪倒在她面前,把她遗失的那封信交还给她,向她承认一切,包括我忍受的全部苦痛和我心生的种种怀疑,本着在我身上燃烧的爱——对她、对受苦者的无限深情——和她拥抱,对她说,我是她的孩子、她的朋友,我的心为她敞开,她可以往那里瞧一瞧,看看里边对她怀有多少最炽烈、最坚贞的感情。我的天哪!我知道,我感觉到,我是她能够倾心相告的最后一个人了,正因为如此,我认为得救的希望就更大,我的话也更具威力······尽管只是模模糊糊、朦朦胧胧的猜测,但我能理解她的郁闷;一想到她可能在我面前红脸,可能害怕受到我的审判,愤怒便在我心里沸腾······可怜的亚历山德拉·米海洛夫娜,这件事的罪人难道是你吗?这就是我将伏在她脚下哭着要对她说的话。正义感在我身上如激浪翻滚,我气愤已极。我不知道自己会采取什么行动,但直到事后我才清醒过来,那是在发生了一个意想不到的情况之后,它几乎在我刚迈出第一步时就把我拉住,使我和她免遭毁灭。当时可把我吓坏了。要不然,她那颗饱受摧残的心难道能重新点燃起希望之火?我差点儿一下子把她置于死地!
\par 事情的经过是这样的:我已经走到离亚历山德拉·米海洛夫娜的起坐室只有两间屋子的地方,这时彼得·亚历山德罗维奇从边门出来,走在我前头,但没有发现我。他也向亚历山德拉·米海洛夫娜的房间走去。我站住了一动不动;他是我此刻最不愿意碰见的人。我想走开,但好奇心突然把我象桩子一般钉在原地。
\par 他在镜子前面站停片刻,整整头发,最使我惊愕不置的是:我忽然听到他在哼一支歌曲。霎时间我童年时代一桩阴暗遥远的回忆在我的脑海中重新浮现。为了便于理解我在此刻产生的那种奇怪的感觉,我得把这桩回忆讲一下。在我刚来到这户人家的第一年,一件事曾深深地震动我的心灵,不过直到现在才使我恍然大悟,因为直到现在,直到此时此刻我才意识到,我对这个人如何会产生莫名其妙的反感!我已经提到过,还在那个时期,只要他在场,我总是有一种沉重的感觉。我已经说过,他那种皱眉蹙额、心事重重的神态,那种经常笼罩着愁云惨雾的面部表情,一向给我留下郁闷的印象;每次我们在亚历山德拉·米海洛夫娜张罗下一起喝过茶以后,我照例心情压抑;再者,我差不多曾经目击前面我谈起过的那两三次阴霾四布、疑云重重的冲突。当时同现在一样,我也是在这个地方、这个时刻遇见他,他也同现在一样到亚历山德拉·米海洛夫娜房间里去。我单独遇见他时总感到一种纯粹孩子气的羞怯,所以每次都象犯了什么过失似地躲在角落里,祈求命运别让他看见我。当时他也同现在一样站在镜子前面,一种非儿童的无名感觉促使我打了个寒战。我觉得他好象在化妆改容。至少我清楚地看到,他走近镜子之前脸上带笑;我看到了过去在他脸上从未看到过的笑容,因为(我记得这是最使我震惊的)他在亚历山德拉·米海洛夫娜面前从来不笑。突然,他刚朝镜子里一瞧,脸色顿时大变。笑容象接到禁令一般立刻消失,取代它的是无论作何等慷慨大度的努力还是情不自禁地从心中冒出来的苦恼,这种决非凡人所能掩饰的苦恼把他的嘴唇一撇,一阵神经质的痛感在他额上刻下一道道皱纹,把他的双眉挤在一起。目光深沉地藏到眼镜后面,——总之,才一眨眼的工夫,他象听到口令似地完全变成了另一个人。记得当时我一个小孩子由于害怕理解我所看见的现象而吓得直哆嗦,从此以后,一个非常不痛快的印象便留在我心里再也出不去。他照了约莫一分钟镜子,低下头来,拱起背脊,装出通常在亚历山德拉·米海洛夫娜面前显现的神态,蹑手蹑脚向她的起坐室走去。这就是使我震惊的一桩回忆。
\par 当时同现在一样,他也以为没有旁人,他也是站在这面镜子前。同那时一样,我怀着不友好、不愉快的心情与他狭路相逢。但我听到了这歌声,实在感到太意外了(要知道任何类似的举动在他身上都是绝对不可思议的,更何况是唱歌),竟在原地站住了呆若木鸡;与此同时,相似的局面又使我想起我童年时代几乎一模一样的一刹那,——那时,我无法表达我的心给怎样一种辛辣的印象猛戳了一下。我周身的神经都为之震颤,这不祥的歌声竟引起我纵声大笑,而倒楣的歌手也惊叫起来,从镜子前倒退两步,象一个被连赃拿获的罪犯面无人色地望着我,恐怖、惊愕、狂怒之状一览无余。他的目光激起了我病态的反应。我冲着他发出一阵神经质的、歇斯底里的狂笑,一路笑着从他旁边擦身而过,走进亚历山德拉·米海洛夫娜的起坐室后哈哈之声依旧不绝。我情知他站在帘幔后面,也许他在犹豫要不要进来,也许恼怒和胆怯把他在原地钉住了,——我抱着挑战性的焦躁情绪等待他作出决定;我敢打赌说他不会进来。果然给我猜中了,他过了半个小时才进来。亚历山德拉·米海洛夫娜在极度的诧异中对我看了半天,一再问我是怎么回事,可是毫无结果。我答不上来,一味笑得上气不接下气。最后,她明白我在发歇斯底里症,所以对我格外关注。我喘过一口气来以后,抓住她的双手连连吻着。这时我才改变主意,这时我才想到,要不是遇见她的丈夫,我简直会把她置于死地。现在她在我眼里等于是一个复活的人。
\par 彼得·亚历山德罗维奇走了进来。
\par 我向他瞥了一眼;他装做我们之间什么也没有发生的样子,照例是那样道貌岸然、阴沉冷漠。但是看他苍白的脸色和微微翕动的嘴唇,我猜到他在竭力掩饰内心的激动。他冷淡地招呼了亚历山德拉·米海洛夫娜,在老位子上默然坐下。当他拿起茶杯的时候,手在发颤。我预感到山雨欲来,一阵莫名的恐惧兜上心头。我本想离去,但不敢撇下望着丈夫自己脸上变色的亚历山德拉·米海洛夫娜。她也感觉到某种不祥的预兆。后来,我忧心忡忡地预测的情况终于发生了。
\par 在一阵深沉的静默中,我举目一看,正好遇上彼得·亚历山德罗维奇的一副眼镜直盯着我瞧。我猝不及防,不禁打了个寒战,差点儿喊出声来,急忙垂下双目。亚历山德拉·米海洛夫娜注意到了我的举动。
\par “您怎么啦?您为什么脸红?”我听到彼得·亚历山德罗维奇在问,他的声音刺耳,语气粗暴。
\par 我不吭声;我的心跳得使我一句话也说不出来。
\par “她为什么要脸红?为什么她动不动就脸红?”他放肆地指着我向亚历山德拉·米海洛夫娜问道。
\par 愤怒几乎使我气闭。我把恳求的目光投向亚历山德拉·米海洛夫娜。她会意了。她苍白的两颊顿时燃烧起来。
\par “安涅塔,”她用完全出乎我意料之外的坚定的声调对我说,“你到自己房间里去;我过一会去找你,晚上我们俩在一起过······”
\par “我在问您,您听见了我的话没有?”彼得·亚历山德罗维奇把嗓门拔得更高,根本不理会妻子在说话。“为什么您看到我就脸红?回答!”
\par “因为您迫使她脸红,同样也迫使我脸红,”亚历山德拉·米海洛夫娜激动得有些结结巴巴地答道。
\par 我惊异地望着亚历山德拉·米海洛夫娜。她居然顶得这样厉害,一开始我完全不能理解。
\par “我迫使您脸红,我?”彼得·亚历山德罗维奇应道,看来他也大惑不解,在我字上特别加强语气。“您是为了我脸红?难道我会迫使您为我脸红?难道您以为,应该您为我脸红而不是我为您脸红?”
\par 这句话对于我是如此明白无误,又是以如此狠心、刻毒的讥刺口吻说出来的,我吓得大叫一声扑向亚历山德拉·米海洛夫娜。她那死灰色的脸上反映出惊愕、痛苦、责备和恐怖。我望着彼得·亚历山德罗维奇,十指交叉作哀求状。看起来他自己也觉失了分寸,但促使他吐出这句话来的暴怒尚未过去。不过,他发现了我无声的哀求后,显得有些尴尬。我的手势说明他们之间迄今秘而不宣的事情有许多我已经了解,说明我清楚地懂得他话中的含义。
\par “安涅塔,您到自己房间里去,”亚历山德拉·米海洛夫娜从椅子上站起来,声音微弱、但是口气坚决地又说了一遍,“我要和彼得·亚历山德罗维奇谈谈······”
\par 表面上她相当镇定;但这种镇定比任何激动更叫我担心。我仿佛没有听见她的话,依旧待在原地一动也不动。我聚精会神地竭力想从她脸上看清此刻她内心的活动。我觉得,她既没有领会我的手势之所指,也没有明白我的叫喊的意思。
\par “这是您一手造成的,太太!”彼得·亚历山德罗维奇抓住我的手,指着妻子说。
\par 我的上帝啊!我从未见过此时在这张全无人色的脸上看到的那种绝望。他拉住我的手把我带出房间。我向他们看了最后的一眼。亚历山德拉·米海洛夫娜用胳膊肘撑在壁炉架上,双手紧紧捧住脑袋。她整个身体的姿势显示着难以忍受的痛苦。我抓住彼得·亚历山德罗维奇的一只手拚命地握着。
\par “看在上帝份上!看在上帝份上!”我气急败坏地说。“可怜可怜吧!”
\par “不用担心,不用担心!”他说,一边异样地看着我。“这是旧病复发,不要紧的。您走吧,走吧。”
\par 我走进自己的房间,倒在沙发上,双手捂住面孔。我在这种状态下足足度过三个小时,却等于走遍了整个地狱。后来,我终于沉不住气,便通过别人去问,我可不可以去见亚历山德拉·米海洛夫娜。回音是廖塔尔太太带来的。彼得·亚历山德罗维奇派她来说,一阵发作已过,目前没有危险,但是亚历山德拉·米海洛夫娜需要安静。我直到凌晨三点以前始终没有就寝,老是在想,一边在房间里来回走着。我的处境比任何时候更为耐人寻味,但我的心情却比较平静,也许因为我觉得自己的过失比谁都大。我躺下睡觉时巴不得明天早晨尽快来临。
\par 可是到了第二天,使我悲哀而又诧异的是:我发现亚历山德拉·米海洛夫娜的态度难以解释地冷淡。起初我以为,昨天我在无意间目击了她跟丈夫之间发生的这场冲突之后,这个心地纯洁高尚的女人不好意思和我待在一起。我知道,这个天真的孩子会在我面前赧颜,也会请求我原谅,因为昨天那不幸的一幕或许伤了我的心。但我旋即注意到她另有挂虑和烦恼,那是以极不自然的方式表现出来的:她向我答话时而生硬、冷淡,时而可以从她的话里体味到特殊的涵义,时而她又忽然对我亲热起来,似乎为那种不可能发自她内心的疾言厉色表示懊悔,她的亲切、温和的话语听来象是责备。后来,我直截了当地问她觉得怎么样,有没有话要对我说?我这样单刀直入的提问使她略感窘迫,但她马上举起一双娴静的大眼睛,现出温柔的笑容望着我,说:
\par “没什么,涅朵琦卡;不过我可以告诉你:你问得这样快,我有点儿心慌。这是因为你问得太快的缘故······请你相信。不过,你听着,我的孩子,我要你实话回答我:你有没有这样的心事,如果别人也这样猝不及防地向你问起这事,你同样会感到心慌?”
\par “没有,”我毫不含糊地正视着她答道。
\par “那很好!你恐怕想象不出,我的朋友,听到这样干脆的回答我是多么感激你。倒不是我怀疑你可能存什么歹念,——绝对不是!即使是一点点猜疑的影子,我也不会原谅自己的。不过,你听我说:我收养你的时候,你是个孩子;可现在你十七岁了。你自己也看到:我身体不好,我自己象个孩子,还需要别人照看。我不可能完全代替你的亲生母亲,虽然我心中对你的爱是绰绰有余的。如果说我为此而忧心忡忡,那当然不是你的过错,而是我的过错。原谅我刚才提的问题,原谅我也许出于无奈没能履行我从爸爸家里。把你接来时向你和他许下的全部诺言。现在我为这件事心中非常不安,过去我也时常放心不下,我的朋友。”
\par 我搂住她哭了起来。
\par “哦,感谢您,感谢您为我所做的一切!”我说着让眼泪洒在她的手上。“不要这样对我说话,不要撕碎我的心。您对我胜过母亲;你们二位——您和公爵——为我这个无依无靠的可怜的孤儿做了这么多,为了这一切,愿上帝赐福给你们!我可怜的、亲爱的亚历山德拉·米海洛夫娜!”
\par “好了,涅朵琦卡,好了!好好搂住我,对,就这样,搂紧些,更紧些!我想告诉你,天知道为什么缘故,我觉得你这是最后一次和我拥抱。”
\par “不,不,”我象个小孩子似地放声大哭道,“不,决不是这样!您会得到幸福的!······来日方长,相信我,我们会幸福的。”
\par “谢谢你,谢谢你这样爱我。现在我身边已没有多少人;大家都离开我了!”
\par “究竟哪些人离开您了?他们都是谁?”
\par “以前我周围也有另一些人;你不知道,涅朵琦卡。后来他们都离开了我,全走了,仿佛过去都是幽灵。我曾经巴巴地等待他们,等了整整一辈子;算了,让上帝保佑他们吧!你瞧,涅朵琦卡,秋已深,很快就要下雪;到下第一场雪的时候,我也要死了,——是的;但我并不伤感。别了!”
\par 她的脸苍白、憔悴;每一侧面颊上各有一块预兆不祥的血色红斑,她的嘴唇在哆嗦,而且给内热烤焦了。
\par 她走到钢琴前弹了几个和弦;这时,有一根弦嘣的一声断了,在一个长长的颤音中凄楚地呻吟······
\par “听到没有,涅朵琦卡,听到没有?”她忽然若有所悟地指着钢琴说。“这根弦绷得太紧,它经不起锤击,所以死了。你听,这临死的声音多么凄苦!”
\par 她费力地说着。脸上反映出她内心的隐痛,眼睛里挤满泪水。
\par “别谈这些吧,涅朵琦卡,我的朋友;够了;你去把孩子带来。”
\par 我把他们带到她屋里。她瞧着两个孩子,似乎缓过一口气来;一小时后让他们离去。
\par “安涅塔,我死了以后,你不会撇下他们不管吧?是不是?”她说得很轻,好象怕有人偷听我们的话。
\par “别说了,您把我的肠子都快绞断了!”我说不出旁的话来回答她。
\par “我不过是说说笑话,”沉默片时后她微微一笑说。“你就相信了?有时候天知道我会胡说些什么。现在我跟小孩子一样;对我什么都不能计较。”
\par 这时,她胆怯地望着我,好象有什么话不敢说出口。我等待着。
\par “你得注意别让他吓着,”她终于眼睛朝下、脸上微泛着红晕说,声音之轻我几乎听不清楚。
\par “谁?”我惊奇地问。
\par “我丈夫。你慢慢儿把一切都告诉他。”
\par “为什么,究竟为什么?”我愈来愈诧异地连声问道。
\par “也许,不告诉他也罢,反正这是将来的事!”她回答时尽可能调皮地看着我,不过,她的嘴唇上仍挂着憨直的微笑,脸上的红斑愈来愈扩大。“别谈这些了;我只是闹着玩儿的。”
\par 我的心给愈攥愈紧,愈来愈疼。
\par “不过,我死了以后,你会爱他们的,——是吗?”她认真地、重又带着有些神秘的表情附加说,“就象爱自己的亲生孩子一样,——你说是不是?要记住:我一向把你当亲生女儿看待,同自己的孩子不加区别。”
\par “是的,是的,”我答道,实在不知道说什么好,眼泪和窘困憋得我顺不过气来。
\par 我的手感到好象被什么烫了一下,还没来得及往回缩,就有一个火热的吻在上面燃烧起来。我惊讶得开不出口。
\par “她怎么啦?她在想什么?他们昨天发生了什么事?”一连串的疑问在我头脑里闪过。
\par 过了一会,她开始抱怨精神疲乏。
\par “我早就有病,只不过不愿吓着你们俩,”她说。“我知道你们俩都爱我,——难道不是吗?······再见,涅朵琦卡;你去吧,不过晚上一定得来看我。你来不?”
\par 我答应一定来,但这时急于离去。我再也无法忍受。
\par 可怜哪,真是可怜!你将被怎样的猜疑送入坟墓?你的心又在遭受怎样一种新愁的伤害和蛀蚀,而你连一个字也不敢提及?我的上帝啊!这长时间的苦痛现在我已经了如指掌。这是怎样一种暗无天日的生活,这是怎样一种羞涩的、无所需求的爱!甚至到了现在,几乎在临终的病榻上,心即将给痛苦撕成两半,她仍象犯了什么罪恶似地不敢有半句怨言,——刚刚在想象中臆造出一种新的哀愁,她便已经向它认命,安之若素!······
\par 傍晚,乘莫斯科来的奥弗罗夫不在,我走进暮霭沉沉的藏书室,打开柜子开始在书堆里翻寻,想挑一本书念给亚历山德拉·米海洛夫娜听。我想给她解解闷,所以准备挑一本内容轻松愉快的······我翻了很久,可是思想不太集中。暮色愈来愈浓;我的忧愁也愈来愈深。我手中又出现了司各脱的《圣罗南之泉》,我把它翻到可以清晰地看到信的痕迹的那一页上;那封信从此一直藏在我怀里,这个秘密标志着我的一生的转折和新的起始,它向我透出一阵阵寒气,还包含着那么多神秘莫测、凶多吉少而且从现在即已对我虎视眈眈的未知数······“我们将来会怎么样呢?”我在想。“我在那里一直感到温暖舒适的一隅将变成空屋子!守护着我的青春期的纯洁、光辉的天使即将离我而去。往后怎么办?”我出神地站着回味此刻显得如此可爱的往昔,力图看透那未知的、威胁着我的未来······现在我回忆起当时在藏书室里的心情,犹如此时此刻的感受,可见它在我的记忆中刻得有多深。
\par 我手里拿着那封信和翻开的书,眼泪湿透了我的面庞。突然,我吓得打了个寒噤:我身旁响起了一个熟悉的声音。与此同时,我感觉到信已从我手中被夺走。我惊呼一声回过头来:在我面前站着彼得·亚历山德罗维奇。他紧紧抓住我的一只手不让我动弹;他用右手将信移至向光处,力图看清最初的几行······我叫喊起来;我宁死也不愿让信落到他的手里。从那得意洋洋的笑容我看得出,他已经弄清楚头几行字句的意思。我正在失去理智······
\par 转瞬间,我几乎昏头昏脑地向他扑去,把信从他手中夺了过来。这一切发生得如此迅速,我简直连自己也还没弄清楚,信怎样又回到了我的手中。但我发现他又想第二次来夺,我急忙把信藏在怀里,倒退三步。
\par 我们四目对视有半分钟之久。我惊魂未定,还在发抖;他面色灰白,颤动不已的嘴唇气得发青。
\par “够了!”他首先打破沉默,不过由于激动而声音微弱。“您想必不希望我使用武力吧;还是自愿把信交给我的好。”
\par 我这才恢复思维的活动,屈辱、羞愧、对暴力的愤慨使我呼吸急促。热泪顺着我发烫的腮颊直淌。我激动得浑身发抖,半晌说不出一句话。
\par “您听见没有?”他说着向我逼近两步······
\par “不许碰我!”我叫喊着竭力躲开他。“您的行为是卑下的、不道德的。您忘了自己的身份!······放我走!······”
\par “怎么?这是什么意思?您干了这样的事······居然还敢如此放肆······拿出来,我命令您!”
\par 他又向我跨近一步,但是朝我望了一下之后,看到我眼睛里充满了决心,不得不停下来考虑考虑。
\par “好!”最后他干巴巴地说,象是作出了决定,但还在勉强克制自己。“这件事慢慢再说,我先问您······”
\par 说到这里,他向四周看了一下。
\par “是谁让您进藏书室来的?为什么这柜子开着?您哪来的钥匙?”
\par “我不回答您,”我说,“我不能告诉您。放我走,放我走!”
\par 我向门口走去。
\par “且慢,”他一把拉住我的手说,“您不能这样一走了之!”
\par 我默默地使劲抽出被他拉住的一只手,重又向门口走去。
\par “好哇。但是我决不允许您在我家里接受您的情人的书信······”
\par 我惊骇地大叫一声,望着他茫无所措。
\par “因此······”
\par “住口!”我喝道。“您怎么可以?······您怎么能对我说这样的话?······我的上帝啊!我的上帝!”
\par “什么?什么?!您还想威胁我?”
\par 我在绝望中面如土色地望着他。我们之间的这场冲突达到了我无法理解的最激烈的程度。我通过眼神恳求他不要再继续下去。我可以不计较对我的侮辱,只要他停止施加压力。他注视着我,看来思想有些动摇。
\par “不要把我逼急了,”我在惊恐中喃喃地说。
\par “不,这种状况必须结束!”他考虑过后终于说。“我不妨向您承认,您的眼神一度使我动摇,”他面露奇怪的笑容附加这几句。“但遗憾的是事情本身很说明问题。我已经读了信的开头。这是一封情书。您不可能说服我改变看法!不,您休想!如果说,我曾经产生短时间的怀疑,这只能证明,在您所有的美好品质中间,我必须添上一条出色的撒谎本领,因此我重申······”
\par 他愈是往下说,便愈是因恼恨而变得面目全非。他的脸呈灰白色,嘴唇扭曲而且发颤,因此最后那几句话他说得很吃力。天在黑下来。我孤立无援地面对着一个不惜欺侮女子的人。何况一切迹象都对我不利;我羞愧难当,茫然若失,无法理解此人为何这样狠心。我没答理他,在极大的恐怖中失魂落魄地冲出藏书室,直至站在亚历山德拉·米海洛夫娜的起坐室门口才定下神来。紧接着,他的脚步声也从后面传来;我已经准备走进去,突然象遭到雷击似地止步不前。
\par “她会怎样呢?”这个念头在我脑中一闪。“这封信!······不,任何后果也比往她心中捅这最后一刀来得好些,”于是我又往回跑。但是来不及了:他已经站在我身旁。
\par “到哪儿去都可以,就是不要在此地,此地不行!”我抓住他的一只手低声说。“饶了她吧!我愿意再到藏书室或者······任何地方都可以!否则您会要了她的命!”
\par “您才会要她的命!”他答道,一边想把我推开。
\par 我的一切希望都成了泡影。我感觉到他恰恰要把这一场冲突搬演到亚历山德拉·米海洛夫娜面前去。
\par “看在上帝份上!”我说着拚命拉住他。但就在这个当儿,门帘掀开了,亚历山德拉·米海洛夫娜已出现在我们面前。她愕然望着我们。她的脸色比平时更加苍白。她几乎站也站不住。看得出,她是听到我们的声音后费了极大的力气才走到门口来的。
\par “谁在这里?你们在这里谈什么?”她极度骇异地望着我们问。
\par 冷场持续了若干秒钟,亚历山德拉·米海洛夫娜的面容惨白如纸。我扑过去紧紧抱住她,把她扶回到屋里去。彼得·亚历山德罗维奇跟在我后面走进来。我把自己的脸埋在亚历山德拉·米海洛夫娜胸前,愈来愈紧地搂住她,屏息待变。
\par “你怎么啦,你们是怎么回事?”亚历山德拉·米海洛夫娜再一次问。
\par “您去问她。昨天您还那样护着她,”彼得·亚历山德罗维奇说着在一把圈椅里颓然坐下。
\par 我把亚历山德拉·米海洛夫娜愈来愈紧地搂在自己怀里。
\par “可是,我的老天爷,这究竟是怎么回事?”她惊骇异常地问。“您在发那么大的脾气;她又吓得眼泪汪汪。安涅塔,你告诉我:你们到底是怎么回事?”
\par “不,请让我先讲,”说着,彼得·亚历山德罗维奇走到我们跟前,抓住我的手把我从亚历山德拉·米海洛夫娜身边拉开。“您站在这里,”他指着房间中央说。“我要当着替代您母亲的人来审判您。”然后,他让亚历山德拉·米海洛夫娜在圈椅里坐好,向她说:“您先坐下来,不要激动。我痛心的是不得不让您参加这一番不愉快的质询;但质询是必要的。”
\par “我的天!你们到底要做什么?”亚历山德拉·米海洛夫娜一边说,一边无限郁闷地时而看看我,时而望望丈夫。我扭绞着双手,预感到致命的时刻迫在眉睫。我已经不指望他发善心。
\par “总之,”彼得·亚历山德罗维奇往下说,“我希望您和我一起来处理。您一向(我不明白原因何在,反正这是您的一个怪癖),您一向认为,例如昨天您还说过······我简直不知道该怎么讲;一想起那些凭空的猜测,我就感到脸红······总之,您护着她,您非难我,指责我过于严厉;您还暗示说我这种过分的严厉乃是另一种感情引起的;您······我不明白,为什么一想起您的凭空臆测,我总是抑制不住汗颜、羞赧;我不明白,为什么我不能当着她的面把这些臆测公开讲出来······总而言之,您······”
\par “哦,您不能这样做!不,您不能讲出来!”亚历山德拉·米海洛夫娜叫了起来,她激动得胸部起伏不已,羞得无地自容。“您饶了她吧。这都是我胡思乱想的结果!现在我一点也不怀疑。原谅我的臆测,请原谅。我有病,对我必须原谅,可不要对她讲,不要讲······安涅塔,”她走过来对我说,“安涅塔,你不要待在这儿,走吧,快走!他是在开玩笑;都怪我不好;这是失了分寸的玩笑······”
\par “总而言之,您为我吃她的醋,”彼得·亚历山德罗维奇不顾她忧虑地阻之再三,竟狠心地抛出这句话来。亚历山德拉·米海洛夫娜大叫一声,面色煞白,好容易扶住一把圈椅才没有倒下。
\par “求上帝宽恕您!”过了半晌她才用微弱的声音说。“我代替他请你宽恕,涅朵琦卡,原谅我;一切都怪我。我有病,我······”
\par “但这是专横、无耻、卑劣的行为!”我狂怒地喊道,这下我全明白了,原来他要在妻子心目中败坏我的名誉是为了这个缘故。“这太可鄙了;您······”
\par “安涅塔!”亚历山德拉·米海洛夫娜嚷着满怀恐怖抓住我的双手。
\par “荒唐至极!纯粹是一出闹剧!”彼得·亚历山德罗维奇说,一边激动得难以形容地向我们走近。“我对您说,这纯粹是一出闹剧,”他面露狞笑注视着妻子,一面继续说,“在整个这出闹剧中受愚弄的只有您一个人。请相信,我们,”他气咻咻地指着我说,“并不害怕这样的对质;请相信,我们已不是那么天真无邪的嫩面皮,要是对我们谈起这等事来,我们已不会见怪,不会脸红,不会捂住耳朵。对不起,我是实话实说,也许流于粗俗,但不得不如此。太太,您可确信这位······小姐的品行是端方正派的?”
\par “上帝啊!您是怎么啦?您太放肆了!”亚历山德拉·米海洛夫娜吓得目瞪口呆,手脚冰冷。
\par “请不要用过头的字眼!”彼得·亚历山德罗维奇不屑地打断她的话。“我不爱这一套。目下这件事简单、明了、庸俗到了极点。我在问您关于她的品行的情况;您是否知道······”
\par 但我不再让他说下去,我抓住他的双手,使劲把他拖到一边。只差短短一分钟的工夫,可能一切都完了。
\par “不要提起信的事!”我说得很快,声音很轻。“您会当场把她置于死地的。对我的责难同时也就是对她的责难。她不可能对我进行审判,因为我全知道······您懂吗,我一切都了解!”
\par 他以无比好奇的目光注视着我,一时竟惶惑起来;只见血往他脸上直涌。
\par “我一切都了解,一切!”我再说一遍。
\par 他还拿不定主意。一句问话已经到了他微微翕动的嘴边,可是我赶在他前头,没让他说出来。
\par “是这么回事,”我急忙向正用忧惧的目光愕然望着我们的亚历山德拉·米海洛夫娜说。“都是我不好。我已经瞒了您四年。我拿走了藏书室的钥匙,四年来一直在偷偷地读那里的书。彼得·亚历山德罗维奇正好撞见我在读一本书······这本书按说不可能、不应该落到我手中。他为我大起恐慌,便在您眼里故意夸大危险!······但我并不打算为自己辩护,”我注意到彼得·亚历山德罗维奇嘴角浮起一丝讥笑,赶紧说,“因为一切都是我的过失。我抵挡不住诱惑,一旦做了坏事,又不好意思承认错误······您问我们之间出了什么事情,这就是全部经过,其他几乎什么也没有了······”
\par “哦呵,好利索!”彼得·亚历山德罗维奇在我身旁低声说了一句。
\par 亚历山德拉·米海洛夫娜十分注意地听我说完;但是她脸上明显地反映出不相信的态度。她看看我,又看看丈夫。三个人谁也不作声。我勉强喘过气来。她低头垂到胸前,一只手捂住眼睛作思考状,显然在玩味我说的每一句话。最后,她抬起头来凝视着我。
\par “涅朵琦卡,我的孩子,我知道你不会撒谎,”她说。“事情是否就是这些,确实就是这些?”
\par “就是这些,”我答道。
\par “真的就是这些?”她转而问丈夫。
\par “是的,就是这些,”他勉强回答说,“就是这些!”
\par 我松了口气。
\par “你愿意向我保证吗,涅朵琦卡?”
\par “我保证,”我不假思索地回答。
\par 但我忍不住朝彼得·亚历山德罗维奇瞥了一眼。他听我作出保证后笑了。我刷地红了脸,我的窘态没有躲过可怜的亚历山德拉·米海洛夫娜的眼睛。她脸上现出令人透不过气来的忧伤。
\par “好吧,”她无可奈何地说。“我相信你们。我不可能不相信你们。”
\par “我认为这番自供已经足够了,”彼得·亚历山德罗维奇说。“您听见没有?不知您有何感想?”
\par 亚历山德拉·米海洛夫娜避而不答。眼前这局面愈来愈使人感到难堪。
\par “我明天就把所有的书都检查一遍,”彼得·亚历山德罗维奇继续说。“我不知道那里还有些什么;不过······”
\par “她读的是一本什么书?”亚历山德拉·米海洛夫娜问。
\par “什么书?您自己回答,”彼得·亚历山德罗维奇对我说。“您交代情况比我清楚,”他暗暗带着讥诮的口吻说。
\par 我窘得一句话也说不出来。亚历山德拉·米海洛夫娜涨红了脸,目光下垂。冷场持续良久。彼得·亚历山德罗维奇悻悻然在屋子里踱来踱去。
\par “我不知道你们到底是怎么回事,”亚历山德拉·米海洛夫娜不好意思地咬着每一个字眼打破静默,“不过,假如事情果真就是这样,”她继续说,力图使自己的话具有一种特殊的涵义,但在丈夫定睛谛视下已经心慌意乱,虽然自己尽可能不去看他,“假如事情就是这样,那末,我认为我们没有理由大家都这样灰心丧气。我的过错比谁都大,都怨我,我为此十分内疚。我忽视了对她的教育,一切都应由我负责。我应当请她原谅,我不能、也不敢责难她。但是,我再说一遍,我们又何必懊丧呢?危险过去了。您瞧她,”她愈说愈上劲,一边把探究的目光投向丈夫,“您瞧她:难道她失于检点的行为留下了什么后果不成?难道我对她——我的孩子、我心爱的女儿——还不了解?难道我不了解,她的心地纯洁而且高尚,在这颗招人心疼的小脑袋里,”她把我贴在自己身边爱抚着继续说,“有着清楚的思路、明晰的智慧,而良心是不敢骗人的······够了,我亲爱的!这件事就到此为止吧!很可能,在我们的忧闷中间隐藏着别的什么因素;也许,芥蒂的阴影只是在短时间内落到我们身上。让我们用深情、和谐来驱散阴影,消除我们的误会。也许,我们之间还有许多话没有谈透,这首先要归咎于我。我最先把心事瞒着你们,我最先产生种种荒唐的猜疑,其实都是我的头脑出了毛病造成的。不过······不过,既然我们已经部分消除了误会,你们俩都应该原谅我,因为······因为,归根到底,我的猜疑也算不得太大的罪过······”
\par 说完,她红着脸怯生生望着丈夫,怏怏然等他开口。在彼得·亚历山德罗维奇听妻子说话的过程中,他嘴唇上渐渐显露嘲弄的笑意。他停止踱步,在妻子正对面站住,双手抄在背后。他似乎在端详、观察、欣赏妻子的窘态;亚历山德拉·米海洛夫娜感觉到丈夫凝神俯视着她,心里着了慌。彼得·亚历山德罗维奇静候片刻,好象在等她往下说。她越发慌了。最后,彼得·亚历山德罗维奇以一阵轻微、持续的冷笑打破难堪的局面:
\par “我觉得您实在可惜,可怜的女人!”他住了笑,终于痛心而严肃地说。“您扮演了一个不能胜任的角色。您想干什么?您想挑动我回答,想用新的猜疑来刺激我,其实应该说还是旧的猜疑,您刚才那番话甚至没有好好加以掩饰,难道不是吗?您那番话的意思是,不应该生她的气,她在读了不道德的书以后仍然是好的,我却认为那些书恐怕已经结出了恶果;说到底,您准备亲自替她作保;是不是这样?作了这番解释之后,您暗示说有别的什么因素;您认为我的怀疑和追逼出于另一种什么感情。您昨天甚至向我暗示——请不要阻止我,我喜欢实话实说——您昨天甚至暗示,说有些人(我记得,根据您的见解,这些人多半是庄重、严厉、直率、聪明、坚强的,反正您慷慨起来什么形容词都舍得用上!),说有些人的爱往往通过怀疑和追逼表现出来(天晓得您为什么要臆造这种怪论!),说他们的态度也恰恰是严峻、暴躁、生硬的。这是不是您昨天的原话,我已经记不太清楚······请不要打断我;我对您教养出来的这位小姐相当了解;什么话她都听得,我第一百次向您重申,对她什么都不用忌讳。您被愚弄了。但我不知道您为什么坚持认为我正是这样的人?天知道为什么您非要我穿上这丑角的戏装不可!我的年龄已不适宜同这位小姐谈情说爱;况且,归根到底,请相信我,太太,我知道自己的义务,不管您何等慷慨大度地原谅我,我仍要说过去说过的话:无论您把卑下的情操捧到多么崇高的程度,罪恶终究是罪恶,劣迹终究是可耻、卑鄙、不光彩的劣迹!够了!够了!但愿再也不要让我听到这些混帐话!”
\par 亚历山德拉·米海洛夫娜哭了。
\par “让我来受这份罪吧,我愿意承担!”她哽咽着抱住我说。“就算我的猜疑是不光彩的,应该遭到您如此无情的嘲弄!可是你,我可怜的孩子,为什么要被罚听这样的辱骂?我又无法保护你!我没有讲话的余地!天哪!我不能沉默,先生!我受不了······您的行为悖情逆理!······”
\par “算了吧,算了吧!”我轻声说,竭力使她激动的心情平静下来,唯恐彼得·亚历山德罗维奇老羞成怒。我还在战战兢兢地为她担忧。
\par “可是,盲目的女人!”他吼叫起来,“您不知道,您看不到······”
\par 他顿了一下。
\par “离开她!”他冲我说,同时想把我被亚历山德拉·米海洛夫娜握住的一只手硬抽出来。“我不许您碰我的妻子;您会玷污她的;您待在此地对她是一种侮辱!其实······其实我何必保持沉默?我应当讲,必须讲!”他一跺脚咆哮道。“我要讲,我要把一切都讲出来。您说您了解一切,我不知道您指的是什么,小姐,我不知道、也不想知道您打算用什么来恫吓我。听着!”他转向亚历山德拉·米海洛夫娜继续说,“您听着。”
\par “住口!”我大叫着向前扑去。“住口,不许说一个字!”
\par “听着!······”
\par “住口,凭着······”
\par “凭着什么,小姐?”他向我看了迅速而尖利的一眼,把我的话头截住。“凭着什么?您听着,”他重又面向亚历山德拉·米海洛夫娜,“我从她手中夺下了情人写给她的一封信!瞧我们家里在发生什么事情!瞧您身边在发生什么事情!这才是您没有看到、没有发觉的事情!”
\par 我险些昏倒在地。亚历山德拉·米海洛夫娜面如土色。
\par “这不可能,”她喃喃地说,声音几乎听不出来。
\par “我看见了那封信,太太;我还拿到过;我读了开头几行,果然不出所料:是情人写给她的。她从我手中把信抢了过去。现在信在她身边,——这是清清楚楚的,确实如此,毫无疑问;如果您还不信,不妨对她瞧瞧,然后再试试看还能不能存半点怀疑的希望。”
\par “涅朵琦卡!”亚历山德拉·米海洛夫娜叫道,身体向我扑过来。“不,你别说,别说!我不知道这回事,我不知道怎么会有这种事······我的上帝,我的上帝!”
\par 她双手掩面号啕大哭。
\par “不!这不可能!”她又喊叫起来。“您弄错了。这······这意味着什么,我知道!”她说,同时目不转睛地望着丈夫,“你们······我······不能,你决不会骗我,你不可能欺骗我!把一切都告诉我,什么也不要隐瞒;他弄错了,是不是?他错了,难道不是吗?他看到的不是情书,他眼花了!是不是这样?是不是这样?喂,安涅塔,我的孩子,我的亲骨肉,为什么不把一切都告诉我?”
\par “回答呀,快回答!”这是站在我身旁的彼得·亚历山德罗维奇的声音。“您回答:我看见您手里拿着的是不是一封信?”
\par “是的!”我激动得气咻咻地答道。
\par “那是不是您的情人写的信?”
\par “是的!”我回答说。
\par “您跟他现在还保持往来?”
\par “是的,是的,是的!”我已经置一切于不顾,对任何问题都给予肯定的回答,唯求结束我们所受的折磨。
\par “她的回答您该听见了吧?那末,现在您还有什么话说?咳,您这颗善良而过于轻信的心哪,相信我吧,”他握住妻子的手说,“相信我吧,把您病态的想象产生的种种错觉统统抛开。现在您看到了,这位······小姐是怎样一个人。我只想证明您的猜疑是毫无根据的。我对此早就有所察觉,我高兴的是终于当着您撕下了她的面具。过去,看到她在您身边,在您的怀抱里,和我们坐在同一张桌旁,看到她待在我们家里,我一直不以为然。您的盲目性使我愤慨。正因为如此,也仅仅因为如此,我才注意她,对她留神;这引起了您的猜疑,您从这种荒谬绝伦的疑心出发,开始编织更加荒谬绝伦的臆想。但现在局面已经澄清,任何疑问都不再是疑问,所以明天,小姐,明天您就不在我家了!”末了几句他是冲我说的。
\par “且住!”亚历山德拉·米海洛夫娜说着从椅子上撑起来。“你们演的这出戏我完全不信。您不必这样恶狠狠瞪着我,也不必嘲笑我。我同样吁请您来参加对我的评审。安涅塔,我的孩子,到我这儿来,把你的手给我,对。我们都有罪过!”她用哭得发颤的声音说,眼睛温顺地望着丈夫。“试问,我们之中谁能甩掉谁的手?安涅塔,我心爱的孩子,把你的手给我;我不见得比你高明,不见得比你好;您待在此地不可能构成对我的侮辱,因为我也一样,也是罪人。”
\par “太太!”彼得·亚历山德罗维奇愕然叫道。“太太!您控制一下自己!不要忘其所以!······”
\par “我什么也不会忘记。请不要打断我,让我把话说完。您看到她手里拿着一封信;您甚至读了信上的字句;您说,而且她也······承认这信是她所爱的人写的。但这难道可以证明她有罪?难道凭这一点您就可以当着您的妻子的面如此对待她,如此欺负她?难道您对这件事已判断清楚?难道您了解事情的真相?”
\par “看来我还得落荒而逃,还得求她宽恕。您是不是要我这样做?”彼得·亚历山德罗维奇暴跳如雷。“听着您的话,我实在耐不住了!您好好想一想您在说些什么!您可知道您在说些什么?您可知道:您在卫护什么,您庇护的是什么人?我可是什么都看得透亮······”
\par “您连最起码的事情也看不到,因为愤怒和傲慢妨碍着您的视力。您看不到我在卫护什么和我要说什么。我并不包庇邪恶。您有没有断定,——一旦断定了,就会清楚地看到,——您有没有断定,也许她象婴儿一样纯洁无邪?是的,我并不包庇邪恶!我急于作保留性的声明,可能这是您十分乐于听的。是的;假如她是个有夫之妇,是有了孩子的母亲,却忘了自己的义务,哦,那我会同意您的看法······您瞧,我的话是有所保留的。请您注意这一点,不要责备我!但事实是,她在不知什么是邪恶的情况下收到了那封信:事实是,她陶醉于缺乏经验的感情,又没有人及时拉住她;事实是,我应当负最大的责任,因为我没有照看好她的心;事实是,这是第一封信;事实是,您粗暴地怀疑她,从而伤害了她散布着芳香的处女感情;事实是,您对那封信肆无忌惮地妄下断语,从而玷污了她的想象;事实是,您看不到闪耀在她脸上的那种白璧无瑕的、童贞的羞恶,我现在却能看到,刚才她走投无路,不知道说什么好,无可奈何地承认您狠毒地逼问的每一件事的时候,我也看到了;是的,是的!您刚才的质问太狠毒、太残忍了;我简直无法相信是您干的;这件事我决不原谅您,永远也不!”
\par “您就饶了我吧,可怜可怜我吧!”我紧紧抱着她大声说。“请您发发慈悲,相信吧,不要把我推开······”
\par 我跪倒在她面前。
\par “归根结蒂,事实是,”她喘吁吁地说下去,“当时我不在她身旁,您用话吓昏了她,于是这可怜的孩子自己也相信她犯了过错,事实是您搅昏了她的良知,扰乱了她的灵魂,破坏了她内心的平静······天哪!您竟要把她从这个家里撵走!您可知道,这是对待什么人的做法?您可知道,如果您把她撵走,那等于把我们俩一起撵走,就是把我也撵走。您听见没有,先生?”
\par 她双目炯炯,胸部起伏不已,精神的亢奋已达到危机的最后阶段。
\par “我已经听够了,太太!”彼得·亚历山德罗维奇终于高声嚷道。“够了!我知道有柏拉图式的感情——不幸的是我对此深有体会,太太,您听见没有?但是,太太,我不能同镀金的邪恶和睦相处!我不懂这一套。必须把漂亮的伪装剥去!如果您自觉有错,如果您问心有愧(这无须我提醒您,太太),说到底,如果您愿意离开我的家······那末,我只须对您说,只须提醒您一点:您不应该忘记实现您的意图,那也是在这个季节,也是在这个时候,此事距今已有······如果您忘记了,我可以提醒您······”
\par 我望着亚历山德拉·米海洛夫娜。她用痉挛的手靠在我身上,悲伤耗竭了她的精力,眼睛在无限的痛苦中半开半闭。她在顷刻之间就可能倒下来。
\par “哦,看在上帝份上,您就饶了她这一回吧!不要把话说绝了,”我喊叫着跪倒在彼得·亚历山德罗维奇面前,竟忘了这是违背我自己意愿的。但已经晚了。我的话音未落,只听得一声微弱的叫喊,可怜的亚历山德拉·米海洛夫娜倒在地板上人事不省。
\par “完了!是您害死了她!”我说。“快去叫人来救她!我在您的书斋里等候您。我有话对您说;我把一切都告诉您······”
\par “什么事?究竟什么事?”
\par “回头再说!”
\par 晕厥和发病持续了两个小时。全宅上下惊慌非凡。大夫没有把握地频频摇头。两小时以后,我走进彼得·亚历山德罗维奇的书斋。他刚从妻子那里回来,在屋子里走来走去,把指甲咬得几乎出血,脸色苍白,心烦意乱。我从未见过他这样的神态。
\par “您有什么事要告诉我?”他生硬地厉声问。“您不是有话要说吗?”
\par “这就是您曾经从我手中夺走的那封信。您还认得出来吗?”
\par “是的。”
\par “您拿去吧。”
\par 他接过信把它凑到灯光下。我留神注视着他。几分钟后,他很快地翻到第四页上【前文提到此信只有一张纸,大概每一面分成左右两半,四页相当于报纸的四版。】,读了信末的署名。我看得出,血一下子直往他头脑里涌上来。
\par “这是什么?”他问我,模样象是发了呆,可见惊讶到什么程度。
\par “三年前我在一本书里发现了这封信。我料想是别人遗忘在那里的,读了以后才知悉一切。从此,信一直由我保存着,因为我没法把它交给任何人。我不能把信交给她。至于您,您不可能不了解此信的内容,这一悲惨的故事全在其中······您为什么要这样装腔作势——我不知道。这对我来说暂时还是个谜。我还不能看透您黑暗的灵魂深处。您想保持凌驾于她之上的优势地位,您确实做到了。但为了什么呢?就为了压倒一个女病人紊乱的想象中的幻影,就为了向她证明:她迷失了方向,您比她来得正经!您达到了目的,因为她的猜疑无非是理智衰竭的头脑里的固定观念,也许是一颗破碎的心对人间判决不公发出的最后的怨诉,而您也参与作出了判决。‘您爱上我,这究竟坏了什么事?’这就是她要说的,这就是她要向您证明的道理。您的虚荣心、您的极端利己主义太残忍了。再见吧!您不必作任何解释!但请您注意,我完全了解您,我把您看透了,这一点可不要忘记!”
\par 我好象处在迷迷糊糊的状态走进自己的房间。彼得·亚历山德罗维奇的帮办奥弗罗夫在门口把我叫住。
\par “我想跟您谈谈,”他彬彬有礼地点了点头说。
\par 我看着他,几乎不明白他对我说的是什么。
\par “以后再说吧,请原谅,我身体不大舒服,”最后我答道,并打他身旁走过去。
\par “那就明天吧,”他鞠躬退去,说时脸上现出微妙的笑意。
\par 不过,这可能是我的错觉。这一切仿佛在我眼前浮掠过去。
\end{document}