\documentclass[12pt, UTF8]{ctexbook}
\usepackage{geometry}
\usepackage[all, PDF]{xy}
\usepackage{indentfirst}
\usepackage{graphicx}
\usepackage{subfigure}
\usepackage{xcolor}
\usepackage{soulutf8}

% subtitle
% Неточка Незванова
\newcommand\subtitle[1]{{\normalsize #1}} %为了保险,最好使用两层大括号

% use russian for author name
\usepackage[OT2,T1]{fontenc}

\setlength{\parindent}{2em}

\geometry{a4paper,centering,scale==0.75, left=2.5cm,right=2cm,top=2.54cm,bottom=2.54cm}

\begin{document}
\title{涅朵奇卡·涅茨瓦诺娃 \\ \subtitle{\fontencoding{OT2}\selectfont Netoqka Nezvanova}}
\author{陀思妥耶夫斯基 \\ \small\fontencoding{OT2}\selectfont F\"edor Miha\u{i}luvhq Dostoevski\u{i}}
\maketitle
\newpage

\section*{一}
\par 我记忆中没有我的生父的印象。他死的时候我才两岁。我母亲又嫁了别人。这次再醮给她带来了很多痛苦,尽管她改嫁是出于爱情。我的继父是个乐师。他的命运很不寻常:这是我认识的人中间最古怪、最奇特的一个。在我童年时代最初的印象中,他留下的痕迹太深刻了,这对我一生都有影响。为了便于理解我要讲的故事,我先在此概述一下他的履历。下面我要讲的一切,都是后来我从大名鼎鼎的小提琴家B那里知道的,他是我继父年轻时的伙伴和密友。
\par 我的继父姓叶非莫夫。他出生于一位非常有钱的地主的村庄,继父的父亲是个穷乐师,度过漫长的漂泊生涯之后在这位地主的庄上落了户,受雇假如他的乐队。这位地主生活极其阔绰,平生最爱音乐,而且爱的成瘾。据说,他从来不离开自己的村庄,连莫斯科也不去,可是有一回突然出国到一处矿泉疗养地去了,而且只去了几个星期,唯一的目的就是去听一位赫赫有名的小提琴家的演奏,因为报上说他要在那个疗养地举行三场音乐会。这位地主拥有一个相当不坏的私人乐队,他几乎把全部收入都花在这上头。我的继父刚进这个乐队时吹单簧管。他在二十二岁那年结识了一个奇怪的人物。在他们那个县份,住着一位富有的伯爵,可是他为了养一个私人戏班子不惜于倾家荡产。这位伯爵因为自己的意大利出生的乐队长行为不端而把他辞退了。乐队长的品性确实不好。他被解雇以后,更是潦倒不堪,老是在乡下小酒店里喝的酩酊大醉,有时候索性祈求施舍,全省谁也不愿意给他一个职位。我的继父竟跟这样一个人交了朋友。这种奇怪的交往实在难以解释,因为谁也看不出我的继父由于学朋友的样在行为方面有什么变化,甚至起初不准他跟那个意大利人厮混的地主,后来对他们的友谊睁一只眼闭一只眼。最后,乐队长突然死了。他是清晨被农民在堤坝旁边的水沟里面发现的。经过验尸,确定他死于中风。他的遗物存放在我的继父那里,我继父当即出示文件,证明他有充分的权力继承这些遗物,因为死者留下一张亲笔所写的字条,指定叶菲莫夫为自己遗产的继承人。遗产包括一件黑色燕尾服和一把小提琴:燕尾服由死者保存的很仔细,因为他始终抱有觅得一席之位的希望;小提琴看上去却很平常。没有人对这比遗产提出什么争议。可是过了若干时日,伯爵乐队里的首席小提琴手带着铂爵的信来见地主。伯爵在信上于叶菲莫夫情商,却他让出意大利人身后留下的那把提琴,因为伯爵很想把它买下来给自己的乐队使用。伯爵愿意出三千卢布,还说已派人去请过叶果尔·叶菲莫夫多次,以便当面了结这笔交易,但他执意不肯。伯爵最后写道,提琴货真,但他的价钱也实足不假,绝不会少一个子儿,并认为叶菲莫夫的顽固是一种多疑的表现,生怕成交是欺负他老实和外行,所以伯爵动了气,请地主开导开导叶菲莫夫。
\par 地主派人把我继父叫去。
\par “你为什么不肯出让提琴?”他问道。“你又用不着它。人家出你三千卢布,这是十足的价钱,要是你以为人家会出更高的价钱,可就错了,伯爵不会欺骗你的。”
\par 叶菲莫夫回答着说,他自己不想去见伯爵,如果要他去,那就只能按主人的意志办;提琴他不愿意卖给伯爵,如果硬要从他这里把提琴抢走,那也只能按住人的意志办。
\par 很明显,他还祥的回答触到了地主性格中最敏感的一根弦。事情是这样的. 地主一向自豪地说他懂得怎样对待他的乐师, 因为他们个个都是真正的艺术家,因此他的乐队并不但笔伯爵的高明,甚至同京城里的乐队相比也不逊色。
\par“好!” ,地主说。“我通知伯爵, 说你不愿卖琴就是不愿, 因为卖与不卖的权利完全在你,懂呜?不过我要问你:你要提琴干嘛? 你的乐器是单簧管. 虽然你的单簧管吹得相当蹩脚。把琴让给我吧。我出三千卢布。(谁知道此琴是这样的一件乐器!)”
\par 叶菲莫夫莞尔一笑。
\par “不,老爷,我不卖给您,”他答道,“当然,您可以······”
\par “难道我在逼你,难道我在强迫你?!”地主终于沉不住气叫了起来,偏偏事情是当着伯爵的乐师之面发生的,来者据此情景可能推断:地主对他的乐队的全体乐师都不给好看待。“滚开,没良心的东西!从今以后别让我再看见你!要是没有我,冲你那支吹得这样糟糕的单簧管,你能上哪儿混饭吃?你在我这里有吃有穿,还领薪俸;你过的是上等人的日子,把你当艺术家看待,可你根本不想明白这一点,简直无知无觉。滚开,别待在此地招我生气!”
\par 地主总是把他生气的对象从自己身边赶走,因为对自己不放心,怕他的火爆性子发作。而他是说什么也不愿意对他的“艺术家”过于严历的(他管自己的乐师们都叫“艺术家”)。
\par 买卖没有成交,事情似乎到此为止了,不料一个月以后,伯爵的小提琴手忽然大启讼端:他本人出首告发我继父应对意大利人之死负责,说我继父怀着自私的目的把他致于死地,为的是占有价值可观的遗产。伯爵的乐师声称遗嘱是在硬逼软骗之下写出来的,并表示能为这项指控提供人证。伯爵劝之再三,地主也为我的继父说情,但什么也不能动摇告发者的主意。人家把情况摊在他面前:法医对乐队长尸体所做的检验是正确的,硬要告发岂非违背明摆着的事实,也许是因为得不到曾经为他洽购的那件名贵乐器而怀恨在心,想泄私愤。伯爵的乐师一意孤行,还赌神罚咒地说自己是对的,说中风并非酗酒引起,而是中毒造成的,要求复查。乍看起来,他的论点似乎颇有道理。不用说,事情还是闹开了。叶菲莫夫被抓起来关进城里的监狱。这桩官司打起来以后,引起了全省的注意。案件进展很快,结果查明伯爵的乐师犯有诬告罪。判决给了他应得的惩罚,但他始终坚持自己的看法是对的。最后,他承认自己没有任何证据,他提出的论点是他自己臆造的,但他捏造所有这些事实是根据一种假设、一种猜想行事的,因为直到复查结束、正式确认叶菲莫夫先生无罪为止,他还坚信不幸的乐队长之死是叶菲莫夫造成的。不过,对此人的判决没有来得及执行,因为他突然患脑炎发了疯,接着就死在监狱医院里。
\par 整个这桩公案的始末,地主的行为是光明磊落的。他为我继父出力,仿佛我继父是他的亲生儿子似的。他曾几次到监狱里去探望我的继父,安慰他,给他钱;得悉叶菲莫夫喜欢抽烟,就给他带去最好的雪茄;宣告我继父无罪释放时,他让乐队全体成员大事庆贺。地主把叶菲莫夫这个案子看作是关系到整个乐队的事情,因为他对自己的乐师的品行即使不比他们的才能看的更重,至少是不相上下。过了整整一年,忽然有消息在省里传开,说是一位著名的法国小提琴家途径省城,打算举行几场音乐会。地主马上开始设法请他来做客。事情进行的很顺,法国人答应前来。对他的来临已经做好一切准备,还邀请了县里几乎所有的知名人士,不料情况陡起变化。
\par 一天早晨,有人来报告说,叶菲莫夫不知去向。开始到各处去寻找,可是杳无踪影。乐队缺了一只单簧管岂不急死人?在叶菲莫夫失踪三天之后,地主忽然接到法国人写来的信,那位小提琴家在信中傲慢地拒绝了地主的邀请,并且表示(当然是拐弯抹角地),今后在跟那些私人拥有乐队的老爷们打交道时要格外谨慎,看到真正的天才处在根本不知道其价值地人监督之下实在太杀风景,临了还说,叶菲莫夫的例子足以证实他所言不虚,此人是位真正地艺术家,是他在俄国遇到的最好的小提琴演奏家。
\par 地主读了这封信大为惊愕。他这一气直气的他发昏章第十一。什么?他对叶菲莫夫如此关怀备至,如此厚爱有加,而就是这个叶菲莫夫竟昧着良心向一位欧洲的艺术家,向他高度敬重其见解的人如此恶毒地诋毁他!此外,这封信在另一点上也是令人不解的;信中说,叶菲莫夫是个有真正天才的艺术家,他是位小提琴家,可是人们竟不能发现他的才华,强迫他演奏另一种乐器。这一切使地主惊讶万分,他当下准备进城去面晤法国人,忽然伯爵派人送来一封便简,请地主立刻到他那里去,并说他了解全部情况,那位路过的演奏家此时在他家里,叶菲莫夫也在,伯爵闻说叶菲莫夫的诽谤后大为震惊,已下令不准他离开,信上最后说,之所以必须请地主前去,还因为叶菲莫夫的指责甚至涉及伯爵本人;兹事体大,必须尽快加以澄清云云。
\par 地主马上赶往伯爵家中,随即同法国人见面,向他介绍了我继父的全部身世,并说他没料想到叶菲莫夫竟有这般了不起的才华,相反,叶菲莫夫在他那里只是一个很蹩脚的单簧管手,现在他头一回听说,这个离他而去的乐师竟是位小提琴家。地主还说,叶菲莫夫并不是农奴,他享有完全的自由,如果他确实受到束缚,任何时候都可以离开他家。法国人感到很奇怪。他们把叶菲莫夫叫来,他简直跟过去判若两人:态度傲慢,答话带着讥笑,并坚持自己向法国人所说的情况属实。这一切使伯爵恼怒到了极点,他当面骂我的继父是混蛋、造谣中伤的小人,应该得到最可耻的下场。
\par “请放心,伯爵大人,敝人跟阁下并非初次相交,对阁下颇有了解,”我的继父说,“多蒙阁下恩典,在下几乎受到刑事处分。敝人知道,阿列克塞·尼基福雷奇——府上过去的乐师——告发在下是受了何人的嗾使。”
\par 听到这样骇人听闻的责难,伯爵肺都快气炸了。他好不容易才控制住自己;但此时厅堂里一位有事来见的伯爵的官员宣称他不能听任这一切不了了之,说叶菲莫夫这种侮辱性的物理态度包含着恶毒的、不公正的指控和污蔑,所以他谨请允许他在伯爵府第里立即逮捕叶菲莫夫。法国人也表示极度的慷慨,说他无法理解这种丧尽天良的负义行为。于是我的继父暴跳如雷地回答,他宁可接受处分、审判,哪怕再来一次刑事侦讯,也强似迄今为止他在地主乐队里过的那种日子,由于他极度贫困,不能更早离开那里;说完,他就跟逮捕他的官员一起走出厅堂。他被锁在一间偏僻的屋子里,说是明天把他押送进城。
\par 将近午夜时分,拘押我继父物的屋子门打开了。进来的是地主。他穿着睡袍,趿着拖鞋,双手打着一盏点亮的灯笼。看来他睡不着,痛心的焦虑迫使他在这般时分离开床衾。叶果尔也没睡,他惊讶地望着进来的地主。地主把灯笼放好,怀着十分激动的心情坐在他对面的一把椅子上。
\par “叶果尔,”他对叶菲莫夫说,“你为什么要这样伤我的心?”
\par 叶菲莫夫不答。地主又问了一遍,他的话流露出某种深刻的感情,一种奇怪的忧伤。
\par “天知道我为什么要这样伤您的心,老爷!”我继父终于一甩手答道。“想必是鬼迷住了我的心窍!我自己也不知道是谁推动着我这样干!反正我不能在您那儿再待下去,不能再待下去······。魔鬼把我缠住了!”
\par “叶果尔!”地主又开言道。“回到我那儿去吧,我把一切都忘掉,什么都原谅你。听着:你可以当我的首席乐师;我给你一份跟旁人不一样的薪金······”
\par “不,老爷,不,您别说了:我不能在您那儿待下去!我告诉您,魔鬼缠上了我。我要是再待下去,会放火烧掉您的房子;有时候我苦闷得只恨爹娘不该把我生下来!现在我自己也不能担保,老爷,您还是别管我吧。这都是从那个魔鬼跟我结交后开始的······”
\par “谁?”地主问。
\par “就是那个像狗一样咽了气的意大利人,这条狗导出都不受欢迎。”
\par “这么说,叶果鲁什卡,是他教你拉琴的喽?”
\par “是的!他教会了我许多东西,把我引向毁灭。我还是从来没有见到他的好。”
\par “难道他是个小提琴高手,叶果鲁什卡?”
\par “不,他自己知道的不多,可是教的挺好。我是自己学会的;他只不过是示范而已,——这比正规的办法容易,要是我撒谎,就让我这支胳臂烂掉。现在我自己也不知道要什么。老爷,您问我:‘叶果尔卡!你要什么?我什么都能给你,’——可是,老爷,我一句话也没法回答您,一位我自己不知道要什么。老爷,您还是别管我吧,下次再说。我要对自己干一件惊人的事情,好让我被远远地打发走,事情才能了结!”
\par “叶果尔!”地主在沉默片刻后说。“我不能这样撇下你不管。既然你不愿意在我那里干下去,你可以走;你是自由人,我不能强留;但我现在不肯就这样离开你。你用你的琴拉一首曲子给我听听,叶果尔,拉吧!看在上帝的份上,拉一首!我不是命令你,你要明白我的意思,我不强迫你;我含着眼泪请求你:叶果鲁什卡,看在上帝的份上,把你为法国人演奏的曲子也拉给我听听!吐一吐你的心曲!你很固执,我也很固执;要知道我也有自己的脾气,叶果鲁什卡!我心中是有你的,你心中也该像我一样才对。除非你自愿且乐意地把为法国人演奏的曲子拉给我听,否则我日子没法过。”
\par “行,那就照办!”叶菲莫夫说。“老爷,我发过誓永远不在您面前演奏,单单不为您演奏,可现在我心上的束缚解除了。我可以为您演奏不过这是第一次,也是最后一次,老爷,以后您在任何时候、任何地方再也听不到我的演奏,哪怕向我许一千卢布的愿也没用。”
\par 于是他拿起琴来,开始演奏他自己所作的俄罗斯民歌变奏曲。据B说,这套变奏曲是他的第一部、也是写的最好的小提琴作品,此后他任何曲子从来都没有演奏的这样好,这样充满灵感。这位地主听音乐本来就没法不动心,这回索性放声痛哭。演奏完毕时,他从椅子上站起来,掏出三百卢布交给我的继父,说:
\par “现在你走吧,叶果尔。我把你从这里放出去,伯爵那一头我去对付;不过你听着:往后你可别跟我遇上。你面前的道路宽又广,万一咱俩在路上碰头,对我对你都不好受。好啦,那就分手吧!······等一下!我对你还有一句临别赠言,很简单:别喝酒,要用功,莫骄傲!我是像你的亲生父亲一样对你说话。注意,我重复一下:要用过,别喝酒,一旦你开始借酒浇愁(叫人愁苦的事将来多的很!)——那就前功尽弃,非完蛋不可,也许你自己会像那个意大利人一样在不定什么地方的水沟里咽气。好啦,现在分手吧!······等一等,吻我一下!”
\par 他们相互吻了一下,随后我的继父就被放出去。
\par 他刚获得自由,先是立即在附近一个县城里把三百卢布胡乱花光,同时跟一帮最堕落、最下流的无赖交上朋友,后来一个人落得穷愁潦倒、孤苦无依,不得不加入一个小地方的流动戏班子,在可怜巴巴的乐队里拉第一个小提琴,或许是唯一小提琴。这一切跟他原先的设想不太吻合,他本来打算尽快到彼得堡去深造,谋到一个好位子,不折不扣地把自己造成一个艺术家。但是,小乐队里的生活很不如意。我继父不久就跟江湖戏班子的班主闹翻,并且离开了那里。那时他彻底泄了气,甚至不顾一切走出了深深刺痛他的自尊心的一步。他写了一封信给前面提到的那位地主,向它描述了自己的境况,请他资助。信的语气还相当要面子,但是没有回音。于是他再写一封,措辞上极尽卑躬屈膝之能事,称地主为自己的恩人,把他尊为真正的艺术鉴赏家,目的还是求他帮助。回音总算来了。地主寄来一百卢布和由他贴身侍从代笔的寥寥数行复信,叫叶菲莫夫今后再也不要向他提出任何请求。继父得到了这点钱,当即想动身去彼得堡,可是付账还债以后,只剩下那么一点点钱,彼得堡之行根本无法考虑。他仍旧留在小地方,重新加入一个小地方乐队,后来在那里又待不下去,如此不断地换来换去,心中念念不忘能快一点去彼得堡,其实却在小地方跑了整整六年。后来,他忽然大起恐慌。他绝望地发现,在不规则的、贫困的生活不断折磨下,他的才华遭到了不知多大的损失,于是在一个早上抛下班主,拿起提琴,几乎靠乞讨走到彼得堡。他在某处的一个楼顶上住下,在那里第一次遇到了B,彼时B刚从德国来,也想为自己开辟前程。他们很快就交了朋友,B至今还满怀深情回忆他们相交一场。他俩都是青年,怀抱一样的希望,有着相同的目标。但B的青春还刚刚开始,他经历的贫困和苦痛尚少;撇开这些不谈,他首先是一个日耳曼人,在奔向目标的道路上能坚持不懈、持之以恒,充分意识到自己的力量,并且对于自己能有多大的作为几乎早有成竹在胸。可是他的伙伴叶菲莫夫已经三十岁;他已经疲倦、困乏,整整七年不得不东飘西泊,在小地方的戏班子和地主的私人乐队里混口饭吃,耐心既完全丧失,最初旺盛的精力也消耗殆尽。过去支撑着他的只有一个永远不变的固定观念——好歹得摆脱窘境,积一笔钱到彼得堡去。但这个观念事模糊的、朦胧的;这是一种不可违拗的内心的召唤,随着岁月的流逝,这呼声在叶菲莫夫心中也不象最初那样清晰了,当他来到彼得堡时,几乎已经处于无意识状态,只是按照夙愿和反复思量这次进京的老习惯行事而,几乎连自己也不知道要在京城里干什么。他的热情近似歇斯底里,带有怄气和阵发的性质,似乎他想用这种热情欺骗自己,借以使自己相信,他身上最初的精力、最初的热情、最初的灵感尚未枯竭。这股不断迸发的劲头个不冷不热、有条有理的B震动很大;他感到目眩神迷,把我继父当作未来伟大的天才音乐家看待。他不能想象这位伙伴将来的命运会是什么别的样子。但不久B就睁开眼睛把他看透了。他清楚地看到,所有这些阵发地狂热、焦躁的情绪无非是想到自己怀才不遇而不自觉地表现出来的绝望的挣扎;说到底,甚至他的才也许一开始就没有什么了不起,而多是盲目和不切实际的自信、浅薄的自负以及不断幻想自己是盖世奇才的白日梦。“但是,”B如此说,“我对我这位伙伴奇异的天性不能不表示惊讶。我眼看着在病态的强烈欲望与内心的软弱无力之间不断进行凶猛的生死搏斗。这个不幸的人整整七年光靠幻想将来成名聊以自慰,甚至在不知不觉中丢掉了我们这些技艺中最起码的东起,甚至丧失了最基本的业务能力。偏偏在他乱糟糟的想象中无时无刻不在为未来构思气吞山河的宏大计划。他不唯要成为第一流的天才,成为世界上数一数二的小提琴家;他不唯已经把自己当作这样的天才,——他还想成为作曲家,事实上他根本不懂得对位法。但最使我惊讶的是,”B又说,“这个人尽管绝对无能,尽管在技艺方面只是极其贫乏,然而他对艺术却有非常深刻、非常明晰、可以说是出于本能的立即。他的艺术感和鉴赏力是那样高超,无怪乎会失去自知之明,看不到本能决定他是个深刻的艺术批评家,却把自己当作艺术大师、天才演奏家。有时候他用毫无学术味道的粗言俗语能对我说出极其深刻的道理,使我大惑不解:他从来不看书报,什么也不学,可是这一切他是通过什么方式领悟的呢?我在自修过程中,”B接着说,“有许多地方得益于他和他的指点。至于我本人,”B继续说,“我对自己的看法是稳定的。我也热爱本行艺术,虽然从我一开始走上这条道路就有自知之明,知道在某种意义上我只能当一名艺术的苦力;但我感到自豪的是没有象一个懒惰的奴隶那样把自己的拿点天赋埋没掉,相反把它扩到了一百倍;如果说,现在人们夸奖我演奏时干净利落,惊叹功夫到家,那末,这一切都得归功于坚持不懈的努力,归功于有自知之明,宁可把自己看的渺小,永远力戒骄傲,力戒过早的自满,力戒懒惰,因为懒惰是这种自满情绪的必然结果。”
\par B也曾尝试向自己最初非常佩服的伙伴进几句忠言,但结果只是白白惹他生气。他们之间的关系开始疏远了。不久B注意到,淡漠、苦闷和无聊开始愈来愈频繁地控制叶菲莫夫,而他热情冲动的次数则愈来愈少,过后出现的是一种阴沉凄凉、灰心丧气的状态。再后来,叶菲莫夫干脆把琴放下,有时一连几个星期不去碰它。这与彻底的堕落已相去不远,很快,这个不幸的人染上了所有的恶习。当初地主告诫他的事情果然发生了,他开始纵酒无度。B瞧着他无法不感到震惊,他的忠告不起作用,他甚至不敢开口。渐渐地,叶菲莫夫落到恬不知耻的地步:他竟心安理得地花B的钱过日子,这样做甚至好像有充分权力似的。其时维持生活的钱行将告罄;B靠教课竭力苦撑,或者受雇在商人、日耳曼人、小官吏家的晚会上演奏,尽管报酬很少,但他们总能给几个钱。叶菲莫夫根本不愿看到伙伴的难处,对他疾言厉色,有时几个星期不跟他说一句话。一次,B用及其婉转的语气劝他不要过于轻视那把琴,以免对此乐器完全荒废;不料叶菲莫夫大发脾气,声言他将故意永远不去碰自己的琴,那副架势好像会有人跪下来求他似的。另一次,B在一个晚会上演奏需要有个搭档,就请叶菲莫夫跟他合作。这一邀竟使叶菲莫夫暴跳如雷。他怒气冲冲地宣称自己不是街头提琴师,不会像B那样卑鄙,在完全不能赏识他的功夫和才华地臭商人面前拉琴,贬低崇高的艺术。B听罢没回答一句话,出门演奏去了,但叶菲莫夫在伙伴走后对这次邀请反复思考,认为这一切无非暗示他在花B的钱过日子,想让他知道,叫他尝试挣点钱。等B回来以后,叶菲莫夫忽然斥责他行为卑鄙,并表示一分钟也不能跟他待在一起。他确实有两天不知去向,但第三天回来了,仿佛什么也没有发生似的,又继续过原来地那种生活。
\par B本来想结束这种不像话的生活,跟他的伙伴一刀两断,仅仅由于旧的习惯和友谊,加上B瞧着那个堕落的人心中老大不忍,才没有这样做。后来,他们终于分道扬镳。命运向B作了微笑:他找到一座强有力的靠山,成功地举行了一场出色的音乐会。那时他已经是个优秀的艺术家,他那蒸蒸日上的名气旋即给他带来歌剧院乐队里面的一个席位,在那里他很快就取得了完全应该取得的成功。分手的时候,他给了叶菲莫夫一些钱,含着眼泪恳求他回到正路上来。直到现在,B想起他来还是抑制不住一种特殊的感情。跟叶菲莫夫相交一场是他青年时代最深刻的印象之一。他们曾一起开始向自己的目标进军,彼此曾有过非常热烈的好感,叶菲莫夫的乖张性格和十分显著的缺点本身甚至使B对他产生更加强烈的感情。B了解他;他把叶菲莫夫看得透亮,事先就知道这一切将以上面告终。分袂之际,他们相互拥抱,两人都哭了。当时叶菲莫夫流着眼泪呜呜咽咽地说,他是个彻底完蛋的,最最不幸的人,这一点他早就知道了,但现在才看清自己的末路。
\par “我没有才华!”他的除了结论,说时面无人色。
\par B大大地为之心动。
\par “听着,叶果尔·彼得罗维奇,”他对我继父说,“你何苦自暴自弃呢?你的绝望只能毁了你自己;你既没有耐性,又没有勇气。刚才你在灰心丧气的情绪控制下说自己没有才华。不对!你有才华,你可以相信我的话。你有才华。单凭你对艺术的感受和理解,我就看出这一点。我可以用你的全部生活向你证明这一点。你不是把你过去的经历告诉过我吗?当初你也曾不自觉地陷于同样绝望地境地。那时,你的第一位老师——你曾经对我讲过他很多故事的那个怪人——最先在你身上激发起对艺术的爱,最先察觉到你的才华。当时你也产生了强烈而痛心的感觉,就跟现在一样。但你自己不知道你是怎么搞的。你在地主家里待不下去,你自己也不知道你到底要什么。你的老师死得太早。他撇下你的时候你只有一些模模糊糊的志向,主要的是没有使你认清自己。\textcolor{green}{你觉得你需要走另一条更宽广的路,应该向另外的目标进军,可是你不懂得该如何去达到目的,于是在苦闷中痛恨当时你周围的一切。你六年贫困的岁月没有虚度;你在学,在想,在认识自己和自己的能力,现在你对艺术,对自己的使命有了理解。我的朋友,需要耐性和勇气}。等待着你的命运要比我的更值得羡慕:你的艺术家气质超过我一百倍,只要上帝把我的耐性的十分之一赐给你就够了。\textcolor{blue}{要用功,别喝酒,正像那位好心的地主对你说过的那样,主要的是你得重新起步,从头开始。什么东西使你苦恼?贫困?但贫困能造就艺术家。事业的起点总是和穷字分不开的。现在还没有人把你放在眼里,谁都不愿意认识你;人世间就是这样。你等着,一旦人们知道你有才能,可气的事情还不止这些呢。妒忌、卑劣的小心眼儿、特别是种种荒唐的蠢事将比贫穷更加变本加厉地往你身上压下来}。\textcolor{red}{才华需要同情,需要有人理解,可是你知道取得一点点成就,你会看到,包围着你的将是些什么样的面孔。他们会把你靠艰苦的劳动、咬紧牙关、挨饿熬夜练出来的功夫说的一文不值,对你嗤之以鼻。你未来的伙伴们不会鼓励你、安慰你;他们不会向你指出你的真善美,但会幸灾乐祸地挑你的每一处毛病,偏偏把你不好和不对的地方指向你,表面上冷冰冰地瞧不起你,心里却象过节一样庆祝你犯的每一个错误(好像有人能不犯错误似的!)。}你生性傲慢,往往在不适当的场合逞骄,可能会得罪自尊心很强的小人,那就糟了——你只有一个人,而他们人多;他们会像针刺那样折磨你。甚至我也已经开始有此感受。你得立刻振作精神!你还不算太穷,你的日子能过下去,不要嫌弃粗活,有柴就劈,象我在不足道地生意人晚会上劈柴一样。但你缺乏耐性,你有急躁的毛病;你不够朴实,耍小聪明太多,想得太多,脑筋动得太多;你口头上大言不惭,临到需要拿起琴弓地时候又胆怯了。\textcolor{red}{你自尊心太强,胆量又太小。勇敢一些,耐性一些,耐心等待,学着点儿,如果你不寄望于自己的力量,那就碰碰运气;你身上有激情,有感觉。也许碰运气能达到目的,即使达不到,也不妨碰碰运气,反正不会失去什么,因为奖赏实在太大了。}这个意义上说,老兄,咱们的运气是了不起的大事。”
\par 叶菲莫夫怀着很深的感情听过去伙伴这番话。但在B往下说的过程中,叶菲莫夫苍白的脸色逐渐泛红,两腮出现了血色;他的眼睛里燃起了少有的勇气和希望。不久,这种高尚的勇气转换为自负,接着又变成平时那份狂妄,最后,当B快要结束这番规劝的时候,叶菲莫夫已经心不在焉,听得不耐烦了。不过他还是热烈地和B握手,向他表示感谢,并且拿出从自暴自弃和灰心丧气迅速跃向极端傲慢狂妄那种老脾气,用过于自信的口气叫他的朋友不必为他的命运操心,说他知道该如何安排自己的未来,他希望不久也能找到一座靠山,举行音乐会,那时便可一下子名利双收。B耸耸肩膀,但没有给过去的伙伴泼冷水他们就此分手,虽然分别的时间不长——这是不言而喻的。叶菲莫夫把给他的钱立刻花的精光,又去要了第二次,然后是第三次,然后是第四次······第十次,最后B实在忍不住了,便推说不在家。从此叶菲莫夫下落不明。
\par 几年过去了。有一次B排练归来,在一条陋巷里肮脏的小酒店门口碰见一个衣衫褴褛的醉汉再叫他的名字。那人是叶菲莫夫。他大大变了样,脸上浮肿、发黄;显然,放荡的生活在他身上打下了不可磨灭的烙印。B非常高兴,没说上两句话,就被他拖进小酒店。到了那意见偏僻、乌黑的小屋子里,B才看清楚他的摸样。叶菲莫夫的衣着几乎全是破烂,一双靴子坏得不像话。衬衣的前胸沾满酒污。他的头发开始斑白、脱落。
\par “你怎么啦?你现在哪里?”B问。
\par 叶菲莫夫窘态毕露,起初甚至有些发慌,语无伦次,答非所问,以致B以为他神经错乱了呢。后来,叶菲莫夫承认,不喝一点伏特加他就没办法说话,可是小酒店里对他早就不相信了。他这样说时脸是红的,尽管竭力做一些豪放的手势给自己壮胆;但造成的印象却是厚颜无耻、娇柔做作,简直令人不忍卒睹;善良的B看到自己原先的忧虑果然完全成为事实,又动了恻隐之心。他先吩咐把伏特加端上来。叶菲莫夫由于感激脸上顿时变样,他激动得含着眼泪准备吻他恩人的手。进餐的时候、B惊讶万分地得悉这个不幸的人结了婚。但更使他惊讶的是了解到,妻子竟然构成了叶菲莫夫的全部灾难和悲哀,结婚彻底摧残了他的才华。
\par “怎么会的呢?”B问。
\par “老弟,我已经两年没拿起提琴了,”叶菲莫夫答道。“简直是个村妇、厨娘、毫无教养的粗俗女人。别提她了!······我们一天到晚打架,旁的什么也不干。”
\par “既然这样,你为什么要结婚?”
\par “当时没东西吃啊。我结识了她:她有千把卢布,我就不管三七二十一结了婚。是她爱上我的。她自己缠住我不放。又没人撺掇她!钱花完了、喝光了,老弟,哪儿还有什么才华!全都落了空!”
\par B看得出,叶菲莫夫似乎急于向他表白自己没有过错。
\par “我把一切扔下了,”叶菲莫夫又说。他向B表示,在这以前他的琴艺几乎达到了完美的境界;虽然从我一开始走上这条道路就有自知之明B是全城数一数二的小提琴家,可是跟他比起来还差得远呢,如果叶菲莫夫愿意的话。
\par “那你为什么不干呢?”B诧异地问。“你该找份事情做啊!”
\par “没意思!”叶菲莫夫一甩手说。“你们那里有谁懂得一点点皮毛?你们懂得什么?懂个屁!你们只会从什么芭蕾音乐中抽一段五去来胡闹一通。优秀的小提琴家你们既没有见过,也没有听过。何必去碰你们呢;你们爱怎么干,就怎么干吧!”
\par 说到这里,叶菲莫夫又把手一甩,身体在椅子上一晃,因为他已经醉得相当厉害。后来他邀请B上他家去;但是B谢绝了,只问了他的住址,答应明天就去看他。叶菲莫夫此刻酒醉饭饱,已经用嘲弄的目光瞧他过去的伙伴,千方百计用话刺他。他们离座起身时,叶菲莫夫连忙把B的贵重皮裘递给他,做出以卑事尊的样子。经过第一间屋子,他就停下来向酒店里的掌柜、伙计和客人介绍B是全城首屈一指和独一无二的小提琴家。总而言之,此时此刻他的表演十分令人恶心。
\par 第二天上午,B在顶楼上找到了他,当时我们全家就住在那么一间屋子里,过着极度穷困的生活。我那时只有四岁,可是我母亲改嫁叶菲莫夫已有两年。我母亲是个不幸的女人。从前她当过家庭教师,受过很好的教育,长得也漂亮,可是由于穷,嫁给了一个老公务员,也就是我的生父。他们在一起只过了一年。我的生父突然去世以后,为数无多的遗产由他的各个继承人瓜分,留下母亲和我,还有就是她分得的一点点钱。带着一个还不会走路的婴儿再去当家庭教师谈何容易。这时,她在一个偶然的机缘下遇到了叶菲莫夫,的确爱上了他。我母亲富于热情和幻想,把叶菲莫夫当作了不起的天才,相信了他前程似锦的那些自命不凡的话,想到有幸成为一个天才的精神支柱和生活指导,我母亲得意非凡,就嫁给了他。不出一个月,她的理想和希望统统化作泡影,她面前留下的只是寒碜的现实。叶菲莫夫跟我母亲结婚也许确实看在她有千把卢布这一点上,等到钱一花完,便叉起两支胳膊,似乎欣然找到一个借口,立即向所有的人宣称,结婚毁了他的才华,说他在窄闷的屋子里面对饥饿的一家子没法工作,说在这样的条件下头脑里不可能产生歌曲和音乐,看来他命中注定得受这份罪云云。后来,好像他自己也确信自己的牢骚发得有理,似乎还为有了新的口实感到高兴。看样子,这位不幸的、毁掉的天才自己在寻找表面的理由。以便把所有的挫折、所有的灾难一古脑儿推到那上头。至于接受这样一个可怕的念头:他在艺术上早已经完蛋,而且是永远地完蛋了——他做不到。他象抵抗恶梦似地向这个可怕的结论作拼死的挣扎,及至现实把他压倒、使他偶尔有儿分钟睁开眼睛的时候,他觉得自己马上就会吓得发疯。对于长期以来构成他全部生活的信念,他不能这样轻易地放弃,直到最后一分钟依然认为那一分钟还没有过去。在彷徨的时刻,他就借熏人的酒味浇胸中之块垒。可能,他自己也不知道这时候妻子对于他是多么需要。这是一个活的借口,的确,我继父对这样一个设想几乎着了魔,认为一旦他把坑了他的妻子葬入坟墓,一切都将走上正轨。可怜的妈妈不理解他。作为一个十足的幻想家,在冷酷无情的现实中刚迈出第一步,她就受不了,变得暴躁易怒,动辄骂人,时刻跟存心折磨她取乐的丈夫吵架,不断逼着他工作。但是,我继父的盲目自大、固定观念和荒谬见解使他变得几乎毫无心肝和麻木不仁。他只是笑着赌咒决不拿起提琴,除非妻子死去,并以狠心的直率态度向她宣布这一决心。不管出现什么情况,妈妈对他的炽热的爱至死不渝,可是这样的生活她忍受不了。她变得老是生病、多灾多难,身心处在没完没了的痛苦之中,陈了这一切不幸之外,一家人的吃饭问题全要她一个人操心。她开始做菜,承接包饭。但是丈夫偷偷地从她那儿把钱都弄走,逼得她常常开不出人家包的饭。当B来访的时候,她正在洗衣服,并且把一件旧的连衫裙重新染色。我们就一直这样在我们的顶楼上凑合着过。
\par 我们家的贫穷使B吃惊。
\par “你呀,完全是胡说八道,”他对我继父说,“这儿哪有什么摧残才华的事情?她明明在养活你,你在干什么?”
\par “什么也不干!”继父回答说。
\par 但B还不知道妈妈所有的灾难。丈夫经常把形形色色捣乱撒野的浪荡鬼一大帮一大帮地带到家里来,那时真可以说是无所不为!
\par B对他过去的伙伴劝说了半天,最后表示,如果叶菲莫夫再不改过,他决不提供任何帮助,还直截了当地说不给他钱,因为他又会把钱喝光,临了他要叶菲莫夫用提琴拉点儿什么给他听听,看看能为叶菲莫夫想些什么办法。等我继父去取琴的时候,B悄悄地要把钱给我母亲,可是她不拿。她这是第一次落到接受施舍的地步!于是B就把钱交给我,可怜的妈妈哭了。继父取了琴来,但要求先给他一点伏特加,说否则没法演奏。于是就派人去买伏特加。酒喝下去以后,劲头就上来了。
\par “看在朋友交情上,我给你拉一首我自己的作品。”他对B说,并从柜子底下拖出厚厚一本尘封的练习簿。
\par “这些都是我自己写的,”他指着本子说。“你不妨瞧瞧!老弟,这可不是你们的那些芭蕾音乐!”
\par B默默地翻阅了几页,然后打开他自己带着的乐谱,要我继父把自己的作品放在一边,从他带来的乐谱中拉几段给他听听。
\par 继父有些动气,不过由于担心失去新的靠山,还是照B的吩咐做了。于是B发现,在他们分手期间,他从前的伙伴确实下了很多功夫,进步不小,虽然他吹说打从结婚以后就没有拿过琴。我那可怜的母亲高兴得什么似的。她瞧着丈夫,重又为他感到自豪。善良的B也衷心欣喜,决定设法安置我继父。那时B已经认识许多有地位的人物,他当即开始去托人情,向人推荐这位可怜的伙伴,但事前要他保证好自为之。B先掏钱把他的衣着搞得象样些,带着他去见几位名人,因为B打算给他谋的位子关键在那些人物身上。叶菲莫夫只是口头上神气十足,其实恐怕他万分乐意地接受了老朋友的建议。据B说,我继父拚命讨好他,生怕失去他的照应,那种阿谀奉迎、卑躬屈膝的丑态使B为他害臊。叶菲莫夫明白别人想把他引上正途,甚至酒也不喝了。后来,在一个剧团的乐队里总算给他谋到一个位子。他以良好的成绩通过了考试,因为一个月刻苦努力他就把一年半荒废的东西补了回来,并保证要继续练功和一丝不苟地对待新的职责。但我们家的境况毫无改善。继父的薪金他一个子儿也不给妈妈,全都自己花掉,跟他很快又交上的一批新朋友一起喝光、吃光。他结交的大都是剧团职员、合唱队员、龙套演员,总之都是些他可以在其中高踞首席的人,对于真正有才能的却避不交往。他已使那些新朋友对他产生某种特殊的敬意,刚认识就向他们宣传,他是个被埋没的人,他有伟大的天才,是妻子断送了他,并说他们的乐队指挥根本不懂音乐。他嘲笑乐队里所有的演奏员,嘲笑排演的剧目,还嘲笑上演的几部歌剧的作者。后来,他开始侈谈一种新的音乐理论,总之,使整个乐队都感到腻烦!他跟同事、指挥一一闹翻,对上司傲慢无礼,成了出名的一个最不安分、最爱争吵、同时又最卑琐的人,弄得人人讨厌。
\par 这样一个不足道的人,这样一个成事不足、败事有余的演奏员,偏偏这样踌躇满志,这样自命不凡、目空一切,确实是非常奇特的怪现象。
\par 最后,继父跟B也翻了脸;他编造了最拙劣的谣言当作明明白白的事实抛出去,对B进行极其恶毒的诽谤。他在乐队里鬼混了半年之后,终于因玩忽职守、行为失检被撵走。但他没有随即离开那个地方。不久,有人看到他身上又跟过去一样破破烂烂,因为比较像样的衣服又都变卖和抵押了。他去找过去的同事,不管人家欢迎不欢迎这样的客人,在他们面前散播流言,搬弄是非,哭诉日子难过,叫所有的人去看看他那恶魔般的妻子。当然,有人爱听,有人乐于给这位遭逐的同伴灌上几杯,让他大放厥词。何况他的嘴皮子总是那样尖刻,往往击中要害,而夹杂在话中的那股冲天的怨气和各种放肆的怪论,自有一部分人欣赏。他被当作一个神经失常的小丑,闲得无聊时不妨叫他胡言乱语扯上一通。人们喜欢当着他的面谈论某一位新来的小提琴家,故意逗他。叶菲莫夫听到这话,脸上就会变色,怯生生地打听是谁来了,谁是崭露头角的天才,并立即开始妒忌他的声誉。大概从这个时候开始,他才真正陷入经常性的精神错乱,也就是形成了一个固定观念:他是首屈一指的小提琴家,至少在彼得堡是无双的,但他命运多舛,怀才不遇,由于种种妒贤忌能的阴谋,他始终不为人所了解,至今没没无闻。他对末了这一点甚至颇为得意,因为有这样一些人就是喜欢把自己看作受欺压者,喜欢把抱怨挂在嘴上,或在暗中崇拜自己得不到承认的伟大聊以自慰。彼得堡所有的小提琴手他个个了如指掌,在他看来,其中没有一人能与他匹敌。凡是知道这个疯疯癫癫的可怜虫的,不管是行家还是逢场作戏的票友,都喜欢在他面前谈论某某才华出众的著名小提琴家,好让他也发表自己的看法。他们没有人能如此巧妙地用如此夸张的漫画手法描绘当代著名的音乐家。甚至遭到他这般挖苦的那些艺术家也有些怕他,因为知道他那张嘴够损的,承认他的指摘有根据,他的见解有道理,如果确实该骂的话。人们惯常在剧场的走廊里和后台看到他。职员们对他不加拦阻,因为少不得这样一个人,于是他成了个国产的忒耳西忒斯【荷马的史诗《伊里亚特》中人物,他别有用心地劝兵临特洛伊城下的希腊人不要打完仗就回去,通常用来比喻喋喋不休地喜欢跟大家抬扛的人。莎士比亚的历史剧《特洛伊罗斯与克瑞西达》也写了这个丑陋和好谩骂的希腊人。】。这样的生活持续了两三年,最后,他连扮演这样一个角色也惹得人人厌烦。接下来我继父遭到毫不含糊的驱逐,在他生前的最后两年似乎销声匿迹了,哪儿也看不到他。不过,B遇见过他两次,可是瞧着他那副狼狈相,在B的心中同情又压倒了厌恶。B招呼他,但我继父动了气,假装没听见,把一顶破旧不堪的帽子拉得遮住眼睛,打旁边走过去。后来,在一个不知什么大节日,一清早有人向B通报,说他以前的伙伴叶菲莫夫来向他拜节。B出去见他。叶菲莫夫喝得醉醺醺地站在那儿,开始行大幅度鞠躬礼,几乎碰到地上,嘴唇不住牵动,固执地不肯走进房间。他的举动的意思是说:我们这种没有才能的人,怎么能同您这样的名流交往,我们这种小人物能到此听差的所在,来向您拜个节,行个礼就走,已经心满意足。总之,这一切恶劣、无聊之至,令人作呕。此后B很久没有见到他,直至发生这幕惨剧结束他可悲的、病态的、堕落的一生为止。他的生命是以可怕的方式结束的。这幕惨剧不仅与我童年时代最初的印象紧紧连在一起,甚至和我的一生也有密切关系。事情是这样发生的……。但我首先必须说明,我的童年时代是怎么一回事,在我最初的印象中留下如此痛苦的痕迹、把我那可怜的妈妈逼死的这个人对于我又意味着什么。

\newpage
\section*{二}
\par
\par
\par
\par
\par
\par
\par
\par
\par
\par
\par
\par
\par
\par
\par
\par
\par
\par
\par
\par
\par
\par
\par
\par
\par
\par
\par
\par
\par
\par
\par
\par
\par
\par
\par
\par
\par
\par
\par
\par
\par
\par
\par
\end{document}